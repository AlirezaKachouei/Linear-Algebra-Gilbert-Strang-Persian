%%%%%%%%%%%%%%%%%%%%%%%%%%%%%%%%%%%%%%%%%%%%%%%%%%%%
%           LaTeX Code for Problem Set 1.1           %
%             Translated to Persian (Farsi)          %
%%%%%%%%%%%%%%%%%%%%%%%%%%%%%%%%%%%%%%%%%%%%%%%%%%%%

\documentclass[12pt,a4paper]{article}

% --- Preamble ---
% For Persian language support using XeLaTeX engine
\usepackage{xepersian}

% Set the main text font and digit font for Persian
% You can replace "XB Niloofar" with any other Persian font installed on your system (e.g., "B Nazanin")
\settextfont{XB Niloofar}
\setdigitfont{XB Niloofar}

% For mathematical symbols and environments
\usepackage{amsmath}
\usepackage{amssymb}
\usepackage{amsfonts}

% For better page layout
\usepackage[left=2.5cm, right=2.5cm, top=2.5cm, bottom=2.5cm]{geometry}

% Title setup
\title{ترجمه پاسخنامه مجموعه مسائل ۱.۱}
\author{صفحه ۸}
\date{\today}


% --- Document Body ---
\begin{document}
	\maketitle
	\RTL{ % Sets the text direction to Right-to-Left for the whole document
		
		\section*{مجموعه مسائل ۱.۱، صفحه ۸}
		
		\begin{enumerate}
			\item ترکیب‌ها (الف) یک خط در $R^3$ (ب) یک صفحه در $R^3$ (ج) تمام فضای $R^3$ را تولید می‌کنند.
			
			\item $v+w=(2,3)$ و $v-w=(6,-1)$ قطرهای متوازی‌الاضلاعی خواهند بود که $v$ و $w$ دو ضلع آن هستند که از مبدأ $(0,0)$ خارج می‌شوند.
			
			\item این مسئله قطرهای $v+w$ و $v-w$ یک متوازی‌الاضلاع را داده و اضلاع آن را می‌خواهد: برعکس مسئله ۲. در این مثال $v=(3,3)$ و $w=(2,-2)$ هستند.
			
			\item $3v+w=(7,5)$ و $cv+dw=(2c+d, c+2d)$.
			
			\item $u+v=(-2,3,1)$ و $u+v+w=(0,0,0)$ و $2u+2v+w=$ (با افزودن پاسخ‌های اول) $=(-2,3,1)$. بردارها $u$، $v$ و $w$ در یک صفحه قرار دارند زیرا ترکیبی از آنها بردار $(0,0,0)$ را نتیجه می‌دهد. به بیانی دیگر: $u=-v-w$ در صفحه‌ای است که توسط $v$ و $w$ ساخته می‌شود.
			
			\item مجموع مؤلفه‌های هر $cv+dw$ صفر است زیرا مجموع مؤلفه‌های $v$ و $w$ صفر است. $c=3$ و $d=9$ بردار $(3,3,-6)$ را می‌دهد. هیچ جوابی برای $cv+dw=(3,3,6)$ وجود ندارد زیرا مجموع $3+3+6$ صفر نیست.
			
			\item نُه ترکیب $c(2,1)+d(0,1)$ با مقادیر $c=0,1,2$ و $d=0,1,2$ بر روی یک شبکه قرار می‌گیرند. اگر تمام اعداد صحیح $c$ و $d$ را در نظر بگیریم، این شبکه کل صفحه را پوشش می‌دهد.
			
			\item قطر دیگر $v-w$ (یا $w-v$) است. جمع کردن قطرها $2v$ (یا $2w$) را نتیجه می‌دهد.
			
			\item گوشه چهارم می‌تواند $(4,4)$ یا $(4,0)$ یا $(-2,2)$ باشد. سه متوازی‌الاضلاع ممکن وجود دارد!
			
			\item $i-j=(1,1,0)$ در صفحه پایه (صفحه x-y) قرار دارد. $i+j+k=(1,1,1)$ گوشه مقابل $(0,0,0)$ است. نقاط درون مکعب در شرایط $0 \le x \le 1$، $0 \le y \le 1$، $0 \le z \le 1$ صدق می‌کنند.
			
			\item چهار گوشه دیگر: $(1,1,0)$، $(1,0,1)$، $(0,1,1)$، $(1,1,1)$. نقطه مرکزی $(\frac{1}{2}, \frac{1}{2}, \frac{1}{2})$ است. مراکز وجوه عبارتند از $(\frac{1}{2}, \frac{1}{2}, 0)$، $(\frac{1}{2}, \frac{1}{2}, 1)$ و $(0, \frac{1}{2}, \frac{1}{2})$، $(1, \frac{1}{2}, \frac{1}{2})$ و $(\frac{1}{2}, 0, \frac{1}{2})$، $(\frac{1}{2}, 1, \frac{1}{2})$.
			
			\item ترکیب‌های $i=(1,0,0)$ و $i+j=(1,1,0)$ صفحه $xy$ را در فضای $xyz$ پر می‌کنند.
			
			\item مجموع = بردار صفر. مجموع = بردار ساعت ۲:۰۰- = بردار ساعت ۸:۰۰. بردار ساعت ۲:۰۰ با محور افقی زاویه $30^\circ$ می‌سازد $=(\cos\frac{\pi}{6}, \sin\frac{\pi}{6})=(\frac{\sqrt{3}}{2}, \frac{1}{2})$.
			
			\item انتقال مبدأ به موقعیت ساعت ۶:۰۰، بردار $j=(0,1)$ را به هر بردار اضافه می‌کند. بنابراین مجموع دوازده بردار از $\mathbf{0}$ به $12j=(0,12)$ تغییر می‌کند.
		\end{enumerate}
		
	
		\begin{enumerate}
			\setcounter{enumi}{14} % This command continues the numbering from 15
			\item نقطه $\frac{3}{4}v + \frac{1}{4}w$ در سه‌چهارم مسیر به سمت $v$ با شروع از $w$ قرار دارد. بردار $\frac{1}{4}v + \frac{1}{4}w$ در نیمه راه به سمت $u=\frac{1}{2}v+\frac{1}{2}w$ است. بردار $v+w$ برابر با $2u$ (گوشه دورتر متوازی‌الاضلاع) است.
			
			\item تمام ترکیب‌ها با شرط $c+d=1$ روی خطی قرار دارند که از $v$ و $w$ می‌گذرد. نقطه $V=-v+2w$ روی آن خط قرار دارد اما خارج از محدوده $w$ است.
			
			\item تمام بردارهای $cv+cw$ روی خطی قرار دارند که از $(0,0)$ و $u=\frac{1}{2}v+\frac{1}{2}w$ می‌گذرد. این خط به سمت بیرون از $v+w$ و به عقب از $(0,0)$ ادامه می‌یابد. با شرط $c \ge 0$، نیمی از این خط حذف شده و یک نیم‌خط باقی می‌ماند که از $(0,0)$ شروع می‌شود.
			
			\item ترکیب‌های $cv+dw$ با شرایط $0 \le c \le 1$ و $0 \le d \le 1$ متوازی‌الاضلاع با اضلاع $v$ و $w$ را پر می‌کنند. برای مثال، اگر $v=(1,0)$ و $w=(0,1)$ باشند، آنگاه $cv+dw$ مربع واحد را پر می‌کند. اما وقتی $v=(a,0)$ و $w=(b,0)$ باشند، این ترکیب‌ها تنها یک پاره‌خط را پر می‌کنند.
			
			\item با شرایط $c \ge 0$ و $d \ge 0$، ما «مخروط» یا «گوه» بی‌نهایت بین $v$ و $w$ را به دست می‌آوریم. برای مثال، اگر $v=(1,0)$ و $w=(0,1)$ باشند، آنگاه مخروط کل ربع صفحه با $x \ge 0$ و $y \ge 0$ است. سوال: اگر $w=-v$ باشد چه اتفاقی می‌افتد؟ مخروط به یک نیم‌فضا باز می‌شود. اما ترکیب‌های $v=(1,0)$ و $w=(-1,0)$ تنها یک خط را پر می‌کنند.
			
			\item (الف) $\frac{1}{3}u+\frac{1}{3}v+\frac{1}{3}w$ مرکز مثلث بین $u, v, w$ است؛ $\frac{1}{2}u+\frac{1}{2}w$ بین $u$ و $w$ قرار دارد. (ب) برای پر کردن مثلث، شرایط $c \ge 0, d \ge 0, e \ge 0$ و $c+d+e=1$ را حفظ کنید.
			
			\item مجموع برابر است با $(v-u)+(w-v)+(u-w)=$ بردار صفر. این سه ضلع یک مثلث در یک صفحه قرار دارند!
			
			\item بردار $\frac{1}{2}(u+v+w)$ خارج از هرم قرار دارد زیرا $c+d+e = \frac{1}{2}+\frac{1}{2}+\frac{1}{2} > 1$.
			
			\item تمام بردارها ترکیب‌هایی از $u, v, w$ هستند که (در یک صفحه نیستند) رسم شده‌اند. با این مشاهده شروع کنید که $cu+dv$ یک صفحه را پر می‌کند، سپس افزودن $ew$ کل فضای $R^3$ را پر می‌کند.
			
			\item ترکیب‌های $u$ و $v$ یک صفحه را پر می‌کنند. ترکیب‌های $v$ و $w$ صفحه‌ای دیگر را پر می‌کنند. این دو صفحه در یک خط یکدیگر را قطع می‌کنند: تنها بردارهای $cv$ در هر دو صفحه قرار دارند.
			
			\item (الف) برای یک خط، $u=v=w=$ هر بردار غیر صفری را انتخاب کنید. (ب) برای یک صفحه، $u$ و $v$ را در جهت‌های مختلف انتخاب کنید. ترکیبی مانند $w=u+v$ در همان صفحه قرار دارد.
		\end{enumerate}
		
		
		\begin{enumerate}
			\setcounter{enumi}{25} % Continues numbering from 26
			\item دو معادله از دو مؤلفه به دست می‌آید: $c + 3d = 14$ و $2c + d = 8$. جواب $c=2$ و $d=4$ است. سپس $2(1,2)+ 4(3,1) = (14,8)$.
			
			\item یک مکعب چهاربعدی دارای $2^4 = 16$ گوشه و $2 \cdot 4 = 8$ وجه سه‌بعدی و ۲۴ وجه دوبعدی و ۳۲ یال است (بر اساس مثال حل شده 2.4 A).
			
			\item شش عدد مجهول $v_1, v_2, v_3, w_1, w_2, w_3$ وجود دارد. شش معادله از مؤلفه‌های $v + w = (4,5,6)$ و $v - w = (2,5,8)$ به دست می‌آیند. با جمع این دو معادله $2v = (6,10,14)$ حاصل می‌شود، بنابراین $v=(3,5,7)$ و $w=(1,0,-1)$ است.
			
			\item واقعیت: برای هر سه بردار $u, v, w$ در صفحه، ترکیبی مانند $cu + dv + ew$ برابر با بردار صفر است (فراتر از حالت بدیهی $c=d=e=0$). بنابراین اگر یک ترکیب $Cu+Dv+Ew$ وجود داشته باشد که بردار $b$ را تولید کند، ترکیبات بسیار بیشتری نیز وجود خواهد داشت — کافی است $c,d,e$ یا $2c,2d,2e$ را به جواب خاص $C,D,E$ اضافه کنید.
			در این مثال $3u - 2v + w = 3(1,3) - 2(2,7) + 1(1,5) = (0,0)$ است. همچنین $-2u+1v+0w = b=(0,1)$ است. با جمع این دو داریم $u-v+w=(0,1)$. در این حالت $c,d,e$ برابر با $3,-2,1$ و $C,D,E$ برابر با $-2,1,0$ هستند.
			آیا ممکن است در مثالی دیگر، بردارها $u, v, w$ نتوانند ترکیب شوند تا $b$ را تولید کنند؟ بله. بردارهای $(1,1), (2,2), (3,3)$ روی یک خط قرار دارند و هیچ ترکیبی از آنها $b$ را تولید نمی‌کند. ما به راحتی می‌توانیم $cu+dv+ew=0$ را حل کنیم اما $Cu+Dv+Ew=b$ را نه.
			
			\item ترکیب‌های $v$ و $w$ صفحه را پر می‌کنند مگر اینکه $v$ و $w$ روی یک خط گذرنده از $(0,0)$ قرار داشته باشند. یک مثال از چهار برداری که ترکیب‌هایشان فضای چهاربعدی را پر می‌کند، «پایه استاندارد» است: $(1,0,0,0), (0,1,0,0), (0,0,1,0),$ و $(0,0,0,1)$.
			
			\item معادلات برای $cu + dv + ew = b$ به صورت زیر هستند:
			\begin{align*}
				2c - d &= 1 \\
				-c + 2d - e &= 0 \\
				-d + 2e &= 0
			\end{align*}
			از معادله سوم داریم $d=2e$. \\
			با جایگذاری در معادله دوم: $-c + 2(2e) - e = 0 \implies -c + 3e = 0 \implies c = 3e$. \\
			با جایگذاری $c$ و $d$ در معادله اول: $2(3e) - (2e) = 1 \implies 6e - 2e = 1 \implies 4e=1$. \\
			بنابراین $e = \frac{1}{4}$. \\
			و در نتیجه: $c = \frac{3}{4}$ و $d = \frac{2}{4} = \frac{1}{2}$.
			
		\end{enumerate}
		
	} % End of Right-to-Left environment
\end{document}