\documentclass[12pt, a4paper]{book}

% فراخوانی بسته‌های لازم
\usepackage{amsmath}         % برای فرمول‌های پیشرفته ریاضی
\usepackage{amsfonts}        % بسته برای فونت‌های ریاضی مانند \mathbb
\usepackage{amssymb}         % برای نمادهای بیشتر ریاضی
\usepackage{graphicx}        % برای افزودن تصاویر
\usepackage{xepersian}       % بسته اصلی برای پارسی‌نویسی
\usepackage{geometry}        % برای تنظیم حاشیه‌ها
\usepackage{setspace}        % برای تنظیم فاصله خطوط
\usepackage{enumitem}        % برای کنترل بهتر لیست‌ها

% تنظیم حاشیه‌های صفحه
\geometry{
	a4paper,
	total={170mm,257mm},
	left=20mm,
	top=20mm,
}

% تنظیم فونت‌های نوشتاری و ریاضی
\settextfont{XB Niloofar}
\setdigitfont{XB Niloofar}
\setmathdigitfont{XB Niloofar}

% تعریف دستورات سفارشی
\newcommand{\R}{\mathbb{R}} % برای نمایش مجموعه اعداد حقیقی

\begin{document}
	
	% اعمال فاصله 1.5 بین خطوط برای خوانایی بهتر
	\onehalfspacing
	
	\chapter{حل معادلات خطی}
	\label{chap:solving_linear_equations}
	
	\section{بردارها و معادلات خطی}
	\label{sec:vectors_linear_equations}
	
	۱. تصویر ستونی از \( A\mathbf{x} = \mathbf{b} \): ترکیبی از \( n \) ستون ماتریس \( A \) بردار \( \mathbf{b} \) را تولید می‌کند.
	
	۲. این یک معادله برداری است \( A\mathbf{x} = x_1\mathbf{a}_1 + \cdots + x_n\mathbf{a}_n = \mathbf{b} \): ستون‌های \( A \) عبارتند از \( \mathbf{a}_1, \mathbf{a}_2, \ldots, \mathbf{a}_n \).
	
	۳. وقتی \( \mathbf{b} = \mathbf{0} \)، ترکیبی از ستون‌ها \( A\mathbf{x} \) صفر می‌شود: یک احتمال \( \mathbf{x} = (0, \ldots, 0) \) است.
	
	۴. تصویر سطری از \( A\mathbf{x} = \mathbf{b} \): \( m \) معادله از \( m \) سطر، \( m \) صفحه را نتیجه می‌دهند که در نقطه \( \mathbf{x} \) یکدیگر را قطع می‌کنند.
	
	۵. یک ضرب داخلی معادله هر صفحه را می‌دهد: \( (\text{سطر } 1) \cdot \mathbf{x} = b_1, \ldots, (\text{سطر } m) \cdot \mathbf{x} = b_m \).
	
	۶. وقتی \( \mathbf{b} = \mathbf{0} \)، تمام صفحات \( (\text{سطر } i) \cdot \mathbf{x} = 0 \) از نقطه مرکزی \( \mathbf{x} = (0, 0, \ldots, 0) \) عبور می‌کنند.
	
	\paragraph{}
	مسئله اصلی جبر خطی، حل یک دستگاه معادلات است. این معادلات خطی هستند، به این معنی که مجهولات فقط در اعداد ضرب می‌شوند—ما هرگز \( x \) ضربدر \( y \) را نمی‌بینیم. اولین دستگاه خطی ما کوچک است. اما خواهید دید که تا کجا پیش می‌رود:
	\begin{align*}
		\text{دو معادله} \\
		\text{دو مجهول}
	\end{align*}
	\begin{align*}
		x - 2y &= 1 \quad\quad (1) \\
		3x + 2y &= 11
	\end{align*}
	
	ما سطر به سطر شروع می‌کنیم. معادله اول \( x - 2y = 1 \) یک خط راست در صفحه \( xy \) ایجاد می‌کند. نقطه \( x=1, y=0 \) روی این خط قرار دارد زیرا در این معادله صدق می‌کند. نقطه \( x=3, y=1 \) نیز روی خط است زیرا \( 3 - 2(1) = 1 \). اگر \( x=101 \) را انتخاب کنیم، \( y=50 \) را پیدا می‌کنیم. شیب این خط خاص \( \frac{1}{2} \) است، زیرا وقتی \( x \) به اندازه ۲ تغییر می‌کند، \( y \) به اندازه ۱ افزایش می‌یابد. اما شیب‌ها در حسابان مهم هستند و اینجا جبر خطی است!
	
	شکل ۲.۱ آن خط اول \( x - 2y = 1 \) را نشان خواهد داد. خط دوم در این «تصویر سطری» از معادله دوم \( 3x + 2y = 11 \) می‌آید. شما نمی‌توانید نقطه \( x=3, y=1 \) را که دو خط در آن یکدیگر را قطع می‌کنند، نادیده بگیرید. آن نقطه \( (3, 1) \) روی هر دو خط قرار دارد و در هر دو معادله صدق می‌کند.
	
	\begin{figure}[h!]
		\centering
		% \includegraphics[width=0.6\textwidth]{figure2_1.png} 
		\fbox{تصویر شکل ۲.۱ در اینجا قرار می‌گیرد}
		\caption{تصویر سطری: نقطه (3, 1) که در آن خطوط یکدیگر را قطع می‌کنند، جواب هر دو معادله است.}
	\end{figure}
	
	\paragraph{سطرها} تصویر سطری دو خط را نشان می‌دهد که در یک نقطه واحد (جواب) به هم می‌رسند.
	
	\paragraph{}
	حالا به تصویر ستونی می‌پردازیم. من می‌خواهم همان دستگاه خطی را به عنوان یک «معادله برداری» تشخیص دهم. به جای اعداد، باید بردارها را ببینیم. اگر دستگاه اصلی را به جای سطرها، به ستون‌هایش تفکیک کنید، یک معادله برداری به دست می‌آورید:
	\[
	\text{ترکیب برابر است با } \mathbf{b} \quad\quad x \begin{bmatrix} 1 \\ 3 \end{bmatrix} + y \begin{bmatrix} -2 \\ 2 \end{bmatrix} = \begin{bmatrix} 1 \\ 11 \end{bmatrix} \quad\quad (2)
	\]
	این معادله دو بردار ستونی در سمت چپ دارد. مسئله این است که ترکیبی از آن بردارها را پیدا کنیم که برابر با بردار سمت راست شود. ما ستون اول را در \( x \) و ستون دوم را در \( y \) ضرب کرده و با هم جمع می‌کنیم. با انتخاب‌های درست \( x=3 \) و \( y=1 \) (همان اعداد قبلی)، این عمل \( 3(\text{ستون } 1) + 1(\text{ستون } 2) = \mathbf{b} \) را تولید می‌کند.
	
	\vspace{5mm}
	\textit{\textbf{(توضیح مترجم: دو دیدگاه برای یک حقیقت)} \\
		توجه کنید که «تصویر سطری» و «تصویر ستونی» دو راه مختلف برای نگاه کردن به یک مسئله واحد هستند.
		\begin{itemize}
			\item \textbf{تصویر سطری (دیدگاه هندسی):} هر معادله یک شیء هندسی (خط، صفحه، ابرصفحه) را تعریف می‌کند. جواب دستگاه، نقطه تلاقی این اشیاء هندسی است. این دیدگاه برای تجسم دستگاه‌های کوچک (۲ یا ۳ بعدی) بسیار مفید است.
			\item \textbf{تصویر ستونی (دیدگاه ترکیبی):} ما به دنبال یافتن ضرایب صحیح ($x$ و $y$) برای ترکیب ستون‌های ماتریس $A$ هستیم تا به بردار $\mathbf{b}$ در سمت راست برسیم. این دیدگاه در جبر خطی بسیار بنیادی و قدرتمندتر است، زیرا به راحتی به ابعاد بالاتر تعمیم می‌یابد و مفاهیمی مانند «فضای ستونی» و «استقلال خطی» را پایه‌ریزی می‌کند.
	\end{itemize}}
	\vspace{5mm}
	
	شکل ۲.۲ «تصویر ستونی» دو معادله در دو مجهول را نشان می‌دهد. بخش اول دو ستون جداگانه و ستون اول ضربدر ۳ را نشان می‌دهد. این ضرب در یک اسکالر (یک عدد) یکی از دو عملیات اساسی در جبر خطی است:
	
	\paragraph{ضرب اسکالر}
	اگر مؤلفه‌های یک بردار \(\mathbf{v}\) برابر \(v_1\) و \(v_2\) باشند، آنگاه \(c\mathbf{v}\) مؤلفه‌های \(cv_1\) و \(cv_2\) را دارد.
	
	عملیات اساسی دیگر، جمع برداری است. ما مؤلفه‌های اول و مؤلفه‌های دوم را به طور جداگانه جمع می‌کنیم. مجموع برداری \( (1, 11) \) است، یعنی بردار مطلوب \(\mathbf{b}\).
	
	\paragraph{جمع برداری}
	سمت راست شکل ۲.۲ این جمع را نشان می‌دهد. دو بردار با رنگ سیاه مشخص شده‌اند. مجموع در امتداد قطر، بردار \( \mathbf{b} = (1, 11) \) در سمت راست معادلات خطی است.
	
	\begin{figure}[h!]
		\centering
		% \includegraphics[width=0.8\textwidth]{figure2_2.png}
		\fbox{تصویر شکل ۲.۲ در اینجا قرار می‌گیرد}
		\caption{تصویر ستونی: ترکیبی از ستون‌ها، سمت راست معادله یعنی (1, 11) را تولید می‌کند.}
	\end{figure}
	
	برای تکرار: سمت چپ معادله برداری یک ترکیب خطی از ستون‌ها است. مسئله این است که ضرایب صحیح \( x=3 \) و \( y=1 \) را پیدا کنیم. ما ضرب اسکالر و جمع برداری را در یک مرحله ترکیب می‌کنیم. این مرحله به طور حیاتی مهم است، زیرا هر دو عملیات اساسی را در بر دارد: ضرب در ۳ و ۱، و سپس جمع.
	
	\paragraph{ترکیب خطی}
	البته جواب \( x=3, y=1 \) همان جوابی است که در تصویر سطری به دست آمد. نمی‌دانم کدام تصویر را ترجیح می‌دهید! گمان می‌کنم که دو خط متقاطع در ابتدا آشناتر هستند. شاید تصویر سطری را بیشتر دوست داشته باشید، اما فقط برای یک روز. ترجیح شخصی من ترکیب بردارهای ستونی است. دیدن ترکیبی از چهار بردار در فضای چهاربعدی بسیار آسان‌تر از تجسم این است که چگونه چهار ابرصفحه ممکن است در یک نقطه به هم برسند. (حتی یک ابرصفحه به تنهایی به اندازه کافی سخت است...)
	
	ماتریس ضرایب در سمت چپ معادلات، ماتریس \( 2 \times 2 \) به نام \( A \) است:
	\[
	\text{ماتریس ضرایب} \quad\quad A = \begin{bmatrix} 1 & -2 \\ 3 & 2 \end{bmatrix}
	\]
	این در جبر خطی بسیار معمول است که به یک ماتریس از دیدگاه سطرها و ستون‌ها نگاه کنیم. سطرهای آن تصویر سطری و ستون‌های آن تصویر ستونی را می‌دهند. اعداد یکسان، تصاویر متفاوت، معادلات یکسان. ما آن معادلات را در یک مسئله ماتریسی \( A\mathbf{x}=\mathbf{b} \) ترکیب می‌کنیم:
	\[
	\text{معادله ماتریسی} \quad\quad A\mathbf{x}=\mathbf{b} \quad \begin{bmatrix} 1 & -2 \\ 3 & 2 \end{bmatrix} \begin{bmatrix} x \\ y \end{bmatrix} = \begin{bmatrix} 1 \\ 11 \end{bmatrix}
	\]
	
	تصویر سطری با دو سطر ماتریس \(A\) سروکار دارد. تصویر ستونی ستون‌ها را ترکیب می‌کند. اعداد \(x=3\) و \(y=1\) به بردار \(\mathbf{x}\) می‌روند. در اینجا ضرب ماتریس در بردار آمده است:
	\begin{align*}
		\text{ضرب داخلی با سطرها} \quad A\mathbf{x} &= \begin{bmatrix} (\text{سطر } 1) \cdot \mathbf{x} \\ (\text{سطر } 2) \cdot \mathbf{x} \end{bmatrix} = \begin{bmatrix} (1)(3) + (-2)(1) \\ (3)(3) + (2)(1) \end{bmatrix} = \begin{bmatrix} 1 \\ 11 \end{bmatrix} \\
		\text{ترکیب ستون‌ها} \quad A\mathbf{x} &= x(\text{ستون } 1) + y(\text{ستون } 2) = 3\begin{bmatrix} 1 \\ 3 \end{bmatrix} + 1\begin{bmatrix} -2 \\ 2 \end{bmatrix} = \begin{bmatrix} 3 \\ 9 \end{bmatrix} + \begin{bmatrix} -2 \\ 2 \end{bmatrix} = \begin{bmatrix} 1 \\ 11 \end{bmatrix}
	\end{align*}
	\(A\mathbf{x}=\mathbf{b}\) به این صورت است.
	
	\subsection*{نگاه به آینده}
	این فصل قصد دارد \( n \) معادله در \( n \) مجهول را (برای هر \( n \)) حل کند. من با سرعت بالا پیش نمی‌روم، زیرا دستگاه‌های کوچکتر امکان مثال‌ها، تصاویر و درک کامل را فراهم می‌کنند. شما آزادید که سریع‌تر پیش بروید، تا زمانی که ضرب ماتریس و معکوس‌گیری برایتان روشن شود. این دو ایده کلید ماتریس‌های معکوس‌پذیر خواهند بود.
	
	من می‌توانم چهار مرحله برای درک حذف گاوسی با استفاده از ماتریس‌ها را لیست کنم.
	\begin{enumerate}
		\item حذف گاوسی با دنباله‌ای از مراحل ماتریسی \( E_{ij} \) از \( A \) به یک ماتریس مثلثی بالا \( U \) می‌رود.
		\item دستگاه مثلثی با جایگذاری پسرو (back substitution) حل می‌شود: از پایین به بالا کار می‌کند.
		\item در زبان ماتریس، \( A \) به \( LU = (\text{مثلثی پایین}) (\text{مثلثی بالا}) \) تجزیه می‌شود.
		\item حذف گاوسی در صورتی موفقیت‌آمیز است که \( A \) معکوس‌پذیر باشد. (اما ممکن است به جابجایی سطرها نیاز داشته باشد.)
	\end{enumerate}
	پرکاربردترین الگوریتم در علوم محاسباتی این مراحل را طی می‌کند (MATLAB آن را \texttt{lu} می‌نامد). سریع‌ترین شکل آن عملگر بک‌اسلش است: \texttt{x = A\textbackslash b}. اما جبر خطی فراتر از ماتریس‌های مربعی معکوس‌پذیر می‌رود! برای ماتریس‌های \( m \times n \)، \( A\mathbf{x} = \mathbf{0} \) ممکن است جواب‌های زیادی داشته باشد. آن جواب‌ها یک فضای برداری را تشکیل می‌دهند. رتبه \( A \) به بعد آن فضای برداری منجر می‌شود. همه اینها در فصل ۳ می‌آید، و من نمی‌خواهم عجله کنم. اما باید به آنجا برسم.
	
	\subsection*{سه معادله در سه مجهول}
	سه مجهول \( x, y, z \) هستند. ما سه معادله خطی داریم:
	\begin{align*}
		x + 2y + 3z &= 6 \quad\quad &(3) \\
		2x + 5y + 2z &= 4 \\
		6x - 3y + z &= 2
	\end{align*}
	این دستگاه را می‌توان به فرم ماتریسی \(A\mathbf{x}=\mathbf{b}\) نوشت:
	\[
	\begin{bmatrix} 1 & 2 & 3 \\ 2 & 5 & 2 \\ 6 & -3 & 1 \end{bmatrix} \begin{bmatrix} x \\ y \\ z \end{bmatrix} = \begin{bmatrix} 6 \\ 4 \\ 2 \end{bmatrix}
	\]
	ما به دنبال اعداد \( x, y, z \) هستیم که همزمان در هر سه معادله صدق کنند. این اعداد مورد نظر ممکن است وجود داشته باشند یا نداشته باشند. برای این دستگاه، آنها وجود دارند. وقتی تعداد مجهولات با تعداد معادلات برابر است، در این مورد \( 3=3 \)، معمولاً یک جواب وجود دارد.
	قبل از حل مسئله، ما آن را به هر دو روش تجسم می‌کنیم:
	
	\paragraph{سطر} تصویر سطری سه صفحه را نشان می‌دهد که در یک نقطه واحد به هم می‌رسند.
	
	\paragraph{ستون} تصویر ستونی سه ستون را ترکیب می‌کند تا \( \mathbf{b} = (6, 4, 2) \) را تولید کند.
	
	در تصویر سطری، هر معادله یک صفحه در فضای سه‌بعدی ایجاد می‌کند. اولین صفحه در شکل ۲.۳ از اولین معادله \( x + 2y + 3z = 6 \) می‌آید. این صفحه محورهای \(x\)، \(y\) و \(z\) را در نقاط \( (6,0,0) \)، \( (0,3,0) \) و \( (0,0,2) \) قطع می‌کند. این سه نقطه در معادله صدق می‌کنند و کل صفحه را تعیین می‌کنند.
	
	\vspace{5mm}
	\textit{\textbf{(توضیح مترجم: یافتن محل برخورد با محورها)} \\
		برای پیدا کردن محل تقاطع یک صفحه با هر یک از محورهای مختصات، کافی است دو متغیر دیگر را صفر قرار دهید. برای مثال، برای یافتن محل تقاطع صفحه \(x + 2y + 3z = 6\) با محور \(x\)، قرار می‌دهیم \(y=0\) و \(z=0\)، که نتیجه می‌دهد \(x=6\). بنابراین نقطه تقاطع \((6,0,0)\) است. این روش برای سایر محورها نیز به همین ترتیب است.}
	\vspace{5mm}
	
	بردار \( (x,y,z) = (0,0,0) \) در معادله \( x + 2y + 3z = 6 \) صدق نمی‌کند. بنابراین آن صفحه شامل مبدأ نیست. صفحه \( x + 2y + 3z = 0 \) از مبدأ عبور می‌کند و با صفحه \( x + 2y + 3z = 6 \) موازی است. وقتی سمت راست به ۶ افزایش می‌یابد، صفحه موازی از مبدأ دور می‌شود.
	
	صفحه دوم توسط معادله دوم \( 2x + 5y + 2z = 4 \) داده می‌شود. این صفحه، صفحه اول را در یک خط \( L \) قطع می‌کند. نتیجه معمول دو معادله در سه مجهول، یک خط \( L \) از جواب‌ها است. (مگر اینکه معادلات موازی بودند، مانند \( x+2y+3z=6 \) و \( x+2y+3z=0 \).)
	
	معادله سوم، صفحه سومی را می‌دهد. این صفحه خط \( L \) را در یک نقطه واحد قطع می‌کند. آن نقطه روی هر سه صفحه قرار دارد و در هر سه معادله صدق می‌کند. کشیدن این نقطه تقاطع سه‌گانه سخت‌تر از تصور آن است. سه صفحه در نقطه جواب (که هنوز پیدا نکرده‌ایم) به هم می‌رسند. فرم ستونی اکنون فوراً نشان می‌دهد که چرا \( z=2 \).
	
	\begin{figure}[h!]
		\centering
		% \includegraphics[width=0.8\textwidth]{figure2_3.png}
		\fbox{تصویر شکل ۲.۳ در اینجا قرار می‌گیرد}
		\caption{تصویر سطری: دو صفحه در یک خط L به هم می‌رسند. سه صفحه در یک نقطه به هم می‌رسند.}
	\end{figure}
	
	تصویر ستونی با فرم برداری معادلات \( A\mathbf{x} = \mathbf{b} \) شروع می‌شود:
	\[
	\text{ترکیب ستون‌ها} \quad x\begin{bmatrix} 1 \\ 2 \\ 6 \end{bmatrix} + y\begin{bmatrix} 2 \\ 5 \\ -3 \end{bmatrix} + z\begin{bmatrix} 3 \\ 2 \\ 1 \end{bmatrix} = \begin{bmatrix} 6 \\ 4 \\ 2 \end{bmatrix} \quad\quad (4)
	\]
	مجهولات، ضرایب \( x, y, z \) هستند. ما می‌خواهیم سه بردار ستونی را در اعداد صحیح \( x, y, z \) ضرب کنیم تا \( \mathbf{b} = (6, 4, 2) \) را تولید کنیم.
	
	شکل ۲.۴ این تصویر ستونی را نشان می‌دهد. ترکیبات خطی از آن ستون‌ها می‌توانند هر بردار \( \mathbf{b} \) را تولید کنند! ترکیبی که \( \mathbf{b} = (6, 4, 2) \) را تولید می‌کند، فقط ۲ برابر ستون سوم است. ضرایبی که ما نیاز داریم \( x=0, y=0 \) و \( z=2 \) هستند.
	
	سه صفحه در تصویر سطری در همان نقطه جواب \( (0,0,2) \) به هم می‌رسند:
	\[
	\text{ترکیب صحیح} \quad 0\begin{bmatrix} 1 \\ 2 \\ 6 \end{bmatrix} + 0\begin{bmatrix} 2 \\ 5 \\ -3 \end{bmatrix} + 2\begin{bmatrix} 3 \\ 2 \\ 1 \end{bmatrix} = \begin{bmatrix} 6 \\ 4 \\ 2 \end{bmatrix}
	\]
	جواب \( (x,y,z) = (0,0,2) \) است.
	
	\begin{figure}[h!]
		\centering
		% \includegraphics[width=0.7\textwidth]{figure2_4.png}
		\fbox{تصویر شکل ۲.۴ در اینجا قرار می‌گیرد}
		\caption{تصویر ستونی: ستون‌ها را با وزن‌های (x, y, z) = (0, 0, 2) ترکیب کنید.}
	\end{figure}
	
	% ... ادامه ترجمه تا ابتدای مجموعه مسائل ...
	
	\newpage
	\section*{مجموعه مسائل ۲.۱}
	\begin{enumerate}[label=\arabic*.]
		\item 
		\textbf{مسائل ۱-۸ درباره تصویر سطری و ستونی \(A\mathbf{x}=\mathbf{b}\) هستند.} \\
		با $A=I$ (ماتریس همانی)، صفحات را در تصویر سطری رسم کنید. سه وجه یک جعبه در جواب $\mathbf{x}=(x,y,z)=(2,3,4)$ به هم می‌رسند:
		\[
		\begin{cases}
			1x+0y+0z = 2 \\
			0x+1y+0z = 3 \\
			0x+0y+1z = 4
		\end{cases}
		\quad \text{یا} \quad
		\begin{bmatrix}
			1 & 0 & 0 \\ 0 & 1 & 0 \\ 0 & 0 & 1
		\end{bmatrix}
		\begin{bmatrix} x \\ y \\ z \end{bmatrix}
		=
		\begin{bmatrix} 2 \\ 3 \\ 4 \end{bmatrix}
		\]
		بردارها را در تصویر ستونی رسم کنید. دو برابر ستون ۱ به علاوه سه برابر ستون ۲ به علاوه چهار برابر ستون ۳ برابر با سمت راست $\mathbf{b}$ است.
		
		\item 
		اگر معادلات مسئله ۱ در ۲، ۳ و ۴ ضرب شوند، به صورت $D\mathbf{x}=B$ در می‌آیند:
		\[
		\begin{cases}
			2x+0y+0z = 4 \\
			0x+3y+0z = 9 \\
			0x+0y+4z = 16
		\end{cases}
		\quad \text{یا} \quad
		D\mathbf{x}=
		\begin{bmatrix}
			2 & 0 & 0 \\ 0 & 3 & 0 \\ 0 & 0 & 4
		\end{bmatrix}
		\begin{bmatrix} x \\ y \\ z \end{bmatrix}
		=
		\begin{bmatrix} 4 \\ 9 \\ 16 \end{bmatrix} = B
		\]
		چرا تصویر سطری یکسان است؟ آیا جواب $X$ همان $x$ است؟ در تصویر ستونی چه چیزی تغییر کرده است—ستون‌ها یا ترکیب صحیح برای به دست آوردن $B$؟
		
		\item
		اگر معادله ۱ به معادله ۲ اضافه شود، کدام یک از اینها تغییر می‌کند: صفحات در تصویر سطری، بردارها در تصویر ستونی، ماتریس ضرایب، جواب؟ معادلات جدید در مسئله ۱ به صورت $x=2, x+y=5, z=4$ خواهند بود.
		
		\item
		نقطه‌ای با $z=2$ روی خط تقاطع صفحات $x+y+3z=6$ و $x-y+z=4$ پیدا کنید. نقطه با $z=0$ را پیدا کنید. نقطه سومی را در نیمه راه بین آن دو پیدا کنید.
		
		\item
		معادله اول به علاوه معادله دوم برابر با معادله سوم است:
		\begin{align*}
			x+y+z &= 2 \\
			x+2y+z &= 3 \\
			2x+3y+2z &= 5
		\end{align*}
		دو صفحه اول در امتداد یک خط به هم می‌رسند. صفحه سوم شامل آن خط است، زیرا اگر $x,y,z$ در دو معادله اول صدق کنند، در سومی نیز صدق می‌کنند. معادلات بی‌نهایت جواب دارند (کل خط $L$). سه جواب روی $L$ پیدا کنید.
		
		\item
		صفحه سوم در مسئله ۵ را به یک صفحه موازی $2x+3y+2z=9$ منتقل کنید. حالا سه معادله جوابی ندارند—چرا؟ دو صفحه اول در امتداد خط $L$ به هم می‌رسند، اما صفحه سوم آن خط را قطع نمی‌کند.
		
		\item
		در مسئله ۵ ستون‌ها (1,1,2) و (1,2,3) و (1,1,2) هستند. این یک «حالت منفرد» است زیرا ستون سوم برابر ستون اول است. دو ترکیب از ستون‌ها را بیابید که $\mathbf{b}=(2,3,5)$ را بدهند. این فقط برای $\mathbf{b}=(4,6,c)$ ممکن است اگر $c = 10$.
		
		\item
		به طور معمول ۴ «صفحه» در فضای ۴ بعدی در یک نقطه به هم می‌رسند. به طور معمول ۴ بردار ستونی در فضای ۴ بعدی می‌توانند ترکیب شوند تا $\mathbf{b}$ را تولید کنند. چه ترکیبی از (1,0,0,0)، (1,1,0,0)، (1,1,1,0)، (1,1,1,1) بردار $\mathbf{b}=(3,3,3,2)$ را تولید می‌کند؟ شما در حال حل چه ۴ معادله‌ای برای $x,y,z,t$ هستید؟
		
		\item
		\textbf{مسائل ۹-۱۴ درباره ضرب ماتریس و بردار هستند.} \\
		هر $A\mathbf{x}$ را با ضرب داخلی سطرها با بردار ستونی محاسبه کنید:
		(الف) $ \begin{bmatrix} 1 & 2 & 4 \\ 0 & 3 & 1 \end{bmatrix} \begin{bmatrix} 2 \\ 1 \\ 0 \end{bmatrix} $
		(ب) $ \begin{bmatrix} 1 & 2 \\ 1 & 0 \\ 1 & 2 \end{bmatrix} \begin{bmatrix} 3 \\ -4 \end{bmatrix} $
		
		\item
		هر $A\mathbf{x}$ در مسئله ۹ را به عنوان ترکیبی از ستون‌ها محاسبه کنید: \\
		(الف) \( 2\begin{bmatrix} 1 \\ 0 \end{bmatrix} + 1\begin{bmatrix} 2 \\ 3 \end{bmatrix} + 0\begin{bmatrix} 4 \\ 1 \end{bmatrix} = \begin{bmatrix} 4 \\ 3 \end{bmatrix} \)
		(ب) \( 3\begin{bmatrix} 1 \\ 1 \\ 1 \end{bmatrix} - 4\begin{bmatrix} 2 \\ 0 \\ 2 \end{bmatrix} = \begin{bmatrix} -5 \\ 3 \\ -5 \end{bmatrix} \)
		
		\item
		دو مؤلفه $A\mathbf{x}$ را به صورت سطری یا ستونی بیابید:
		\[ \begin{bmatrix} 2 & 3 \\ 5 & 1 \end{bmatrix} \begin{bmatrix} 4 \\ 2 \end{bmatrix} \quad \text{و} \quad \begin{bmatrix} 3 & 6 \\ 6 & 12 \end{bmatrix} \begin{bmatrix} 2 \\ -1 \end{bmatrix} \quad \text{و} \quad \begin{bmatrix} 1 & 0 \\ 0 & 1 \end{bmatrix} \begin{bmatrix} 4 \\ 2 \end{bmatrix} \]
		
		\item
		$A$ ضربدر $\mathbf{x}$ را برای یافتن سه مؤلفه $A\mathbf{x}$ ضرب کنید:
		\[ \begin{bmatrix} 0 & 0 & 1 \\ 0 & 1 & 0 \\ 1 & 0 & 0 \end{bmatrix} \begin{bmatrix} x \\ y \\ z \end{bmatrix} \quad \text{و} \quad \begin{bmatrix} 2 & 1 & 3 \\ 1 & 2 & 3 \\ 3 & 3 & 6 \end{bmatrix} \begin{bmatrix} 1 \\ 1 \\ -1 \end{bmatrix} \quad \text{و} \quad \begin{bmatrix} 2 & 1 \\ 1 & 2 \\ 3 & 3 \end{bmatrix} \begin{bmatrix} 1 \\ 1 \end{bmatrix} \]
		
		\item
		(الف) یک ماتریس با $m$ سطر و $n$ ستون در یک بردار با $n$ مؤلفه ضرب می‌شود تا یک بردار با $m$ مؤلفه تولید کند.
		(ب) صفحات از $m$ معادله $A\mathbf{x}=\mathbf{b}$ در فضای $n$ بعدی هستند. ترکیب ستون‌های $A$ در فضای $m$ بعدی است.
		
		\item
		معادله $2x+3y+z+5t=8$ را به صورت ماتریس $A$ (چند سطر؟) ضربدر بردار ستونی $\mathbf{x}=(x,y,z,t)$ برای تولید $\mathbf{b}$ بنویسید. جواب‌های $\mathbf{x}$ یک صفحه یا «ابرصفحه» در فضای ۴ بعدی را پر می‌کنند. این صفحه ۳ بعدی و بدون حجم ۴ بعدی است.
		
		\item
		\textbf{مسائل ۱۵-۲۲ به دنبال ماتریس‌هایی هستند که به روش‌های خاصی روی بردارها عمل می‌کنند.} \\
		(الف) ماتریس همانی ۲ در ۲ چیست؟ $I$ ضربدر $\begin{bmatrix} x \\ y \end{bmatrix}$ برابر با $\begin{bmatrix} x \\ y \end{bmatrix}$ است.
		(ب) ماتریس تعویض ۲ در ۲ چیست؟ $P$ ضربدر $\begin{bmatrix} x \\ y \end{bmatrix}$ برابر با $\begin{bmatrix} y \\ x \end{bmatrix}$ است.
		
		\item
		(الف) چه ماتریس ۲ در ۲ به نام $R$ هر بردار را ۹۰ درجه می‌چرخاند؟ $R$ ضربدر $\begin{bmatrix} x \\ y \end{bmatrix}$ برابر است با $\begin{bmatrix} -y \\ x \end{bmatrix}$.
		(ب) چه ماتریس ۲ در ۲ به نام $R^2$ هر بردار را ۱۸۰ درجه می‌چرخاند؟
		
		\item
		ماتریس $P$ را بیابید که $(x,y,z)$ را به $(y,z,x)$ ضرب می‌کند. ماتریس $Q$ را بیابید که $(y,z,x)$ را به $(x,y,z)$ برمی‌گرداند.
		
		\item
		چه ماتریس ۲ در ۲ به نام $E$ مؤلفه اول را از مؤلفه دوم کم می‌کند؟ چه ماتریس ۳ در ۳ همین کار را انجام می‌دهد?
		\[ E \begin{bmatrix} 3 \\ 5 \end{bmatrix} = \begin{bmatrix} 3 \\ 2 \end{bmatrix} \quad \text{و} \quad E \begin{bmatrix} 3 \\ 5 \\ 7 \end{bmatrix} = \begin{bmatrix} 3 \\ 2 \\ 7 \end{bmatrix} \]
		
		\item
		چه ماتریس ۳ در ۳ به نام $E$ بردار $(x,y,z)$ را به $(x,y,z+x)$ ضرب می‌کند؟ چه ماتریسی $E^{-1}$ بردار $(x,y,z)$ را به $(x,y,z-x)$ ضرب می‌کند؟ اگر $(3,4,5)$ را در $E$ و سپس در $E^{-1}$ ضرب کنید، دو نتیجه ($\begin{bmatrix} 3 \\ 4 \\ 8 \end{bmatrix}$) و ($\begin{bmatrix} 3 \\ 4 \\ 5 \end{bmatrix}$) هستند.
		
		\item
		چه ماتریس ۲ در ۲ به نام $P_1$ بردار $(x,y)$ را بر روی محور $x$ تصویر می‌کند تا $(x,0)$ را تولید کند؟ چه ماتریسی به نام $A$ بر روی محور $y$ تصویر می‌کند تا $(0,y)$ را تولید کند؟ اگر $(5,7)$ را در $P_1$ و سپس در $P_2$ ضرب کنید، ($\begin{bmatrix} 5 \\ 0 \end{bmatrix}$) و ($\begin{bmatrix} 0 \\ 0 \end{bmatrix}$) را به دست می‌آورید.
		
		\item
		چه ماتریس ۲ در ۲ به نام $R$ هر بردار را ۴۵ درجه می‌چرخاند؟ بردار $(1,0)$ به $(\sqrt{2}/2, \sqrt{2}/2)$ می‌رود. بردار $(0,1)$ به $(-\sqrt{2}/2, \sqrt{2}/2)$ می‌رود. این‌ها ماتریس را تعیین می‌کنند. این بردارهای خاص را در صفحه $xy$ رسم کنید و $R$ را بیابید.
		
		\item
		ضرب داخلی $(1,4,5)$ و $(x,y,z)$ را به صورت ضرب ماتریسی $A\mathbf{x}$ بنویسید. ماتریس $A$ یک سطر دارد. جواب‌های $A\mathbf{x}=0$ بر روی یک صفحه عمود بر بردار $(1,4,5)$ قرار دارند. ستون‌های $A$ فقط در فضای یک بعدی هستند.
		
		\item
		در نشانه‌گذاری MATLAB، دستوراتی را بنویسید که ماتریس $A$ و بردارهای ستونی $\mathbf{x}$ و $\mathbf{b}$ زیر را تعریف می‌کنند. چه دستوری آزمایش می‌کند که آیا $A\mathbf{x}=\mathbf{b}$ است یا نه؟
		\[ A = \begin{bmatrix} 1 & 2 & 1 \\ 3 & 6 & 1 \\ 2 & 5 & 2 \end{bmatrix} \quad \mathbf{x} = \begin{bmatrix} 1 \\ 1 \\ -1 \end{bmatrix} \quad \mathbf{b} = \begin{bmatrix} 2 \\ 8 \\ 5 \end{bmatrix} \]
		
		\item
		دستورات MATLAB به صورت \texttt{A = eye(3)} و \texttt{v = [3:5]'} ماتریس همانی ۳ در ۳ و بردار ستونی $(3,4,5)$ را تولید می‌کنند. خروجی‌های \texttt{A*v} و \texttt{v'*v} چیست؟ (کامپیوتر لازم نیست!) اگر \texttt{w*A} را درخواست کنید، چه اتفاقی می‌افتد؟
		
		\item
		اگر ماتریس ۴ در ۴ تمام-یک \texttt{A = ones(4)} و بردار ستونی \texttt{v = ones(4,1)} را ضرب کنید، \texttt{A*v} چیست؟ (کامپیوتر لازم نیست.) اگر \texttt{B = eye(4) + ones(4)} را در \texttt{w = zeros(4,1) + 2*ones(4,1)} ضرب کنید، \texttt{B*w} چیست؟
		
		\item
		\textbf{سوالات ۲۶-۲۸ تصاویر سطری و ستونی را در ۲، ۳ و ۴ بعد مرور می‌کنند.} \\
		تصاویر سطری و ستونی را برای معادلات $x-2y=0, x+y=6$ رسم کنید.
		
		\item
		برای دو معادله خطی در سه مجهول $x,y,z$، تصویر سطری (۲ یا ۳) (خط یا صفحه) را در فضای (۲ یا ۳) بعدی نشان می‌دهد. تصویر ستونی در فضای (۲ یا ۳) بعدی است. جواب‌ها به طور معمول بر روی یک خط قرار دارند.
		
		\item
		برای چهار معادله خطی در دو مجهول $x$ و $y$، تصویر سطری چهار خط را نشان می‌دهد. تصویر ستونی در فضای چهار بعدی است. معادلات جوابی ندارند مگر اینکه بردار سمت راست ترکیبی از دو ستون باشد.
		
		\item
		با بردار $u_0=(1,0)$ شروع کنید. بارها و بارها در همان ماتریس «مارکوف» $A = \begin{bmatrix} .8 & .3 \\ .2 & .7 \end{bmatrix}$ ضرب کنید. سه بردار بعدی $u_1, u_2, u_3$ هستند:
		\[ u_1 = \begin{bmatrix} .8 & .3 \\ .2 & .7 \end{bmatrix} \begin{bmatrix} 1 \\ 0 \end{bmatrix} = \begin{bmatrix} .8 \\ .2 \end{bmatrix} \quad u_2 = A u_1 = \begin{bmatrix} .70 \\ .30 \end{bmatrix} \quad u_3 = A u_2 = \begin{bmatrix} .65 \\ .35 \end{bmatrix} \]
		چه ویژگی‌ای را برای هر چهار بردار $u_0, u_1, u_2, u_3$ مشاهده می‌کنید؟
		
		\item
		\textbf{مسائل چالشی} \\
		مسئله ۲۹ را از $u_0=(1,0)$ تا $u_7$ و همچنین از $v_0=(0,1)$ تا $v_7$ ادامه دهید. درباره $u_7$ و $v_7$ چه چیزی مشاهده می‌کنید؟ ...
		
		\item
		یک ماتریس جادویی ۳ در ۳ به نام $M_3$ با درایه‌های ۱، ۲، ...، ۹ ابداع کنید. تمام سطرها و ستون‌ها و قطرها به ۱۵ جمع می‌شوند. سطر اول می‌تواند ۸، ۳، ۴ باشد. $M_3$ ضربدر $(1,1,1)$ چیست؟ $M_4$ ضربدر $(1,1,1,1)$ چیست اگر یک ماتریس جادویی ۴ در ۴ درایه‌های ۱، ...، ۱۶ داشته باشد؟
		
		\item
		فرض کنید $u$ و $v$ دو ستون اول یک ماتریس ۳ در ۳ به نام $A$ هستند. کدام ستون‌های سوم $w$ این ماتریس را منفرد می‌کنند؟ یک تصویر ستونی نوعی از $A\mathbf{x}=\mathbf{b}$ را در آن حالت منفرد و یک تصویر سطری نوعی (برای یک $\mathbf{b}$ تصادفی) توصیف کنید.
		
		\item
		ضرب در $A$ یک «تبدیل خطی» است. این کلمات به این معناست: اگر $w$ ترکیبی از $u$ و $v$ باشد، آنگاه $Aw$ همان ترکیب از $Au$ و $Av$ است. این «خطی بودن» $A(cu+dv)=cAu+dAv$ است که نام «جبر خطی» را به ما می‌دهد. \\
		\textbf{مسئله:} اگر $u=\begin{bmatrix} 1 \\ 0 \end{bmatrix}$ و $v=\begin{bmatrix} 0 \\ 1 \end{bmatrix}$، آنگاه $Au$ و $Av$ ستون‌های $A$ هستند. $w=cu+dv$ را ترکیب کنید. اگر $w=\begin{bmatrix} 5 \\ 7 \end{bmatrix}$، $Aw$ چگونه به $Au$ و $Av$ مرتبط است؟
		
		\item
		از چهار معادله $-x_{i+1} + 2x_i - x_{i-1} = i$ (برای $i=1,2,3,4$ با $x_0=x_5=0$) شروع کنید. این معادلات را در فرم ماتریسی $A\mathbf{x}=\mathbf{b}$ بنویسید. آیا می‌توانید آنها را برای $x_1, x_2, x_3, x_4$ حل کنید؟
		
		\item
		یک ماتریس سودوکو ۹ در ۹ به نام $S$ اعداد ۱، ...، ۹ را در هر سطر و هر ستون و در هر بلوک ۳ در ۳ دارد. برای بردار تمام-یک $\mathbf{x}=(1,...,1)$، $S\mathbf{x}$ چیست؟ \\
		یک سوال بهتر: کدام تعویض‌های سطر، ماتریس سودوکو دیگری تولید می‌کند؟ همچنین، کدام تعویض‌های بلوک‌های سطری، ماتریس سودوکو دیگری می‌دهد؟
	\end{enumerate}
	
\end{document}