\documentclass[12pt, a4paper]{book}

% فراخوانی بسته‌های لازم
\usepackage{amsmath}         % برای فرمول‌های پیشرفته ریاضی
\usepackage{amsfonts}        % بسته برای فونت‌های ریاضی مانند \mathbb
\usepackage{amssymb}         % برای نمادهای بیشتر ریاضی
\usepackage{graphicx}        % برای افزودن تصاویر
\usepackage{xepersian}       % بسته اصلی برای پارسی‌نویسی
\usepackage{geometry}        % برای تنظیم حاشیه‌ها
\usepackage{setspace}        % برای تنظیم فاصله خطوط

% تنظیم حاشیه‌های صفحه
\geometry{
	a4paper,
	total={170mm,257mm},
	left=20mm,
	top=20mm,
}

% تنظیم فونت‌های نوشتاری و ریاضی
% توجه: این فونت‌ها باید روی سیستم شما نصب باشند
\settextfont{XB Niloofar}
\setdigitfont{XB Niloofar}
\setmathdigitfont{XB Niloofar}

\begin{document}
	
	% اعمال فاصله 1.5 بین خطوط برای خوانایی بهتر
	\onehalfspacing
	
	\chapter{مقدمه‌ای بر بردارها}
	
	قلب جبر خطی در دو عمل نهفته است - هر دو با بردارها سروکار دارند. ما بردارها را با هم جمع می‌کنیم تا $\mathbf{v} + \mathbf{w}$ را به دست آوریم. آن‌ها را در اعداد $c$ و $d$ ضرب می‌کنیم تا $c\mathbf{v}$ و $d\mathbf{w}$ را به دست آوریم. ترکیب این دو عمل (جمع کردن $c\mathbf{v}$ با $d\mathbf{w}$) به ما \textbf{ترکیب خطی} $c\mathbf{v} + d\mathbf{w}$ را می‌دهد.
	
	\vspace{5mm}
	\textit{\textbf{(توضیح مترجم: درک عمیق‌تر ترکیب خطی)} \\
		فکر کنید بردارها مانند دستورالعمل‌های حرکتی هستند. بردار $\mathbf{v}$ می‌گوید: «یک قدم به شرق برو» و بردار $\mathbf{w}$ می‌گوید: «یک قدم به شمال برو». عمل $c\mathbf{v}$ یعنی «$c$ قدم در جهت شرق بردار» (که به آن \textbf{مقیاس‌دهی} یا scaling می‌گویند) و $d\mathbf{w}$ یعنی «$d$ قدم در جهت شمال بردار». \\
		یک \textbf{ترکیب خطی} مانند $c\mathbf{v} + d\mathbf{w}$، نتیجه‌ی ترکیب این دو دستورالعمل است. برای مثال، $2\mathbf{v} + 3\mathbf{w}$ یعنی «دو قدم به شرق برو و سپس سه قدم به شمال برو». با تغییر مقادیر $c$ و $d$، ما می‌توانیم به هر نقطه‌ای در یک صفحه دو بعدی برسیم. این ایده، سنگ بنای تمام جبر خطی است.}
	\vspace{5mm}
	
	\textbf{ترکیب خطی}\\
	مثال: $\mathbf{v} + \mathbf{w} = \begin{bmatrix} v_1 \\ v_2 \end{bmatrix} + \begin{bmatrix} w_1 \\ w_2 \end{bmatrix} = \begin{bmatrix} v_1 + w_1 \\ v_2 + w_2 \end{bmatrix}$ ترکیبی با $c=1$ و $d=1$ است.
	
	ترکیب‌های خطی در این موضوع از اهمیت فوق‌العاده‌ای برخوردارند! گاهی ما به دنبال یک ترکیب خاص هستیم، مانند انتخاب مشخص $c = 2$ و $d = 1$ که حاصل آن $c\mathbf{v} + d\mathbf{w} = (4, 5)$ می‌شود. در مواقع دیگر، ما به دنبال \textbf{تمام} ترکیب‌های ممکن از $\mathbf{v}$ و $\mathbf{w}$ هستیم (که از تمام مقادیر ممکن برای $c$ و $d$ به دست می‌آیند).
	
	بردارهای $c\mathbf{v}$ روی یک خط قرار می‌گیرند. وقتی $\mathbf{w}$ روی آن خط نباشد، ترکیب‌های $c\mathbf{v} + d\mathbf{w}$ تمام صفحه دو بعدی را پر می‌کنند. با شروع از چهار بردار $\mathbf{u}, \mathbf{v}, \mathbf{w}, \mathbf{z}$ در فضای چهار بعدی، ترکیب‌های آن‌ها $c\mathbf{u} + d\mathbf{v} + e\mathbf{w} + f\mathbf{z}$ به احتمال زیاد فضا را پر می‌کنند، اما نه همیشه. این بردارها و ترکیب‌هایشان ممکن است در یک صفحه یا روی یک خط قرار گیرند.
	
	فصل اول این ایده‌های اصلی را که همه چیز بر پایه آن‌ها بنا می‌شود، توضیح می‌دهد. ما با بردارهای دو بعدی و سه بعدی که ترسیم آن‌ها منطقی است، شروع می‌کنیم. سپس به ابعاد بالاتر می‌رویم. ویژگی واقعاً تأثیرگذار جبر خطی این است که چگونه این گام را به نرمی به فضای $n$-بعدی برمی‌دارد. تصویر ذهنی شما کاملاً صحیح باقی می‌ماند، حتی اگر ترسیم یک بردار ده بعدی غیرممکن باشد.
	
	این مسیری است که کتاب به سمت آن می‌رود (به سوی فضای $n$-بعدی). گام‌های اولیه، عملیات بخش‌های ۱.۱ و ۱.۲ هستند. سپس بخش ۱.۳ سه ایده بنیادی را تشریح می‌کند.
	\begin{enumerate}
		\item جمع برداری $\mathbf{v} + \mathbf{w}$ و ترکیب‌های خطی $c\mathbf{v} + d\mathbf{w}$.
		\item ضرب داخلی $\mathbf{v} \cdot \mathbf{w}$ دو بردار و طول $||\mathbf{v}|| = \sqrt{\mathbf{v} \cdot \mathbf{v}}$.
		\item ماتریس‌ها $A$، معادلات خطی $A\mathbf{x} = \mathbf{b}$، و جواب‌ها $\mathbf{x} = A^{-1}\mathbf{b}$.
	\end{enumerate}
	
	\section{بردارها و ترکیب‌های خطی}
	
	\begin{enumerate}
		\item $3\mathbf{v} + 5\mathbf{w}$ یک ترکیب خطی نوعی از بردارهای $\mathbf{v}$ و $\mathbf{w}$ است.
		\item برای $\mathbf{v} = \begin{bmatrix} 1 \\ 1 \end{bmatrix}$ و $\mathbf{w} = \begin{bmatrix} 2 \\ 3 \end{bmatrix}$، آن ترکیب برابر است با $3\begin{bmatrix} 1 \\ 1 \end{bmatrix} + 5\begin{bmatrix} 2 \\ 3 \end{bmatrix} = \begin{bmatrix} 3 \\ 3 \end{bmatrix} + \begin{bmatrix} 10 \\ 15 \end{bmatrix} = \begin{bmatrix} 13 \\ 18 \end{bmatrix}$.
		\item بردار $\begin{bmatrix} 2 \\ 3 \end{bmatrix}$ در صفحه $xy$ به اندازه $x=2$ به سمت راست و به اندازه $y=3$ به سمت بالا می‌رود.
		\item ترکیب‌های $c\begin{bmatrix} 1 \\ 0 \end{bmatrix} + d\begin{bmatrix} 0 \\ 1 \end{bmatrix}$ تمام صفحه $xy$ را پر می‌کنند. آن‌ها هر بردار $\begin{bmatrix} x \\ y \end{bmatrix}$ را تولید می‌کنند.
		\item ترکیب‌های $c\begin{bmatrix} 1 \\ 1 \\ 0 \end{bmatrix} + d\begin{bmatrix} 0 \\ 1 \\ 1 \end{bmatrix}$ یک صفحه را در فضای $xyz$ پر می‌کنند.
		\item اما دستگاه معادلات $ \begin{cases} c+2d=1 \\ c+3d=0 \\ c+4d=0 \end{cases} $ جوابی ندارد، زیرا سمت راست آن $\begin{bmatrix} 1 \\ 0 \\ 0 \end{bmatrix}$ در آن صفحه قرار ندارد.
	\end{enumerate}
	
	\subsection*{بردار ستونی v}
	«شما نمی‌توانید سیب و پرتقال را با هم جمع کنید.» به طرزی عجیب، این دلیل وجود بردارهاست. ما دو عدد مجزا $v_1$ و $v_2$ داریم. این زوج، یک بردار دو بعدی $\mathbf{v}$ را تولید می‌کند:
	\begin{itemize}
		\item $v_1$ = مؤلفه اول $\mathbf{v}$
		\item $v_2$ = مؤلفه دوم $\mathbf{v}$
	\end{itemize}
	ما $\mathbf{v}$ را به صورت یک ستون می‌نویسیم، نه یک سطر. نکته اصلی تا اینجا این است که یک حرف واحد $\mathbf{v}$ (به صورت بولد ایتالیک) برای این زوج از اعداد $v_1$ و $v_2$ (به صورت ایتالیک معمولی) داشته باشیم.
	
	\vspace{5mm}
	\textit{\textbf{(توضیح مترجم: چرا بردار ستونی؟)} \\
		نمایش بردارها به صورت ستونی یک قرارداد بسیار مهم در جبر خطی است. در فصول آینده خواهید دید که این نمایش، ضرب ماتریس در بردار را بسیار طبیعی و منطقی می‌کند. وقتی معادله‌ای مانند $A\mathbf{x} = \mathbf{b}$ را می‌نویسیم، بردار $\mathbf{x}$ باید به صورت ستونی باشد تا این ضرب معنای درستی پیدا کند. پس از همین ابتدا به این نمایش عادت کنید.}
	\vspace{5mm}
	
	حتی اگر ما $v_1$ را با $v_2$ جمع نکنیم، ما بردارها را با هم جمع می‌کنیم. مؤلفه‌های اول $\mathbf{v}$ و $\mathbf{w}$ از مؤلفه‌های دوم جدا می‌مانند:
	
	\textbf{جمع برداری}
	\[ \mathbf{v} = \begin{bmatrix} v_1 \\ v_2 \end{bmatrix} \quad \text{و} \quad \mathbf{w} = \begin{bmatrix} w_1 \\ w_2 \end{bmatrix} \quad \text{جمع می‌شوند به} \quad \mathbf{v} + \mathbf{w} = \begin{bmatrix} v_1 + w_1 \\ v_2 + w_2 \end{bmatrix} \]
	تفریق نیز از همین ایده پیروی می‌کند: مؤلفه‌های $\mathbf{v} - \mathbf{w}$ برابر با $v_1 - w_1$ و $v_2 - w_2$ هستند.
	
	عمل اصلی دیگر، \textbf{ضرب اسکالر} است. بردارها را می‌توان در ۲ یا در ۱- یا در هر عدد $c$ ضرب کرد. برای یافتن $2\mathbf{v}$، هر مؤلفه $\mathbf{v}$ را در ۲ ضرب می‌کنیم:
	
	\textbf{ضرب اسکالر}
	\[ 2\mathbf{v} = \begin{bmatrix} 2v_1 \\ 2v_2 \end{bmatrix} = \mathbf{v} + \mathbf{v} \quad \quad -\mathbf{v} = \begin{bmatrix} -v_1 \\ -v_2 \end{bmatrix} \]
	مؤلفه‌های $c\mathbf{v}$ برابر با $cv_1$ و $cv_2$ هستند. عدد $c$ یک «اسکالر» نامیده می‌شود. \\
	\textit{(توضیح مترجم: به یک عدد تنها که برای مقیاس‌دهی (بزرگ یا کوچک کردن) طول یک بردار استفاده می‌شود، اسکالر می‌گویند. این واژه از کلمه scale به معنای مقیاس گرفته شده است. ضرب در یک اسکالر مثبت جهت بردار را تغییر نمی‌دهد، اما ضرب در اسکالر منفی جهت آن را ۱۸۰ درجه برعکس می‌کند.)}
	
	توجه کنید که جمع $\mathbf{-v}$ و $\mathbf{v}$ بردار صفر است. این $\mathbf{0}$ است که با عدد صفر یکی نیست! بردار $\mathbf{0}$ مؤلفه‌های ۰ و ۰ را دارد. مرا ببخشید که بر تفاوت بین یک بردار و مؤلفه‌هایش پافشاری می‌کنم. جبر خطی بر پایه این عملیات بنا شده است: $\mathbf{v} + \mathbf{w}$ و $c\mathbf{v}$ و $d\mathbf{w}$ -- جمع کردن بردارها و ضرب کردن در اسکالرها.
	
	\subsection{ترکیب‌های خطی}
	اکنون ما جمع را با ضرب اسکالر ترکیب می‌کنیم تا یک «ترکیب خطی» از $\mathbf{v}$ و $\mathbf{w}$ بسازیم. $\mathbf{v}$ را در $c$ و $\mathbf{w}$ را در $d$ ضرب کنید. سپس $c\mathbf{v} + d\mathbf{w}$ را جمع کنید.
	\begin{quote}
		جمع $c\mathbf{v}$ و $d\mathbf{w}$ یک \textbf{ترکیب خطی} $c\mathbf{v} + d\mathbf{w}$ است.
	\end{quote}
	چهار ترکیب خطی خاص عبارتند از: جمع، تفاضل، صفر، و یک مضرب اسکالر $c\mathbf{v}$:
	\begin{itemize}
		\item $1\mathbf{v} + 1\mathbf{w}$ (جمع بردارها در شکل ۱.۱ الف)
		\item $1\mathbf{v} - 1\mathbf{w}$ (تفاضل بردارها در شکل ۱.۱ ب)
		\item $0\mathbf{v} + 0\mathbf{w}$ (بردار صفر)
		\item $c\mathbf{v} + 0\mathbf{w}$ (بردار $c\mathbf{v}$ در جهت $\mathbf{v}$)
	\end{itemize}
	بردار صفر همیشه یک ترکیب ممکن است (ضرایب آن صفر هستند). هر بار که ما یک «فضای» برداری را می‌بینیم، آن بردار صفر نیز در آن گنجانده شده است. \\
	\textit{(توضیح مترجم: در جبر خطی، وقتی از «فضا» صحبت می‌کنیم، منظور مجموعه‌ای از بردارهاست که تحت دو عمل اصلی جمع برداری و ضرب اسکالر بسته است. یعنی هر ترکیب خطی از بردارهای آن مجموعه، خود برداری در همان مجموعه خواهد بود. وجود بردار صفر در هر فضای برداری یک اصل است، زیرا نقطه شروع یا مبدأ تمام حرکات است.)}
	
	شکل‌ها نشان می‌دهند که چگونه می‌توانید بردارها را تجسم کنید. برای جبر، ما فقط به مؤلفه‌ها (مانند ۴ و ۲) نیاز داریم. آن بردار $\mathbf{v}$ با یک فلش نمایش داده می‌شود. فلش به اندازه $v_1 = 4$ واحد به راست و $v_2 = 2$ واحد به بالا می‌رود. این فلش در نقطه‌ای که مختصات $x, y$ آن ۴ و ۲ است، به پایان می‌رسد. این نقطه نمایش دیگری از بردار است -- بنابراین ما سه راه برای توصیف $\mathbf{v}$ داریم:
	\begin{description}
		\item[نمایش بردار v]
		\item[دو عدد:] مؤلفه‌های بردار مانند $(4,2)$.
		\item[یک فلش از مبدأ (۰,۰):] یک پیکان که از مرکز مختصات شروع شده و به نقطه $(4,2)$ اشاره می‌کند. این نمایش، جهت و اندازه بردار را به خوبی نشان می‌دهد.
		\item[یک نقطه در صفحه:] خود نقطه انتهایی $(4,2)$ نماینده بردار است، با این فرض که نقطه شروع، مبدأ است.
	\end{description}
	ما با استفاده از اعداد جمع می‌کنیم. ما $\mathbf{v} + \mathbf{w}$ را با استفاده از فلش‌ها تجسم می‌کنیم:
	\begin{description}
		\item[جمع برداری (قانون سر به دم یا متوازی‌الاضلاع):] برای جمع دو بردار $\mathbf{v}$ و $\mathbf{w}$، فلش بردار $\mathbf{w}$ را به انتهای (سر) فلش بردار $\mathbf{v}$ منتقل می‌کنیم. بردار حاصل جمع، فلشی است که از ابتدای $\mathbf{v}$ به انتهای $\mathbf{w}$ کشیده می‌شود. این بردار، قطر متوازی‌الاضلاعی است که $\mathbf{v}$ و $\mathbf{w}$ دو ضلع مجاور آن هستند.
	\end{description}
	
	\begin{figure}[h!]
		\centering
		% برای استفاده، این خط را از حالت کامنت خارج کرده و نام فایل تصویر خود را جایگزین کنید
		% \includegraphics[width=0.8\textwidth]{figure1-1.png} 
		\fbox{تصویر شکل ۱.۱ در اینجا قرار می‌گیرد}
		\caption{جمع برداری $\mathbf{v} + \mathbf{w} = (3, 4)$ قطر یک متوازی‌الاضلاع را تولید می‌کند. بردار معکوس $\mathbf{w}$ برابر با $-\mathbf{w}$ است. ترکیب خطی در سمت راست $\mathbf{v} - \mathbf{w} = (5, 0)$ است.}
	\end{figure}
	
	ما در امتداد $\mathbf{v}$ و سپس در امتداد $\mathbf{w}$ حرکت می‌کنیم. یا مسیر میان‌بر را در امتداد $\mathbf{v} + \mathbf{w}$ طی می‌کنیم. ما همچنین می‌توانیم در امتداد $\mathbf{w}$ و سپس $\mathbf{v}$ برویم. به عبارت دیگر، $\mathbf{w} + \mathbf{v}$ همان جواب $\mathbf{v} + \mathbf{w}$ را می‌دهد (خاصیت جابجایی). این‌ها مسیرهای متفاوتی در طول متوازی‌الاضلاع هستند.
	
	\subsection{بردارها در سه بعد}
	یک بردار با دو مؤلفه به یک نقطه در صفحه $xy$ متناظر است. مؤلفه‌های $\mathbf{v}$ مختصات آن نقطه هستند: $x=v_1$ و $y=v_2$. فلش در این نقطه $(v_1, v_2)$ به پایان می‌رسد، وقتی که از $(0,0)$ شروع شود. اکنون ما اجازه می‌دهیم بردارها سه مؤلفه داشته باشند $(v_1, v_2, v_3)$.
	
	صفحه $xy$ با فضای سه بعدی $xyz$ جایگزین می‌شود. در اینجا بردارهای نوعی آمده‌اند (همچنان بردارهای ستونی اما با سه مؤلفه):
	\[ \mathbf{v} = \begin{bmatrix} 1 \\ 1 \\ -1 \end{bmatrix} \quad \text{و} \quad \mathbf{w} = \begin{bmatrix} 2 \\ 3 \\ 4 \end{bmatrix} \quad \text{و} \quad \mathbf{v} + \mathbf{w} = \begin{bmatrix} 3 \\ 4 \\ 3 \end{bmatrix} \]
	بردار $\mathbf{v}$ به یک فلش در فضای سه‌بعدی متناظر است. معمولاً فلش از «مبدأ»، جایی که محورهای $xyz$ به هم می‌رسند و مختصات $(0,0,0)$ است، شروع می‌شود. فلش در نقطه‌ای با مختصات $v_1, v_2, v_3$ به پایان می‌رسد. یک تطابق کامل بین بردار ستونی، فلش از مبدأ و نقطه‌ای که فلش در آن به پایان می‌رسد، وجود دارد.
	\begin{quote}
		بردار $(x, y)$ در صفحه با بردار $(x, y, 0)$ در فضای سه‌بعدی متفاوت است! بردار اول در فضای دو بعدی $\mathbb{R}^2$ زندگی می‌کند، در حالی که دومی یک بردار در فضای سه بعدی $\mathbb{R}^3$ است که ارتفاع آن صفر است.
	\end{quote}
	
	\begin{figure}[h!]
		\centering
		% \includegraphics[width=0.7\textwidth]{figure1-2.png}
		\fbox{تصویر شکل ۱.۲ در اینجا قرار می‌گیرد}
		\caption{بردارهای $\begin{bmatrix} 3 \\ 2 \end{bmatrix}$ و $\begin{bmatrix} 3 \\ 2 \\ 4 \end{bmatrix}$ به نقاط $(x,y)$ و $(x,y,z)$ متناظر هستند.}
	\end{figure}
	
	از این به بعد $\mathbf{v} = \begin{bmatrix} 1 \\ 1 \\ -1 \end{bmatrix}$ به صورت $\mathbf{v} = (1, 1, -1)$ نیز نوشته می‌شود.
	دلیل استفاده از فرم سطری (درون پرانتز) صرفه‌جویی در فضاست. اما $\mathbf{v} = (1, 1, -1)$ یک بردار سطری نیست! در واقعیت یک بردار ستونی است که موقتاً دراز کشیده است. بردار سطری $[1 \ 1 \ -1]$ کاملاً متفاوت است، حتی اگر همان سه مؤلفه را داشته باشد. آن بردار سطری ۱ در ۳ «ترانهاده» بردار ستونی ۳ در ۱، یعنی $\mathbf{v}$ است. \\
	\textit{(توضیح مترجم: ترانهاده (Transpose) یک ماتریس یا بردار، عملی است که در آن جای سطرها و ستون‌ها با هم عوض می‌شود. ترانهاده بردار ستونی $\mathbf{v}$ را با $\mathbf{v}^T$ نمایش می‌دهیم. پس $\mathbf{v} = \begin{bmatrix} 1 \\ 1 \\ -1 \end{bmatrix}$ و $\mathbf{v}^T = [1 \ 1 \ -1]$. این تمایز در محاسبات ماتریسی حیاتی است.)}
	
	در سه بعد، $\mathbf{v} + \mathbf{w}$ همچنان مؤلفه به مؤلفه پیدا می‌شود. جمع دارای مؤلفه‌های $v_1+w_1$ و $v_2+w_2$ و $v_3+w_3$ است. شما می‌بینید که چگونه بردارها را در ۴ یا ۵ یا $n$ بعد جمع کنید. وقتی $\mathbf{w}$ از انتهای $\mathbf{v}$ شروع می‌شود، ضلع سوم $\mathbf{v} + \mathbf{w}$ است. مسیر دیگر در اطراف متوازی‌الاضلاع $\mathbf{w} + \mathbf{v}$ است. سؤال: آیا هر چهار ضلع در یک صفحه قرار دارند؟ بله، همیشه. و جمع $\mathbf{v} + \mathbf{w} - \mathbf{v} - \mathbf{w}$ کاملاً یک دور کامل می‌زند تا بردار \textbf{صفر} را تولید کند.
	
	یک ترکیب خطی نوعی از سه بردار در سه بعد، $\mathbf{u} + 4\mathbf{v} - 2\mathbf{w}$ است.
	
	\subsection*{پرسش‌های مهم}
	برای یک بردار $\mathbf{u}$، تنها ترکیب‌های خطی، مضارب $c\mathbf{u}$ هستند. برای دو بردار، ترکیب‌ها $c\mathbf{u} + d\mathbf{v}$ هستند. برای سه بردار، ترکیب‌ها $c\mathbf{u} + d\mathbf{v} + e\mathbf{w}$ هستند.
	آیا گام بزرگ را از یک ترکیب به \textbf{همه} ترکیب‌ها برمی‌دارید؟ همه مقادیر $c$ و $d$ و $e$ مجاز هستند. فرض کنید بردارهای $\mathbf{u}, \mathbf{v}, \mathbf{w}$ در فضای سه بعدی هستند:
	\begin{enumerate}
		\item تصویر تمام ترکیب‌های $c\mathbf{u}$ چیست؟
		\item تصویر تمام ترکیب‌های $c\mathbf{u} + d\mathbf{v}$ چیست؟
		\item تصویر تمام ترکیب‌های $c\mathbf{u} + d\mathbf{v} + e\mathbf{w}$ چیست؟
	\end{enumerate}
	پاسخ‌ها به بردارهای خاص $\mathbf{u}, \mathbf{v}, \mathbf{w}$ بستگی دارد. اگر آن‌ها بردارهای صفر بودند، آنگاه هر ترکیبی صفر می‌شد. اگر آن‌ها بردارهای غیرصفر نوعی باشند (مؤلفه‌ها به صورت تصادفی انتخاب شده باشند)، در اینجا سه پاسخ آمده است. این کلید موضوع ماست:
	\begin{enumerate}
		\item \textbf{خط:} ترکیب‌های $c\mathbf{u}$ یک \textbf{خط} را که از مبدأ $(0,0,0)$ می‌گذرد، پر می‌کنند. \textit{(این خط تمام نقاطی را شامل می‌شود که با کشیدن یا فشرده کردن بردار $\mathbf{u}$ به دست می‌آیند.)}
		\item \textbf{صفحه:} ترکیب‌های $c\mathbf{u} + d\mathbf{v}$ یک \textbf{صفحه} را که از مبدأ $(0,0,0)$ می‌گذرد، پر می‌کنند. \textit{(این در صورتی است که $\mathbf{u}$ و $\mathbf{v}$ در یک راستا نباشند. این صفحه شامل تمام نقاطی است که با حرکت در جهت $\mathbf{u}$ و سپس حرکت در جهت $\mathbf{v}$ قابل دسترسی هستند.)}
		\item \textbf{فضا:} ترکیب‌های $c\mathbf{u} + d\mathbf{v} + e\mathbf{w}$ \textbf{فضای سه بعدی} را پر می‌کنند. \textit{(این در صورتی است که $\mathbf{w}$ در صفحه‌ای که توسط $\mathbf{u}$ و $\mathbf{v}$ ساخته شده، قرار نداشته باشد. در این حالت، ما سه جهت مستقل داریم که با ترکیب آن‌ها می‌توان به هر نقطه‌ای در فضا رسید.)}
	\end{enumerate}
	بردار صفر $(0,0,0)$ روی خط است زیرا $c$ می‌تواند صفر باشد. روی صفحه است زیرا $c$ و $d$ هر دو می‌توانند صفر باشند. خط بردارهای $c\mathbf{u}$ بی‌نهایت طولانی است (به جلو و عقب). این صفحه تمام $c\mathbf{u} + d\mathbf{v}$ (ترکیب دو بردار در فضای سه بعدی) است که من به خصوص از شما می‌خواهم به آن فکر کنید.
	
	جمع کردن تمام $c\mathbf{u}$ روی یک خط با تمام $d\mathbf{v}$ روی خط دیگر، صفحه را در شکل ۱.۳ پر می‌کند. وقتی بردار سوم $\mathbf{w}$ را شامل می‌کنیم، مضارب $e\mathbf{w}$ یک خط سوم را می‌دهند. فرض کنید آن خط سوم در صفحه $\mathbf{u}$ و $\mathbf{v}$ نباشد. آنگاه ترکیب تمام $e\mathbf{w}$ با تمام $c\mathbf{u} + d\mathbf{v}$ کل فضای سه بعدی را پر می‌کند.
	
	این وضعیت نوعی است! خط، سپس صفحه، سپس فضا. اما احتمالات دیگری نیز وجود دارد. وقتی $\mathbf{w}$ به طور اتفاقی ترکیبی خطی از دو بردار دیگر باشد (مثلاً $\mathbf{w} = c\mathbf{u} + d\mathbf{v}$)، آنگاه بردار سوم $\mathbf{w}$ در صفحه دو بردار اول قرار دارد. در این حالت، $\mathbf{w}$ هیچ «جهت جدیدی» اضافه نمی‌کند و ترکیب‌های $\mathbf{u}, \mathbf{v}, \mathbf{w}$ از آن صفحه $uv$ خارج نخواهند شد. ما کل فضای سه بعدی را به دست نمی‌آوریم. به این حالت \textbf{وابستگی خطی} می‌گویند. لطفاً به موارد خاص در مسئله ۱ فکر کنید.
	
	\begin{figure}[h!]
		\centering
		% \includegraphics[width=0.9\textwidth]{figure1-3.png}
		\fbox{تصویر شکل ۱.۳ در اینجا قرار می‌گیرد}
		\caption{الف) خطی که شامل تمام $c\mathbf{u}$ است. ب) صفحه‌ای که شامل خطوط گذرنده از $\mathbf{u}$ و $\mathbf{v}$ است.}
	\end{figure}
	
	\newpage
	\subsection*{مروری بر ایده‌های کلیدی}
	\begin{enumerate}
		\item یک بردار $\mathbf{v}$ در فضای دو بعدی دارای دو مؤلفه $v_1$ و $v_2$ است.
		\item $\mathbf{v} + \mathbf{w} = (v_1+w_1, v_2+w_2)$ و $c\mathbf{v} = (cv_1, cv_2)$ مؤلفه به مؤلفه پیدا می‌شوند.
		\item یک ترکیب خطی از سه بردار $\mathbf{u}$ و $\mathbf{v}$ و $\mathbf{w}$ به صورت $c\mathbf{u} + d\mathbf{v} + e\mathbf{w}$ است.
		\item تمام ترکیب‌های خطی از $\mathbf{u}$، یا $\mathbf{u}$ و $\mathbf{v}$، یا $\mathbf{u}, \mathbf{v}, \mathbf{w}$ را در نظر بگیرید. در سه بعد، این ترکیب‌ها به طور معمول یک خط، سپس یک صفحه، و سپس کل فضای $\mathbb{R}^3$ را پر می‌کنند.
	\end{enumerate}
	
	\subsection*{مثال‌های حل شده}
	\subsubsection*{مثال ۱.۱ الف}
	ترکیب‌های خطی از $\mathbf{v} = (1,1,0)$ و $\mathbf{w} = (0,1,1)$ یک صفحه را در $\mathbb{R}^3$ پر می‌کنند. آن صفحه را توصیف کنید. برداری پیدا کنید که ترکیبی از $\mathbf{v}$ و $\mathbf{w}$ نباشد—یعنی روی صفحه نباشد.
	
	\textbf{راه حل} \\
	صفحه $\mathbf{v}$ و $\mathbf{w}$ شامل تمام ترکیب‌های $c\mathbf{v} + d\mathbf{w}$ است. بردارها در آن صفحه هر $c$ و $d$ را مجاز می‌دانند.
	\[ \text{ترکیب‌ها} \quad c\mathbf{v} + d\mathbf{w} = c\begin{bmatrix} 1 \\ 1 \\ 0 \end{bmatrix} + d\begin{bmatrix} 0 \\ 1 \\ 1 \end{bmatrix} = \begin{bmatrix} c \\ c+d \\ d \end{bmatrix} \quad \text{یک صفحه را پر می‌کنند.} \]
	برای پیدا کردن مشخصه این صفحه، به رابطه بین مؤلفه‌ها نگاه می‌کنیم. اگر یک بردار دلخواه $\begin{bmatrix} x \\ y \\ z \end{bmatrix}$ در این صفحه باشد، آنگاه باید $x=c, y=c+d, z=d$ باشد. با جایگذاری $c$ و $d$ در معادله دوم، به دست می‌آوریم: $y = x+z$. این معادله، توصیف دقیق این صفحه است. هر برداری که در این رابطه صدق نکند، روی صفحه نیست.
	
	برای مثال، بردار $(1,2,3)$ را بررسی می‌کنیم. آیا $2 = 1+3$ است؟ خیر. پس این بردار در صفحه نیست.
	بردار $(5,7,2)$ را بررسی می‌کنیم. آیا $7 = 5+2$ است؟ بله. پس این بردار در صفحه قرار دارد.
	
	توصیف دیگری از این صفحه گذرنده از $(0,0,0)$ این است که بدانیم بردار $\mathbf{n} = (1, -1, 1)$ بر این صفحه عمود است. بخش ۱.۲ این زاویه ۹۰ درجه را با آزمون ضرب داخلی تأیید خواهد کرد: $\mathbf{v} \cdot \mathbf{n} = 0$ و $\mathbf{w} \cdot \mathbf{n} = 0$. بردارهای عمود بر هم ضرب داخلی صفر دارند.
	
	\subsubsection*{مثال ۱.۱ ب}
	برای $\mathbf{v}=(1,0)$ و $\mathbf{w}=(0,1)$، تمام نقاط $c\mathbf{v}$ را با (۱) اعداد صحیح $c$ و (۲) اعداد غیرمنفی $c \ge 0$ توصیف کنید. سپس تمام بردارهای $d\mathbf{w}$ را اضافه کرده و تمام $c\mathbf{v} + d\mathbf{w}$ را توصیف کنید.
	
	\textbf{راه حل}
	\begin{enumerate}
		\item بردارهای $c\mathbf{v} = (c,0)$ با اعداد صحیح $c$ نقاطی با فواصل مساوی در امتداد محور $x$ (جهت $\mathbf{v}$) هستند. آن‌ها شامل $(-2,0), (-1,0), (0,0), (1,0), (2,0)$ می‌شوند.
		\item بردارهای $c\mathbf{v}$ با $c \ge 0$ یک نیم‌خط را پر می‌کنند. این نیم‌خط، محور $x$ مثبت است. این نیم‌خط از $(0,0)$ که در آن $c=0$ است شروع می‌شود. این شامل $(100,0)$ و $(\pi, 0)$ است اما شامل $(-100,0)$ نیست.
	\end{enumerate}
	\begin{enumerate}
		\item[۱'] افزودن تمام بردارهای $d\mathbf{w} = (0,d)$ یک خط عمودی را از میان آن نقاط $c\mathbf{v}$ با فواصل مساوی عبور می‌دهد. ما بی‌نهایت خط موازی از (عدد صحیح $c$، هر عدد $d$) داریم.
		\item[۲'] افزودن تمام بردارهای $d\mathbf{w}$ یک خط عمودی را از هر $c\mathbf{v}$ روی نیم‌خط عبور می‌دهد. اکنون ما یک نیم‌صفحه داریم. نیمه راست صفحه $xy$ دارای هر $x \ge 0$ و هر $y$ است.
	\end{enumerate}
	
	\subsubsection*{مثال ۱.۱ ج}
	دو معادله برای $c$ و $d$ پیدا کنید به طوری که ترکیب خطی $c\mathbf{v} + d\mathbf{w}$ برابر با $\mathbf{b}$ شود:
	\[ \mathbf{v} = \begin{bmatrix} 2 \\ -1 \end{bmatrix} \quad \mathbf{w} = \begin{bmatrix} -1 \\ 2 \end{bmatrix} \quad \mathbf{b} = \begin{bmatrix} 1 \\ 0 \end{bmatrix} \]
	در کاربرد ریاضیات، بسیاری از مسائل دو بخش دارند:
	\begin{enumerate}
		\item \textbf{بخش مدل‌سازی:} مسئله را با مجموعه‌ای از معادلات بیان کنید.
		\item \textbf{بخش محاسباتی:} آن معادلات را با یک الگوریتم سریع و دقیق حل کنید.
	\end{enumerate}
	در اینجا از ما فقط بخش اول (معادلات) خواسته شده است. فصل ۲ به بخش دوم (حل) اختصاص دارد. مثال ما در یک مدل بنیادی برای جبر خطی قرار می‌گیرد:
	\begin{quote}
		$n$ عدد $c_1, \dots, c_n$ را پیدا کنید به طوری که $c_1\mathbf{v}_1 + \dots + c_n\mathbf{v}_n = \mathbf{b}$.
	\end{quote}
	معادله برداری $c\mathbf{v} + d\mathbf{w} = \mathbf{b}$ به این صورت است:
	\[ c\begin{bmatrix} 2 \\ -1 \end{bmatrix} + d\begin{bmatrix} -1 \\ 2 \end{bmatrix} = \begin{bmatrix} 1 \\ 0 \end{bmatrix} \]
	\textit{(توضیح مترجم: برای اینکه تساوی برداری برقرار باشد، باید مؤلفه‌های متناظر با هم برابر باشند. یعنی مؤلفه اول (بالایی) سمت چپ باید با مؤلفه اول سمت راست، و مؤلفه دوم (پایینی) سمت چپ با مؤلفه دوم سمت راست برابر باشد.)}
	
	\[ \begin{bmatrix} 2c - d \\ -c + 2d \end{bmatrix} = \begin{bmatrix} 1 \\ 0 \end{bmatrix} \]
	این برابری به ما دو معادله خطی مجزا می‌دهد:
	\[
	\begin{cases}
		2c - d = 1 \\
		-c + 2d = 0
	\end{cases}
	\]
	هر معادله یک خط را در صفحه $cd$ توصیف می‌کند. نقطه‌ای که این دو خط یکدیگر را قطع می‌کنند، جواب دستگاه است.
	چرا این را به عنوان یک معادله ماتریسی هم نبینیم، چون مسیری است که به آن سمت می‌رویم:
	\[ \text{ماتریس ۲ در ۲} \quad \begin{bmatrix} 2 & -1 \\ -1 & 2 \end{bmatrix} \begin{bmatrix} c \\ d \end{bmatrix} = \begin{bmatrix} 1 \\ 0 \end{bmatrix} \]
	در اینجا، ستون‌های ماتریس همان بردارهای $\mathbf{v}$ و $\mathbf{w}$ هستند. این نمایش، جوهر اصلی ارتباط بین ترکیب‌های خطی و دستگاه معادلات خطی است.
	
	\newpage
	\section*{مجموعه مسائل ۱.۱}
	
	\subsection*{مسائل ۱-۹ درباره جمع برداری و ترکیب‌های خطی هستند.}
	\begin{enumerate}
		\item تمام ترکیب‌های خطی بردارهای زیر را به صورت هندسی توصیف کنید (خط، صفحه یا تمام $\mathbb{R}^3$).
		\item بردارهای $\mathbf{v}=\begin{bmatrix} 4 \\ 1 \end{bmatrix}$ و $\mathbf{w}=\begin{bmatrix} -2 \\ 2 \end{bmatrix}$ و $\mathbf{v}+\mathbf{w}$ و $\mathbf{v}-\mathbf{w}$ را در یک صفحه $xy$ رسم کنید.
		\item اگر $\mathbf{v}+\mathbf{w}=\begin{bmatrix} 5 \\ 1 \end{bmatrix}$ و $\mathbf{v}-\mathbf{w}=\begin{bmatrix} 1 \\ 5 \end{bmatrix}$، بردارها $\mathbf{v}$ و $\mathbf{w}$ را محاسبه و رسم کنید.
		\item از $\mathbf{v}=\begin{bmatrix} 2 \\ 1 \end{bmatrix}$ و $\mathbf{w}=\begin{bmatrix} 1 \\ 2 \end{bmatrix}$، مؤلفه‌های $3\mathbf{v}+\mathbf{w}$ و $c\mathbf{v}+d\mathbf{w}$ را بیابید.
		\item $\mathbf{u}+\mathbf{v}+\mathbf{w}$ و $2\mathbf{u}+2\mathbf{v}+\mathbf{w}$ را محاسبه کنید. از کجا می‌دانید $\mathbf{u, v, w}$ در یک صفحه قرار دارند؟ اینها در یک صفحه قرار دارند زیرا $\mathbf{w} = c\mathbf{u}+d\mathbf{v}$. مقادیر $c$ و $d$ را بیابید.
		\item مجموع مؤلفه‌های هر ترکیب خطی از $\mathbf{v}=(1,-2,1)$ و $\mathbf{w}=(0,1,-1)$ برابر با \_\_\_ است. $c$ و $d$ را طوری بیابید که $c\mathbf{v}+d\mathbf{w}=(3,3,-6)$. چرا $(3,3,6)$ غیرممکن است؟
		\item در صفحه $xy$ تمام این نه ترکیب خطی را مشخص کنید: $c\begin{bmatrix} 2 \\ 1 \end{bmatrix} + d\begin{bmatrix} 0 \\ 1 \end{bmatrix}$ با $c=0,1,2$ و $d=0,1,2$.
		\item متوازی‌الاضلاع در شکل ۱.۱ قطر $\mathbf{v}+\mathbf{w}$ را دارد. قطر دیگر آن چیست؟ جمع این دو قطر چیست؟ آن جمع برداری را رسم کنید.
		\item اگر سه گوشه یک متوازی‌الاضلاع $(1,1)$، $(4,2)$ و $(1,3)$ باشند، سه گوشه چهارم ممکن کدامند؟ دو مورد از آن‌ها را رسم کنید.
	\end{enumerate}
	
	\begin{figure}[h!]
		\centering
		% \includegraphics[width=0.8\textwidth]{figure1-4.png}
		\fbox{تصویر شکل ۱.۴ در اینجا قرار می‌گیرد}
		\caption{مکعب واحد از $\mathbf{i, j, k}$ و دوازده بردار ساعت.}
	\end{figure}
	
	\subsection*{مسائل ۱۰-۱۴ درباره بردارهای ویژه روی مکعب‌ها و ساعت‌ها در شکل ۱.۴ هستند.}
	\begin{enumerate}
		\setcounter{enumi}{9}
		\item کدام نقطه از مکعب $\mathbf{i}+\mathbf{j}$ است؟ کدام نقطه جمع برداری $\mathbf{i}=(1,0,0)$ و $\mathbf{j}=(0,1,0)$ و $\mathbf{k}=(0,0,1)$ است؟ تمام نقاط $(x,y,z)$ در مکعب را توصیف کنید.
		\item چهار گوشه این مکعب واحد عبارتند از $(0,0,0)$، $(1,0,0)$، $(0,1,0)$ و $(0,0,1)$. چهار گوشه دیگر کدامند؟ مختصات نقطه مرکزی مکعب را بیابید. نقاط مرکزی شش وجه عبارتند از \_\_\_. مکعب چند یال دارد؟
		\item \textbf{سوال مروری.} در فضای $xyz$، صفحه تمام ترکیب‌های خطی $\mathbf{i}=(1,0,0)$ و $\mathbf{i}+\mathbf{j}=(1,1,0)$ کجاست؟
		\item 
		(الف) جمع $V$ دوازده برداری که از مرکز ساعت به ساعت‌های 00:1، 00:2، ...، 00:12 می‌روند، چیست؟
		(ب) اگر بردار ساعت 00:2 حذف شود، چرا ۱۱ بردار باقی‌مانده به 00:8 جمع می‌شوند؟
		(ج) مؤلفه‌های $x, y$ آن بردار ساعت 00:2 یعنی $\mathbf{v}=(\cos\theta, \sin\theta)$ چیست؟
		\item فرض کنید دوازده بردار به جای مرکز $(0,0)$ از ساعت 00:6در پایین شروع شوند. بردار به سمت ساعت 00:12 دو برابر شده و به $(0,2)$ تبدیل می‌شود. دوازده بردار جدید به \_\_\_ جمع می‌شوند.
	\end{enumerate}
	
	\begin{figure}[h!]
		\centering
		% \includegraphics[width=0.9\textwidth]{figure1-5.png}
		\fbox{تصویر شکل ۱.۵ در اینجا قرار می‌گیرد}
		\caption{مسائل ۱۵-۱۹ در یک صفحه}
	\end{figure}
	
	\subsection*{مسائل ۱۵-۱۹ با ترکیب‌های خطی $\mathbf{v}$ و $\mathbf{w}$ پیش می‌روند (شکل ۱.۵ الف).}
	\begin{enumerate}
		\setcounter{enumi}{14}
		\item شکل ۱.۵ الف، $\frac{1}{2}\mathbf{v} + \frac{1}{2}\mathbf{w}$ را نشان می‌دهد. نقاط $\frac{3}{4}\mathbf{v} + \frac{1}{4}\mathbf{w}$ و $\frac{1}{4}\mathbf{v} + \frac{3}{4}\mathbf{w}$ و $\mathbf{v}+\mathbf{w}$ را مشخص کنید.
		\item نقطه $\mathbf{-v} + \mathbf{2w}$ و هر ترکیب دیگر $c\mathbf{v}+d\mathbf{w}$ با $c+d=1$ را مشخص کنید. خط تمام ترکیب‌هایی که $c+d=1$ دارند را رسم کنید.
		\item $\frac{1}{3}\mathbf{v}+\frac{1}{3}\mathbf{w}$ و $\frac{2}{3}\mathbf{v}+\frac{2}{3}\mathbf{w}$ را مشخص کنید. ترکیب‌های $c\mathbf{v}+c\mathbf{w}$ چه خطی را پر می‌کنند؟
		\item با محدودیت $0 \le c \le 1$ و $0 \le d \le 1$، تمام ترکیب‌های $c\mathbf{v}+d\mathbf{w}$ را سایه بزنید.
		\item فقط با محدودیت $c \ge 0$ و $d \ge 0$، «مخروط» تمام ترکیب‌های $c\mathbf{v}+d\mathbf{w}$ را رسم کنید.
	\end{enumerate}
	
	\subsection*{مسائل ۲۰-۲۵ در فضای سه بعدی}
	\begin{enumerate}
		\setcounter{enumi}{19}
		\item $\frac{1}{3}\mathbf{u}+\frac{1}{3}\mathbf{v}+\frac{1}{3}\mathbf{w}$ و $\frac{1}{2}\mathbf{u}+\frac{1}{2}\mathbf{w}$ را در شکل ۱.۵ ب مشخص کنید. \textbf{مسئله چالشی:} تحت چه محدودیت‌هایی بر $c, d, e$ ترکیب‌های $c\mathbf{u}+d\mathbf{v}+e\mathbf{w}$ مثلث خط‌چین را پر می‌کنند؟ برای ماندن در مثلث، یک شرط $c \ge 0, d \ge 0, e \ge 0$ است.
		\item سه ضلع مثلث خط‌چین عبارتند از $\mathbf{v}-\mathbf{u}$ و $\mathbf{w}-\mathbf{v}$ و $\mathbf{u}-\mathbf{w}$. جمع آن‌ها \_\_\_ است. جمع سر به دم برداری $(3,1)$ به علاوه $(-1,1)$ به علاوه $(-2,-2)$ را در اطراف یک مثلث صفحه‌ای رسم کنید.
		\item هرم ترکیب‌های $c\mathbf{u}+d\mathbf{v}+e\mathbf{w}$ با $c \ge 0, d \ge 0, e \ge 0$ و $c+d+e \le 1$ را سایه بزنید. بردار $\frac{1}{2}(\mathbf{u}+\mathbf{v}+\mathbf{w})$ را به عنوان داخل یا خارج این هرم مشخص کنید.
		\item اگر به تمام ترکیب‌های آن $\mathbf{u, v, w}$ نگاه کنید، آیا برداری وجود دارد که نتوان از $c\mathbf{u}+d\mathbf{v}+e\mathbf{w}$ تولید کرد؟ پاسخ متفاوت است اگر $\mathbf{u, v, w}$ همگی در \_\_\_ باشند.
		\item کدام بردارها هم ترکیب‌هایی از $\mathbf{u}$ و $\mathbf{v}$ هستند و هم ترکیب‌هایی از $\mathbf{v}$ و $\mathbf{w}$؟
		\item بردارهای $\mathbf{u,v,w}$ را طوری رسم کنید که ترکیب‌هایشان $c\mathbf{u}+d\mathbf{v}+e\mathbf{w}$ فقط یک خط را پر کنند. بردارهای $\mathbf{u,v,w}$ را پیدا کنید که ترکیب‌هایشان فقط یک صفحه را پر کنند.
		\item چه ترکیبی از $c\begin{bmatrix} 1 \\ 2 \end{bmatrix} + d\begin{bmatrix} 3 \\ 1 \end{bmatrix}$ بردار $\begin{bmatrix} 14 \\ 8 \end{bmatrix}$ را تولید می‌کند؟ این سوال را به صورت دو معادله برای ضرایب $c$ و $d$ در ترکیب خطی بیان کنید.
	\end{enumerate}
	
	\subsection*{مسائل چالشی}
	\begin{enumerate}
		\setcounter{enumi}{26}
		\item یک مکعب در ۴ بعد چند گوشه دارد؟ چند وجه سه بعدی؟ چند یال؟ یک گوشه نوعی $(0,0,1,0)$ است. یک یال نوعی به $(0,1,0,0)$ می‌رود.
		\item بردارهای $\mathbf{v}$ و $\mathbf{w}$ را طوری بیابید که $\mathbf{v}+\mathbf{w}=(4,5,6)$ و $\mathbf{v}-\mathbf{w}=(2,5,8)$. این یک مسئله با \_\_\_ عدد مجهول و تعداد مساوی معادله برای یافتن آن اعداد است.
		\item دو ترکیب متفاوت از سه بردار $\mathbf{u}=(1,3)$ و $\mathbf{v}=(2,7)$ و $\mathbf{w}=(1,5)$ را بیابید که $\mathbf{b}=(0,1)$ را تولید کنند. سوال کمی ظریف: اگر من هر سه بردار $\mathbf{u,v,w}$ را در صفحه بگیرم، آیا همیشه دو ترکیب متفاوت وجود خواهد داشت که $\mathbf{b}=(0,1)$ را تولید کنند؟
		\item ترکیب‌های خطی $\mathbf{v}=(a,b)$ و $\mathbf{w}=(c,d)$ صفحه را پر می‌کنند مگر اینکه \_\_\_.
		\ چهار بردار $\mathbf{u,v,w,z}$ با چهار مؤلفه هر کدام بیابید به طوری که ترکیب‌هایشان $c\mathbf{u}+d\mathbf{v}+e\mathbf{w}+f\mathbf{z}$ تمام بردارهای $(b_1,b_2,b_3,b_4)$ را در فضای چهار بعدی تولید کنند.
		\item سه معادله برای $c,d,e$ بنویسید به طوری که $c\mathbf{u}+d\mathbf{v}+e\mathbf{w}=\mathbf{b}$. آیا می‌توانید به نحوی $c,d,e$ را برای این $\mathbf{b}$ پیدا کنید؟
		\[ \mathbf{u} = \begin{bmatrix} 2 \\ -1 \\ 0 \end{bmatrix} \quad \mathbf{v} = \begin{bmatrix} -1 \\ 2 \\ -1 \end{bmatrix} \quad \mathbf{w} = \begin{bmatrix} 0 \\ -1 \\ 2 \end{bmatrix} \quad \mathbf{b} = \begin{bmatrix} 1 \\ 0 \\ 0 \end{bmatrix} \]
	\end{enumerate}
	
\end{document}