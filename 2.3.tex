\documentclass[12pt, a4paper]{book}

% فراخوانی بسته‌های لازم
\usepackage{amsmath}         % برای فرمول‌های پیشرفته ریاضی
\usepackage{amsfonts}        % بسته برای فونت‌های ریاضی مانند \mathbb
\usepackage{amssymb}         % برای نمادهای بیشتر ریاضی
\usepackage{graphicx}        % برای افزودن تصاویر
\usepackage{xepersian}       % بسته اصلی برای پارسی‌نویسی
\usepackage{geometry}        % برای تنظیم حاشیه‌ها
\usepackage{setspace}        % برای تنظیم فاصله خطوط
\usepackage{array}           % برای امکانات پیشرفته در جدول‌ها و آرایه‌ها

% تنظیم حاشیه‌های صفحه
\geometry{
	a4paper,
	total={170mm,257mm},
	left=20mm,
	top=20mm,
}

% تنظیم فونت‌های نوشتاری و ریاضی
% توجه: این فونت‌ها باید روی سیستم شما نصب باشند
\settextfont{XB Niloofar}
\setdigitfont{XB Niloofar}
\setmathdigitfont{XB Niloofar}

\begin{document}
	
	% اعمال فاصله 1.5 بین خطوط برای خوانایی بهتر
	\onehalfspacing
	
	\chapter{حل معادلات خطی}
	\section{حذف با استفاده از ماتریس‌ها}
	
	\begin{enumerate}
		\item گام اول، معادلات $A\mathbf{x}=\mathbf{b}$ را در یک ماتریس $E_{21}$ ضرب می‌کند تا به $E_{21}A\mathbf{x} = E_{21}\mathbf{b}$ برسیم.
		\item ماتریس $E_{21}A$ در جایگاه سطر دوم و ستون اول خود یک صفر دارد، زیرا $x_1$ از معادله دوم حذف شده است.
		\item ماتریس $E_{21}$ همان ماتریس همانی (با ۱ روی قطر اصلی) است که مضرب $a_{21}/a_{11}$ از درایه واقع در سطر دوم و ستون اول آن کم شده است.
		\item ضرب ماتریس در ماتریس، معادل $n$ بار ضرب ماتریس در بردار است: $EA = [E\mathbf{a}_1 \ \dots \ E\mathbf{a}_n]$.
		\item ما باید $E\mathbf{b}$ را نیز محاسبه کنیم! بنابراین $E$ در حال ضرب شدن در \textbf{ماتریس الحاقی} $[A \ \mathbf{b}] = [\mathbf{a}_1 \ \dots \ \mathbf{a}_n \ \mathbf{b}]$ است.
		\item فرآیند حذف، $A\mathbf{x}=\mathbf{b}$ را به ترتیب در ماتریس‌های $E_{21}, E_{31}, \dots, E_{n1}$، سپس $E_{32}, E_{42}, \dots, E_{n2}$ و الی آخر ضرب می‌کند.
		\item ماتریس تعویض سطر، $E_{ij}$ نیست، بلکه $P_{ij}$ است. برای یافتن $P_{ij}$، کافی است سطرهای $i$ و $j$ ماتریس همانی $I$ را با هم تعویض کنید.
	\end{enumerate}
	
	این بخش اولین مثال‌های ما از ضرب ماتریسی را ارائه می‌دهد. طبیعتاً با ماتریس‌هایی شروع می‌کنیم که صفرهای زیادی دارند. هدف ما این است که ببینیم ماتریس‌ها «کاری انجام می‌دهند». ماتریس $E$ بر روی بردار $\mathbf{b}$ یا ماتریس $A$ «اثر می‌کند» تا بردار جدید $E\mathbf{b}$ یا ماتریس جدید $EA$ را تولید کند.
	
	اولین مثال‌های ما «ماتریس‌های حذفی» خواهند بود. این ماتریس‌ها مراحل حذف را اجرا می‌کنند. یعنی معادله $j$-ام را در $l_{ij}$ ضرب کرده و از معادله $i$-ام کم می‌کنند (این کار متغیر $x_j$ را از معادله $i$-ام حذف می‌کند). ما به تعداد زیادی از این ماتریس‌های ساده $E_{ij}$ نیاز داریم، یکی برای هر درایه غیرصفری که باید در پایین قطر اصلی صفر شود.
	
	خوشبختانه در فصول بعدی، تمام این ماتریس‌های $E_{ij}$ را نخواهیم دید. آن‌ها برای شروع مثال‌های خوبی هستند، اما تعدادشان بیش از حد زیاد است. این ماتریس‌ها می‌توانند در یک ماتریس کلی $E$ ترکیب شوند که تمام مراحل را یک‌جا انجام می‌دهد. بهترین راه این است که معکوس همه‌ی آن‌ها یعنی $(E_{ij})^{-1}$ را در یک ماتریس کلی $L = E^{-1}$ ترکیب کنیم. هدف صفحات بعدی همین است:
	
	\begin{enumerate}
		\item دیدن اینکه چگونه هر مرحله از حذف، یک ضرب ماتریسی است.
		\item سرهم کردن تمام آن مراحل $E_{ij}$ در یک ماتریس حذفی واحد $E$.
		\item دیدن اینکه چگونه هر $E_{ij}$ با ماتریس معکوس خود $E_{ij}^{-1}$ وارون می‌شود.
		\item سرهم کردن تمام آن معکوس‌ها $E_{ij}^{-1}$ (با ترتیب درست) در ماتریس $L$.
	\end{enumerate}
	
	ویژگی خاص ماتریس $L$ این است که تمام مضرب‌های $l_{ij}$ دقیقاً در جای خود قرار می‌گیرند. این اعداد در ماتریس $E$ (حذف پیش‌رو از $A$ به $U$) درهم‌ریخته هستند، اما در ماتریس $L$ (خنثی کردن حذف و بازگشت از $U$ به $A$) کاملاً مرتب هستند. عمل معکوس‌گیری، مراحل و ماتریس‌های $E_{ij}$ را در ترتیب مخالف قرار می‌دهد و همین کار از درهم‌ریختگی جلوگیری می‌کند.
	
	این بخش ماتریس‌های $E_{ij}$ را پیدا می‌کند. بخش ۲.۴ چهار روش برای ضرب ماتریس‌ها را ارائه می‌دهد. بخش ۲.۵ هر مرحله را معکوس می‌کند (برای ماتریس‌های حذفی، ما همین‌جا می‌توانیم $E_{ij}^{-1}$ را ببینیم). سپس آن معکوس‌ها در ماتریس $L$ قرار می‌گیرند.
	
	\subsection*{ماتریس‌ها در بردارها و $A\mathbf{x}=\mathbf{b}$}
	
	مثال ۳ در ۳ در بخش قبل، فرم کوتاه $A\mathbf{x}=\mathbf{b}$ را دارد:
	
	\begin{align*}
		2x_1 + 4x_2 - 2x_3 &= 2 \\
		4x_1 + 9x_2 - 3x_3 &= 8 \\
		-2x_1 - 3x_2 + 7x_3 &= 10
	\end{align*}
	این دستگاه معادلات، معادل است با:
	\[
	\begin{bmatrix}
		2 & 4 & -2 \\
		4 & 9 & -3 \\
		-2 & -3 & 7
	\end{bmatrix}
	\begin{bmatrix}
		x_1 \\ x_2 \\ x_3
	\end{bmatrix}
	=
	\begin{bmatrix}
		2 \\ 8 \\ 10
	\end{bmatrix} \quad (1)
	\]
	نه عدد سمت چپ در ماتریس $A$ قرار می‌گیرند. این ماتریس فقط کنار $\mathbf{x}$ نمی‌نشیند، بلکه $A$ در $\mathbf{x}$ ضرب می‌شود. قانون «ضرب $A$ در $\mathbf{x}$» دقیقاً طوری انتخاب شده است که سه معادله اصلی را نتیجه دهد.
	
	\textbf{مروری بر ضرب $A$ در $\mathbf{x}$:} یک ماتریس ضربدر یک بردار، یک بردار را نتیجه می‌دهد. وقتی تعداد معادلات (سه) با تعداد مجهولات (سه) برابر باشد، ماتریس مربع است. ماتریس ما ۳ در ۳ است. یک ماتریس مربع عمومی $n$ در $n$ است. در این صورت بردار $\mathbf{x}$ در فضای $n$-بعدی $\mathbb{R}^n$ قرار دارد.
	
	\textit{(توضیح مترجم: دو دیدگاه برای ضرب $A\mathbf{x}$ وجود دارد که هر دو بسیار مهم هستند.)}
	
	\textbf{نکته کلیدی:} $A\mathbf{x}=\mathbf{b}$ هم نمایش سطری و هم نمایش ستونی معادلات را نشان می‌دهد.
	
	\textbf{نمایش ستونی}
	\[ A\mathbf{x} = x_1 \begin{bmatrix} 2 \\ 4 \\ -2 \end{bmatrix} + x_2 \begin{bmatrix} 4 \\ 9 \\ -3 \end{bmatrix} + x_3 \begin{bmatrix} -2 \\ -3 \\ 7 \end{bmatrix} = \begin{bmatrix} 2 \\ 8 \\ 10 \end{bmatrix} \quad (2) \]
	$A\mathbf{x}$ یک ترکیب خطی از ستون‌های ماتریس $A$ است. برای محاسبه هر مؤلفه از $A\mathbf{x}$، ما از \textbf{نمایش سطری} ضرب ماتریسی استفاده می‌کنیم. مؤلفه‌های $A\mathbf{x}$ حاصل ضرب داخلی سطرهای $A$ با بردار $\mathbf{x}$ هستند.
	
	فرمول کوتاه برای این ضرب داخلی با $\mathbf{x}$ از «نماد سیگما» ($\Sigma$) استفاده می‌کند.
	مؤلفه $i$-ام $A\mathbf{x}$ برابر است با $(\text{سطر } i) \cdot \mathbf{x} = a_{i1}x_1 + a_{i2}x_2 + \dots + a_{in}x_n$.
	این عبارت گاهی با نماد سیگما به صورت $\sum_{j=1}^{n} a_{ij}x_j$ نوشته می‌شود. $\Sigma$ دستوری برای جمع زدن است. با $j=1$ شروع کرده و با $j=n$ پایان می‌یابد. این جمع با $a_{i1}x_1$ آغاز و با $a_{in}x_n$ ختم می‌شود و حاصلضرب داخلی $(\text{سطر } i) \cdot \mathbf{x}$ را تولید می‌کند.
	
	یک نکته در مورد نمادگذاری ماتریس که باید تکرار شود: درایه واقع در سطر ۱ و ستون ۱ (گوشه بالا سمت چپ) $a_{11}$ است. درایه سطر ۱ و ستون ۳، $a_{13}$ است. درایه سطر ۳ و ستون ۱، $a_{31}$ است (شماره سطر قبل از شماره ستون می‌آید). واژه «درایه» برای ماتریس معادل واژه «مؤلفه» برای بردار است. قانون کلی: $a_{ij} = A(i,j)$ در سطر $i$ و ستون $j$ قرار دارد.
	
	\subsubsection*{مثال ۱}
	این ماتریس دارای $a_{ij} = 2i+j$ است. در نتیجه $a_{11}=3$ و $a_{12}=4$ و $a_{21}=5$.
	در اینجا $A\mathbf{x}$ به روش سطری با اعداد و حروف نشان داده شده است:
	\[
	\begin{bmatrix} 3 & 4 \\ 5 & 6 \end{bmatrix}
	\begin{bmatrix} 2 \\ 1 \end{bmatrix}
	=
	\begin{bmatrix} 3 \cdot 2 + 4 \cdot 1 \\ 5 \cdot 2 + 6 \cdot 1 \end{bmatrix}
	=
	\begin{bmatrix} 10 \\ 16 \end{bmatrix}
	\]
	یک سطر ضربدر یک ستون، یک ضرب داخلی را نتیجه می‌دهد.
	
	\subsection*{شکل ماتریسی یک مرحله از حذف}
	$A\mathbf{x}=\mathbf{b}$ فرم مناسبی برای معادله اولیه است. در مورد مراحل حذف چطور؟ در این مثال، معادله اول ۲ بار ضرب شده و از معادله دوم کم می‌شود. در سمت راست، مؤلفه اول $\mathbf{b}$ دو بار ضرب شده و از مؤلفه دوم کم می‌شود.
	
	\textbf{گام اول}
	\[ \mathbf{b} = \begin{bmatrix} 2 \\ 8 \\ 10 \end{bmatrix} \quad \text{تغییر می‌کند به} \quad \mathbf{b}_{\text{new}} = \begin{bmatrix} 2 \\ 4 \\ 10 \end{bmatrix} \]
	ما می‌خواهیم این تفریق را با یک ماتریس انجام دهیم! همین نتیجه $\mathbf{b}_{\text{new}} = E\mathbf{b}$ زمانی حاصل می‌شود که ما یک «ماتریس حذفی» $E$ را در $\mathbf{b}$ ضرب کنیم. این ماتریس، ۲ برابر $b_1$ را از $b_2$ کم می‌کند:
	\[
	\text{ماتریس حذفی } E = \begin{bmatrix} 1 & 0 & 0 \\ -2 & 1 & 0 \\ 0 & 0 & 1 \end{bmatrix}
	\]
	ضرب در $E$، دو برابر سطر ۱ را از سطر ۲ کم می‌کند. سطر ۱ و ۳ بدون تغییر باقی می‌مانند:
	\[
	\begin{bmatrix} 1 & 0 & 0 \\ -2 & 1 & 0 \\ 0 & 0 & 1 \end{bmatrix}
	\begin{bmatrix} 2 \\ 8 \\ 10 \end{bmatrix}
	=
	\begin{bmatrix} 2 \\ 8-2(2) \\ 10 \end{bmatrix}
	=
	\begin{bmatrix} 2 \\ 4 \\ 10 \end{bmatrix}
	\]
	سطر اول و سوم ماتریس $E$ از ماتریس همانی $I$ آمده‌اند. آن‌ها اعداد اول و سوم (۲ و ۱۰) را تغییر نمی‌دهند. مؤلفه دوم جدید، عدد ۴ است که بعد از مرحله حذف ظاهر شد. این همان $b_2 - 2b_1$ است.
	
	توصیف «ماتریس‌های مقدماتی» یا «ماتریس‌های حذفی» مانند این $E$ آسان است.
	\begin{quote}
		با ماتریس همانی $I$ شروع کنید. یکی از صفرهای آن را به مضرب $-l$ تغییر دهید.
	\end{quote}
	ماتریس همانی روی قطر اصلی خود ۱ و در بقیه جاها ۰ دارد. در نتیجه $I\mathbf{b}=\mathbf{b}$ برای هر $\mathbf{b}$ برقرار است.
	ماتریس مقدماتی یا حذفی $E_{ij}$ یک درایه غیرصفر اضافی $-l$ در موقعیت $i,j$ دارد. در نتیجه $E_{ij}$ مضرب $l$ از سطر $j$ را از سطر $i$ کم می‌کند.
	
	\subsubsection*{مثال ۲}
	ماتریس $E_{31}$ درایه $-l$ را در موقعیت ۳,۱ دارد:
	\[
	\text{همانی } I = \begin{bmatrix} 1 & 0 & 0 \\ 0 & 1 & 0 \\ 0 & 0 & 1 \end{bmatrix} \quad
	\text{حذفی } E_{31} = \begin{bmatrix} 1 & 0 & 0 \\ 0 & 1 & 0 \\ -l & 0 & 1 \end{bmatrix}
	\]
	وقتی $I$ را در $\mathbf{b}$ ضرب کنید، خود $\mathbf{b}$ را به دست می‌آورید. اما $E_{31}$ مضرب $l$ از مؤلفه اول را از مؤلفه سوم کم می‌کند. با $l=4$ این مثال $9-4=5$ را نتیجه می‌دهد:
	\[
	\mathbf{b} = \begin{bmatrix} 1 \\ 2 \\ 9 \end{bmatrix} \quad \text{و} \quad
	E\mathbf{b} = \begin{bmatrix} 1 & 0 & 0 \\ 0 & 1 & 0 \\ -4 & 0 & 1 \end{bmatrix}
	\begin{bmatrix} 1 \\ 2 \\ 9 \end{bmatrix}
	=
	\begin{bmatrix} 1 \\ 2 \\ 5 \end{bmatrix}
	\]
	در مورد سمت چپ معادله $A\mathbf{x}=\mathbf{b}$ چطور؟ هر دو طرف باید در این $E_{31}$ ضرب شوند.
	هدف $E_{31}$ این است که در موقعیت (۳,۱) ماتریس، یک صفر ایجاد کند.
	نمادگذاری با این هدف متناسب است. با $A$ شروع کنید. ماتریس‌های $E$ را اعمال کنید تا زیر لولاها صفر ایجاد شود (اولین $E$ ماتریس $E_{21}$ است). در نهایت به یک ماتریس بالا مثلثی $U$ برسید. اکنون با جزئیات به این مراحل نگاه می‌کنیم.
	
	یک نکته کوچک. بردار $\mathbf{x}$ ثابت می‌ماند. جواب $\mathbf{x}$ با حذف تغییر نمی‌کند. (شاید این نکته‌ای بیش از یک نکته کوچک باشد.) این ماتریس ضرایب است که تغییر می‌کند. وقتی با $A\mathbf{x}=\mathbf{b}$ شروع می‌کنیم و در $E$ ضرب می‌کنیم، نتیجه $EA\mathbf{x}=E\mathbf{b}$ است.
	ماتریس جدید $EA$ نتیجه ضرب $E$ در $A$ است.
	
	\textbf{اعتراف:} ماتریس‌های حذفی $E_{ij}$ مثال‌های عالی هستند، اما شما آن‌ها را بعداً نخواهید دید. آن‌ها نشان می‌دهند که یک ماتریس چگونه بر روی سطرها عمل می‌کند. با انجام چندین مرحله حذف، ما خواهیم دید که چگونه ماتریس‌ها را ضرب کنیم (و ترتیب ماتریس‌های $E$ مهم می‌شود). ضرب‌ها و معکوس‌ها به ویژه برای ماتریس‌های $E$ واضح هستند. این دو ایده هستند که کتاب از آن‌ها استفاده خواهد کرد.
	
	\subsection*{ضرب ماتریسی}
	سوال بزرگ این است: چگونه دو ماتریس را در هم ضرب کنیم؟ وقتی ماتریس اول $E$ است، ما می‌دانیم که برای $EA$ چه انتظاری داشته باشیم. این $E$ خاص، ۲ برابر سطر ۱ را از سطر ۲ کم می‌کند. مضرب $l=2$ است:
	\[
	\begin{bmatrix} 1 & 0 & 0 \\ -2 & 1 & 0 \\ 0 & 0 & 1 \end{bmatrix}
	\begin{bmatrix} 2 & 4 & -2 \\ 4 & 9 & -3 \\ -2 & -3 & 7 \end{bmatrix}
	=
	\begin{bmatrix} 2 & 4 & -2 \\ 0 & 1 & 1 \\ -2 & -3 & 7 \end{bmatrix} \quad (3)
	\]
	این مرحله سطر ۱ و ۳ ماتریس $A$ را تغییر نمی‌دهد. آن سطرها در $EA$ بدون تغییر باقی می‌مانند - فقط سطر ۲ متفاوت است. دو برابر سطر اول از سطر دوم کم شده است. ضرب ماتریسی با حذف مطابقت دارد - و دستگاه معادلات جدید $EA\mathbf{x} = E\mathbf{b}$ است.
	
	$EA\mathbf{x}$ ساده است اما شامل یک ایده ظریف است. با $A\mathbf{x}=\mathbf{b}$ شروع کنید. ضرب هر دو طرف در $E$ نتیجه می‌دهد $E(A\mathbf{x})=E\mathbf{b}$. با ضرب ماتریسی، این همچنین $(EA)\mathbf{x}=E\mathbf{b}$ است.
	اولی $E$ ضربدر $A\mathbf{x}$ بود، دومی $EA$ ضربدر $\mathbf{x}$. آن‌ها یکسان هستند.
	پرانتز لازم نیست. ما فقط می‌نویسیم $EA\mathbf{x}$.
	
	\textit{(توضیح مترجم: این ویژگی بسیار مهم به عنوان \textbf{قانون شرکت‌پذیری} شناخته می‌شود. این قانون به ما اجازه می‌دهد که ترتیب عملیات را در ضرب‌های متوالی تغییر دهیم.)}
	
	این قانون به ماتریسی مانند $C$ با چندین بردار ستونی نیز تعمیم می‌یابد. هنگام ضرب $EAC$ می‌توانید ابتدا $AC$ را انجام دهید یا ابتدا $EA$ را. این نکته «قانون شرکت‌پذیری» است مانند $3 \times (4 \times 5) = (3 \times 4) \times 5$. ۳ را در ۲۰ ضرب کنید یا ۱۲ را در ۵ ضرب کنید. هر دو پاسخ ۶۰ است. این قانون آنقدر واضح به نظر می‌رسد که تصور نادرست بودن آن دشوار است.
	
	«قانون جابجایی» $3 \times 4 = 4 \times 3$ حتی واضح‌تر به نظر می‌رسد. اما $EA$ معمولاً با $AE$ متفاوت است. وقتی $E$ از سمت راست ضرب می‌شود، بر ستون‌های $A$ اثر می‌کند - نه سطرها. $AE$ در واقع ۲ برابر ستون ۲ را از ستون ۱ کم می‌کند. بنابراین $EA \neq AE$.
	
	\begin{itemize}
		\item \textbf{قانون شرکت‌پذیری برقرار است:} $A(BC) = (AB)C$
		\item \textbf{قانون جابجایی برقرار نیست:} اغلب $AB \neq BA$
	\end{itemize}
	
	یک الزام دیگر در ضرب ماتریسی وجود دارد. فرض کنید $B$ فقط یک ستون دارد (این ستون $\mathbf{b}$ است). قانون ماتریس-ماتریس برای $EB$ باید با قانون ماتریس-بردار برای $E\mathbf{b}$ مطابقت داشته باشد. حتی بیشتر، ما باید بتوانیم ماتریس‌های $EB$ را ستون به ستون ضرب کنیم:
	\begin{quote}
		اگر $B$ چندین ستون $\mathbf{b}_1, \mathbf{b}_2, \mathbf{b}_3$ داشته باشد، ستون‌های $EB$ عبارتند از $E\mathbf{b}_1, E\mathbf{b}_2, E\mathbf{b}_3$.
	\end{quote}
	\textbf{ضرب ماتریسی:} $E[\mathbf{b}_1 \ \mathbf{b}_2 \ \mathbf{b}_3] = [E\mathbf{b}_1 \ E\mathbf{b}_2 \ E\mathbf{b}_3] \quad (4)$
	
	این برای ضرب ماتریسی در معادله (۳) صادق است. اگر ستون ۳ ماتریس $A$ را در $E$ ضرب کنید، به درستی ستون ۳ ماتریس $EA$ را به دست می‌آورید:
	\[
	\begin{bmatrix} 1 & 0 & 0 \\ -2 & 1 & 0 \\ 0 & 0 & 1 \end{bmatrix}
	\begin{bmatrix} -2 \\ -3 \\ 7 \end{bmatrix}
	=
	\begin{bmatrix} -2 \\ 1 \\ 7 \end{bmatrix}
	\]
	$E(\text{ستون } j \text{ از } A) = \text{ستون } j \text{ از } EA$.
	این الزام با ستون‌ها سروکار دارد، در حالی که حذف بر روی سطرها اعمال می‌شود. بخش بعدی هر درایه از هر ضرب $AB$ را توصیف می‌کند. زیبایی ضرب ماتریسی این است که هر سه رویکرد (سطرها، ستون‌ها، ماتریس‌های کامل) نتیجه درستی می‌دهند.
	
	\subsection*{ماتریس $P_{ij}$ برای تعویض سطر}
	برای کم کردن سطر $j$ از سطر $i$ از $E_{ij}$ استفاده می‌کنیم. برای تعویض یا «جایگشت» آن سطرها از ماتریس دیگری به نام $P_{ij}$ (ماتریس جایگشت) استفاده می‌کنیم. تعویض سطر زمانی لازم است که صفر در موقعیت لولا باشد. پایین‌تر، آن ستون لولا ممکن است شامل یک درایه غیرصفر باشد. با تعویض دو سطر، ما یک لولا داریم و حذف به جلو می‌رود.
	
	چه ماتریسی $P_{23}$ سطر ۲ را با سطر ۳ تعویض می‌کند؟ ما می‌توانیم آن را با تعویض سطرهای ماتریس همانی $I$ پیدا کنیم:
	\[
	\text{ماتریس جایگشت } P_{23} = \begin{bmatrix} 1 & 0 & 0 \\ 0 & 0 & 1 \\ 0 & 1 & 0 \end{bmatrix}
	\]
	این یک ماتریس تعویض سطر است. ضرب در $P_{23}$ مؤلفه‌های ۲ و ۳ هر بردار ستونی را تعویض می‌کند. بنابراین، سطرهای ۲ و ۳ هر ماتریسی را نیز تعویض می‌کند:
	\[
	\begin{bmatrix} 1 & 0 & 0 \\ 0 & 0 & 1 \\ 0 & 1 & 0 \end{bmatrix}
	\begin{bmatrix} a \\ b \\ c \end{bmatrix}
	=
	\begin{bmatrix} a \\ c \\ b \end{bmatrix}
	\quad \text{و} \quad
	\begin{bmatrix} 1 & 0 & 0 \\ 0 & 0 & 1 \\ 0 & 1 & 0 \end{bmatrix}
	\begin{bmatrix} 2 & 4 & 1 \\ 0 & 0 & 3 \\ 0 & 6 & 5 \end{bmatrix}
	=
	\begin{bmatrix} 2 & 4 & 1 \\ 0 & 6 & 5 \\ 0 & 0 & 3 \end{bmatrix}
	\]
	در سمت راست، $P_{23}$ کاری را انجام می‌دهد که برای آن ساخته شده است. با وجود صفر در موقعیت لولای دوم و «۶» در زیر آن، این تعویض، ۶ را به جایگاه لولا منتقل می‌کند.
	
	ماتریس‌ها عمل می‌کنند. آن‌ها فقط یک جا نمی‌نشینند. ما به زودی با ماتریس‌های جایگشت دیگری آشنا خواهیم شد که می‌توانند ترتیب چندین سطر را تغییر دهند. سطرهای ۱, ۲, ۳ می‌توانند به ۳, ۱, ۲ منتقل شوند. $P_{23}$ ما یک ماتریس جایگشت خاص است - این ماتریس سطرهای ۲ و ۳ را تعویض می‌کند.
	
	\begin{quote}
		\textbf{ماتریس تعویض سطر} \\
		$P_{ij}$ ماتریس همانی است که سطرهای $i$ و $j$ آن جابجا شده‌اند.
		وقتی این «ماتریس جایگشت» $P_{ij}$ در ماتریسی ضرب می‌شود، سطرهای $i$ و $j$ آن را تعویض می‌کند.
	\end{quote}
	
	معمولاً تعویض سطر لازم نیست. به احتمال زیاد حذف فقط از $E_{ij}$ها استفاده می‌کند. اما $P_{ij}$ها در صورت نیاز آماده هستند تا یک لولا را به قطر اصلی منتقل کنند.
	
	\subsection*{ماتریس الحاقی}
	این کتاب در نهایت بسیار فراتر از حذف می‌رود. ماتریس‌ها انواع کاربردهای عملی دارند که در آن‌ها ضرب می‌شوند. بهترین نقطه شروع ما یک ماتریس مربع $E$ ضربدر یک ماتریس مربع $A$ بود، زیرا این را در حذف دیدیم - و می‌دانیم برای $EA$ چه جوابی را انتظار داشته باشیم.
	
	گام بعدی این است که یک ماتریس مستطیلی را مجاز بدانیم. این ماتریس هنوز از معادلات اصلی ما می‌آید، اما اکنون شامل سمت راست یعنی $\mathbf{b}$ نیز می‌شود.
	
	\textbf{ایده کلیدی:} حذف، عملیات سطری یکسانی را بر روی $A$ و $\mathbf{b}$ انجام می‌دهد. ما می‌توانیم $\mathbf{b}$ را به عنوان یک ستون اضافی در نظر بگیریم و آن را در طول حذف دنبال کنیم. ماتریس $A$ با ستون اضافی $\mathbf{b}$ بزرگ‌تر یا «الحاقی» می‌شود:
	\[
	\text{ماتریس الحاقی } [A \ \mathbf{b}] = \left[ \begin{array}{ccc|c} 2 & 4 & -2 & 2 \\ 4 & 9 & -3 & 8 \\ -2 & -3 & 7 & 10 \end{array} \right]
	\]
	حذف بر روی کل سطرهای این ماتریس عمل می‌کند. سمت چپ و راست هر دو در $E$ ضرب می‌شوند تا ۲ برابر معادله ۱ از معادله ۲ کم شود. با $[A \ \mathbf{b}]$ این مراحل با هم اتفاق می‌افتند:
	\[
	\begin{bmatrix} 1 & 0 & 0 \\ -2 & 1 & 0 \\ 0 & 0 & 1 \end{bmatrix}
	\left[ \begin{array}{ccc|c} 2 & 4 & -2 & 2 \\ 4 & 9 & -3 & 8 \\ -2 & -3 & 7 & 10 \end{array} \right]
	=
	\left[ \begin{array}{ccc|c} 2 & 4 & -2 & 2 \\ 0 & 1 & 1 & 4 \\ -2 & -3 & 7 & 10 \end{array} \right]
	\]
	سطر دوم جدید شامل ۰, ۱, ۱, ۴ است. معادله دوم جدید $x_2 + x_3 = 4$ است. ضرب ماتریسی به صورت سطری و همزمان به صورت ستونی عمل می‌کند:
	
	\begin{description}
		\item[سطرها:] هر سطر از $E$ بر $[A \ \mathbf{b}]$ اثر می‌کند تا یک سطر از $[EA \ E\mathbf{b}]$ را نتیجه دهد.
		\item[ستون‌ها:] $E$ بر هر ستون از $[A \ \mathbf{b}]$ اثر می‌کند تا یک ستون از $[EA \ E\mathbf{b}]$ را نتیجه دهد.
	\end{description}
	
	دوباره به کلمه «اثر می‌کند» توجه کنید. این ضروری است. ماتریس‌ها کاری انجام می‌دهند! ماتریس $A$ بر $\mathbf{x}$ اثر می‌کند تا $\mathbf{b}$ را تولید کند. ماتریس $E$ بر $A$ عمل می‌کند تا $EA$ را به دست آورد. کل فرآیند حذف، دنباله‌ای از عملیات سطری است که همان ضرب‌های ماتریسی هستند. $A$ به $E_{21}A$ می‌رود، که به $E_{31}E_{21}A$ می‌رود. در نهایت $E_{32}E_{31}E_{21}A$ یک ماتریس مثلثی است.
	
	سمت راست در ماتریس الحاقی گنجانده شده است. نتیجه نهایی یک دستگاه معادلات مثلثی است. قبل از نوشتن قوانین برای تمام ضرب‌های ماتریسی (شامل ضرب قطعه‌ای)، برای تمرین‌های مربوط به ضرب در $E$ توقف می‌کنیم.
	
	\subsection*{مروری بر ایده‌های کلیدی}
	\begin{enumerate}
		\item $A\mathbf{x} = x_1(\text{ستون ۱}) + \dots + x_n(\text{ستون } n)$. و $(A\mathbf{x})_i = \sum_{j=1}^{n} a_{ij}x_j$.
		\item ماتریس همانی $= I$, ماتریس حذفی $= E_{ij}$ با استفاده از $l_{ij}$، ماتریس تعویض $= P_{ij}$.
		\item ضرب $A\mathbf{x}=\mathbf{b}$ در $E_{21}$، مضرب $l_{21}$ از معادله ۱ را از معادله ۲ کم می‌کند. عدد $-l_{21}$ درایه (۲,۱) ماتریس حذفی $E_{21}$ است.
		\item برای ماتریس الحاقی $[A \ \mathbf{b}]$، آن مرحله حذف $[E_{21}A \ E_{21}\mathbf{b}]$ را نتیجه می‌دهد.
		\item وقتی $A$ در هر ماتریس $B$ ضرب می‌شود، هر ستون از $B$ را جداگانه ضرب می‌کند.
	\end{enumerate}
	
	\subsection*{مثال‌های حل شده}
	\subsubsection*{مثال ۲.۳ الف}
	چه ماتریس ۳ در ۳ به نام $E_{21}$، چهار برابر سطر ۱ را از سطر ۲ کم می‌کند؟ چه ماتریسی به نام $P_{32}$ سطر ۲ و ۳ را تعویض می‌کند؟ اگر $A$ را از سمت راست به جای چپ ضرب کنید، نتایج $AE_{21}$ و $AP_{32}$ را توصیف کنید.
	
	\textbf{راه حل} \\
	با انجام این عملیات روی ماتریس همانی $I$، به دست می‌آوریم:
	\[
	E_{21} = \begin{bmatrix} 1 & 0 & 0 \\ -4 & 1 & 0 \\ 0 & 0 & 1 \end{bmatrix} \quad \text{و} \quad
	P_{32} = \begin{bmatrix} 1 & 0 & 0 \\ 0 & 0 & 1 \\ 0 & 1 & 0 \end{bmatrix}
	\]
	ضرب در $E_{21}$ از سمت راست، ۴ برابر ستون ۲ را از ستون ۱ کم می‌کند.
	ضرب در $P_{32}$ از سمت راست، ستون‌های ۲ و ۳ را تعویض می‌کند.
	
	\subsubsection*{مثال ۲.۳ ب}
	ماتریس الحاقی $[A \ \mathbf{b}]$ را با یک ستون اضافی بنویسید:
	\begin{align*}
		x + 2y + 2z &= 1 \\
		4x + 8y + 9z &= 3 \\
		3y + 2z &= 1
	\end{align*}
	ابتدا $E_{21}$ و سپس $P_{32}$ را اعمال کنید تا به یک دستگاه مثلثی برسید. با جایگذاری پس‌رو حل کنید. چه ماتریس ترکیبی $P_{32}E_{21}$ هر دو مرحله را یکجا انجام می‌دهد؟
	
	\textbf{راه حل} \\
	$E_{21}$ عدد ۴ را در ستون اول حذف می‌کند. اما صفر در ستون دوم نیز ظاهر می‌شود:
	\[
	[A \ \mathbf{b}] = \left[ \begin{array}{ccc|c} 1 & 2 & 2 & 1 \\ 4 & 8 & 9 & 3 \\ 0 & 3 & 2 & 1 \end{array} \right]
	\quad \xrightarrow{E_{21}} \quad
	E_{21}[A \ \mathbf{b}] = \left[ \begin{array}{ccc|c} 1 & 2 & 2 & 1 \\ 0 & 0 & 1 & -1 \\ 0 & 3 & 2 & 1 \end{array} \right]
	\]
	اکنون $P_{32}$ سطرهای ۲ و ۳ را تعویض می‌کند. جایگذاری پس‌رو ابتدا $z$، سپس $y$ و $x$ را به دست می‌دهد.
	\[
	P_{32}E_{21}[A \ \mathbf{b}] = \left[ \begin{array}{ccc|c} 1 & 2 & 2 & 1 \\ 0 & 3 & 2 & 1 \\ 0 & 0 & 1 & -1 \end{array} \right]
	\quad \rightarrow \quad
	\begin{cases}
		z = -1 \\
		3y + 2(-1) = 1 \Rightarrow 3y = 3 \Rightarrow y=1 \\
		x + 2(1) + 2(-1) = 1 \Rightarrow x=1
	\end{cases}
	\]
	جواب $\mathbf{x} = (1, 1, -1)$ است.
	برای ماتریس $P_{32}E_{21}$ که هر دو مرحله را یکجا انجام می‌دهد، $P_{32}$ را بر $E_{21}$ اعمال کنید.
	\[
	\text{یک ماتریس برای هر دو گام} \quad
	P_{32}E_{21} = P_{32} \begin{bmatrix} 1 & 0 & 0 \\ -4 & 1 & 0 \\ 0 & 0 & 1 \end{bmatrix} = \begin{bmatrix} 1 & 0 & 0 \\ 0 & 0 & 1 \\ -4 & 1 & 0 \end{bmatrix}
	\]
	
	\subsubsection*{مثال ۲.۳ ج}
	این ماتریس‌ها را به دو روش ضرب کنید. اول، سطرهای $A$ در ستون‌های $B$. دوم، ستون‌های $A$ در سطرهای $B$. این روش غیرمعمول دو ماتریس تولید می‌کند که جمع آن‌ها $AB$ می‌شود. چند ضرب معمولی مجزا لازم است؟
	\[
	AB = \begin{bmatrix} 3 & 4 \\ 1 & 5 \end{bmatrix}
	\begin{bmatrix} 2 & 0 \\ 1 & 2 \end{bmatrix}
	=
	\begin{bmatrix} 10 & 8 \\ 7 & 10 \end{bmatrix}
	\]
	\textbf{راه حل} \\
	\textbf{روش اول (سطر در ستون):} این روش استاندارد ضرب داخلی است.
	\[ (\text{سطر ۱}) \cdot (\text{ستون ۱}) = \begin{bmatrix} 3 & 4 \end{bmatrix} \begin{bmatrix} 2 \\ 1 \end{bmatrix} = 6+4=10 \]
	\[ (\text{سطر ۲}) \cdot (\text{ستون ۱}) = \begin{bmatrix} 1 & 5 \end{bmatrix} \begin{bmatrix} 2 \\ 1 \end{bmatrix} = 2+5=7 \]
	در کل ۴ ضرب داخلی نیاز داریم که هر کدام شامل ۲ ضرب معمولی است، پس در مجموع $2 \times 2 \times 2 = 8$ ضرب معمولی لازم است.
	
	\textbf{روش دوم (ستون در سطر):} همان $AB$ از جمع حاصلضرب ستون‌های $A$ در سطرهای $B$ به دست می‌آید. یک ستون ضربدر یک سطر، یک ماتریس (با رتبه ۱) تولید می‌کند.
	\[
	AB = \begin{bmatrix} 3 \\ 1 \end{bmatrix} \begin{bmatrix} 2 & 0 \end{bmatrix} + \begin{bmatrix} 4 \\ 5 \end{bmatrix} \begin{bmatrix} 1 & 2 \end{bmatrix}
	= \begin{bmatrix} 6 & 0 \\ 2 & 0 \end{bmatrix} + \begin{bmatrix} 4 & 8 \\ 5 & 10 \end{bmatrix}
	= \begin{bmatrix} 10 & 8 \\ 7 & 10 \end{bmatrix}
	\]
	
	\newpage
	\section*{مجموعه مسائل ۲.۳}
	
	\subsection*{مسائل ۱-۱۵ در مورد ماتریس‌های حذفی هستند.}
	\begin{enumerate}
		\item ماتریس‌های ۳ در ۳ را بنویسید که این مراحل حذف را تولید می‌کنند:
		\begin{itemize}
			\item[(الف)] $E_{21}$ پنج برابر سطر ۱ را از سطر ۲ کم می‌کند.
			\item[(ب)] $E_{32}$ منفی هفت برابر سطر ۲ را از سطر ۳ کم می‌کند.
			\item[(ج)] $P$ سطرهای ۱ و ۲ را تعویض می‌کند، سپس سطرهای ۲ و ۳ را.
		\end{itemize}
		\item در مسئله ۱، اعمال $E_{21}$ و سپس $E_{32}$ بر روی $\mathbf{b}=(1,0,0)$ نتیجه می‌دهد $E_{32}E_{21}\mathbf{b} = \_\_\_$. اعمال $E_{32}$ قبل از $E_{21}$ نتیجه می‌دهد $E_{21}E_{32}\mathbf{b} = \_\_\_$. وقتی $E_{32}$ اول می‌آید، سطر \textbf{اول} هیچ تأثیری از سطر \textbf{دوم} نمی‌پذیرد.
		\item کدام سه ماتریس $E_{21}, E_{31}, E_{32}$ ماتریس $A$ را به شکل مثلثی $U$ در می‌آورند؟
		\[ A = \begin{bmatrix} 1 & 1 & 0 \\ 4 & 6 & 1 \\ -2 & 2 & 0 \end{bmatrix} \]
		این ماتریس‌های $E$ را ضرب کنید تا یک ماتریس $M$ به دست آید که حذف را انجام می‌دهد: $MA=U$.
		\item بردار $\mathbf{b}=(1,0,0)$ را به عنوان ستون چهارم در مسئله ۳ اضافه کنید تا $[A \ \mathbf{b}]$ تولید شود. مراحل حذف را روی این ماتریس الحاقی انجام دهید تا $A\mathbf{x}=\mathbf{b}$ حل شود.
		\item فرض کنید $a_{33}=7$ و لولای سوم ۵ است. اگر $a_{33}$ را به ۱۱ تغییر دهید، لولای سوم \textbf{۹} است. اگر $a_{33}$ را به \textbf{۲} تغییر دهید، لولای سوم وجود ندارد.
		\item اگر هر ستون از $A$ مضربی از $(1,1,1)$ باشد، آنگاه $A\mathbf{x}$ همیشه مضربی از $(1,1,1)$ است. یک مثال ۳ در ۳ بزنید. حذف چند لولا تولید می‌کند؟
		\item فرض کنید $E$ هفت برابر سطر ۱ را از سطر ۳ کم می‌کند.
		\begin{itemize}
			\item[(الف)] برای معکوس کردن این مرحله، باید \textbf{۷} برابر سطر \textbf{۱} را به سطر \textbf{۳} \textbf{اضافه} کنید.
			\item[(ب)] چه «ماتریس معکوسی» $E^{-1}$ این مرحله معکوس را انجام می‌دهد (به طوری که $E^{-1}E=I$ باشد)؟
			\item[(ج)] اگر مرحله معکوس اول اعمال شود (و سپس $E$) نشان دهید که $EE^{-1}=I$.
		\end{itemize}
		\item دترمینان ماتریس $M = \begin{bmatrix} a & b \\ c & d \end{bmatrix}$ برابر با $\det M = ad-bc$ است. $C$ برابر سطر ۱ را از سطر ۲ کم کنید تا یک ماتریس جدید $M^*$ تولید شود. نشان دهید که $\det M^* = \det M$ برای هر $C$. وقتی $C=c/a$ باشد، حاصلضرب لولاها برابر با دترمینان است: $(a)(d-Cb)$ برابر با $ad-bc$ است.
		\item (الف) $E_{21}$ سطر ۱ را از سطر ۲ کم می‌کند و سپس $P_{23}$ سطرهای ۲ و ۳ را تعویض می‌کند. چه ماتریسی $M=P_{23}E_{21}$ هر دو مرحله را یکجا انجام می‌دهد؟
		(ب) $P_{23}$ سطرهای ۲ و ۳ را تعویض می‌کند و سپس $E_{31}$ سطر ۱ را از سطر ۳ کم می‌کند. چه ماتریسی $M=E_{31}P_{23}$ هر دو مرحله را یکجا انجام می‌دهد؟ توضیح دهید چرا ماتریس‌های $M$ یکسان هستند اما ماتریس‌های $E$ متفاوتند.
		\item (الف) چه ماتریس ۳ در ۳ به نام $E_{13}$ سطر ۳ را به سطر ۱ اضافه می‌کند؟
		(ب) چه ماتریسی سطر ۱ را به سطر ۳ و همزمان سطر ۳ را به سطر ۱ اضافه می‌کند؟
		(ج) چه ماتریسی سطر ۱ را به سطر ۳ اضافه می‌کند و سپس سطر ۳ را به سطر ۱ اضافه می‌کند؟
		\item ماتریسی بسازید که $a_{11}=a_{22}=a_{33}=1$ باشد اما حذف بدون تعویض سطر دو لولای منفی تولید کند. (لولای اول ۱ است.)
		\item این ماتریس‌ها را ضرب کنید:
		\[ \begin{bmatrix} 1 & 0 & 0 \\ a & 1 & 0 \\ b & c & 1 \end{bmatrix} \begin{bmatrix} 1 & 2 & 3 \\ 0 & 1 & 4 \\ 0 & 0 & 1 \end{bmatrix} \]
		\item این حقایق را توضیح دهید. اگر ستون سوم $B$ همگی صفر باشد، ستون سوم $EB$ نیز همگی صفر است (برای هر $E$). اگر سطر سوم $B$ همگی صفر باشد، سطر سوم $EB$ ممکن است صفر نباشد.
		\item این ماتریس ۴ در ۴ به ماتریس‌های حذفی $E_{21}$ و $E_{32}$ و $E_{43}$ نیاز خواهد داشت. آن ماتریس‌ها کدامند؟
		\[ A = \begin{bmatrix} 2 & -1 & 0 & 0 \\ -1 & 2 & -1 & 0 \\ 0 & -1 & 2 & -1 \\ 0 & 0 & -1 & 2 \end{bmatrix} \]
		\item ماتریس ۳ در ۳ را بنویسید که در آن $a_{ij}=2i-3j$. این ماتریس $a_{32}=0$ دارد، اما حذف هنوز به $E_{32}$ برای تولید صفر در موقعیت ۳,۲ نیاز دارد. کدام مرحله قبلی صفر اصلی را از بین می‌برد و $E_{32}$ چیست؟
	\end{enumerate}
	
	\subsection*{مسائل ۱۶-۲۳ در مورد ساخت و ضرب ماتریس‌ها هستند.}
	\begin{enumerate}
		\setcounter{enumi}{15}
		\item این مسائل قدیمی را به شکل ماتریسی ۲ در ۲ یعنی $A\mathbf{x}=\mathbf{b}$ بنویسید و حل کنید:
		\begin{itemize}
			\item[(الف)] سن X دو برابر سن Y است و مجموع سن آن‌ها ۳۳ است.
			\item[(ب)] نقاط $(x,y)=(2,5)$ و $(3,7)$ روی خط $y=mx+c$ قرار دارند. $m$ و $c$ را بیابید.
		\end{itemize}
		\item سهمی $y=a+bx+cx^2$ از نقاط $(x,y)=(1,4)$، $(2,8)$ و $(3,14)$ می‌گذرد. یک معادله ماتریسی برای مجهولات $(a,b,c)$ بیابید و آن را حل کنید.
		\item این ماتریس‌ها را در ترتیب‌های $EF$ و $FE$ ضرب کنید:
		\[ E = \begin{bmatrix} 1 & 0 \\ 1 & 1 \end{bmatrix}, \quad F = \begin{bmatrix} 1 & 1 \\ 0 & 1 \end{bmatrix} \]
		همچنین $E^2=EE$ و $F^3=FFF$ را محاسبه کنید. می‌توانید $F^{100}$ را حدس بزنید.
		\item این ماتریس‌های تعویض سطر را در ترتیب‌های $PQ$ و $QP$ و $P^2$ ضرب کنید:
		\[ P = \begin{bmatrix} 0 & 1 \\ 1 & 0 \end{bmatrix}, \quad Q = \begin{bmatrix} 0 & 1 \\ 1 & 0 \end{bmatrix} \]
		یک ماتریس غیرقطری دیگر بیابید که مربع آن $M^2=I$ باشد.
		\item (الف) فرض کنید تمام ستون‌های $B$ یکسان هستند. آنگاه تمام ستون‌های $EB$ نیز یکسان هستند، زیرا هر کدام برابر است با $E$ ضربدر \_\_\_.
		(ب) فرض کنید تمام سطرهای $B$ برابر با $[1 \ 2 \ 4]$ هستند. با یک مثال نشان دهید که تمام سطرهای $EB$ برابر با $[1 \ 2 \ 4]$ نیستند. درست است که آن سطرها \_\_\_ هستند.
		\item اگر $E$ سطر ۱ را به سطر ۲ اضافه کند و $F$ سطر ۲ را به سطر ۱ اضافه کند، آیا $EF$ با $FE$ برابر است؟
		\item درایه‌های $A$ و $\mathbf{x}$ به ترتیب $a_{ij}$ و $x_j$ هستند. بنابراین مؤلفه اول $A\mathbf{x}$ برابر است با $\sum a_{1j}x_j = a_{11}x_1 + \dots + a_{1n}x_n$. اگر $E_{21}$ سطر ۱ را از سطر ۲ کم کند، فرمولی برای موارد زیر بنویسید:
		\begin{itemize}
			\item[(الف)] مؤلفه سوم $A\mathbf{x}$.
			\item[(ب)] درایه (۲,۱) از $E_{21}A$.
			\item[(ج)] درایه (۲,۱) از $E_{21}(E_{21}A)$.
			\item[(د)] مؤلفه اول $E_{21}A\mathbf{x}$.
		\end{itemize}
		\item ماتریس حذفی $E = \begin{bmatrix} 1 & 0 \\ -2 & 1 \end{bmatrix}$ دو برابر سطر ۱ ماتریس $A$ را از سطر ۲ آن کم می‌کند. نتیجه $EA$ است. تأثیر $E(EA)$ چیست؟ در ترتیب مخالف $AE$، ما دو برابر \_\_\_ از $A$ را از \_\_\_ کم می‌کنیم. (مثال بزنید.)
	\end{enumerate}
	
	\subsection*{مسائل ۲۴-۲۷ شامل ستون b در ماتریس الحاقی [A b] هستند.}
	\begin{enumerate}
		\setcounter{enumi}{23}
		\item حذف را بر روی ماتریس الحاقی ۲ در ۳ یعنی $[A \ \mathbf{b}]$ اعمال کنید. دستگاه مثلثی $U\mathbf{x}=\mathbf{c}$ چیست؟ جواب $\mathbf{x}$ چیست؟
		\[ A = \begin{bmatrix} 2 & 3 \\ 4 & 5 \end{bmatrix}, \quad \mathbf{b} = \begin{bmatrix} 4 \\ 6 \end{bmatrix} \]
		\item حذف را بر روی ماتریس الحاقی ۳ در ۴ یعنی $[A \ \mathbf{b}]$ اعمال کنید. از کجا می‌دانید این دستگاه جوابی ندارد؟ عدد آخر یعنی ۶ را طوری تغییر دهید که جواب وجود داشته باشد.
		\[ A = \begin{bmatrix} 1 & 2 & 2 \\ 2 & 4 & 5 \\ 0 & 1 & 1 \end{bmatrix}, \quad \mathbf{b} = \begin{bmatrix} 1 \\ 4 \\ 6 \end{bmatrix} \]
		\item معادلات $A\mathbf{x}=\mathbf{b}$ و $A\mathbf{x}^*=\mathbf{b}^*$ ماتریس $A$ یکسانی دارند. برای حل همزمان هر دو معادله، از چه ماتریس الحاقی دوگانه‌ای باید در حذف استفاده کنید؟
		هر دو معادله زیر را با کار بر روی یک ماتریس ۲ در ۴ حل کنید:
		\[ \begin{bmatrix} 1 & 4 \\ 2 & 7 \end{bmatrix} \begin{bmatrix} x \\ y \end{bmatrix} = \begin{bmatrix} 1 \\ 0 \end{bmatrix} \quad \text{و} \quad \begin{bmatrix} 1 & 4 \\ 2 & 7 \end{bmatrix} \begin{bmatrix} x^* \\ y^* \end{bmatrix} = \begin{bmatrix} 0 \\ 1 \end{bmatrix} \]
		\item اعداد $a, b, c, d$ را در این ماتریس الحاقی طوری انتخاب کنید که (الف) جوابی وجود نداشته باشد (ب) بی‌نهایت جواب وجود داشته باشد.
		\[ [A \ \mathbf{b}] = \left[ \begin{array}{ccc|c} 1 & 2 & 3 & a \\ 0 & 4 & 5 & b \\ 0 & 0 & d & c \end{array} \right] \]
		کدام یک از اعداد $a, b, c$ یا $d$ تأثیری بر قابل حل بودن دستگاه ندارند؟
	\end{enumerate}
	
	\subsection*{مسائل چالشی}
	\begin{enumerate}
		\setcounter{enumi}{27}
		\item اگر $AB=I$ و $BC=I$ باشد، از قانون شرکت‌پذیری استفاده کنید تا ثابت کنید $A=C$.
		\item ماتریس مثلثی $E$ را بیابید که «ماتریس پاسکال» را به یک ماتریس پاسکال کوچک‌تر کاهش دهد:
		\[ E \begin{bmatrix} 1 & 1 & 1 \\ 1 & 2 & 3 \\ 1 & 3 & 6 \end{bmatrix} = \begin{bmatrix} 1 & 1 & 1 \\ 0 & 1 & 2 \\ 0 & 2 & 5 \end{bmatrix} \]
		کدام ماتریس $M$ (که حاصلضرب چند ماتریس $E$ است) ماتریس پاسکال را کاملاً به $I$ کاهش می‌دهد؟
		\item $M = \begin{bmatrix} a & b \\ c & d \end{bmatrix}$ را به صورت حاصلضرب عوامل زیادی از نوع $A = \begin{bmatrix} 1 & 1 \\ 0 & 1 \end{bmatrix}$ و $B = \begin{bmatrix} 1 & 0 \\ 1 & 1 \end{bmatrix}$ بنویسید.
		(الف) چه ماتریسی $E$ سطر ۱ را از سطر ۲ کم می‌کند تا سطر ۲ ماتریس $EM$ کوچک‌تر شود؟
		(ب) چه ماتریسی $F$ سطر ۲ از $EM$ را از سطر ۱ کم می‌کند تا سطر ۱ از $FEM$ کاهش یابد؟
		(ج) این فرآیند را با $E$ها و $F$ها ادامه دهید تا حاصلضرب (تعداد زیادی $E$ و $F$) در ($M$) برابر با ($A$ یا $B$) شود.
		(د) $E$ و $F$ معکوس‌های $A$ و $B$ هستند! انتقال همه $E$ها و $F$ها به سمت راست، نتیجه مطلوب $M = $ حاصلضرب $A$ها و $B$ها را به شما می‌دهد. این برای ماتریس‌های صحیح $M$ با درایه‌های مثبت که $ad-bc=1$ دارند، امکان‌پذیر است.
		\item ماتریس‌های حذفی $E_{21}$ سپس $E_{32}$ و سپس $E_{43}$ را برای تبدیل $K$ به $U$ بیابید:
		\[ K = \begin{bmatrix} 1 & -a & 0 & 0 \\ 0 & 1 & -b & 0 \\ 0 & 0 & 1 & -c \\ 0 & 0 & 0 & 1 \end{bmatrix} \]
		این سه مرحله را روی ماتریس همانی $I$ اعمال کنید تا حاصلضرب $E_{43}E_{32}E_{21}$ را به دست آورید.
	\end{enumerate}
	
\end{document}