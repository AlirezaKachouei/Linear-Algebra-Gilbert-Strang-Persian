\documentclass[12pt, a4paper]{book}

% فراخوانی بسته‌های لازم
\usepackage{amsmath}         % برای فرمول‌های پیشرفته ریاضی
\usepackage{amsfonts}        % بسته برای فونت‌های ریاضی مانند \mathbb
\usepackage{amssymb}         % برای نمادهای بیشتر ریاضی
\usepackage{graphicx}        % برای افزودن تصاویر
\usepackage{xepersian}       % بسته اصلی برای پارسی‌نویسی
\usepackage{geometry}        % برای تنظیم حاشیه‌ها
\usepackage{setspace}        % برای تنظیم فاصله خطوط
\usepackage{amsthm}          % برای محیط‌های اثبات و تعریف

% تنظیم حاشیه‌های صفحه
\geometry{
	a4paper,
	total={170mm,257mm},
	left=20mm,
	top=20mm,
}

% تنظیم فونت‌های نوشتاری و ریاضی
% توجه: این فونت‌ها باید روی سیستم شما نصب باشند
\settextfont{XB Niloofar}
\setdigitfont{XB Niloofar}
\setmathdigitfont{XB Niloofar}

% تعریف محیط توضیح مترجم
\newtheoremstyle{translator}
{10pt} % Space above
{10pt} % Space below
{\itshape} % Body font
{} % Indent amount
{\bfseries} % Theorem head font
{:} % Punctuation after theorem head
{.5em} % Space after theorem head
{} % Theorem head spec (can be left empty, meaning `normal')
\theoremstyle{translator}
\newtheorem*{translator}{توضیح مترجم}

\begin{document}
	
	% اعمال فاصله 1.5 بین خطوط برای خوانایی بهتر
	\onehalfspacing
	
	\chapter{حل معادلات خطی}
	
	\section{ایده‌ی حذف}
	
	\begin{enumerate}
		\item برای حالت $m = n = 3$، سه معادله $A\mathbf{x}=\mathbf{b}$ و سه مجهول $x_1, x_2, x_3$ وجود دارد.
		\item دو معادله اول عبارتند از $a_{11}x_1 + \dots = b_1$ و $a_{21}x_1 + \dots = b_2$.
		\item معادله اول را در $a_{21}/a_{11}$ ضرب کرده و از معادله دوم کم می‌کنیم: در این صورت $x_1$ حذف می‌شود.
		\item درایه گوشه‌ای $a_{11}$ اولین «\textbf{پاشنه}» (pivot) و نسبت $a_{21}/a_{11}$ اولین «\textbf{ضریب}» (multiplier) است.
		\item با کم کردن حاصلضرب ضریب $a_{i1}/a_{11}$ در معادله اول از هر معادله باقیمانده $i$، متغیر $x_1$ را حذف می‌کنیم.
		\item اکنون $n-1$ معادله آخر، شامل $n-1$ مجهول $x_2, \dots, x_n$ هستند. این فرآیند را برای حذف $x_2$ تکرار می‌کنیم.
		\item اگر در مکان پاشنه، عدد صفر ظاهر شود، فرآیند حذف با شکست مواجه می‌شود. جابجا کردن دو معادله ممکن است این مشکل را برطرف کند.
	\end{enumerate}
	
	این فصل یک روش نظام‌مند برای حل معادلات خطی را توضیح می‌دهد. این روش «\textbf{حذف}» (elimination) نام دارد و شما می‌توانید آن را فوراً در مثال ۲ در ۲ ما مشاهده کنید. قبل از حذف، $x$ و $y$ در هر دو معادله ظاهر می‌شوند. پس از حذف، مجهول اول یعنی $x$ از معادله دوم $8y=8$ ناپدید شده است:
	
	\vspace{5mm}
	\textbf{قبل از حذف} \hspace{2cm} \textbf{بعد از حذف}
	\[
	\begin{cases}
		x - 2y = 1 \\
		3x + 2y = 11
	\end{cases}
	\quad
	\begin{array}{l}
		\xrightarrow{\text{(معادله ۱ را در ۳ ضرب کن)}} \\
		\xrightarrow{\text{(برای حذف 3x تفریق کن)}}
	\end{array}
	\quad
	\begin{cases}
		x - 2y = 1 \\
		8y = 8
	\end{cases}
	\]
	
	معادله جدید $8y=8$ فوراً نتیجه می‌دهد $y=1$. با جایگذاری $y=1$ در معادله اول، به $x - 2 = 1$ می‌رسیم. بنابراین $x=3$ و جواب $(x, y) = (3, 1)$ کامل می‌شود.
	
	حذف یک \textbf{دستگاه بالا-مثلثی} (upper triangular system) تولید می‌کند—این هدف ماست. ضرایب غیرصفر ۱، ۲- و ۸ یک مثلث را تشکیل می‌دهند. این دستگاه از پایین به بالا حل می‌شود؛ ابتدا $y=1$ و سپس $x=3$. این فرآیند سریع «\textbf{جایگذاری پَس‌رو}» (back substitution) نامیده می‌شود. این روش برای دستگاه‌های بالا-مثلثی با هر اندازه‌ای، پس از آنکه فرآیند حذف یک مثلث ایجاد کرد، به کار می‌رود.
	
	\begin{translator}
		هدف نهایی حذف، تبدیل یک دستگاه معادلات مربعی به یک دستگاه بالا-مثلثی است. در این نوع دستگاه، اولین معادله می‌تواند تمام متغیرها را داشته باشد، اما هر معادله بعدی یک متغیر کمتر دارد تا جایی که آخرین معادله فقط یک متغیر دارد. این ساختار مثلثی باعث می‌شود حل دستگاه بسیار ساده شود.
	\end{translator}
	
	\textbf{نکته مهم:} معادلات اصلی همان جواب $x=3$ و $y=1$ را دارند. شکل ۲.۵ هر دستگاه را به صورت یک جفت خط نشان می‌دهد که در نقطه جواب $(3,1)$ یکدیگر را قطع می‌کنند. پس از حذف، خطوط همچنان در همان نقطه یکدیگر را ملاقات می‌کنند. هر مرحله با معادلات صحیح کار کرده است.
	
	چگونه از اولین جفت خطوط به دومین جفت رسیدیم؟ ما ۳ برابر معادله اول را از معادله دوم کم کردیم. مرحله‌ای که $x$ را از معادله ۲ حذف می‌کند، عملیات بنیادین در این فصل است. ما آنقدر از آن استفاده می‌کنیم که با دقت بیشتری به آن نگاه می‌کنیم:
	
	\begin{quote}
		\textbf{برای حذف x:} مضربی از معادله ۱ را از معادله ۲ کم کنید.
	\end{quote}
	
	سه برابرِ $x - 2y = 1$ برابر با $3x - 6y = 3$ است. وقتی این از $3x + 2y = 11$ کم می‌شود، سمت راست برابر با ۸ می‌شود. نکته اصلی این است که $3x$ با $3x$ خنثی می‌شود. آنچه در سمت چپ باقی می‌ماند $2y - (-6y)$ یا $8y$ است، و $x$ حذف می‌شود. دستگاه مثلثی شد.
	
	از خود بپرسید که آن ضریب $\ell_{21} = 3$ چگونه پیدا شد. معادله اول شامل $1x$ است. بنابراین اولین \textbf{پاشنه} ۱ بود (ضریب $x$). معادله دوم شامل $3x$ است، بنابراین ضریب ۳ بود. سپس تفریق $3x-3x$ صفر و مثلث را تولید کرد.
	
	شما قاعده ضریب را خواهید دید اگر من معادله اول را به $4x - 8y = 4$ تغییر دهم. (همان خط مستقیم اما پاشنه اول ۴ می‌شود.) ضریب صحیح اکنون $\ell_{21} = 3/4$ است. برای یافتن ضریب، ضریب «۳» که باید حذف شود را بر پاشنه «۴» تقسیم کنید:
	\[
	\begin{cases}
		4x - 8y = 4 \\
		3x + 2y = 11
	\end{cases}
	\quad
	\begin{array}{l}
		\text{معادله ۱ را در } 3/4 \text{ ضرب کن} \rightarrow 3x - 6y = 3 \\
		\text{از معادله ۲ کم کن} \rightarrow 8y = 8.
	\end{array}
	\]
	
	دستگاه نهایی مثلثی است و معادله آخر هنوز $y=1$ را می‌دهد. جایگذاری پس‌رو $4x - 8 = 4$ و $4x = 12$ و $x=3$ را تولید می‌کند. ما اعداد را تغییر دادیم اما خطوط یا جواب را تغییر ندادیم. برای پیدا کردن آن ضریب $\ell_{ij}$ بر پاشنه تقسیم کنید:
	\begin{description}
		\item[پاشنه] اولین درایه غیرصفر در سطری است که عمل حذف را انجام می‌دهد.
		\item[ضریب] $\ell_{ij} = \frac{\text{(درایه‌ای که باید حذف شود)}}{\text{(پاشنه)}}$. برای مثال ما $\ell_{21} = \frac{3}{4}$.
	\end{description}
	معادله دوم جدید با پاشنه دوم، که ۸ است، شروع می‌شود. ما از آن برای حذف $y$ از معادله سوم، اگر وجود داشت، استفاده می‌کردیم. برای حل $n$ معادله ما به $n$ پاشنه نیاز داریم. پاشنه‌ها پس از حذف روی قطر مثلث قرار دارند.
	
	شما می‌توانستید آن معادلات را برای $x$ و $y$ بدون خواندن این کتاب حل کنید. این یک مسئله فوق‌العاده ساده است، اما ما کمی بیشتر با آن می‌مانیم. حتی برای یک دستگاه ۲ در ۲، حذف ممکن است با شکست مواجه شود. با درک شکست احتمالی (زمانی که نمی‌توانیم مجموعه کاملی از پاشنه‌ها را پیدا کنیم)، شما کل فرآیند حذف را درک خواهید کرد.
	
	\begin{figure}[h!]
		\centering
		\fbox{تصویری از دو خط متقاطع، و سپس یکی از خط‌ها افقی می‌شود}
		\caption{شکل ۲.۵: حذف $x$ خط دوم را افقی می‌کند. سپس $8y = 8$ به سادگی $y=1$ را نتیجه می‌دهد.}
	\end{figure}
	
	\subsection*{شکست فرآیند حذف}
	به طور معمول، فرآیند حذف پاشنه‌هایی را تولید می‌کند که ما را به جواب می‌رسانند. اما شکست ممکن است. در یک مرحله، روش ممکن است از ما بخواهد که بر صفر تقسیم کنیم. ما نمی‌توانیم این کار را انجام دهیم. فرآیند باید متوقف شود. ممکن است راهی برای تنظیم و ادامه وجود داشته باشد—یا ممکن است شکست اجتناب‌ناپذیر باشد. مثال ۱ با نداشتن جواب برای $0y=8$ با شکست مواجه می‌شود. مثال ۲ با داشتن جواب‌های بسیار زیاد برای $0y=0$ با شکست مواجه می‌شود. مثال ۳ با جابجا کردن معادلات موفق می‌شود.
	
	\begin{translator}
		شکست در فرآیند حذف به دو دسته اصلی تقسیم می‌شود:
		\begin{enumerate}
			\item \textbf{شکست دائمی (دستگاه منفرد):} زمانی رخ می‌دهد که یک معادله به صورت $0=k$ (که $k \neq 0$) درآید. این یعنی دستگاه هیچ جوابی ندارد. یا زمانی که یک معادله به صورت $0=0$ درآید، که به معنای وجود بی‌نهایت جواب است. در هر دو حالت، ما تعداد کافی پاشنه برای یافتن یک جواب یکتا نداریم.
			\item \textbf{شکست موقت:} زمانی رخ می‌دهد که یک صفر در جایگاه پاشنه قرار دارد، اما یک درایه غیرصفر در زیر آن در همان ستون وجود دارد. با جابجا کردن سطر فعلی با آن سطر پایینی، می‌توانیم یک پاشنه غیرصفر به دست آورده و فرآیند را ادامه دهیم.
		\end{enumerate}
	\end{translator}
	
	\subsubsection*{مثال ۱: شکست دائمی بدون جواب}
	حذف این موضوع را روشن می‌کند:
	\[
	\begin{cases}
		x - 2y = 1 \\
		3x - 6y = 11
	\end{cases}
	\xrightarrow{\text{۳ برابر معادله ۱ را از معادله ۲ کم کن}}
	\begin{cases}
		x - 2y = 1 \\
		0y = 8
	\end{cases}
	\]
	هیچ جوابی برای $0y = 8$ وجود ندارد. به طور معمول ما سمت راست یعنی ۸ را بر پاشنه دوم تقسیم می‌کنیم، اما این دستگاه پاشنه دوم ندارد. (صفر هرگز به عنوان پاشنه مجاز نیست!) تصاویر سطری و ستونی در شکل ۲.۶ نشان می‌دهند که چرا شکست اجتناب‌ناپذیر بود. اگر جوابی وجود نداشته باشد، حذف با رسیدن به معادله‌ای مانند $0y=8$ آن واقعیت را کشف خواهد کرد.
	
	\textbf{تصویر سطری} از شکست، خطوط موازی را نشان می‌دهد—که هرگز یکدیگر را قطع نمی‌کنند. یک جواب باید روی هر دو خط قرار داشته باشد. بدون نقطه تقاطع، معادلات جوابی ندارند.
	
	\textbf{تصویر ستونی} دو ستون $\begin{bmatrix} 1 \\ 3 \end{bmatrix}$ و $\begin{bmatrix} -2 \\ -6 \end{bmatrix}$ را در یک جهت نشان می‌دهد. تمام ترکیب‌های این ستون‌ها در امتداد یک خط قرار می‌گیرند. اما بردار سمت راست $\mathbf{b} = \begin{bmatrix} 1 \\ 11 \end{bmatrix}$ در جهت دیگری است. هیچ ترکیبی از ستون‌ها نمی‌تواند این سمت راست را تولید کند؛ بنابراین، هیچ جوابی وجود ندارد.
	
	\begin{figure}[h!]
		\centering
		\fbox{تصویر سطری: دو خط موازی. تصویر ستونی: دو بردار هم‌راستا و بردار b خارج از آن راستا}
		\caption{شکل ۲.۶: تصویر سطری و ستونی برای مثال ۱: هیچ جوابی وجود ندارد.}
	\end{figure}
	
	\subsubsection*{مثال ۲: شکست با بی‌نهایت جواب}
	بردار سمت راست $\mathbf{b} = (1, 11)$ را به $(1, 3)$ تغییر دهید.
	\[
	\begin{cases}
		x - 2y = 1 \\
		3x - 6y = 3
	\end{cases}
	\xrightarrow{\text{۳ برابر معادله ۱ را از معادله ۲ کم کن}}
	\begin{cases}
		x - 2y = 1 \\
		0y = 0
	\end{cases}
	\quad \text{(هنوز فقط یک پاشنه)}
	\]
	هر مقدار $y$ در معادله $0y = 0$ صدق می‌کند. در واقع فقط یک معادله $x - 2y = 1$ وجود دارد. مجهول $y$ «\textbf{آزاد}» است. پس از اینکه $y$ به طور آزاد انتخاب شد، $x$ به صورت $x = 1+2y$ تعیین می‌شود.
	
	در \textbf{تصویر سطری}، خطوط موازی به یک خط واحد تبدیل شده‌اند. هر نقطه روی آن خط در هر دو معادله صدق می‌کند. ما یک خط کامل از جواب‌ها را در شکل ۲.۷ داریم.
	
	در \textbf{تصویر ستونی}، $\mathbf{b} = \begin{bmatrix} 1 \\ 3 \end{bmatrix}$ اکنون با ستون ۱ یکسان است. بنابراین می‌توانیم $x=1$ و $y=0$ را انتخاب کنیم. ما همچنین می‌توانیم $x=0$ و $y=-1/2$ را انتخاب کنیم؛ ستون ۲ ضربدر $-1/2$ برابر با $\mathbf{b}$ است. هر $(x,y)$ که مسئله سطری را حل کند، مسئله ستونی را نیز حل می‌کند.
	
	\begin{figure}[h!]
		\centering
		\fbox{تصویر سطری: دو خط منطبق بر هم. تصویر ستونی: دو بردار هم‌راستا و بردار b در همان راستا}
		\caption{شکل ۲.۷: تصویر سطری و ستونی برای مثال ۲: بی‌نهایت جواب.}
	\end{figure}
	
	\begin{quote}
		\textbf{شکست:} برای $n$ معادله، ما $n$ پاشنه به دست نمی‌آوریم. حذف به معادله $0 \neq 0$ (بدون جواب) یا $0=0$ (جواب‌های زیاد) منجر می‌شود.
	\end{quote}
	\begin{quote}
		\textbf{موفقیت:} با $n$ پاشنه حاصل می‌شود. اما ممکن است مجبور شویم $n$ معادله را جابجا کنیم.
	\end{quote}
	
	حذف می‌تواند به طریق سومی نیز با مشکل مواجه شود—اما این بار می‌توان آن را برطرف کرد. فرض کنید جایگاه پاشنه اول شامل صفر باشد. ما صفر را به عنوان پاشنه مجاز نمی‌دانیم. وقتی معادله اول هیچ جمله‌ای شامل $x$ ندارد، می‌توانیم آن را با یک معادله پایین‌تر جابجا کنیم:
	
	\subsubsection*{مثال ۳: شکست موقت (صفر در پاشنه) با یک جابجایی سطر}
	یک جابجایی سطر (Permutation) دو پاشنه تولید می‌کند:
	\[
	\begin{cases}
		0x + 2y = 4 \\
		3x - 2y = 5
	\end{cases}
	\xrightarrow{\text{دو معادله را جابجا کن}}
	\begin{cases}
		3x - 2y = 5 \\
		2y = 4
	\end{cases}
	\]
	دستگاه جدید از قبل مثلثی است. این مثال کوچک برای جایگذاری پس‌رو آماده است. معادله آخر $y=2$ را می‌دهد، و سپس معادله اول $x=3$ را می‌دهد. تصویر سطری نرمال است (دو خط متقاطع). تصویر ستونی نیز نرمال است (بردارهای ستونی در یک جهت نیستند). پاشنه‌های ۳ و ۲ نرمال هستند—اما یک جابجایی سطر لازم بود.
	
	مثال‌های ۱ و ۲ \textbf{منفرد} (singular) هستند—پاشنه دومی وجود ندارد. مثال ۳ \textbf{غیرمنفرد} (nonsingular) است—یک مجموعه کامل از پاشنه‌ها و دقیقاً یک جواب وجود دارد. معادلات منفرد یا هیچ جوابی ندارند یا بی‌نهایت جواب دارند. پاشنه‌ها باید غیرصفر باشند زیرا ما باید بر آنها تقسیم کنیم.
	
	\subsection*{سه معادله در سه مجهول}
	برای درک حذف گاوسی، باید فراتر از دستگاه‌های ۲ در ۲ بروید. سه در سه برای دیدن الگو کافی است. فعلاً ماتریس‌ها مربعی هستند—تعداد مساوی سطر و ستون. در اینجا یک دستگاه ۳ در ۳ آمده است که به طور خاص ساخته شده تا تمام مراحل حذف به اعداد صحیح و نه کسر منجر شوند:
	\[
	(۱) \quad
	\begin{cases}
		\mathbf{2}x + 4y - 2z = 2 \\
		4x + 9y - 3z = 8 \\
		-2x - 3y + 7z = 10
	\end{cases}
	\]
	مراحل چیست؟ اولین پاشنه عدد پررنگ \textbf{۲} (بالا سمت چپ) است. زیر آن پاشنه می‌خواهیم ۴ را حذف کنیم. اولین ضریب نسبت $\ell_{21} = 4/2 = 2$ است. معادله پاشنه را در $\ell_{21} = 2$ ضرب کرده و تفریق کنید. تفریق $4x$ را از معادله دوم حذف می‌کند:
	
	\textbf{مرحله ۱:} ۲ برابر معادله ۱ را از معادله ۲ کم کنید. این کار $y+z=4$ را باقی می‌گذارد.
	
	ما همچنین $-2x$ را از معادله ۳ حذف می‌کنیم—هنوز با استفاده از پاشنه اول. راه سریع این است که معادله ۱ را به معادله ۳ اضافه کنیم. آنگاه $2x$ با $-2x$ خنثی می‌شود. ما دقیقاً همین کار را می‌کنیم، اما قاعده در این کتاب تفریق به جای جمع است. الگوی نظام‌مند ضریب $\ell_{31} = -2/2 = -1$ را دارد. کم کردن $-1$ برابر یک معادله همانند اضافه کردن آن است:
	
	\textbf{مرحله ۲:} $-1$ برابر معادله ۱ را از معادله ۳ کم کنید. این کار $y+5z=12$ را باقی می‌گذارد.
	
	دو معادله جدید فقط شامل $y$ و $z$ هستند. پاشنه دوم (پررنگ) ۱ است:
	\[
	\text{x حذف شده است}
	\quad
	\begin{cases}
		\mathbf{1}y + 1z = 4 \\
		1y + 5z = 12
	\end{cases}
	\]
	ما به یک دستگاه ۲ در ۲ رسیده‌ایم. مرحله نهایی $y$ را حذف می‌کند تا آن را ۱ در ۱ کند:
	
	\textbf{مرحله ۳:} معادله دوم جدید را از معادله سوم جدید کم کنید. ضریب $\ell_{32} = 1/1 = 1$ است. آنگاه $4z=8$.
	
	دستگاه اصلی $A\mathbf{x} = \mathbf{b}$ به یک دستگاه بالا-مثلثی $U\mathbf{x} = \mathbf{c}$ تبدیل شده است:
	\[
	(۲) \quad
	\underbrace{
		\begin{cases}
			2x + 4y - 2z = 2 \\
			4x + 9y - 3z = 8 \\
			-2x - 3y + 7z = 10
		\end{cases}
	}_{A\mathbf{x}=\mathbf{b}}
	\quad \text{تبدیل شده به} \quad
	\underbrace{
		\begin{cases}
			\mathbf{2}x + 4y - 2z = 2 \\
			\phantom{2x+} \mathbf{1}y + 1z = 4 \\
			\phantom{2x+4y-} \mathbf{4}z = 8
		\end{cases}
	}_{U\mathbf{x}=\mathbf{c}}
	\]
	هدف محقق شد—\textbf{حذف پیش‌رو} (forward elimination) از $A$ به $U$ کامل است. به پاشنه‌ها یعنی ۲، ۱، ۴ در امتداد قطر $U$ توجه کنید. پاشنه‌های ۱ و ۴ در دستگاه اصلی پنهان بودند. حذف آنها را آشکار کرد. $U\mathbf{x} = \mathbf{c}$ برای \textbf{جایگذاری پس‌رو} آماده است، که سریع است:
	\begin{itemize}
		\item ($4z=8$ نتیجه می‌دهد $z=2$)
		\item ($y+z=4$ نتیجه می‌دهد $y=2$)
		\item (معادله ۱ نتیجه می‌دهد $x=-1$)
	\end{itemize}
	جواب $(\mathbf{x, y, z) = (-1, 2, 2)}$ است. تصویر سطری سه صفحه از سه معادله دارد. تمام صفحات از این جواب عبور می‌کنند. صفحات اصلی شیب‌دار هستند، اما صفحه آخر $4z=8$ پس از حذف، افقی است.
	
	تصویر ستونی ترکیبی $A\mathbf{x}$ از بردارهای ستونی را نشان می‌دهد که سمت راست $\mathbf{b}$ را تولید می‌کند. ضرایب آن ترکیب ۱-، ۲ و ۲ هستند (جواب):
	\[
	(۳) \quad
	(-1) \begin{bmatrix} 2 \\ 4 \\ -2 \end{bmatrix}
	+ (2) \begin{bmatrix} 4 \\ 9 \\ -3 \end{bmatrix}
	+ (2) \begin{bmatrix} -2 \\ -3 \\ 7 \end{bmatrix}
	=
	\begin{bmatrix} 2 \\ 8 \\ 10 \end{bmatrix}
	= \mathbf{b}.
	\]
	اعداد $x, y, z$ ستون‌های ۱، ۲ و ۳ را در $A\mathbf{x}=\mathbf{b}$ و همچنین در دستگاه مثلثی $U\mathbf{x} = \mathbf{c}$ ضرب می‌کنند.
	
	\subsection*{حذف از A به U}
	برای یک مسئله ۴ در ۴، یا یک مسئله $n$ در $n$، حذف به همان روش پیش می‌رود. در اینجا کل ایده، ستون به ستون از $A$ به $U$، زمانی که حذف گاوسی موفقیت‌آمیز است، آمده است.
	
	\textbf{ستون ۱.} از معادله اول برای ایجاد صفر در زیر پاشنه اول استفاده کنید.
	
	\textbf{ستون ۲.} از معادله جدید دوم برای ایجاد صفر در زیر پاشنه دوم استفاده کنید.
	
	\textbf{ستون‌های ۳ تا n.} به همین ترتیب ادامه دهید تا تمام $n$ پاشنه و ماتریس بالا-مثلثی $U$ را پیدا کنید.
	\[
	(۴) \quad
	\text{هدف رسیدن به ماتریس } U \text{ است:} \quad
	U = \begin{bmatrix}
		\bullet & x & x & x \\
		0 & \bullet & x & x \\
		0 & 0 & \bullet & x \\
		0 & 0 & 0 & \bullet
	\end{bmatrix}
	\]
	نتیجه حذف پیش‌رو یک دستگاه بالا-مثلثی است. اگر مجموعه کاملی از $n$ پاشنه (هیچ‌کدام صفر نباشند!) وجود داشته باشد، دستگاه غیرمنفرد است. سوال: کدام متغیرها در سمت چپ در فرآیند حذف تغییر نخواهند کرد زیرا پاشنه‌ها مشخص هستند؟ در اینجا یک مثال نهایی برای نشان دادن دستگاه اصلی $A\mathbf{x}=\mathbf{b}$، دستگاه مثلثی $U\mathbf{x}=\mathbf{c}$، و جواب $(x,y,z)$ از جایگذاری پس‌رو آمده است:
	\[
	\begin{array}{lclcl}
		\multicolumn{1}{c}{\mathbf{A\mathbf{x} = \mathbf{b}}} & & \multicolumn{1}{c}{\mathbf{U\mathbf{x} = \mathbf{c}}} & & \multicolumn{1}{c}{\mathbf{x}} \\
		\begin{cases}
			x+y+z=6 \\
			x+2y+2z=9 \\
			x+2y+3z=10
		\end{cases}
		& \xrightarrow{\text{حذف پیش‌رو}} &
		\begin{cases}
			x+y+z=6 \\
			\phantom{x+}y+z=3 \\
			\phantom{x+y+}z=1
		\end{cases}
		& \xrightarrow{\text{جایگذاری پس‌رو}} &
		\begin{cases}
			x=3 \\
			y=2 \\
			z=1
		\end{cases}
	\end{array}
	\]
	تمام ضرایب ۱ هستند. تمام پاشنه‌ها ۱ هستند. تمام صفحات در جواب $(3,2,1)$ یکدیگر را قطع می‌کنند. ستون‌های $A$ با ضرایب ۳، ۲ و ۱ ترکیب می‌شوند تا $\mathbf{b}=(6,9,10)$ را بدهند. مثلث $U\mathbf{x}=\mathbf{c}=(6,3,1)$ را نشان می‌دهد.
	
	\subsection*{مروری بر ایده‌های کلیدی}
	\begin{enumerate}
		\item یک دستگاه خطی ($A\mathbf{x}=\mathbf{b}$) پس از حذف به یک دستگاه بالا-مثلثی ($U\mathbf{x}=\mathbf{c}$) تبدیل می‌شود.
		\item ما $\ell_{ij}$ برابر معادله $j$ را از معادله $i$ کم می‌کنیم تا درایه $(i,j)$ صفر شود.
		\item ضریب برابر است با $\ell_{ij} = \frac{\text{درایه‌ای که باید در سطر i حذف شود}}{\text{پاشنه در سطر j}}$. پاشنه‌ها نمی‌توانند صفر باشند!
		\item وقتی در جایگاه پاشنه صفر وجود دارد، اگر یک درایه غیرصفر زیر آن باشد، سطرها را جابجا کنید.
		\item دستگاه بالا-مثلثی $U\mathbf{x}=\mathbf{c}$ با جایگذاری پس‌رو (شروع از پایین) حل می‌شود.
		\item وقتی شکست دائمی است، $A\mathbf{x}=\mathbf{b}$ یا جوابی ندارد یا بی‌نهایت جواب دارد.
	\end{enumerate}
	
	\newpage
	\subsection*{مثال‌های حل شده}
	\subsubsection*{مثال ۲.۲ الف}
	وقتی حذف روی این ماتریس $A$ اعمال می‌شود، پاشنه‌های اول و دوم چه هستند؟ ضریب $\ell_{21}$ در مرحله اول (که $\ell_{21}$ برابر سطر ۱ از سطر ۲ کم می‌شود) چیست؟
	\[ A = \begin{bmatrix} 1 & 1 & 0 \\ 1 & 2 & 1 \\ 0 & 1 & 2 \end{bmatrix} \xrightarrow{\ell_{21}=1} \begin{bmatrix} 1 & 1 & 0 \\ 0 & 1 & 1 \\ 0 & 1 & 2 \end{bmatrix} \xrightarrow{\ell_{32}=1} \begin{bmatrix} 1 & 1 & 0 \\ 0 & 1 & 1 \\ 0 & 0 & 1 \end{bmatrix} = U \]
	چه درایه‌ای در موقعیت ۲،۲ (به جای ۲) باعث جابجایی سطرهای ۲ و ۳ می‌شود؟ چرا ضریب پایین سمت چپ $\ell_{31}=0$ است، یعنی صفر برابر سطر ۱ از سطر ۳ کم می‌شود؟ اگر درایه گوشه‌ای $a_{33}=2$ را به $a_{33}=1$ تغییر دهید، چرا حذف با شکست مواجه می‌شود؟
	
	\textbf{پاسخ:}
	پاشنه اول ۱ است. ضریب $\ell_{21}$ برابر $1/1=1$ است. وقتی ۱ برابر سطر ۱ از سطر ۲ کم می‌شود، پاشنه دوم به عنوان یک ۱ دیگر آشکار می‌شود. اگر درایه میانی اصلی به جای ۲، ۱ بود، این امر مجبور به جابجایی سطر می‌کرد (زیرا پس از مرحله اول، درایه (2,2) صفر می‌شد).
	
	ضریب $\ell_{31}$ صفر است زیرا $a_{31}=0$ است. یک صفر در ابتدای یک سطر نیازی به حذف ندارد. این ماتریس $A$ یک «ماتریس نواری» است. همه چیز خارج از نوار صفر باقی می‌ماند.
	
	پاشنه آخر در ماتریس $U$ برابر ۱ است. این پاشنه از تفریق سطر دوم جدید از سطر سوم جدید به دست می‌آید: $2-1=1$. اگر درایه گوشه‌ای اصلی $a_{33}=2$ به $a_{33}=1$ تغییر می‌کرد، آنگاه پاشنه سوم $1-1=0$ می‌شد. پاشنه سومی وجود نخواهد داشت و حذف با شکست مواجه می‌شود.
	
	\subsubsection*{مثال ۲.۲ ب}
	فرض کنید $A$ از قبل یک ماتریس مثلثی است (بالا-مثلثی یا پایین-مثلثی). پاشنه‌های آن را کجا می‌بینید؟ چه زمانی $A\mathbf{x}=\mathbf{b}$ دقیقاً یک جواب برای هر $\mathbf{b}$ دارد؟
	
	\textbf{پاسخ:}
	پاشنه‌های یک ماتریس مثلثی از قبل در امتداد قطر اصلی قرار دارند. حذف (یا حل) زمانی موفقیت‌آمیز است که تمام آن اعداد قطری غیرصفر باشند. وقتی $A$ بالا-مثلثی است از جایگذاری پس‌رو استفاده کنید، و وقتی $A$ پایین-مثلثی است از جایگذاری پیش‌رو (از بالا به پایین) استفاده کنید.
	
	\subsubsection*{مثال ۲.۲ ج}
	از حذف برای رسیدن به ماتریس‌های بالا-مثلثی $U$ استفاده کنید. با جایگذاری پس‌رو حل کنید یا توضیح دهید چرا این غیرممکن است. پاشنه‌ها چه هستند (هرگز صفر نیستند)؟ در صورت لزوم معادلات را جابجا کنید. تنها تفاوت در $-x$ در معادله آخر است.
	\[
	\begin{array}{cc}
		\textbf{موفقیت‌آمیز} & \textbf{ناموفق} \\
		\begin{cases}
			x+y+z=7 \\
			x+y-z=5 \\
			x-y+z=3
		\end{cases}
		&
		\begin{cases}
			x+y+z=7 \\
			x+y-z=5 \\
			-x-y+z=3
		\end{cases}
	\end{array}
	\]
	
	\textbf{پاسخ:}
	برای دستگاه اول، معادله ۱ را از معادلات ۲ و ۳ کم کنید (ضرایب $\ell_{21}=1$ و $\ell_{31}=1$ هستند). درایه ۲،۲ صفر می‌شود، بنابراین معادلات ۲ و ۳ را جابجا کنید:
	\[
	\begin{cases}
		x+y+z=7 \\
		0y-2z=-2 \\
		-2y+0z=-4
	\end{cases}
	\xrightarrow{\text{جابجایی سطر ۲ و ۳}}
	\begin{cases}
		x+y+z=7 \\
		-2y+0z=-4 \\
		-2z=-2
	\end{cases}
	\]
	سپس جایگذاری پس‌رو $z=1$ و $y=2$ و $x=4$ را می‌دهد. پاشنه‌ها ۱، ۲- و ۲- هستند.
	
	برای دستگاه دوم، معادله ۱ را از معادله ۲ مثل قبل کم کنید (ضریب $\ell_{21}=1$). معادله ۱ را به معادله ۳ اضافه کنید (ضریب $\ell_{31}=-1$). این کار در درایه ۲،۲ و همچنین زیر آن (درایه ۳،۲) صفر باقی می‌گذارد:
	\[
	\begin{cases}
		x+y+z=7 \\
		0y-2z=-2 \\
		0y+2z=10
	\end{cases}
	\xrightarrow{\text{اضافه کردن سطر جدید ۲ به ۳}}
	\begin{cases}
		x+y+z=7 \\
		0y-2z=-2 \\
		0z=8
	\end{cases}
	\]
	معادله آخر $0z=8$ است که هیچ جوابی ندارد. هیچ پاشنه‌ای در ستون ۲ وجود ندارد. سه صفحه یکدیگر را قطع نمی‌کنند. صفحه ۱ با صفحه ۲ در یک خط تلاقی دارد. صفحه ۱ با صفحه ۳ در یک خط موازی دیگر تلاقی دارد. هیچ جوابی وجود ندارد.
	
	اگر «۳» را در معادله سوم اصلی به «۵-» تغییر دهیم، آنگاه حذف به $0=0$ منجر می‌شود و بی‌نهایت جواب وجود دارد. در این حالت سه صفحه در امتداد یک خط کامل با هم تلاقی دارند. معادله دوم ($0y-2z=-2$) نتیجه می‌دهد $z=1$. سپس معادله اول ($x+y+z=7$) به $x+y=6$ تبدیل می‌شود. نبود پاشنه در ستون ۲، متغیر $y$ را آزاد می‌کند. آنگاه $x=6-y$.
	
	\newpage
	\section*{مجموعه مسائل ۲.۲}
	\subsection*{مسائل ۱-۱۰ در مورد حذف در دستگاه‌های ۲ در ۲ هستند.}
	
	\begin{enumerate}
		\item چه ضریبی $\ell_{21}$ از معادله ۱ باید از معادله ۲ کم شود؟
		\[
		\begin{cases}
			2x + 3y = 1 \\
			10x + 9y = 11
		\end{cases}
		\]
		پس از حذف، دستگاه بالا-مثلثی را بنویسید و دو پاشنه را مشخص کنید.
		
		\item دستگاه مثلثی مسئله ۱ را با جایگذاری پس‌رو، ابتدا $y$ و سپس $x$ حل کنید. تأیید کنید که $x$ برابر $(2, 10)$ به علاوه $y$ برابر $(3, 9)$ مساوی با $(1, 11)$ است. اگر سمت راست به $(4, 44)$ تغییر کند، جواب جدید چیست؟
		
		\item چه ضریبی از معادله ۱ باید از معادله ۲ کم شود؟
		\[
		\begin{cases}
			2x - 4y = 6 \\
			-x + 5y = 0
		\end{cases}
		\]
		پس از این مرحله حذف، دستگاه مثلثی را حل کنید. اگر سمت راست به $(-6, 0)$ تغییر کند، جواب جدید چیست؟
		
		\item چه ضریبی $\ell$ از معادله ۱ باید از معادله ۲ کم شود تا $cx$ حذف شود؟
		\[
		\begin{cases}
			ax+by=f \\
			cx+dy=g
		\end{cases}
		\]
		پاشنه اول $a$ است (فرض می‌شود غیرصفر است). حذف چه فرمولی را برای پاشنه دوم تولید می‌کند؟ $y$ چیست؟ پاشنه دوم زمانی وجود ندارد که $ad=bc$ باشد: حالت منفرد.
		
		\item یک سمت راست انتخاب کنید که هیچ جوابی ندهد و یک سمت راست دیگر که بی‌نهایت جواب بدهد. دو مورد از آن جواب‌ها کدامند؟
		\[
		\text{دستگاه منفرد} \quad
		\begin{cases}
			3x + 2y = 10 \\
			6x + 4y = \underline{\phantom{00}}
		\end{cases}
		\]
		
		\item ضریب $b$ را طوری انتخاب کنید که این دستگاه منفرد شود. سپس یک سمت راست $g$ انتخاب کنید که آن را قابل حل کند. دو جواب در آن حالت منفرد پیدا کنید.
		\[
		\begin{cases}
			2x + by = 16 \\
			4x + 8y = g
		\end{cases}
		\]
		
		\item برای کدام اعداد $a$ حذف با شکست مواجه می‌شود (۱) دائمی (۲) موقت؟
		\[
		\begin{cases}
			ax + 3y = -3 \\
			4x + 6y = 6
		\end{cases}
		\]
		پس از رفع شکست موقت با جابجایی سطر، $x$ و $y$ را حل کنید.
		
		\item برای کدام سه عدد $k$ حذف با شکست مواجه می‌شود؟ کدام یک با جابجایی سطر رفع می‌شود؟ در هر مورد، تعداد جواب‌ها ۰، ۱ یا $\infty$ است؟
		\[
		\begin{cases}
			kx + 3y = 6 \\
			3x + ky = -6
		\end{cases}
		\]
		
		\item چه آزمونی بر روی $b_1$ و $b_2$ تصمیم می‌گیرد که آیا این دو معادله اجازه جواب دارند؟ چند جواب خواهند داشت؟ تصویر ستونی را برای $\mathbf{b}=(1,2)$ و $(1,0)$ رسم کنید.
		\[
		\begin{cases}
			3x - 2y = b_1 \\
			6x - 4y = b_2
		\end{cases}
		\]
		
		\item در صفحه $xy$، خطوط $x+y=5$ و $x+2y=6$ و معادله $y = \underline{\phantom{00}}$ که از حذف به دست می‌آید را رسم کنید. خط $5x-4y=c$ از جواب این دستگاه عبور خواهد کرد اگر $c = \underline{\phantom{00}}$.
	\end{enumerate}
	
	\subsection*{مسائل ۱۱-۲۰ به مطالعه حذف در دستگاه‌های ۳ در ۳ (و شکست احتمالی) می‌پردازند.}
	\begin{enumerate}
		\setcounter{enumi}{10}
		\item (توصیه می‌شود) یک دستگاه معادلات خطی نمی‌تواند دقیقاً دو جواب داشته باشد. چرا؟
		\begin{itemize}
			\item[(الف)] اگر $(x,y,z)$ و $(X,Y,Z)$ دو جواب باشند، جواب دیگر چیست؟
			\item[(ب)] اگر ۲۵ صفحه در دو نقطه یکدیگر را قطع کنند، در کجا دیگر یکدیگر را قطع می‌کنند؟
		\end{itemize}
		
		\item این دستگاه را با دو عملیات سطری به فرم بالا-مثلثی کاهش دهید:
		\[
		\begin{cases}
			2x + 3y + z = 8 \\
			4x + 7y + 5z = 20 \\
			-2y + 2z = 0
		\end{cases}
		\]
		پاشنه‌ها را مشخص کنید. با جایگذاری پس‌رو برای $z,y,x$ حل کنید.
		
		\item حذف را اعمال کنید (پاشنه‌ها را مشخص کنید) و با جایگذاری پس‌رو حل کنید:
		\[
		\begin{cases}
			2x - 3y = 3 \\
			4x - 5y + z = 7 \\
			2x - y - 3z = 5
		\end{cases}
		\]
		سه عملیات سطری را لیست کنید: کم کردن \_\_\_ برابر سطر \_\_\_ از سطر \_\_\_.
		
		\item کدام عدد $d$ باعث جابجایی سطر می‌شود، و دستگاه مثلثی (غیرمنفرد) برای آن $d$ چیست؟ کدام $d$ این دستگاه را منفرد می‌کند (بدون پاشنه سوم)؟
		\[
		\begin{cases}
			2x + 5y + z = 0 \\
			4x + dy + z = 2 \\
			y - z = 3
		\end{cases}
		\]
		
		\item کدام عدد $b$ بعداً به جابجایی سطر منجر می‌شود؟ کدام $b$ به یک پاشنه گمشده منجر می‌شود؟ در آن حالت منفرد یک جواب غیرصفر $x,y,z$ پیدا کنید.
		\[
		\begin{cases}
			x + by = 0 \\
			x - 2y - z = 0 \\
			y + z = 0
		\end{cases}
		\]
		
		\item (الف) یک دستگاه ۳ در ۳ بسازید که برای رسیدن به فرم مثلثی و یک جواب به دو جابجایی سطر نیاز داشته باشد.
		(ب) یک دستگاه ۳ در ۳ بسازید که برای ادامه به جابجایی سطر نیاز داشته باشد، اما بعداً با شکست مواجه شود.
		
		\item اگر سطرهای ۱ و ۲ یکسان باشند، با حذف چقدر می‌توانید پیش بروید (با اجازه جابجایی سطر)؟ اگر ستون‌های ۱ و ۲ یکسان باشند، کدام پاشنه گمشده است؟
		\[
		\begin{array}{cc}
			\textbf{سطرهای برابر} & \textbf{ستون‌های برابر} \\
			\begin{cases}
				2x - y + z = 0 \\
				2x - y + z = 0 \\
				4x + y + z = 2
			\end{cases}
			&
			\begin{cases}
				2x + 2y + z = 0 \\
				4x + 4y + z = 0 \\
				6x + 6y + z = 2
			\end{cases}
		\end{array}
		\]
		
		\item یک مثال ۳ در ۳ بسازید که ۹ ضریب مختلف در سمت چپ داشته باشد، اما سطرهای ۲ و ۳ در حذف صفر شوند. دستگاه شما با $\mathbf{b}=(1, 10, 100)$ چند جواب دارد و با $\mathbf{b}=(0,0,0)$ چند جواب دارد؟
		
		\item کدام عدد $q$ این دستگاه را منفرد می‌کند و کدام سمت راست $t$ به آن بی‌نهایت جواب می‌دهد؟ جوابی را پیدا کنید که $z=1$ دارد.
		\[
		\begin{cases}
			x + 4y - 2z = 1 \\
			x + 7y - 6z = 6 \\
			3y + qz = t
		\end{cases}
		\]
		
		\item سه صفحه می‌توانند نقطه تلاقی نداشته باشند، حتی اگر هیچ صفحه‌ای موازی نباشد. دستگاه منفرد است اگر سطر ۳ ماتریس $A$ یک \textbf{ترکیب خطی} از دو سطر اول باشد. یک معادله سوم پیدا کنید که نتواند همراه با $x+y+z=0$ و $x-2y-z=1$ حل شود.
		
		\item پاشنه‌ها و جواب را برای هر دو دستگاه ($A\mathbf{x}=\mathbf{b}$ و $K\mathbf{x}=\mathbf{b}$) پیدا کنید:
		\[
		\begin{array}{cc}
			A\mathbf{x}=\mathbf{b} & K\mathbf{x}=\mathbf{b} \\
			\begin{cases}
				2x + y \qquad = 0 \\
				x + 2y + z \quad = 0 \\
				y + 2z + t = 0 \\
				z + 2t = 5
			\end{cases}
			& \qquad
			\begin{cases}
				2x - y \qquad = 0 \\
				-x + 2y - z \quad = 0 \\
				-y + 2z - t = 0 \\
				-z + 2t = 5
			\end{cases}
		\end{array}
		\]
		
		\item اگر مسئله ۲۱ را با الگوی ۱، ۲، ۱ یا الگوی ۱-، ۲، ۱- گسترش دهید، پاشنه پنجم چیست؟ پاشنه $n$-ام چیست؟ ($K$ ماتریسی با درایه‌های قطری ۲، و درایه‌های روی قطرهای فرعی ۱- یا ۱+ است).
		
		\item اگر حذف به $x+y=1$ و $2y=3$ منجر شود، سه مسئله اصلی ممکن را پیدا کنید.
		
		\item برای کدام دو عدد $a$ حذف در $A = \begin{bmatrix} a & 2 \\ a & a \end{bmatrix}$ با شکست مواجه می‌شود؟
		
		\item برای کدام سه عدد $a$ حذف نمی‌تواند سه پاشنه بدهد؟ $A = \begin{bmatrix} a & 2 & 3 \\ a & a & 4 \\ a & a & a \end{bmatrix}$ برای سه مقدار $a$ منفرد است.
		
		\item به دنبال ماتریسی بگردید که مجموع سطرهای آن ۴ و ۸ و مجموع ستون‌های آن ۲ و $s$ باشد:
		\[
		\text{ماتریس} = \begin{bmatrix} a & b \\ c & d \end{bmatrix} \quad
		\begin{array}{ll}
			a+b=4 & a+c=2 \\
			c+d=8 & b+d=s
		\end{array}
		\]
		چهار معادله تنها در صورتی قابل حل هستند که $s = \underline{\phantom{00}}$. سپس دو ماتریس متفاوت پیدا کنید که مجموع سطرها و ستون‌های صحیحی داشته باشند. \textbf{امتیاز اضافی:} دستگاه ۴ در ۴ $A\mathbf{x}=\mathbf{b}$ را با $\mathbf{x}=(a,b,c,d)^T$ بنویسید و $A$ را با حذف، مثلثی کنید.
		
		\item حذف به ترتیب معمول چه ماتریس $U$ و چه جوابی برای این دستگاه «پایین-مثلثی» می‌دهد؟ ما در واقع با جایگذاری پیش‌رو حل می‌کنیم:
		\[
		\begin{cases}
			3x \qquad \quad = 3 \\
			6x + 2y \quad = 8 \\
			9x - 2y + z = 9
		\end{cases}
		\]
	\end{enumerate}
	
	\subsection*{مسائل چالشی}
	\begin{enumerate}
		\setcounter{enumi}{27}
		\item یک دستور MATLAB به صورت `A(2, :) = ...` برای سطر جدید ۲ بسازید تا ۳ برابر سطر ۱ را از سطر موجود ۲ کم کند، اگر ماتریس A از قبل مشخص باشد.
		
		\item به صورت تجربی میانگین اندازه پاشنه‌های اول، دوم و سوم را از دستور `[L, U] = lu(rand(3))` در MATLAB پیدا کنید. میانگین اندازه `abs(U(1,1))` بالاتر از ۰.۵ است زیرا `lu` بزرگترین پاشنه موجود در ستون ۱ را انتخاب می‌کند. در اینجا `A=rand(3)` درایه‌های تصادفی بین ۰ و ۱ دارد.
		
		\item اگر درایه گوشه‌ای آخر $A(5,5)=11$ و پاشنه آخر $A$ (یعنی $U(5,5)$) برابر $4$ باشد، چه درایه متفاوتی برای $A(5,5)$ می‌توانست $A$ را منفرد کند؟
		
		\item فرض کنید حذف $A$ را بدون جابجایی سطر به $U$ می‌برد. آنگاه سطر $j$ از $U$ ترکیبی از کدام سطرهای $A$ است؟ اگر $A\mathbf{x}=0$ باشد، آیا $U\mathbf{x}=0$ است؟ اگر $A\mathbf{x}=\mathbf{b}$ باشد، آیا $U\mathbf{x}=\mathbf{b}$ است؟ اگر $A$ از ابتدا پایین-مثلثی باشد، ماتریس بالا-مثلثی $U$ چیست؟
		
		\item با ۱۰۰ معادله $A\mathbf{x}=0$ برای ۱۰۰ مجهول $\mathbf{x}=(x_1, \dots, x_{100})$ شروع کنید. فرض کنید حذف، معادله صدم را به $0=0$ کاهش می‌دهد، بنابراین دستگاه «منفرد» است.
		\begin{itemize}
			\item[(الف)] حذف، ترکیب‌های خطی از سطرها را می‌گیرد. بنابراین این دستگاه منفرد خاصیت منفرد زیر را دارد: یک ترکیب خطی از ۱۰۰ سطر برابر با \textbf{سطر صفر} است.
			\item[(ب)] دستگاه‌های منفرد $A\mathbf{x}=0$ بی‌نهایت جواب دارند. این به این معنی است که یک ترکیب خطی از ۱۰۰ ستون برابر با \textbf{ستون صفر} است.
			\item[(ج)] یک ماتریس منفرد ۱۰۰ در ۱۰۰ بدون درایه صفر ابداع کنید.
			\item[(د)] برای ماتریس خود، تصویر سطری و تصویر ستونی $A\mathbf{x}=0$ را با کلمات توصیف کنید. لازم نیست فضای ۱۰۰ بعدی را رسم کنید.
		\end{itemize}
	\end{enumerate}
	
\end{document}