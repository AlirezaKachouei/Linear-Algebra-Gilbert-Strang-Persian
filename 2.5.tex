\documentclass[12pt, a4paper]{book}

% فراخوانی بسته‌های لازم
\usepackage{amsmath}         % برای فرمول‌های پیشرفته ریاضی
\usepackage{amsfonts}        % بسته برای فونت‌های ریاضی مانند \mathbb
\usepackage{amssymb}         % برای نمادهای بیشتر ریاضی
\usepackage{graphicx}        % برای افزودن تصاویر
\usepackage{xepersian}       % بسته اصلی برای پارسی‌نویسی
\usepackage{geometry}        % برای تنظیم حاشیه‌ها
\usepackage{setspace}        % برای تنظیم فاصله خطوط
\usepackage{array}           % برای امکانات پیشرفته در جدول‌ها و آرایه‌ها
\usepackage{enumitem}        % برای کنترل بیشتر بر لیست‌ها

% تنظیم حاشیه‌های صفحه
\geometry{
	a4paper,
	total={170mm,257mm},
	left=20mm,
	top=20mm,
}

% تنظیم فونت‌های نوشتاری و ریاضی
% توجه: این فونت‌ها باید روی سیستم شما نصب باشند
\settextfont{XB Niloofar}
\setdigitfont{XB Niloofar}
\setmathdigitfont{XB Niloofar}

\begin{document}
	
	% اعمال فاصله 1.5 بین خطوط برای خوانایی بهتر
	\onehalfspacing
	
	\section{ماتریس‌های معکوس}
	
	\begin{enumerate}[label=\arabic*.]
		\item اگر ماتریس مربع $A$ معکوس داشته باشد، آنگاه هم $A^{-1}A=I$ و هم $AA^{-1}=I$ برقرار است.
		\item الگوریتم آزمون معکوس‌پذیری، حذف است: $A$ باید $n$ لولای (غیرصفر) داشته باشد.
		\item آزمون جبری برای معکوس‌پذیری، دترمینان $A$ است: $\det(A)$ نباید صفر باشد.
		\item معادله‌ای که معکوس‌پذیری را می‌آزماید $A\mathbf{x}=\mathbf{0}$ است: $\mathbf{x}=\mathbf{0}$ باید تنها جواب باشد.
		\item اگر $A$ و $B$ (هم‌اندازه) معکوس‌پذیر باشند، آنگاه $AB$ نیز معکوس‌پذیر است: $(AB)^{-1} = B^{-1}A^{-1}$.
		\item $AA^{-1}=I$ شامل $n$ معادله برای $n$ ستون از $A^{-1}$ است. روش گاوس-جردن ماتریس الحاقی $[A \ I]$ را به $[I \ A^{-1}]$ تبدیل می‌کند.
		\item در صفحه آخر کتاب، ۱۴ شرط معادل برای معکوس‌پذیر بودن ماتریس مربع $A$ ارائه شده است.
	\end{enumerate}
	
	فرض کنید $A$ یک ماتریس مربع است. ما به دنبال یک «ماتریس معکوس» $A^{-1}$ با همان اندازه هستیم، به طوری که حاصل‌ضرب $A^{-1}$ در $A$ برابر با $I$ شود. هر کاری که $A$ انجام می‌دهد، $A^{-1}$ آن را \textbf{خنثی می‌کند}. حاصل‌ضرب آن‌ها ماتریس همانی است—که هیچ تأثیری بر یک بردار ندارد، بنابراین $A^{-1}A\mathbf{x} = \mathbf{x}$. اما $A^{-1}$ ممکن است وجود نداشته باشد.
	
	\textit{(توضیح مترجم: شهود اصلی پشت ماتریس معکوس، مفهوم «عملیات معکوس» است. همان‌طور که تقسیم، عمل ضرب را خنثی می‌کند و ریشه دوم، عمل توان دو را، ماتریس معکوس نیز تبدیل خطی اعمال‌شده توسط ماتریس اصلی را خنثی کرده و ما را به بردار اولیه بازمی‌گرداند.)}
	
	کاری که یک ماتریس عمدتاً انجام می‌دهد، ضرب شدن در یک بردار $\mathbf{x}$ است. ضرب کردن $A\mathbf{x}=\mathbf{b}$ در $A^{-1}$ نتیجه می‌دهد $A^{-1}A\mathbf{x} = A^{-1}\mathbf{b}$. این یعنی $\mathbf{x} = A^{-1}\mathbf{b}$. حاصل‌ضرب $A^{-1}A$ مانند ضرب کردن در یک عدد و سپس تقسیم بر همان عدد است. یک عدد اگر صفر نباشد، معکوس دارد. ماتریس‌ها پیچیده‌تر و جالب‌تر هستند. ماتریس $A^{-1}$ «معکوس A» نامیده می‌شود.
	
	\begin{quote}
		\textbf{تعریف:} ماتریس $A$ \textbf{معکوس‌پذیر (invertible)} است اگر ماتریسی مانند $A^{-1}$ وجود داشته باشد که $A$ را «معکوس» کند:
		\[ A^{-1}A = I \quad \text{و} \quad AA^{-1} = I \quad \text{(معکوس دوطرفه)} \quad (۱) \]
	\end{quote}
	
	همه ماتریس‌ها معکوس ندارند. این اولین سوالی است که در مورد یک ماتریس مربع می‌پرسیم: آیا $A$ معکوس‌پذیر است؟ منظور ما این نیست که بلافاصله $A^{-1}$ را محاسبه کنیم. در بیشتر مسائل هرگز آن را محاسبه نمی‌کنیم! در اینجا شش «نکته» در مورد $A^{-1}$ آورده شده است.
	
	\textbf{نکته ۱}
	معکوس وجود دارد اگر و تنها اگر فرآیند حذف، $n$ لولا تولید کند (جابجایی سطرها مجاز است). حذف، معادله $A\mathbf{x}=\mathbf{b}$ را بدون استفاده صریح از ماتریس $A^{-1}$ حل می‌کند.
	
	\textbf{نکته ۲}
	ماتریس $A$ نمی‌تواند دو معکوس متفاوت داشته باشد. فرض کنید $BA=I$ و همچنین $AC=I$. آنگاه $B=C$، طبق این «اثبات با پرانتز»:
	\[ B(AC) = (BA)C \implies BI = IC \implies B = C \quad (۲) \]
	این نشان می‌دهد که یک \textbf{معکوس چپ} $B$ (که از سمت چپ ضرب می‌شود) و یک \textbf{معکوس راست} $C$ (که از سمت راست در $A$ ضرب می‌شود تا $AC=I$ را بدهد) باید ماتریس یکسانی باشند.
	
	\textbf{نکته ۳}
	اگر $A$ معکوس‌پذیر باشد، تنها جواب برای $A\mathbf{x}=\mathbf{b}$ برابر است با $\mathbf{x}=A^{-1}\mathbf{b}$.
	
	\textbf{نکته ۴ (مهم)}
	فرض کنید بردار غیرصفری مانند $\mathbf{x}$ وجود داشته باشد که $A\mathbf{x}=\mathbf{0}$. آنگاه $A$ نمی‌تواند معکوس داشته باشد. هیچ ماتریسی نمی‌تواند $\mathbf{0}$ را به $\mathbf{x}$ برگرداند.
	\begin{quote}
		اگر $A$ معکوس‌پذیر باشد، آنگاه $A\mathbf{x}=\mathbf{0}$ فقط می‌تواند جواب صفر $\mathbf{x}=A^{-1}\mathbf{0}=\mathbf{0}$ را داشته باشد.
	\end{quote}
	
	\textbf{نکته ۵}
	یک ماتریس ۲ در ۲ معکوس‌پذیر است اگر و تنها اگر $ad-bc$ صفر نباشد:
	\begin{quote}
		\textbf{معکوس ماتریس ۲ در ۲:}
		\[
		\begin{bmatrix} a & b \\ c & d \end{bmatrix}^{-1} = \frac{1}{ad-bc} \begin{bmatrix} d & -b \\ -c & a \end{bmatrix} \quad (۳)
		\]
	\end{quote}
	این عدد $ad-bc$ \textbf{دترمینان} ماتریس $A$ است. یک ماتریس معکوس‌پذیر است اگر دترمینان آن صفر نباشد (این مفهوم در فصل ۵ به تفصیل بررسی خواهد شد). آزمون وجود $n$ لولا معمولاً قبل از ظهور دترمینان مشخص می‌شود.
	
	\textbf{نکته ۶}
	یک ماتریس قطری معکوس دارد به شرطی که هیچ درایه قطری آن صفر نباشد:
	\[
	\text{اگر } A = \begin{bmatrix} d_1 & & \\ & \ddots & \\ & & d_n \end{bmatrix} \quad \text{آنگاه } \quad A^{-1} = \begin{bmatrix} 1/d_1 & & \\ & \ddots & \\ & & 1/d_n \end{bmatrix}
	\]
	
	\subsubsection*{مثال ۱}
	ماتریس ۲ در ۲ $A=\begin{bmatrix} 1 & 2 \\ 1 & 2 \end{bmatrix}$ معکوس‌پذیر نیست. این ماتریس آزمون نکته ۵ را رد می‌کند، زیرا $ad-bc$ برابر $2-2=0$ است. آزمون نکته ۴ را رد می‌کند، زیرا $A\mathbf{x}=\mathbf{0}$ برای $\mathbf{x}=(2, -1)^T$. و آزمون نکته ۱ را رد می‌کند چون دو لولا ندارد. حذف، سطر دوم این ماتریس را به یک سطر صفر تبدیل می‌کند.
	
	\subsection*{معکوس حاصل‌ضرب AB}
	برای دو عدد غیرصفر $a$ و $b$، مجموع $a+b$ ممکن است معکوس‌پذیر باشد یا نباشد. اعداد $a=3$ و $b=-3$ معکوس‌های $1/3$ و $-1/3$ را دارند. مجموع آن‌ها $a+b=0$ معکوسی ندارد. اما حاصل‌ضرب آن‌ها $ab=-9$ معکوس دارد.
	
	برای دو ماتریس $A$ و $B$ وضعیت مشابه است. گفتن چیزی در مورد معکوس‌پذیری $A+B$ دشوار است. اما حاصل‌ضرب $AB$ معکوس دارد، اگر و تنها اگر دو عامل $A$ و $B$ به طور جداگانه معکوس‌پذیر باشند (و هم‌اندازه باشند). نکته مهم این است که $A^{-1}$ و $B^{-1}$ به ترتیب معکوس ظاهر می‌شوند:
	\begin{quote}
		اگر $A$ و $B$ معکوس‌پذیر باشند، آنگاه $AB$ نیز معکوس‌پذیر است. معکوس حاصل‌ضرب $AB$ عبارت است از:
		\[ (AB)^{-1} = B^{-1}A^{-1} \quad (۴) \]
	\end{quote}
	برای اینکه ببینیم چرا ترتیب معکوس می‌شود، $AB$ را در $B^{-1}A^{-1}$ ضرب می‌کنیم. در داخل آن $BB^{-1}=I$ وجود دارد:
	\[ (AB)(B^{-1}A^{-1}) = A(BB^{-1})A^{-1} = A(I)A^{-1} = AA^{-1} = I \]
	ما پرانتزها را جابجا کردیم تا ابتدا $BB^{-1}$ را ضرب کنیم. به طور مشابه، ضرب $B^{-1}A^{-1}$ در $AB$ برابر با $I$ می‌شود.
	
	$B^{-1}A^{-1}$ یک قانون اساسی ریاضیات را نشان می‌دهد: \textbf{معکوس‌ها به ترتیب معکوس می‌آیند}.
	این موضوع منطقی هم هست: اگر اول جوراب و بعد کفش بپوشید، اولین چیزی که باید درآورید کفش است. همین ترتیب معکوس برای سه ماتریس یا بیشتر نیز اعمال می‌شود:
	\[ \text{ترتیب معکوس} \quad (ABC)^{-1} = C^{-1}B^{-1}A^{-1} \quad (۵) \]
	
	\subsubsection*{مثال ۲}
	معکوس یک ماتریس حذف. اگر $E$ پنج برابر سطر ۱ را از سطر ۲ کم کند، آنگاه $E^{-1}$ پنج برابر سطر ۱ را به سطر ۲ اضافه می‌کند:
	\[ E = \begin{bmatrix} 1 & 0 & 0 \\ -5 & 1 & 0 \\ 0 & 0 & 1 \end{bmatrix} \quad \text{و} \quad E^{-1} = \begin{bmatrix} 1 & 0 & 0 \\ 5 & 1 & 0 \\ 0 & 0 & 1 \end{bmatrix} \]
	$EE^{-1}$ را ضرب کنید تا ماتریس همانی $I$ به دست آید. همچنین $E^{-1}E$ را ضرب کنید تا $I$ به دست آید. برای ماتریس‌های مربع، اگر $AC=I$ باشد، آنگاه به طور خودکار $CA=I$ نیز برقرار است.
	
	\subsubsection*{مثال ۳}
	فرض کنید $F$ چهار برابر سطر ۲ را از سطر ۳ کم می‌کند و $F^{-1}$ آن را دوباره اضافه می‌کند. حالا $F$ را در ماتریس $E$ مثال ۲ ضرب کنید تا $FE$ به دست آید. همچنین $E^{-1}$ را در $F^{-1}$ ضرب کنید تا $(FE)^{-1}$ پیدا شود. به ترتیب‌های $FE$ و $E^{-1}F^{-1}$ توجه کنید!
	\[
	FE = \begin{bmatrix} 1 & 0 & 0 \\ 0 & 1 & 0 \\ 0 & -4 & 1 \end{bmatrix}
	\begin{bmatrix} 1 & 0 & 0 \\ -5 & 1 & 0 \\ 0 & 0 & 1 \end{bmatrix}
	= \begin{bmatrix} 1 & 0 & 0 \\ -5 & 1 & 0 \\ 20 & -4 & 1 \end{bmatrix}
	\]
	\[
	E^{-1}F^{-1} = \begin{bmatrix} 1 & 0 & 0 \\ 5 & 1 & 0 \\ 0 & 0 & 1 \end{bmatrix}
	\begin{bmatrix} 1 & 0 & 0 \\ 0 & 1 & 0 \\ 0 & 4 & 1 \end{bmatrix}
	= \begin{bmatrix} 1 & 0 & 0 \\ 5 & 1 & 0 \\ 0 & 4 & 1 \end{bmatrix}
	\]
	نتیجه زیبا و درست است. حاصل‌ضرب $FE$ شامل «۲۰» است اما معکوس آن اینطور نیست. $E$ پنج برابر سطر ۱ را از سطر ۲ کم می‌کند. سپس $F$ چهار برابر سطر ۲ جدید را (که توسط سطر ۱ تغییر کرده) از سطر ۳ کم می‌کند. در این ترتیب $FE$، سطر ۳ از سطر ۱ تأثیر می‌پذیرد.
	
	در ترتیب $E^{-1}F^{-1}$، این اثر رخ نمی‌دهد. ابتدا $F^{-1}$ چهار برابر سطر ۲ را به سطر ۳ اضافه می‌کند. پس از آن، $E^{-1}$ پنج برابر سطر ۱ را به سطر ۲ اضافه می‌کند. هیچ ۲۰ وجود ندارد، زیرا سطر ۳ دوباره تغییر نمی‌کند. به همین دلیل است که بخش بعدی $A=LU$ را انتخاب می‌کند تا از ماتریس مثلثی $U$ به $A$ بازگردد. مضرب‌ها کاملاً در جای خود در ماتریس پایین مثلثی $L$ قرار می‌گیرند.
	
	\subsection*{محاسبه $A^{-1}$ با حذف گاوس-جردن}
	اشاره کردم که ممکن است $A^{-1}$ به صراحت مورد نیاز نباشد. معادله $A\mathbf{x}=\mathbf{b}$ با $\mathbf{x}=A^{-1}\mathbf{b}$ حل می‌شود. اما محاسبه $A^{-1}$ و ضرب آن در $\mathbf{b}$ نه ضروری و نه کارآمد است. حذف مستقیماً به $\mathbf{x}$ می‌رسد. و حذف همچنین روشی برای محاسبه $A^{-1}$ است، همانطور که اکنون نشان می‌دهیم. ایده گاوس-جردن حل $AA^{-1}=I$ و یافتن هر ستون از $A^{-1}$ است.
	
	\textit{(توضیح مترجم: روش گاوس-جردن در واقع مانند حل همزمان $n$ دستگاه معادله است. هر دستگاه به شکل $A\mathbf{x}_i=\mathbf{e}_i$ است که $\mathbf{x}_i$ ستون $i$-ام ماتریس معکوس و $\mathbf{e}_i$ ستون $i$-ام ماتریس همانی است. با قرار دادن تمام ستون‌های $\mathbf{e}_i$ در کنار ماتریس $A$ به صورت $[A|I]$، ما تمام این دستگاه‌ها را به یکباره حل می‌کنیم.)}
	
	$A$ ستون اول $A^{-1}$ (آن را $\mathbf{x}_1$ بنامید) را ضرب می‌کند تا ستون اول $I$ (آن را $\mathbf{e}_1$ بنامید) را بدهد. این معادله ما $A\mathbf{x}_1=\mathbf{e}_1=(1,0,0)^T$ است. دو معادله دیگر نیز وجود خواهد داشت. هر یک از ستون‌های $\mathbf{x}_1, \mathbf{x}_2, \mathbf{x}_3$ از $A^{-1}$ در $A$ ضرب می‌شود تا یک ستون از $I$ تولید شود:
	\[ A A^{-1} = A \begin{bmatrix} \mathbf{x}_1 & \mathbf{x}_2 & \mathbf{x}_3 \end{bmatrix} = \begin{bmatrix} \mathbf{e}_1 & \mathbf{e}_2 & \mathbf{e}_3 \end{bmatrix} = I \quad (۷) \]
	
	معمولاً «ماتریس الحاقی» $[A \ \mathbf{b}]$ یک ستون اضافی $\mathbf{b}$ دارد. اکنون ما سه سمت راست $\mathbf{e}_1, \mathbf{e}_2, \mathbf{e}_3$ داریم (وقتی $A$ ماتریس ۳ در ۳ است). آن‌ها ستون‌های $I$ هستند، بنابراین ماتریس الحاقی در واقع ماتریس قطعه‌ای $[A \ I]$ است. از این فرصت برای معکوس کردن ماتریس مورد علاقه‌ام، $K$، با ۲ روی قطر اصلی و ۱- در کنار ۲ها استفاده می‌کنم:
	\[
	\left[ \begin{array}{ccc|ccc}
		2 & -1 & 0 & 1 & 0 & 0 \\
		-1 & 2 & -1 & 0 & 1 & 0 \\
		0 & -1 & 2 & 0 & 0 & 1
	\end{array} \right] \xrightarrow{(\frac{1}{2}R_1)+R_2}
	\left[ \begin{array}{ccc|ccc}
		2 & -1 & 0 & 1 & 0 & 0 \\
		0 & 3/2 & -1 & 1/2 & 1 & 0 \\
		0 & -1 & 2 & 0 & 0 & 1
	\end{array} \right] \xrightarrow{(\frac{2}{3}R_2)+R_3}
	\left[ \begin{array}{ccc|ccc}
		2 & -1 & 0 & 1 & 0 & 0 \\
		0 & 3/2 & -1 & 1/2 & 1 & 0 \\
		0 & 0 & 4/3 & 1/3 & 2/3 & 1
	\end{array} \right]
	\]
	ما در نیمه راه رسیدن به $K^{-1}$ هستیم. ماتریس در سه ستون اول $U$ (بالا مثلثی) است. لولاها $2, 3/2, 4/3$ روی قطر آن هستند. گاوس با جایگذاری پس‌رو کار را تمام می‌کرد. سهم جردن ادامه دادن با حذف است! او تا انتها به فرم پلکانی کاهش‌یافته $R=I$ می‌رود. سطرها به سطرهای بالای خود اضافه می‌شوند تا در بالای لولاها صفر ایجاد شود:
	\[
	\xrightarrow{\dots} \left[ \begin{array}{ccc|ccc}
		2 & 0 & 0 & 3/2 & 1 & 1/2 \\
		0 & 3/2 & 0 & 3/4 & 3/2 & 3/4 \\
		0 & 0 & 4/3 & 1/3 & 2/3 & 1
	\end{array} \right]
	\]
	گام نهایی گاوس-جردن تقسیم هر سطر بر لولای آن است. لولاهای جدید همه ۱ می‌شوند. ما به $I$ در نیمه اول ماتریس رسیده‌ایم، زیرا $K$ معکوس‌پذیر است. سه ستون $K^{-1}$ در نیمه دوم $[I \ K^{-1}]$ قرار دارند:
	\[
	\left[ \begin{array}{ccc|ccc}
		1 & 0 & 0 & 3/4 & 1/2 & 1/4 \\
		0 & 1 & 0 & 1/2 & 1 & 1/2 \\
		0 & 0 & 1 & 1/4 & 1/2 & 3/4
	\end{array} \right] = [I \ \mathbf{x}_1 \ \mathbf{x}_2 \ \mathbf{x}_3] = [I \ K^{-1}]
	\]
	با شروع از ماتریس ۳ در ۶ $[K \ I]$، به $[I \ K^{-1}]$ رسیدیم. کل فرآیند گاوس-جردن برای هر ماتریس معکوس‌پذیر $A$ در یک خط به این صورت است:
	\begin{quote}
		\textbf{گاوس-جردن:} با انجام عملیات سطری روی $[A \ I]$، آن را به $[I \ A^{-1}]$ تبدیل کنید.
	\end{quote}
	
	\subsection*{ماتریس منفرد در مقابل ماتریس معکوس‌پذیر}
	
	به سوال اصلی باز می‌گردیم. کدام ماتریس‌ها معکوس دارند؟ در ابتدای این بخش آزمون لولا پیشنهاد شد: $A^{-1}$ دقیقاً زمانی وجود دارد که $A$ مجموعه کاملی از $n$ لولا داشته باشد. (جابجایی سطرها مجاز است.) اکنون می‌توانیم این را با حذف گاوس-جردن اثبات کنیم:
	
	\begin{enumerate}
		\item با $n$ لولا، حذف تمام معادلات $A\mathbf{x}_i=\mathbf{e}_i$ را حل می‌کند. ستون‌های $\mathbf{x}_i$ در $A^{-1}$ قرار می‌گیرند. آنگاه $AA^{-1}=I$ و $A^{-1}$ حداقل یک معکوس راست است.
		\item حذف در واقع دنباله‌ای از ضرب‌ها در ماتریس‌های $E$, $P$ و $D^{-1}$ است:
		\[ \text{معکوس چپ } C = (D^{-1} \cdots E \cdots P \cdots E) \implies CA=I \]
		این استدلال نشان می‌دهد که اگر $AC=I$ باشد، آنگاه $CA=I$ و $C=A^{-1}$.
	\end{enumerate}
	
	\textit{(توضیح مترجم: تمام شرایط معکوس‌پذیری (داشتن $n$ لولا، دترمینان غیرصفر، نداشتن جواب غیرصفر برای $A\mathbf{x}=\mathbf{0}$) به یکدیگر وابسته‌اند. آنها جنبه‌های مختلفی از یک مفهوم واحد را توصیف می‌کنند: اینکه آیا تبدیل خطی نمایش داده شده توسط ماتریس $A$ اطلاعاتی را از بین می‌برد یا خیر. اگر اطلاعات از بین برود (مثلاً یک بردار غیرصفر را به صفر تصویر کند)، ماتریس منفرد است و نمی‌توان آن تبدیل را معکوس کرد.)}
	
	\subsection*{شناخت یک ماتریس معکوس‌پذیر}
	
	معمولاً برای تصمیم‌گیری در مورد معکوس‌پذیر بودن یک ماتریس به کار نیاز است. اما برای برخی ماتریس‌ها می‌توانید به سرعت ببینید که معکوس‌پذیر هستند زیرا هر عدد $a_{ii}$ روی قطر اصلی آن‌ها بر بقیه قسمت‌های آن سطر غلبه دارد.
	
	\begin{quote}
		\textbf{ماتریس‌های قطری غالب (Diagonally dominant)} معکوس‌پذیر هستند. در هر سطر $i$:
		\[ |a_{ii}| > \sum_{j \neq i} |a_{ij}| \]
	\end{quote}
	این شرط نشان می‌دهد که $A\mathbf{x}=\mathbf{0}$ تنها زمانی ممکن است که $\mathbf{x}=\mathbf{0}$. بنابراین $A$ معکوس‌پذیر است.
	
	\subsection*{مروری بر ایده‌های کلیدی}
	\begin{enumerate}
		\item ماتریس معکوس روابط $AA^{-1}=I$ و $A^{-1}A=I$ را برقرار می‌کند.
		\item $A$ معکوس‌پذیر است اگر و تنها اگر $n$ لولا داشته باشد (جابجایی سطرها مجاز است).
		\item \textbf{مهم:} اگر $A\mathbf{x}=\mathbf{0}$ برای یک بردار غیرصفر $\mathbf{x}$ برقرار باشد، آنگاه $A$ معکوس ندارد.
		\item معکوس $AB$ حاصل‌ضرب معکوس $B^{-1}A^{-1}$ است. و $(ABC)^{-1} = C^{-1}B^{-1}A^{-1}$.
		\item روش گاوس-جردن $AA^{-1}=I$ را حل می‌کند تا $n$ ستون $A^{-1}$ را پیدا کند. ماتریس الحاقی $[A \ I]$ به $[I \ A^{-1}]$ کاهش سطری می‌یابد.
		\item ماتریس‌های قطری غالب معکوس‌پذیر هستند. هر $|a_{ii}|$ بر سطر خود غلبه دارد.
	\end{enumerate}
	
	
	\newpage
	\section*{مجموعه مسائل ۲.۵}
	\subsection*{مسائل ۱-۴۴}
	\begin{enumerate}
		\item \textbf{(مسائل ۱-۲۱ درباره ویژگی‌های معکوس هستند.)}\\
		معکوس ماتریس‌های $A, B, C$ را پیدا کنید (مستقیماً یا از فرمول ۲ در ۲):
		\[ A = \begin{bmatrix} 0 & 1/3 \\ 1/4 & 0 \end{bmatrix}, \quad B = \begin{bmatrix} 2 & 0 \\ -1 & 1/2 \end{bmatrix}, \quad C = \begin{bmatrix} 3 & 4 \\ 5 & 7 \end{bmatrix} \]
		
		\item برای این «ماتریس‌های جایگشت» $P^{-1}$ را با آزمون و خطا پیدا کنید (با ۱ و ۰). (راهنمایی: $P^{-1}$ همان ترانهاده $P$ است.)
		\[ P = \begin{bmatrix} 0 & 0 & 1 \\ 0 & 1 & 0 \\ 1 & 0 & 0 \end{bmatrix} \quad \text{و} \quad P = \begin{bmatrix} 0 & 1 & 0 \\ 0 & 0 & 1 \\ 1 & 0 & 0 \end{bmatrix} \]
		
		\item برای ستون اول $(x,y)$ و ستون دوم $(t,z)$ از $A^{-1}$ حل کنید (این ایده اصلی گاوس-جردن است):
		\[ \begin{bmatrix} 10 & 20 \\ 20 & 50 \end{bmatrix} \begin{bmatrix} x \\ y \end{bmatrix} = \begin{bmatrix} 1 \\ 0 \end{bmatrix} \quad \text{و} \quad \begin{bmatrix} 10 & 20 \\ 20 & 50 \end{bmatrix} \begin{bmatrix} t \\ z \end{bmatrix} = \begin{bmatrix} 0 \\ 1 \end{bmatrix} \]
		
		\item نشان دهید که $\begin{bmatrix} 1 & 2 \\ 3 & 6 \end{bmatrix}$ معکوس‌پذیر نیست با تلاش برای حل $A\mathbf{x}_1=(1,0)^T$.
		
		\item یک ماتریس بالا مثلثی $U$ (غیرقطری) با $U^2=I$ پیدا کنید که $U=U^{-1}$ را نتیجه دهد.
		
		\item (الف) اگر $A$ معکوس‌پذیر باشد و $AB=AC$، به سرعت ثابت کنید که $B=C$.
		(ب) اگر $A = \begin{bmatrix} 1 & 1 \\ 1 & 1 \end{bmatrix}$، دو ماتریس متفاوت $B, C$ پیدا کنید که $AB=AC$.
		
		\item (مهم) اگر $A$ دارای (سطر ۱ + سطر ۲ = سطر ۳) باشد، نشان دهید که $A$ معکوس‌پذیر نیست:
		\begin{itemize}
			\item[(الف)] توضیح دهید چرا $A\mathbf{x}=(1,0,0)$ نمی‌تواند جوابی داشته باشد. (راهنمایی: معادله ۱ + معادله ۲ - معادله ۳ را در نظر بگیرید)
			\item[(ب)] کدام سمت راست‌ها $\mathbf{b}=(b_1, b_2, b_3)^T$ ممکن است به $A\mathbf{x}=\mathbf{b}$ اجازه جواب دهند؟
			\item[(ج)] در حذف، چه اتفاقی برای معادله ۳ می‌افتد؟
		\end{itemize}
		
		\item اگر $A$ دارای (ستون ۱ + ستون ۲ = ستون ۳) باشد، نشان دهید که $A$ معکوس‌پذیر نیست:
		\begin{itemize}
			\item[(الف)] یک جواب غیرصفر $\mathbf{x}$ برای $A\mathbf{x}=\mathbf{0}$ پیدا کنید. ماتریس ۳ در ۳ است.
			\item[(ب)] حذف، (ستون ۱ + ستون ۲ = ستون ۳) را حفظ می‌کند. توضیح دهید چرا لولای سومی وجود ندارد.
		\end{itemize}
		
		\item فرض کنید $A$ معکوس‌پذیر است و شما دو سطر اول آن را جابجا می‌کنید تا به $B$ برسید. آیا ماتریس جدید $B$ معکوس‌پذیر است؟ چگونه $B^{-1}$ را از $A^{-1}$ پیدا می‌کنید؟ (راهنمایی: $B=PA$)
		
		\item معکوس ماتریس‌های قطعه‌ای زیر را پیدا کنید:
		\[ A = \begin{bmatrix} 2 & 0 & 0 & 0 \\ 0 & 3 & 0 & 0 \\ 0 & 0 & 4 & 0 \\ 0 & 0 & 0 & 5 \end{bmatrix} \quad B = \begin{bmatrix} 3 & 2 & 0 & 0 \\ 4 & 3 & 0 & 0 \\ 0 & 0 & 6 & 5 \\ 0 & 0 & 7 & 6 \end{bmatrix} \]
		
		\item (الف) ماتریس‌های معکوس‌پذیر $A$ و $B$ را پیدا کنید به طوری که $A+B$ معکوس‌پذیر نباشد.
		(ب) ماتریس‌های منفرد $A$ و $B$ را پیدا کنید به طوری که $A+B$ معکوس‌پذیر باشد.
		
		\item اگر حاصل‌ضرب $C=AB$ معکوس‌پذیر باشد ($A$ و $B$ مربع هستند)، آنگاه خود $A$ نیز معکوس‌پذیر است. فرمولی برای $A^{-1}$ که شامل $C^{-1}$ و $B$ است، پیدا کنید.
		
		\item اگر حاصل‌ضرب $M=ABC$ از سه ماتریس مربع معکوس‌پذیر باشد، آنگاه $B$ معکوس‌پذیر است. (همچنین $A$ و $C$.) فرمولی برای $B^{-1}$ پیدا کنید که شامل $M^{-1}$ و $A$ و $C$ باشد.
		
		\item اگر سطر ۱ از $A$ را به سطر ۲ اضافه کنید تا $B$ به دست آید ($B=EA$)، چگونه $B^{-1}$ را از $A^{-1}$ پیدا می‌کنید؟
		
		\item ثابت کنید ماتریسی با یک ستون صفر نمی‌تواند معکوس داشته باشد.
		
		\item $\begin{bmatrix} a & b \\ c & d \end{bmatrix}$ را در $\begin{bmatrix} d & -b \\ -c & a \end{bmatrix}$ ضرب کنید. معکوس هر ماتریس چیست اگر $ad \neq bc$؟
		
		\item (الف) چه ماتریس ۳ در ۳ به نام $E$ همان تأثیر این سه مرحله را دارد؟ سطر ۱ را از سطر ۲ کم کنید، سطر ۱ را از سطر ۳ کم کنید، سپس سطر ۲ را از سطر ۳ کم کنید.
		(ب) چه ماتریس واحدی به نام $L$ همان تأثیر این سه مرحله معکوس را دارد؟ سطر ۲ را به سطر ۳ اضافه کنید، سطر ۱ را به سطر ۳ اضافه کنید، سپس سطر ۱ را به سطر ۲ اضافه کنید.
		
		\item اگر $B$ معکوس $A^2$ باشد ($A^2B=I$)، نشان دهید که $AB$ معکوس $A$ است.
		
		\item اعداد $a$ و $b$ را پیدا کنید که معکوس ماتریس $n \times n$ زیر را در فرم $aI+bJ$ بدهد (که $J$ ماتریس تمام یک است):
		\[ ( (n-1)I - J )^{-1} = aI + bJ \quad (\text{برای } n=5 \text{ در سوال اصلی}) \]
		
		\item نشان دهید که $A= (n-1)I - J$ (برای اندازه $n=4$) معکوس‌پذیر نیست. (راهنمایی: بردار تمام یک را در آن ضرب کنید.)
		
		\item شانزده ماتریس ۲ در ۲ وجود دارد که درایه‌های آنها ۱ و ۰ است. چند تای آنها معکوس‌پذیر هستند؟
		
		\item \textbf{(مسائل ۲۲-۲۸ در مورد روش گاوس-جردن برای محاسبه $A^{-1}$ هستند.)}\\
		$I$ را به $A^{-1}$ تبدیل کنید همانطور که $A$ را به $I$ کاهش می‌دهید (با عملیات سطری):
		\[ [A \ I] = \left[ \begin{array}{cc|cc} 1 & 3 & 1 & 0 \\ 2 & 7 & 0 & 1 \end{array} \right] \quad \text{و} \quad [A \ I] = \left[ \begin{array}{cc|cc} 1 & 4 & 1 & 0 \\ 3 & 9 & 0 & 1 \end{array} \right] \]
		
		\item مثال ۳ در ۳ متن را با علامت‌های مثبت در $A$ دنبال کنید. بالا و پایین لولاها را حذف کنید تا $[A \ I]$ به $[I \ A^{-1}]$ کاهش یابد:
		\[ [A \ I] = \left[ \begin{array}{ccc|ccc} 2 & 1 & 0 & 1 & 0 & 0 \\ 1 & 2 & 1 & 0 & 1 & 0 \\ 0 & 1 & 2 & 0 & 0 & 1 \end{array} \right] \]
		
		\item از حذف گاوس-جردن روی $[U \ I]$ استفاده کنید تا معکوس بالا مثلثی $U^{-1}$ را پیدا کنید:
		\[ UU^{-1}=I \implies \begin{bmatrix} 1 & a & b \\ 0 & 1 & c \\ 0 & 0 & 1 \end{bmatrix} [U^{-1}] = \begin{bmatrix} 1 & 0 & 0 \\ 0 & 1 & 0 \\ 0 & 0 & 1 \end{bmatrix} \]
		
		\item $A^{-1}$ و $B^{-1}$ را (در صورت وجود) با حذف روی $[A \ I]$ و $[B \ I]$ پیدا کنید:
		\[ A = \begin{bmatrix} 2 & 1 & 1 \\ 1 & 2 & 1 \\ 1 & 1 & 2 \end{bmatrix} \quad \text{و} \quad B = \begin{bmatrix} 2 & -1 & -1 \\ -1 & 2 & -1 \\ -1 & -1 & 2 \end{bmatrix} \]
		
		\item چه سه ماتریس $E_{21}$ و $E_{12}$ و $D^{-1}$ ماتریس $A=\begin{bmatrix} 1 & 2 \\ 2 & 6 \end{bmatrix}$ را به ماتریس همانی کاهش می‌دهند؟ $A^{-1}=D^{-1}E_{12}E_{21}$ را محاسبه کنید.
		
		\item این ماتریس‌ها را با روش گاوس-جردن با شروع از $[A \ I]$ معکوس کنید:
		\[ A = \begin{bmatrix} 1 & 0 & 0 \\ 2 & 1 & 3 \\ 0 & 0 & 1 \end{bmatrix} \quad \text{و} \quad A = \begin{bmatrix} 1 & 1 & 1 \\ 1 & 2 & 2 \\ 1 & 2 & 3 \end{bmatrix} \]
		
		\item سطرها را جابجا کرده و با گاوس-جردن ادامه دهید تا $A^{-1}$ را پیدا کنید:
		\[ [A \ I] = \left[ \begin{array}{cc|cc} 0 & 2 & 1 & 0 \\ 2 & 2 & 0 & 1 \end{array} \right] \]
		
		\item \textbf{(مسائل ۲۹-۳۸ مسائل مفهومی درباره معکوس‌پذیری هستند.)}\\
		درست یا غلط (با یک مثال نقض اگر غلط و یک دلیل اگر درست است):
		\begin{itemize}
			\item[(الف)] یک ماتریس ۴ در ۴ با یک سطر صفر معکوس‌پذیر نیست.
			\item[(ب)] هر ماتریسی با ۱ روی قطر اصلی معکوس‌پذیر است.
			\item[(ج)] اگر $A$ معکوس‌پذیر باشد، آنگاه $A^{-1}$ و $A^2$ نیز معکوس‌پذیر هستند.
		\end{itemize}
		
		\item (توصیه شده) ثابت کنید که $A$ معکوس‌پذیر است اگر $a \neq 0$ و $a \neq b$. سپس مقادیر $c$ را طوری پیدا کنید که $C$ معکوس‌پذیر نباشد:
		\[ A = \begin{bmatrix} a & b & b \\ a & a & b \\ a & a & a \end{bmatrix}, \quad C = \begin{bmatrix} c & c & c \\ c & c & c \\ c & c & c \end{bmatrix} \]
		
		\item این ماتریس یک معکوس قابل توجه دارد. $A^{-1}$ را با حذف روی $[A \ I]$ پیدا کنید. آن را حل کرده و برای ماتریس ۵ در ۵ مشابهی حدس بزنید.
		\[ A = \begin{bmatrix} 1 & -1 & 1 & -1 \\ 0 & 1 & -1 & 1 \\ 0 & 0 & 1 & -1 \\ 0 & 0 & 0 & 1 \end{bmatrix}, \quad \text{معادله } A\mathbf{x}=(1, 1, 1, 1)^T \text{ را حل کنید.} \]
		
		\item فرض کنید ماتریس‌های $P$ و $Q$ همان سطرهای $I$ را اما به هر ترتیبی دارند. آنها «ماتریس‌های جایگشت» هستند. نشان دهید که $P-Q$ با حل $(P-Q)\mathbf{x}=\mathbf{0}$ برای یک $\mathbf{x}$ غیرصفر، منفرد است.
		
		\item معکوس این ماتریس‌های قطعه‌ای را پیدا و بررسی کنید (با فرض وجود):
		\[ \begin{bmatrix} I & 0 \\ C & I \end{bmatrix}, \quad \begin{bmatrix} A & 0 \\ C & D \end{bmatrix}, \quad \begin{bmatrix} 0 & I \\ I & D \end{bmatrix} \]
		
		\item آیا یک ماتریس ۴ در ۴ $A$ می‌تواند معکوس‌پذیر باشد اگر هر سطر شامل اعداد ۰, ۱, ۲, ۳ به ترتیبی باشد؟ چه می‌شود اگر هر سطر از $B$ شامل ۰, ۱, ۲, ۳- به ترتیبی باشد؟ (راهنمایی: مجموع درایه‌های هر سطر را بررسی کنید.)
		
		\item در مثال حل شده ۲.۵ ج، ماتریس مثلثی پاسکال $L$ دارای $L^{-1}=DLD$ است، که در آن ماتریس قطری $D$ درایه‌های متناوب ۱, ۱-, ۱, ۱- دارد. از این رابطه نتیجه بگیرید که $(LD)^{-1}=LD$ است.
		
		\item ماتریس‌های هیلبرت دارای $H_{ij} = 1/(i+j-1)$ هستند. از متلب معکوس دقیق ۶ در ۶ `invhilb(6)` را بخواهید. سپس از آن بخواهید `inv(hilb(6))` را محاسبه کند. چگونه این دو می‌توانند متفاوت باشند، در حالی که کامپیوتر هرگز اشتباه نمی‌کند؟
		
		\item (الف) از `inv(P)` برای معکوس کردن ماتریس متقارن ۴ در ۴ متلب $P=\text{pascal}(4)$ استفاده کنید.
		(ب) ماتریس پایین مثلثی پاسکال $L = \text{abs}(\text{pascal}(4,1))$ را ایجاد کرده و $P = LL^T$ را آزمایش کنید.
		
		\item اگر $A=\text{ones}(4,4)$ (ماتریس تمام یک) و $\mathbf{b}=\text{rand}(4,1)$، متلب چگونه به شما می‌گوید که $A\mathbf{x}=\mathbf{b}$ جوابی ندارد؟ برای $\mathbf{b}$ خاص $\mathbf{b}=\text{ones}(4,1)$، کدام جواب برای $A\mathbf{x}=\mathbf{b}$ توسط `A\b` پیدا می‌شود؟
		
		\item \textbf{(مسائل چالشی)}\\
		$A$ یک ماتریس ۴ در ۴ دو قطری با ۱ روی قطر و a-, b-, c- روی ابرقطر است. $A^{-1}$ را پیدا کنید.
		\[ A = \begin{bmatrix} 1 & -a & 0 & 0 \\ 0 & 1 & -b & 0 \\ 0 & 0 & 1 & -c \\ 0 & 0 & 0 & 1 \end{bmatrix} \]
		
		\item فرض کنید $E_1, E_2, E_3$ ماتریس‌های همانی ۴ در ۴ هستند، با این تفاوت که $E_1$ دارای $a,b,c$ در ستون ۱، $E_2$ دارای $d,e$ در ستون ۲ و $E_3$ دارای $f$ در ستون ۳ (همگی زیر قطر) است. $L=E_1E_2E_3$ را ضرب کنید تا نشان دهید همه این غیرصفرها در $L$ کپی می‌شوند.
		
		\item ماتریس‌های تفاضل دوم اگر با $T_{11}=1$ شروع شوند (به جای $K_{11}=2$) معکوس‌های زیبایی دارند. ماتریس ۴ در ۴ $T$ و معکوس آن را در نظر بگیرید:
		\[ T = \begin{bmatrix} 1 & -1 & 0 & 0 \\ -1 & 2 & -1 & 0 \\ 0 & -1 & 2 & -1 \\ 0 & 0 & -1 & 2 \end{bmatrix}, \quad T^{-1} = \begin{bmatrix} 4 & 3 & 2 & 1 \\ 3 & 3 & 2 & 1 \\ 2 & 2 & 2 & 1 \\ 1 & 1 & 1 & 1 \end{bmatrix} \]
		سوال: لولاهای $T$ چیست؟ معکوس آن چگونه به دست می‌آید؟
		
		\item ترتیب معکوس $UL$ (با $T=LU$) چه ماتریسی $T^*$ می‌دهد؟ معکوس $T^*$ چیست؟
		
		\item در اینجا دو ماتریس تفاضل دیگر، هر دو مهم، وجود دارد. آیا آنها معکوس‌پذیر هستند؟ (راهنمایی: مجموع هر سطر را بررسی کنید.)
		\[ C = \begin{bmatrix} 2 & -1 & 0 & -1 \\ -1 & 2 & -1 & 0 \\ 0 & -1 & 2 & -1 \\ -1 & 0 & -1 & 2 \end{bmatrix}, \quad F = \begin{bmatrix} 2 & -1 & 0 & 0 \\ -1 & 2 & -1 & 0 \\ 0 & -1 & 2 & -1 \\ 0 & 0 & -1 & 1 \end{bmatrix} \]
		
		\item حذف برای یک ماتریس قطعه‌ای: وقتی سطر قطعه‌ای اول $[A \ B]$ را در $CA^{-1}$ ضرب کرده و از سطر قطعه‌ای دوم $[C \ D]$ کم می‌کنید، «مکمل شور» $S$ ظاهر می‌شود. $S=D-CA^{-1}B$. لولاهای قطعه‌ای $A$ و $S$ هستند. اگر آنها معکوس‌پذیر باشند، ماتریس قطعه‌ای نیز معکوس‌پذیر است. $S$ را برای ماتریس زیر پیدا کنید:
		\[ \begin{bmatrix} A & B \\ C & D \end{bmatrix} = \left[ \begin{array}{c|cc} 2 & 3 & 3 \\ \hline 4 & 1 & 0 \\ 4 & 0 & 1 \end{array} \right] \]
		
		\item این رابطه $A(I + BA) = (I+ AB)A$ چگونه معکوس‌های $I+BA$ و $I+AB$ را به هم مرتبط می‌کند؟ نشان دهید که اگر $A$ معکوس‌پذیر باشد، یکی از این دو معکوس‌پذیر است اگر و تنها اگر دیگری نیز باشد.
	\end{enumerate}
	
\end{document}