\documentclass[12pt, a4paper]{book}

% فراخوانی بسته‌های لازم
\usepackage{amsmath}         % برای فرمول‌های پیشرفته ریاضی
\usepackage{amsfonts}        % بسته برای فونت‌های ریاضی مانند \mathbb
\usepackage{amssymb}         % برای نمادهای بیشتر ریاضی
\usepackage{graphicx}        % برای افزودن تصاویر
\usepackage{xepersian}       % بسته اصلی برای پارسی‌نویسی
\usepackage{geometry}        % برای تنظیم حاشیه‌ها
\usepackage{setspace}        % برای تنظیم فاصله خطوط
\usepackage{array}           % برای امکانات پیشرفته در جدول‌ها و آرایه‌ها
\usepackage{enumitem}        % برای کنترل بیشتر بر لیست‌ها

% تنظیم حاشیه‌های صفحه
\geometry{
	a4paper,
	total={170mm,257mm},
	left=20mm,
	top=20mm,
}

% تنظیم فونت‌های نوشتاری و ریاضی
% توجه: این فونت‌ها باید روی سیستم شما نصب باشند
\settextfont{XB Niloofar}
\setdigitfont{XB Niloofar}
\setmathdigitfont{XB Niloofar}

\begin{document}
	
	% اعمال فاصله 1.5 بین خطوط برای خوانایی بهتر
	\onehalfspacing
	
	\section*{۲.۶ حذف = تجزیه: A = LU}
	
	\begin{enumerate}
		\item هر گام حذفی $E_{ij}$ با $L_{ij}$ معکوس می‌شود. در خارج از قطر اصلی، $-l_{ij}$ به $+l_{ij}$ تغییر می‌کند.
		\item کل فرآیند حذف پیشرو (بدون تعویض سطر) با $L$ معکوس می‌شود:
		\[ L = (L_{21}L_{31} \cdots L_{n1})(L_{32} \cdots L_{n2})(L_{43} \cdots L_{n3}) \cdots (L_{n,n-1}) \]
		\item ماتریس حاصل‌ضرب $L$ همچنان پایین‌مثلثی است. هر مضرب $l_{ij}$ در سطر $i$ و ستون $j$ قرار دارد.
		\item ماتریس اصلی $A$ از $U$ با رابطه $A = LU$ بازیابی می‌شود = (پایین‌مثلثی) ضربدر (بالامثلثی).
		\item حذف روی $A\mathbf{x} = \mathbf{b}$ به $U\mathbf{x} = \mathbf{c}$ می‌رسد. سپس جایگذاری پسرو (back-substitution) دستگاه $U\mathbf{x} = \mathbf{c}$ را حل می‌کند.
		\item حل یک دستگاه مثلثی به $n^2/2$ عمل ضرب-تفریق نیاز دارد. حذف برای یافتن $U$ حدود $n^3/3$ عمل نیاز دارد.
	\end{enumerate}
	
	دانشجویان اغلب می‌گویند که درس‌های ریاضی بیش از حد نظری هستند. خب، این بخش اینطور نیست. این بخش تقریباً به طور کامل عملی است. هدف، توصیف حذف گاوسی به مفیدترین شکل ممکن است. بسیاری از ایده‌های کلیدی جبر خطی، وقتی با دقت به آن‌ها نگاه می‌کنید، در واقع تجزیه یک ماتریس هستند. ماتریس اصلی $A$ به حاصل‌ضرب دو یا سه ماتریس خاص تبدیل می‌شود. اولین تجزیه - که در عمل نیز مهم‌ترین است - اکنون از فرآیند حذف به دست می‌آید. عامل‌های $L$ و $U$ ماتریس‌های مثلثی هستند. تجزیه‌ای که از حذف به دست می‌آید $A=LU$ است.
	
	ما از قبل با $U$، ماتریس بالامثلثی که لولاها روی قطر اصلی آن قرار دارند، آشنا هستیم. گام‌های حذف، $A$ را به $U$ تبدیل می‌کنند. ما نشان خواهیم داد که چگونه معکوس کردن این گام‌ها (بازگرداندن $U$ به $A$) با یک ماتریس پایین‌مثلثی $L$ انجام می‌شود. درایه‌های $L$ دقیقاً همان مضارب $l_{ij}$ هستند—که سطر لولای $j$ را قبل از تفریق از سطر $i$ ضرب می‌کردند.
	
	با یک مثال ۲ در ۲ شروع می‌کنیم. ماتریس $A$ شامل درایه‌های ۲، ۱، ۶، ۸ است. عددی که باید حذف شود ۶ است. ۳ برابر سطر اول را از سطر دوم کم می‌کنیم. این گام در جهت پیشرو، ماتریس $E_{21}$ با مضرب $l_{21}=3$ است. گام بازگشت از $U$ به $A$ با ماتریس $L=E_{21}^{-1}$ (یک عمل جمع با استفاده از ۳+) انجام می‌شود:
	
	\textbf{پیشرو از $A$ به $U$:}
	\[ E_{21}A = \begin{bmatrix} 1 & 0 \\ -3 & 1 \end{bmatrix} \begin{bmatrix} 2 & 1 \\ 6 & 8 \end{bmatrix} = \begin{bmatrix} 2 & 1 \\ 0 & 5 \end{bmatrix} = U \]
	\textbf{بازگشت از $U$ به $A$:}
	\[ E_{21}^{-1}U = \begin{bmatrix} 1 & 0 \\ 3 & 1 \end{bmatrix} \begin{bmatrix} 2 & 1 \\ 0 & 5 \end{bmatrix} = \begin{bmatrix} 2 & 1 \\ 6 & 8 \end{bmatrix} = A \]
	خط دوم همان تجزیه $LU=A$ است. به جای $E_{21}^{-1}$، ما $L$ می‌نویسیم. حال به سراغ ماتریس‌های بزرگ‌تر با تعداد زیادی ماتریس $E$ می‌رویم. در این صورت $L$ شامل معکوس همه آن‌ها خواهد بود.
	
	هر گام از $A$ به $U$ با یک ماتریس $E_{ij}$ ضرب می‌شود تا درایه $(i,j)$ را صفر کند. برای واضح نگه داشتن مطلب، ما به متداول‌ترین حالت—که هیچ تعویض سطری در کار نیست—می‌پردازیم. اگر $A$ یک ماتریس ۳ در ۳ باشد، ما آن را در $E_{21}$ و $E_{31}$ و $E_{32}$ ضرب می‌کنیم. مضارب $l_{ij}$ درایه‌های (۲,۱)، (۳,۱) و (۳,۲) را صفر می‌کنند—همگی زیر قطر اصلی. حذف با ماتریس بالامثلثی $U$ به پایان می‌رسد.
	
	حالا این ماتریس‌های $E$ را به طرف دیگر معادله منتقل می‌کنیم، جایی که معکوس‌هایشان در $U$ ضرب می‌شوند:
	\[ (E_{32}E_{31}E_{21})A=U \implies A = (E_{21}^{-1}E_{31}^{-1}E_{32}^{-1})U = LU \]
	معکوس‌ها باید به ترتیب مخالف قرار گیرند. حاصل‌ضرب این سه معکوس، $L$ است. ما به $A=LU$ رسیده‌ایم. اکنون برای درک آن توقف می‌کنیم.
	
	\subsection*{توضیح و مثال‌ها}
	\textbf{نکته اول:} هر ماتریس معکوس $E^{-1}$ پایین‌مثلثی است. درایه خارج از قطر آن $l_{ij}$ است تا عمل تفریق حاصل از $-l_{ij}$ را خنثی کند. قطرهای اصلی $E$ و $E^{-1}$ شامل ۱ هستند.
	مثال بالا دارای $l_{21}=3$ و $E = \begin{bmatrix} 1 & 0 \\ -3 & 1 \end{bmatrix}$ و $L = E^{-1} = \begin{bmatrix} 1 & 0 \\ 3 & 1 \end{bmatrix}$ بود.
	
	\textbf{نکته دوم:} معادله بالا یک ماتریس پایین‌مثلثی (حاصل‌ضرب $E_{ij}$ها) را نشان می‌دهد که در $A$ ضرب می‌شود. همچنین نشان می‌دهد که تمام $E_{ij}^{-1}$ها در $U$ ضرب می‌شوند تا $A$ را بازگردانند. این حاصل‌ضرب پایین‌مثلثی از معکوس‌ها، همان $L$ است.
	
	یک دلیل برای کار با معکوس‌ها این است که ما می‌خواهیم $A$ را تجزیه کنیم، نه $U$ را. «فرم معکوس» به ما $A=LU$ می‌دهد. دلیل دیگر این است که ما یک چیز اضافی به دست می‌آوریم، تقریباً بیش از آنچه استحقاقش را داریم. این نکته سوم است که نشان می‌دهد $L$ دقیقاً درست است.
	
	\textbf{نکته سوم:} هر مضرب $l_{ij}$ مستقیماً و بدون تغییر در جایگاه $i,j$ خود در حاصل‌ضرب معکوس‌ها که همان $L$ است، قرار می‌گیرد. معمولاً ضرب ماتریسی همه اعداد را با هم مخلوط می‌کند. در اینجا این اتفاق نمی‌افتد. ترتیب ماتریس‌های معکوس برای ثابت نگه داشتن $l$ها درست است. دلیل آن در ادامه در معادله (۲) آورده شده است.
	
	از آنجا که هر $E^{-1}$ روی قطر اصلی خود ۱ دارد، نکته خوب نهایی این است که $L$ نیز همین ویژگی را دارد.
	
	\subsubsection*{$A=LU$}
	\textbf{مثال ۱}
	این حذف بدون تعویض سطر است. ماتریس بالامثلثی $U$ لولاها را روی قطر اصلی خود دارد. ماتریس پایین‌مثلثی $L$ همگی ۱ روی قطر اصلی خود دارد. مضارب $l_{ij}$ زیر قطر اصلی $L$ قرار دارند.
	
	حذف، $1/2$ برابر سطر ۱ را از سطر ۲ کم می‌کند. گام آخر $2/3$ برابر سطر ۲ را از سطر ۳ کم می‌کند. ماتریس پایین‌مثلثی $L$ دارای $l_{21} = 1/2$ و $l_{32} = 2/3$ است. ضرب $LU$ ماتریس $A$ را تولید می‌کند:
	\[ A = \begin{bmatrix} 2 & 1 & 0 \\ 1 & 2 & 1 \\ 0 & 3 & 4 \end{bmatrix} = \begin{bmatrix} 1 & 0 & 0 \\ 1/2 & 1 & 0 \\ 0 & 2/3 & 1 \end{bmatrix} \begin{bmatrix} 2 & 1 & 0 \\ 0 & 3/2 & 1 \\ 0 & 0 & 10/3 \end{bmatrix} = LU \]
	مضرب (۳,۱) صفر است زیرا درایه (۳,۱) در $A$ صفر است. نیازی به عملیات نیست.
	
	\textbf{مثال ۲} درایه بالا سمت چپ را از ۲ در $A$ به ۱ در $B$ تغییر دهید. همه لولاها ۱ می‌شوند. همه مضارب ۱ هستند. این الگو وقتی $B$ یک ماتریس ۴ در ۴ باشد ادامه می‌یابد:
	\textbf{الگوی خاص}
	\[ B = \begin{bmatrix} 1 & 1 & 1 & 1 \\ 1 & 2 & 2 & 2 \\ 1 & 2 & 3 & 3 \\ 1 & 2 & 3 & 4 \end{bmatrix} = \begin{bmatrix} 1 & 0 & 0 & 0 \\ 1 & 1 & 0 & 0 \\ 1 & 1 & 1 & 0 \\ 1 & 1 & 1 & 1 \end{bmatrix} \begin{bmatrix} 1 & 1 & 1 & 1 \\ 0 & 1 & 1 & 1 \\ 0 & 0 & 1 & 1 \\ 0 & 0 & 0 & 1 \end{bmatrix} = LU \]
	این مثال‌های $LU$ چیز دیگری را نیز نشان می‌دهند که در عمل بسیار مهم است. فرض کنید هیچ تعویض سطری وجود ندارد. چه زمانی می‌توانیم وجود صفرها را در $L$ و $U$ پیش‌بینی کنیم؟
	\begin{quote}
		وقتی یک سطر از $A$ با صفرها شروع می‌شود، همان سطر از $L$ نیز چنین است. \\
		وقتی یک ستون از $A$ با صفرها شروع می‌شود، همان ستون از $U$ نیز چنین است.
	\end{quote}
	اگر یک سطر با صفر شروع شود، به گام حذف نیازی نداریم. $L$ یک صفر خواهد داشت که باعث صرفه‌جویی در زمان محاسبات می‌شود. به طور مشابه، صفرهای ابتدای یک ستون در $U$ باقی می‌مانند. اما لطفاً توجه داشته باشید: صفرها در وسط یک ماتریس به احتمال زیاد در حین پیشرفت حذف، با مقادیر غیرصفر پر می‌شوند.
	
	اکنون توضیح می‌دهیم که چرا $L$ مضارب $l_{ij}$ را بدون هیچ‌گونه درهم‌ریختگی در جایگاه خود دارد.
	دلیل کلیدی اینکه چرا $A$ برابر $LU$ است: از خود بپرسید آیا سطرهای لولا که از سطرهای پایین‌تر کم می‌شوند، سطرهای اصلی $A$ هستند؟ خیر، حذف احتمالاً آن‌ها را تغییر داده است. آیا آن‌ها سطرهای $U$ هستند؟ بله، سطرهای لولا دیگر هرگز تغییر نمی‌کنند. هنگام محاسبه سطر سوم $U$, ما مضاربی از سطرهای قبلی $U$ را کم می‌کنیم (نه سطرهای $A$!):
	\[ \text{سطر ۳ از } U = (\text{سطر ۳ از } A) - l_{31}(\text{سطر ۱ از } U) - l_{32}(\text{سطر ۲ از } U) \quad (۱) \]
	این معادله را بازنویسی کنید تا ببینید که سطر $[ l_{31} \ l_{32} \ 1 ]$ در ماتریس $U$ ضرب می‌شود:
	\[ (\text{سطر ۳ از } A) = l_{31}(\text{سطر ۱ از } U) + l_{32}(\text{سطر ۲ از } U) + 1(\text{سطر ۳ از } U) \quad (۲) \]
	این دقیقاً سطر ۳ از $A = LU$ است. آن سطر از $L$ شامل $l_{31}, l_{32}, 1$ است. همه سطرها، صرف نظر از اندازه $A$, به این شکل هستند. بدون تعویض سطر، ما $A=LU$ را داریم.
	
	\subsection*{تعادل بهتر با LDU}
	تجزیه $A=LU$ «نامتقارن» است زیرا $U$ لولاها را روی قطر اصلی خود دارد در حالی که $L$ دارای ۱ است. تغییر این وضعیت آسان است. $U$ را بر یک ماتریس قطری $D$ که شامل لولاهاست تقسیم کنید. این کار یک ماتریس مثلثی جدید با ۱ روی قطر اصلی باقی می‌گذارد:
	\textbf{تجزیه $U$}
	\[ U = \begin{bmatrix} d_1 & \cdots \\ & \ddots \\ & & d_n \end{bmatrix} \begin{bmatrix} 1 & u_{12}/d_1 & \cdots \\ & \ddots & \\ & & 1 \end{bmatrix} \]
	مرسوم است (اما کمی گیج‌کننده) که برای این ماتریس مثلثی جدید از همان حرف $U$ استفاده شود. این ماتریس روی قطر اصلی خود ۱ دارد (مانند $L$). به جای $LU$ معمولی، فرم جدید $D$ را در وسط دارد: پایین‌مثلثی $L$ ضربدر قطری $D$ ضربدر بالامثلثی $U$.
	
	تجزیه مثلثی را می‌توان به صورت $A=LU$ یا $A=LDU$ نوشت. هرگاه $LDU$ را دیدید، فرض بر این است که $U$ روی قطر اصلی خود ۱ دارد. هر سطر بر اولین درایه غیرصفر خود—لولا—تقسیم شده است. آنگاه با $L$ و $U$ در $LDU$ به طور یکسان رفتار می‌شود:
	\[ \begin{bmatrix} 2 & 8 \\ 6 & 21 \end{bmatrix} = \begin{bmatrix} 1 & 0 \\ 3 & 1 \end{bmatrix} \begin{bmatrix} 2 & 8 \\ 0 & -3 \end{bmatrix} \text{ بیشتر تجزیه می‌شود به } \begin{bmatrix} 1 & 0 \\ 3 & 1 \end{bmatrix} \begin{bmatrix} 2 & 0 \\ 0 & -3 \end{bmatrix} \begin{bmatrix} 1 & 4 \\ 0 & 1 \end{bmatrix} \quad (۳) \]
	لولاهای ۲ و ۳- به $D$ رفتند. تقسیم سطرها بر ۲ و ۳-، سطرهای $[1 \ 4]$ و $[0 \ 1]$ را در $U$ جدید با ۱‌های قطری باقی گذاشت. مضرب ۳ همچنان در $L$ است.
	
	\textit{(توضیح مترجم: تجزیه $A=LDU$ به ویژه برای ماتریس‌های متقارن اهمیت دارد. اگر $A$ متقارن باشد، آنگاه $U$ برابر با $L^T$ (ترانهاده $L$) خواهد بود و تجزیه به صورت $A=LDL^T$ در می‌آید که بسیار زیبا و مفید است.)}
	
	سخنرانی‌های خود من گاهی در این نقطه متوقف می‌شوند. من به بخش ۲.۷ می‌روم. پاراگراف‌های بعدی نشان می‌دهند که کدهای حذف چگونه سازماندهی شده‌اند و چقدر زمان می‌برند. اگر MATLAB (یا هر نرم‌افزار دیگری) در دسترس باشد، می‌توانید با شمارش ثانیه‌ها زمان محاسبات را اندازه‌گیری کنید.
	
	\subsection*{یک دستگاه مربعی = دو دستگاه مثلثی}
	ماتریس $L$ حافظه ما از حذف گاوسی است. این ماتریس اعدادی را نگه می‌دارد که سطرهای لولا را قبل از تفریق از سطرهای پایین‌تر ضرب کرده‌اند. چه زمانی به این رکورد نیاز داریم و چگونه از آن در حل $A\mathbf{x}=\mathbf{b}$ استفاده می‌کنیم؟
	
	به محض اینکه یک سمت راست $\mathbf{b}$ وجود داشته باشد، به $L$ نیاز داریم. عامل‌های $L$ و $U$ کاملاً توسط سمت چپ (ماتریس $A$) تعیین شده‌اند. در سمت راست $A\mathbf{x}=\mathbf{b}$، ما از $L^{-1}$ و سپس $U^{-1}$ استفاده می‌کنیم. آن گام \textbf{حل (Solve)} با دو ماتریس مثلثی سر و کار دارد.
	
	\begin{enumerate}
		\item \textbf{تجزیه (Factor)} (به $L$ و $U$, با حذف روی ماتریس سمت چپ $A$).
		\item \textbf{حل (Solve)} (حذف پیشرو روی $\mathbf{b}$ با استفاده از $L$, سپس جایگذاری پسرو برای $\mathbf{x}$ با استفاده از $U$).
	\end{enumerate}
	
	قبلاً ما همزمان روی $A$ و $\mathbf{b}$ کار می‌کردیم. مشکلی در این مورد وجود ندارد—کافی است ماتریس الحاقی $[A \ \mathbf{b}]$ را تشکیل دهیم. اما بیشتر کدهای کامپیوتری دو طرف را جدا نگه می‌دارند. حافظه حذف در $L$ و $U$ نگهداری می‌شود تا هر زمان که بخواهیم $\mathbf{b}$ را پردازش کنیم. راهنمای کاربر LAPACK اشاره می‌کند که «این وضعیت آنقدر رایج و صرفه‌جویی آنقدر مهم است که هیچ تمهیدی برای حل یک دستگاه منفرد تنها با یک زیرروال در نظر گرفته نشده است.»
	
	گام \textbf{حل} چگونه روی $\mathbf{b}$ کار می‌کند؟ ابتدا، حذف پیشرو را روی سمت راست اعمال کنید (مضارب در $L$ ذخیره شده‌اند، اکنون از آن‌ها استفاده کنید). این کار $\mathbf{b}$ را به یک سمت راست جدید $\mathbf{c}$ تغییر می‌دهد. ما در واقع در حال حل $L\mathbf{c} = \mathbf{b}$ هستیم. سپس جایگذاری پسرو، $U\mathbf{x}=\mathbf{c}$ را مثل همیشه حل می‌کند. دستگاه اصلی $A\mathbf{x}=\mathbf{b}$ به دو دستگاه مثلثی تجزیه شده است:
	
	\textbf{حل پیشرو و پسرو}
	\[ \text{ابتدا } L\mathbf{c}=\mathbf{b} \text{ را حل کنید و سپس } U\mathbf{x}=\mathbf{c} \text{ را حل کنید.} \quad (۴) \]
	برای اینکه ببینید $\mathbf{x}$ درست است، $U\mathbf{x}=\mathbf{c}$ را در $L$ ضرب کنید. آنگاه $LU\mathbf{x} = L\mathbf{c}$ همان $A\mathbf{x}=\mathbf{b}$ است.
	
	برای تأکید: هیچ چیز جدیدی در مورد این گام‌ها وجود ندارد. این دقیقاً همان کاری است که ما همیشه انجام داده‌ایم. ما در واقع دستگاه مثلثی $L\mathbf{c}=\mathbf{b}$ را همزمان با پیشرفت حذف حل می‌کردیم. سپس جایگذاری پسرو $\mathbf{x}$ را تولید می‌کرد. یک مثال نشان می‌دهد که ما در عمل چه می‌کردیم.
	
	\textbf{مثال ۳}
	حذف پیشرو (رو به پایین) روی $A\mathbf{x}=\mathbf{b}$ به $U\mathbf{x}=\mathbf{c}$ ختم می‌شود:
	\[ A\mathbf{x}=\mathbf{b} \qquad \begin{cases} u + 2v = 5 \\ 4u + 9v = 21 \end{cases} \quad \text{به} \quad \begin{cases} u + 2v = 5 \\ v=1 \end{cases} \qquad U\mathbf{x}=\mathbf{c} \]
	مضرب ۴ بود که در $L$ ذخیره می‌شود. سمت راست از آن ۴ برای تبدیل ۲۱ به ۱ استفاده کرد:
	دستگاه پایین‌مثلثی $L\mathbf{c}=\mathbf{b}$:
	\[ \begin{bmatrix} 1 & 0 \\ 4 & 1 \end{bmatrix} \begin{bmatrix} c_1 \\ c_2 \end{bmatrix} = \begin{bmatrix} 5 \\ 21 \end{bmatrix} \quad \text{که نتیجه می‌دهد} \quad \mathbf{c} = \begin{bmatrix} 5 \\ 1 \end{bmatrix} \]
	دستگاه بالامثلثی $U\mathbf{x}=\mathbf{c}$:
	\[ \begin{bmatrix} 1 & 2 \\ 0 & 1 \end{bmatrix} \begin{bmatrix} u \\ v \end{bmatrix} = \begin{bmatrix} 5 \\ 1 \end{bmatrix} \quad \text{که نتیجه می‌دهد} \quad \mathbf{x} = \begin{bmatrix} 3 \\ 1 \end{bmatrix} \]
	$L$ و $U$ می‌توانند در $n^2$ مکان حافظه‌ای که در ابتدا $A$ را نگه می‌داشتند، ذخیره شوند (اکنون $A$ قابل فراموش شدن است).
	
	\subsection*{هزینه حذف}
	یک سوال بسیار عملی، هزینه—یا زمان محاسبات—است. ما می‌توانیم ۱۰۰۰ معادله را روی یک کامپیوتر شخصی حل کنیم. اگر $n=100,000$ باشد چه؟ (آیا $A$ یک ماتریس متراکم است یا خلوت؟) دستگاه‌های بزرگ همیشه در محاسبات علمی به وجود می‌آیند، جایی که یک مسئله سه‌بعدی به راحتی می‌تواند به یک میلیون مجهول منجر شود. ما می‌توانیم محاسبات را یک شب تا صبح اجرا کنیم، اما نمی‌توانیم آن را برای ۱۰۰ سال رها کنیم.
	
	مرحله اول حذف، صفرهایی را زیر لولای اول در ستون ۱ ایجاد می‌کند. برای یافتن هر درایه جدید زیر سطر لولا، یک ضرب و یک تفریق لازم است. ما این مرحله اول را $n^2$ ضرب و $n^2$ تفریق حساب می‌کنیم. در واقع کمتر است، $n^2-n$, زیرا سطر ۱ تغییر نمی‌کند.
	
	مرحله بعدی ستون دوم را زیر لولای دوم پاک می‌کند. ماتریس کاری اکنون به اندازه $n-1$ است. این مرحله را با $(n-1)^2$ ضرب و تفریق تخمین می‌زنیم. ماتریس‌ها با پیشرفت حذف کوچک‌تر می‌شوند. شمارش تقریبی برای رسیدن به $U$ مجموع مربعات $n^2+(n-1)^2+\dots+2^2+1^2$ است.
	
	یک فرمول دقیق $\frac{1}{3}n(n+\frac{1}{2})(n+1)$ برای این مجموع مربعات وجود دارد. وقتی $n$ بزرگ است، $\frac{1}{2}$ و $1$ مهم نیستند. عددی که اهمیت دارد $\frac{1}{3}n^3$ است. مجموع مربعات مانند انتگرال $x^2$ است! انتگرال از ۰ تا $n$ برابر $\frac{1}{3}n^3$ است:
	\begin{quote}
		\textbf{حذف روی $A$ به حدود $\frac{1}{3}n^3$ ضرب و $\frac{1}{3}n^3$ تفریق نیاز دارد.}
	\end{quote}
	در مورد سمت راست $\mathbf{b}$ چطور؟ در حرکت پیشرو، ما مضاربی از $b_1$ را از مؤلفه‌های پایینی $b_2, \dots, b_n$ کم می‌کنیم. این $n-1$ گام است. مرحله دوم تنها $n-2$ گام طول می‌کشد، زیرا $b_1$ دیگر درگیر نیست. آخرین مرحله حذف پیشرو یک گام طول می‌کشد.
	
	حالا جایگذاری پسرو را شروع کنید. محاسبه $x_n$ از یک گام استفاده می‌کند (تقسیم بر آخرین لولا). مجهول بعدی از دو گام استفاده می‌کند. وقتی به $x_1$ برسیم، به $n$ گام نیاز خواهد داشت ($n-1$ جایگذاری برای مجهولات دیگر، سپس تقسیم بر اولین لولا). شمارش کل در سمت راست، از $\mathbf{b}$ به $\mathbf{c}$ و به $\mathbf{x}$—پیشرو به پایین و پسرو به بالا—دقیقاً $n^2$ است:
	\[ [(n-1) + (n-2) + \dots + 1] + [1 + 2 + \dots + (n-1) + n] = n^2 \quad (۵) \]
	برای دیدن این مجموع، $(n-1)$ را با ۱ و $(n-2)$ را با ۲ جفت کنید. این جفت‌سازی‌ها $n$ جمله باقی می‌گذارند که هر کدام برابر $n$ است. این $n^2$ را می‌سازد. هزینه سمت راست بسیار کمتر از سمت چپ است!
	\begin{quote}
		\textbf{حل:} هر سمت راست به $n^2$ ضرب و $n^2$ تفریق نیاز دارد.
	\end{quote}
	یک \textbf{ماتریس نواری (band matrix)} $B$ فقط $w$ قطر غیرصفر در زیر و بالای قطر اصلی خود دارد. درایه‌های صفر خارج از نوار در حذف صفر باقی می‌مانند (آن‌ها در $L$ و $U$ صفر هستند). پاک کردن ستون اول به $w^2$ ضرب و تفریق نیاز دارد ($w$ صفر باید زیر لولا تولید شود که هر کدام از یک سطر لولای به طول $w$ استفاده می‌کنند). سپس پاک کردن تمام $n$ ستون برای رسیدن به $U$ به بیش از $nw^2$ نیاز ندارد. این باعث صرفه‌جویی زیادی در زمان می‌شود:
	\begin{itemize}
		\item \textbf{ماتریس نواری}:
		\item $A$ به $U$: $\frac{1}{3}n^3$ به $nw^2$ کاهش می‌یابد.
		\item حل: $n^2$ به $2nw$ کاهش می‌یابد.
	\end{itemize}
	یک ماتریس سه‌قطری (با پهنای نوار $w=1$) محاسبات بسیار سریعی را ممکن می‌سازد. صفرهای آن را ذخیره نکنید!
	
	وب‌سایت کتاب کدهای آموزشی برای تجزیه $A$ به $LU$ و حل $A\mathbf{x}=\mathbf{b}$ دارد. کدهای حرفه‌ای برای کاهش خطای گردکردن، در هر ستون به دنبال بزرگ‌ترین لولای موجود می‌گردند تا سطرها را تعویض کنند.
	
	دستور `backslash` در MATLAB یعنی `x = A\b`، مراحل تجزیه و حل را برای رسیدن به `x` ترکیب می‌کند.
	
	حل $A\mathbf{x}=\mathbf{b}$ چقدر طول می‌کشد؟ برای یک ماتریس تصادفی از مرتبه $n=1000$، زمان معمول روی یک کامپیوتر شخصی ۱ ثانیه است. وقتی $n$ دو برابر می‌شود، زمان تقریباً ۸ برابر می‌شود. برای کدهای حرفه‌ای به netlib.org مراجعه کنید.
	
	طبق این قانون $n^3$، ماتریس‌هایی که ۱۰ برابر بزرگ‌تر هستند (مرتبه ۱۰۰۰۰) هزار ثانیه طول می‌کشند. ماتریس‌های مرتبه ۱۰۰,۰۰۰ یک میلیون ثانیه طول می‌کشند. این بدون یک ابررایانه بسیار گران است، اما به یاد داشته باشید که این ماتریس‌ها متراکم هستند. بیشتر ماتریس‌ها در عمل \textbf{خلوت} (sparse) هستند (بسیاری از درایه‌ها صفر هستند). در آن صورت $A=LU$ بسیار سریع‌تر است.
	
	\subsection*{مروری بر ایده‌های کلیدی}
	\begin{enumerate}
		\item حذف گاوسی (بدون تعویض سطر) ماتریس $A$ را به حاصل‌ضرب $L$ در $U$ تجزیه می‌کند.
		\item ماتریس پایین‌مثلثی $L$ شامل اعداد $l_{ij}$ است که سطرهای لولا را در مسیر از $A$ به $U$ ضرب می‌کنند. حاصل‌ضرب $LU$ آن سطرها را دوباره جمع می‌کند تا $A$ را بازیابی کند.
		\item در سمت راست ما $L\mathbf{c}=\mathbf{b}$ (پیشرو) و $U\mathbf{x}=\mathbf{c}$ (پسرو) را حل می‌کنیم.
		\item \textbf{تجزیه}: $\frac{1}{2}(n^3-n)$ عمل ضرب و تفریق در سمت چپ وجود دارد.
		\item \textbf{حل}: $n^2$ عمل ضرب و تفریق در سمت راست وجود دارد.
		\item برای یک ماتریس نواری، $\frac{1}{3}n^3$ را به $nw^2$ و $n^2$ را به $2wn$ تغییر دهید.
	\end{enumerate}
	
	\subsection*{مثال‌های حل شده}
	\subsubsection*{مثال ۲.۶ الف}
	ماتریس پاسکال پایین‌مثلثی $L$ شامل «مثلث خیام-پاسکال» معروف است. گاوس-جردن در مثال حل شده ۲.۵ ج $L$ را معکوس کرد. در اینجا ما ماتریس پاسکال را تجزیه می‌کنیم.
	
	ماتریس پاسکال متقارن $P$ حاصل‌ضرب ماتریس‌های پاسکال مثلثی $L$ و $U$ است. $P$ متقارن، مثلث خیام-پاسکال را به صورت کج دارد، به طوری که هر درایه مجموع درایه بالایی و درایه سمت چپ است. در MATLAB، ماتریس پاسکال متقارن $n \times n$ با دستور `pascal(n)` ساخته می‌شود.
	
	\textbf{مسئله:} تجزیه پایین‌مثلثی-بالامثلثی شگفت‌انگیز $P=LU$ را برقرار کنید.
	\[ \text{pascal(4)} = \begin{bmatrix} 1 & 1 & 1 & 1 \\ 1 & 2 & 3 & 4 \\ 1 & 3 & 6 & 10 \\ 1 & 4 & 10 & 20 \end{bmatrix} = \begin{bmatrix} 1 & 0 & 0 & 0 \\ 1 & 1 & 0 & 0 \\ 1 & 2 & 1 & 0 \\ 1 & 3 & 3 & 1 \end{bmatrix} \begin{bmatrix} 1 & 1 & 1 & 1 \\ 0 & 1 & 2 & 3 \\ 0 & 0 & 1 & 3 \\ 0 & 0 & 0 & 1 \end{bmatrix} = LU \]
	سپس سطر و ستون بعدی را برای ماتریس‌های پاسکال ۵ در ۵ پیش‌بینی و بررسی کنید.
	
	\textbf{راه حل:} شما می‌توانید $LU$ را ضرب کنید تا به $P$ برسید. بهتر است با $P$ متقارن شروع کرده و با حذف به $U$ بالامثلثی برسید:
	\[
	\begin{bmatrix} 1 & 1 & 1 & 1 \\ 1 & 2 & 3 & 4 \\ 1 & 3 & 6 & 10 \\ 1 & 4 & 10 & 20 \end{bmatrix} \to \begin{bmatrix} 1 & 1 & 1 & 1 \\ 0 & 1 & 2 & 3 \\ 0 & 2 & 5 & 9 \\ 0 & 3 & 9 & 19 \end{bmatrix} \to \begin{bmatrix} 1 & 1 & 1 & 1 \\ 0 & 1 & 2 & 3 \\ 0 & 0 & 1 & 3 \\ 0 & 0 & 3 & 10 \end{bmatrix} \to \begin{bmatrix} 1 & 1 & 1 & 1 \\ 0 & 1 & 2 & 3 \\ 0 & 0 & 1 & 3 \\ 0 & 0 & 0 & 1 \end{bmatrix} = U
	\]
	مضارب $l_{ij}$ که در این گام‌ها استفاده شدند، کاملاً در $L$ قرار می‌گیرند. بنابراین $P=LU$ یک مثال بسیار تمیز است. توجه کنید که هر لولا روی قطر اصلی $U$ برابر با ۱ است.
	
	بخش بعدی نشان خواهد داد که چگونه تقارن یک رابطه خاص بین $L$ و $U$ مثلثی ایجاد می‌کند. برای ماتریس پاسکال، $U$ «ترانهاده» $L$ است.
	
	شما ممکن است انتظار داشته باشید که دستور `lu(pascal(4))` در MATLAB این $L$ و $U$ را تولید کند. این اتفاق نمی‌افتد زیرا زیرروال `lu` بزرگترین لولای موجود در هر ستون را انتخاب می‌کند. لولای دوم از ۱ به ۳ تغییر خواهد کرد. اما یک «تجزیه چولسکی» هیچ تعویض سطری انجام نمی‌دهد: `U = chol(pascal(4))`.
	\textit{(توضیح مترجم: تجزیه چولسکی یک نوع خاص و بسیار کارآمد از تجزیه $LU$ برای ماتریس‌های متقارن و معین مثبت است. این تجزیه به صورت $A=R^TR$ یا $A=LL^T$ است و چون تعویض سطر ندارد، برای کاربردهایی که حفظ ساختار ماتریس مهم است، ایده‌آل است.)}
	
	اثبات کامل $P=LU$ برای تمام اندازه‌های پاسکال کاملاً شگفت‌انگیز است. مقاله «ماتریس‌های پاسکال» در صفحه وب دوره در web.mit.edu/18.06 موجود است که از طریق OpenCourseWare MIT در ocw.mit.edu نیز قابل دسترسی است. این ماتریس‌های پاسکال خواص قابل توجه زیادی دارند—ما دوباره آن‌ها را خواهیم دید.
	
	\subsubsection*{مثال ۲.۶ ب}
	مسئله این است: $P\mathbf{x}=\mathbf{b}=(1,0,0,0)^T$ را حل کنید. این سمت راست برابر با اولین ستون ماتریس همانی $I$ است، به این معنی که $\mathbf{x}$ اولین ستون $P^{-1}$ خواهد بود. این همان روش گاوس-جردن است که ستون‌های $PP^{-1}=I$ را مطابقت می‌دهد. ما از قبل عامل‌های $L$ و $U$ ماتریس $P$ را می‌دانیم:
	\textbf{دو دستگاه مثلثی} $L\mathbf{c}=\mathbf{b}$ (پیشرو) و $U\mathbf{x}=\mathbf{c}$ (پسرو).
	
	\textbf{راه حل:} دستگاه پایین‌مثلثی $L\mathbf{c}=\mathbf{b}$ از بالا به پایین حل می‌شود:
	\begin{align*}
		c_1 &= 1 \\
		c_1 + c_2 &= 0 \\
		c_1 + 2c_2 + c_3 &= 0 \\
		c_1 + 3c_2 + 3c_3 + c_4 &= 0
	\end{align*}
	که نتیجه می‌دهد:
	\[ c_1 = +1, \quad c_2 = -1, \quad c_3 = +1, \quad c_4 = -1 \]
	حذف پیشرو ضرب در $L^{-1}$ است. این کار دستگاه بالامثلثی $U\mathbf{x}=\mathbf{c}$ را تولید می‌کند. جواب $\mathbf{x}$ مثل همیشه با جایگذاری پسرو، از پایین به بالا به دست می‌آید:
	\begin{align*}
		x_1 + x_2 + x_3 + x_4 &= 1 \\
		x_2 + 2x_3 + 3x_4 &= -1 \\
		x_3 + 3x_4 &= 1 \\
		x_4 &= -1
	\end{align*}
	که نتیجه می‌دهد:
	\[ x_1 = +4, \quad x_2 = -6, \quad x_3 = +4, \quad x_4 = -1 \]
	من در آن $\mathbf{x}$ یک الگو می‌بینم، اما نمی‌دانم از کجا می‌آید. دستور `inv(pascal(4))` را امتحان کنید.
	
	\newpage
	\section*{مجموعه مسائل ۲.۶}
	
	\begin{enumerate}
		\item[] \textbf{مسائل ۱-۱۴ تجزیه $A = LU$ (و همچنین $A = LDU$) را محاسبه می‌کنند.}
		\item \textbf{(مهم)} حذف پیشرو دستگاه $\begin{bmatrix} 1 & 1 \\ 1 & 2 \end{bmatrix} \mathbf{x} = \mathbf{b}$ را به یک دستگاه مثلثی $\begin{bmatrix} 1 & 1 \\ 0 & 1 \end{bmatrix} \mathbf{x} = \mathbf{c}$ تبدیل می‌کند:
		\[ \begin{cases} x+y=5 \\ x+2y=7 \end{cases} \to \begin{cases} x+y=5 \\ y=2 \end{cases} \]
		آن گام $l_{21} = \underline{1}$ برابر سطر ۱ را از سطر ۲ کم کرد. گام معکوس $l_{21}$ برابر سطر ۱ را به سطر ۲ اضافه می‌کند. ماتریس برای آن گام معکوس $L = \underline{\begin{bmatrix} 1 & 0 \\ 1 & 1 \end{bmatrix}}$ است. این $L$ را در دستگاه مثلثی $\begin{bmatrix} 1 & 1 \\ 0 & 1 \end{bmatrix} \mathbf{x} = \begin{bmatrix} 5 \\ 2 \end{bmatrix}$ ضرب کنید تا $\underline{A\mathbf{x}} = \underline{\mathbf{b}}$ به دست آید. به صورت حرفی، $L$ در $U\mathbf{x}=\mathbf{c}$ ضرب می‌شود تا \underline{$A\mathbf{x}=\mathbf{b}$} را بدهد.
		
		\item دستگاه‌های مثلثی ۲ در ۲ $L\mathbf{c}=\mathbf{b}$ و $U\mathbf{x}=\mathbf{c}$ را از مسئله ۱ بنویسید. بررسی کنید که $\mathbf{c}=(5,2)^T$ اولی را حل می‌کند. $\mathbf{x}$ را بیابید که دومی را حل کند.
		
		\item (به ۳ در ۳ بروید) حذف پیشرو $A\mathbf{x}=\mathbf{b}$ را به $U\mathbf{x}=\mathbf{c}$ مثلثی تغییر می‌دهد:
		\[ \begin{cases} x+y+z=5 \\ x+2y+3z=7 \\ x+3y+6z=11 \end{cases} \to \begin{cases} x+y+z=5 \\ y+2z=2 \\ 2y+5z=6 \end{cases} \to \begin{cases} x+y+z=5 \\ y+2z=2 \\ z=2 \end{cases} \]
		معادله $z=2$ در $U\mathbf{x}=\mathbf{c}$ از معادله اصلی $x+3y+6z=11$ در $A\mathbf{x}=\mathbf{b}$ با کم کردن $l_{31}=\underline{1}$ برابر معادله ۱ و $l_{32}=\underline{2}$ برابر معادله نهایی ۲ به دست می‌آید. این را معکوس کنید تا $[1 \ 3 \ 6 \ 11]$ در سطر آخر $A$ و $\mathbf{b}$ از $[1 \ 1 \ 1 \ 5]$ نهایی و $[0 \ 1 \ 2 \ 2]$ و $[0 \ 0 \ 1 \ 2]$ در $U$ و $\mathbf{c}$ بازیابی شود:
		\[ \text{سطر ۳ از } [A \ \mathbf{b}] = (l_{31} \text{ سطر ۱ } + l_{32} \text{ سطر ۲ } + 1 \text{ سطر ۳}) \text{ از } [U \ \mathbf{c}] \]
		در نماد ماتریسی این ضرب در $L$ است. بنابراین $A=LU$ و $\mathbf{b}=L\mathbf{c}$.
		
		\item دستگاه‌های مثلثی ۳ در ۳ $L\mathbf{c}=\mathbf{b}$ و $U\mathbf{x}=\mathbf{c}$ از مسئله ۳ چه هستند؟ بررسی کنید که $\mathbf{c}=(5,2,2)^T$ اولی را حل می‌کند. کدام $\mathbf{x}$ دومی را حل می‌کند؟
		
		\item چه ماتریس $E$ ماتریس $A$ را به شکل مثلثی $EA=U$ در می‌آورد؟ با ضرب در $E^{-1}=L$ ماتریس $A$ را به $LU$ تجزیه کنید:
		\[ A = \begin{bmatrix} 2 & 1 & 0 \\ 0 & 4 & 2 \\ 6 & 3 & 5 \end{bmatrix} \]
		
		\item کدام دو ماتریس حذف $E_{21}$ و $E_{32}$ ماتریس $A$ را به شکل بالامثلثی $E_{32}E_{21}A=U$ در می‌آورند؟ با ضرب در $E_{32}^{-1}$ و $E_{21}^{-1}$ ماتریس $A$ را به $LU = E_{21}^{-1}E_{32}^{-1}U$ تجزیه کنید:
		\[ A = \begin{bmatrix} 1 & 1 & 1 \\ 2 & 4 & 5 \\ 0 & 4 & 0 \end{bmatrix} \]
		
		\item کدام سه ماتریس حذف $E_{21}, E_{31}, E_{32}$ ماتریس $A$ را به شکل بالامثلثی $E_{32}E_{31}E_{21}A=U$ در می‌آورند؟ با ضرب در $E_{32}^{-1}, E_{31}^{-1}$ و $E_{21}^{-1}$ ماتریس $A$ را به $L$ ضربدر $U$ تجزیه کنید:
		\[ A = \begin{bmatrix} 1 & 0 & 0 \\ 2 & 2 & 2 \\ 3 & 4 & 5 \end{bmatrix} \quad L = E_{21}^{-1}E_{31}^{-1}E_{32}^{-1} \]
		این مسئله‌ای است که نشان می‌دهد چگونه معکوس‌های $E_{ij}^{-1}$ ضرب می‌شوند تا $L$ را بدهند. شما این را بهتر می‌بینید وقتی $A$ از قبل پایین‌مثلثی با ۱ روی قطر اصلی باشد. آنگاه $U=I$ است!
		\[ A=L = \begin{bmatrix} 1 & 0 & 0 \\ a & 1 & 0 \\ b & c & 1 \end{bmatrix} \]
		ماتریس‌های حذف $E_{21}, E_{31}, E_{32}$ به ترتیب شامل $-a$، $-b$ و $-c$ هستند.
		\begin{itemize}
			\item[(الف)] $E_{32}E_{31}E_{21}$ را ضرب کنید تا ماتریس واحد $E$ را بیابید که $EA=I$ را تولید می‌کند.
			\item[(ب)] $E_{21}^{-1}E_{31}^{-1}E_{32}^{-1}$ را ضرب کنید تا $L$ را بازگردانید.
		\end{itemize}
		مضارب $a, b, c$ در $E$ مخلوط شده‌اند اما در $L$ کامل هستند.
		
		\item وقتی صفر در یک موقعیت لولا ظاهر می‌شود، $A=LU$ ممکن نیست! (ما به لولاهای غیرصفر در $U$ نیاز داریم.) مستقیماً نشان دهید چرا این دو تجزیه غیرممکن هستند:
		\[ \begin{bmatrix} 0 & 1 \\ 2 & 3 \end{bmatrix} = \begin{bmatrix} 1 & 0 \\ l & 1 \end{bmatrix} \begin{bmatrix} d & e \\ 0 & f \end{bmatrix} \quad \text{و} \quad \begin{bmatrix} 1 & 1 & 0 \\ 1 & 1 & 2 \\ 1 & 2 & 1 \end{bmatrix} = LDU \]
		این ماتریس‌ها به یک تعویض سطر نیاز دارند. این کار از یک «ماتریس جایگشت» $P$ استفاده می‌کند.
		
		\item کدام عدد $c$ منجر به صفر در موقعیت لولای دوم می‌شود؟ یک تعویض سطر لازم است و $A=LU$ ممکن نخواهد بود. کدام $c$ در موقعیت لولای سوم صفر تولید می‌کند؟ آنگاه یک تعویض سطر نمی‌تواند کمک کند و حذف با شکست مواجه می‌شود:
		\[ A = \begin{bmatrix} 1 & 2 & 0 \\ 2 & 4 & c \\ 0 & 1 & 1 \end{bmatrix} \quad \text{و} \quad A = \begin{bmatrix} 1 & 2 & 0 \\ 2 & c & 1 \\ 0 & 1 & 1 \end{bmatrix} \]
		
		\item $L$ و $D$ (ماتریس قطری لولاها) برای این ماتریس $A$ چه هستند؟ $U$ در $A=LU$ چیست و $U$ جدید در $A=LDU$ چیست؟
		\[ A = \begin{bmatrix} 2 & 4 & 8 \\ 0 & 3 & 9 \\ 0 & 0 & 7 \end{bmatrix} \quad \text{(از قبل مثلثی)} \]
		
		\item $A$ و $B$ متقارن هستند (زیرا $4=4$). تجزیه‌های سه‌گانه $LDU$ آن‌ها را بیابید و بگویید $U$ برای این ماتریس‌های متقارن چگونه به $L$ مرتبط است:
		\[ A = \begin{bmatrix} 2 & 4 \\ 4 & 11 \end{bmatrix} \quad B = \begin{bmatrix} 1 & 4 & 0 \\ 4 & 12 & 4 \\ 0 & 4 & 0 \end{bmatrix} \quad \text{(متقارن)} \]
		
		\item \textbf{(توصیه می‌شود)} $L$ و $U$ را برای ماتریس متقارن $A$ محاسبه کنید:
		\[ A = \begin{bmatrix} a & a & a & a \\ a & b & b & b \\ a & b & c & c \\ a & b & c & d \end{bmatrix} \]
		چهار شرط روی $a, b, c, d$ بیابید تا $A=LU$ با چهار لولا داشته باشیم.
		
		\item این ماتریس نامتقارن همان $L$ مسئله ۱۳ را خواهد داشت:
		\[ A = \begin{bmatrix} a & r & r & r \\ a & b & s & s \\ a & b & c & t \\ a & b & c & d \end{bmatrix} \]
		$L$ و $U$ را برای آن بیابید. چهار شرط روی $a, b, c, d, r, s, t$ بیابید تا $A=LU$ با چهار لولا داشته باشیم.
		
		\item[] \textbf{مسائل ۱۵-۱۶ از L و U (بدون نیاز به A) برای حل $A\mathbf{x}=\mathbf{b}$ استفاده می‌کنند.}
		\item دستگاه مثلثی $L\mathbf{c}=\mathbf{b}$ را حل کنید تا $\mathbf{c}$ را بیابید. سپس $U\mathbf{x}=\mathbf{c}$ را حل کنید تا $\mathbf{x}$ را بیابید:
		\[ L = \begin{bmatrix} 1 & 0 \\ 4 & 1 \end{bmatrix} \quad \text{و} \quad U = \begin{bmatrix} 2 & 4 \\ 0 & 1 \end{bmatrix} \quad \text{و} \quad \mathbf{b} = \begin{bmatrix} 2 \\ 11 \end{bmatrix} \]
		برای اطمینان، $LU$ را ضرب کرده و $A\mathbf{x}=\mathbf{b}$ را به طور معمول حل کنید. وقتی $\mathbf{c}$ را می‌بینید، دور آن دایره بکشید.
		
		\item $L\mathbf{c}=\mathbf{b}$ را حل کنید تا $\mathbf{c}$ را بیابید. سپس $U\mathbf{x}=\mathbf{c}$ را حل کنید تا $\mathbf{x}$ را بیابید. $A$ چه بوده است؟
		\[ L = \begin{bmatrix} 1 & 0 & 0 \\ 1 & 1 & 0 \\ 1 & 1 & 1 \end{bmatrix} \quad \text{و} \quad U = \begin{bmatrix} 1 & 1 & 1 \\ 0 & 1 & 1 \\ 0 & 0 & 1 \end{bmatrix} \quad \text{و} \quad \mathbf{b} = \begin{bmatrix} 4 \\ 5 \\ 6 \end{bmatrix} \]
		
		\item (الف) وقتی گام‌های حذف معمول را روی $L$ اعمال می‌کنید، به چه ماتریسی می‌رسید؟
		(ب) وقتی همان گام‌ها را روی $I$ اعمال می‌کنید، چه ماتریسی به دست می‌آورید؟
		(ج) وقتی همان گام‌ها را روی $LU$ اعمال می‌کنید، چه ماتریسی به دست می‌آورید؟
		
		\item اگر $A=LDU$ و همچنین $A=L_1D_1U_1$ با همه عامل‌های معکوس‌پذیر، آنگاه $L=L_1$ و $D=D_1$ و $U=U_1$. «سه عامل یکتا هستند.»
		معادله $L_1^{-1}LD = D_1U_1U^{-1}$ را استنتاج کنید. آیا دو طرف مثلثی هستند یا قطری؟ استنتاج کنید که $L=L_1$ و $U=U_1$ (همه آن‌ها روی قطر اصلی ۱ دارند). سپس $D=D_1$.
		
		\item ماتریس‌های سه‌قطری به جز روی قطر اصلی و دو قطر مجاور آن، درایه‌های صفر دارند. این‌ها را به $A=LU$ و $A=LDL^T$ تجزیه کنید:
		\[ A = \begin{bmatrix} 1 & 1 & 0 \\ 1 & 2 & 1 \\ 0 & 1 & 2 \end{bmatrix} \quad \text{و} \quad A = \begin{bmatrix} a & a & 0 \\ a & a+b & b \\ 0 & b & b+c \end{bmatrix} \]
		
		\item وقتی $T$ سه‌قطری است، عامل‌های $L$ و $U$ آن تنها دو قطر غیرصفر دارند. چگونه از دانستن صفرهای $T$ در یک کد برای حذف گاوسی استفاده می‌کنید؟ $L$ و $U$ را بیابید.
		\[ T = \begin{bmatrix} 1 & 1 & 0 \\ 1 & 2 & 1 \\ 0 & 1 & 2 \end{bmatrix} \quad \text{(سه‌قطری)} \]
		
		\item اگر $A$ و $B$ در موقعیت‌های مشخص شده با x غیرصفر باشند، کدام صفرها (مشخص شده با ۰) در عامل‌های $L$ و $U$ آن‌ها صفر باقی می‌مانند؟
		\[ A = \begin{bmatrix} x & x & x & 0 \\ x & x & 0 & x \\ x & 0 & x & x \\ 0 & x & x & x \end{bmatrix} \quad B = \begin{bmatrix} x & x & x & x \\ x & x & 0 & 0 \\ x & 0 & x & 0 \\ x & 0 & 0 & x \end{bmatrix} \]
		
		\item فرض کنید شما به سمت بالا حذف می‌کنید (تقریباً بی‌سابقه). از سطر آخر برای تولید صفر در ستون آخر استفاده کنید (لولا ۱ است). سپس از سطر دوم برای تولید صفر بالای لولای دوم استفاده کنید. عامل‌ها را در ترتیب غیرمعمول $A=UL$ بیابید.
		\[ A = \begin{bmatrix} 5 & 3 & 1 \\ 3 & 3 & 1 \\ 1 & 1 & 1 \end{bmatrix} \quad \text{(بالا ضربدر پایین)} \]
		
		\item (ساده اما مهم) اگر $A$ لولاهای ۵، ۹، ۳ را بدون تعویض سطر داشته باشد، لولاهای زیرماتریس ۲ در ۲ بالا سمت چپ $A_2$ (بدون سطر ۳ و ستون ۳) چه هستند؟
		
		\item[] \textbf{مسائل چالشی}
		
		\item کدام ماتریس‌های معکوس‌پذیر تجزیه $A=LU$ را ممکن می‌سازند (حذف بدون تعویض سطر)؟ سوال خوبی است! به هر یک از زیرماتریس‌های مربعی بالا سمت چپ $A_k$ از $A$ نگاه کنید.
		\textbf{پاسخ:} تمام زیرماتریس‌های $k \times k$ بالا سمت چپ $A_k$ باید معکوس‌پذیر باشند (اندازه‌های $k=1, \dots, n$).
		این پاسخ را توضیح دهید: $A_k$ به \underline{$L_kU_k$} تجزیه می‌شود زیرا $LU = \begin{bmatrix} L_k & 0 \\ * & * \end{bmatrix} \begin{bmatrix} U_k & * \\ 0 & * \end{bmatrix}$.
		
		\item برای ماتریس تفاضل دوم ۶ در ۶ با قطر ثابت $K$، لولاها و مضارب را در $K=LU$ قرار دهید. ($L$ و $U$ تنها دو قطر غیرصفر خواهند داشت، زیرا $K$ سه قطر دارد.) یک فرمول برای درایه $(i,j)$ از $L^{-1}$ با استفاده از نرم‌افزاری مانند MATLAB با `inv(L)` یا با جستجوی یک الگوی خوب بیابید.
		\[ K = \begin{bmatrix} 2 & -1 & & & & \\ -1 & 2 & -1 & & & \\ & \ddots & \ddots & \ddots & & \\ & & & -1 & 2 & -1 \\ & & & & -1 & 2 \end{bmatrix} \quad \text{ماتریس ۱-, ۲, ۱-} \]
		این ماتریس با دستور `toeplitz([2 -1 0 0 0 0])` ساخته می‌شود.
		
		\item اگر $K^{-1}$ را چاپ کنید، چندان خوب به نظر نمی‌رسد (۶ در ۶). اما اگر $7K^{-1}$ را چاپ کنید، آن ماتریس فوق‌العاده به نظر می‌رسد. $7K^{-1}$ را با دست بنویسید، با پیروی از این الگو:
		\begin{enumerate}
			\item سطر ۱ و ستون ۱ عبارتند از $(6, 5, 4, 3, 2, 1)$.
			\item روی قطر اصلی و بالای آن، سطر $i$ برابر $i$ ضربدر سطر ۱ است.
			\item روی قطر اصلی و زیر آن، ستون $j$ برابر $j$ ضربدر ستون ۱ است.
		\end{enumerate}
		$K$ را در آن $7K^{-1}$ ضرب کنید تا $7I$ تولید شود. در اینجا $4K^{-1}$ برای $n=3$ آمده است:
		\[ \text{برای } n=3, \quad 4K^{-1} = \begin{bmatrix} 3 & 2 & 1 \\ 2 & 4 & 2 \\ 1 & 2 & 3 \end{bmatrix} \]
	\end{enumerate}
	
\end{document}