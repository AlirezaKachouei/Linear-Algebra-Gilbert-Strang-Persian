\documentclass[12pt, a4paper]{book}

% فراخوانی بسته‌های لازم
\usepackage{amsmath}         % برای فرمول‌های پیشرفته ریاضی
\usepackage{amsfonts}        % بسته برای فونت‌های ریاضی مانند \mathbb
\usepackage{amssymb}         % برای نمادهای بیشتر ریاضی
\usepackage{graphicx}        % برای افزودن تصاویر
\usepackage{xepersian}       % بسته اصلی برای پارسی‌نویسی
\usepackage{geometry}        % برای تنظیم حاشیه‌ها
\usepackage{setspace}        % برای تنظیم فاصله خطوط
\usepackage{array}           % برای امکانات پیشرفته در جدول‌ها و آرایه‌ها
\usepackage{enumitem}        % برای کنترل بیشتر بر لیست‌ها

% تنظیم حاشیه‌های صفحه
\geometry{
	a4paper,
	total={170mm,257mm},
	left=20mm,
	top=20mm,
}

% تنظیم فونت‌های نوشتاری و ریاضی
% توجه: این فونت‌ها باید روی سیستم شما نصب باشند
\settextfont{XB Niloofar}
\setdigitfont{XB Niloofar}
\setmathdigitfont{XB Niloofar}

\begin{document}
	
	% اعمال فاصله 1.5 بین خطوط برای خوانایی بهتر
	\onehalfspacing
	
	\section{قواعد عملیات ماتریسی}
	
	\begin{enumerate}
		\item ماتریس $A$ با $n$ ستون در ماتریس $B$ با $n$ سطر ضرب می‌شود: $A_{m \times n} B_{n \times p} = C_{m \times p}$.
		\item هر درایه در $AB=C$ یک حاصل‌ضرب داخلی است: $C_{ij} = (\text{سطر } i \text{ از } A) \cdot (\text{ستون } j \text{ از } B)$.
		\item این قاعده به گونه‌ای انتخاب شده است که ضرب $(AB)$ در $C$ برابر با ضرب $A$ در $(BC)$ باشد. همچنین $(AB)\mathbf{x} = A(B\mathbf{x})$.
		\item راه‌های بیشتر برای محاسبه $AB$: (ماتریس $A$ ضربدر ستون‌های $B$)، (سطرهای $A$ ضربدر ماتریس $B$)، (جمع حاصل‌ضرب ستون‌ها در سطرها).
		\item معمولاً $AB = BA$ برقرار نیست. در بیشتر موارد، $A$ با $B$ خاصیت جابجایی ندارد.
		\item ماتریس‌ها را می‌توان به صورت قطعه‌ای (بلوکی) ضرب کرد: $A = [A_1 \ A_2]$ ضربدر $B = \begin{bmatrix} B_1 \\ B_2 \end{bmatrix}$ برابر با $A_1B_1 + A_2B_2$ است.
	\end{enumerate}
	
	من با حقایق پایه‌ای شروع می‌کنم. یک ماتریس، آرایه‌ای مستطیلی از اعداد یا «درایه‌ها» است. وقتی $A$ دارای $m$ سطر و $n$ ستون باشد، یک ماتریس «$m$ در $n$» نامیده می‌شود. ماتریس‌ها را می‌توان با هم جمع کرد اگر ابعادشان یکسان باشد. آن‌ها را می‌توان در هر عدد ثابت $c$ ضرب کرد. در اینجا مثال‌هایی از $A+B$ و $2A$ برای ماتریس‌های ۳ در ۲ آورده شده است:
	\[
	\begin{bmatrix} 1 & 2 \\ 3 & 4 \\ 5 & 6 \end{bmatrix} +
	\begin{bmatrix} 6 & 5 \\ 4 & 3 \\ 2 & 1 \end{bmatrix} =
	\begin{bmatrix} 7 & 7 \\ 7 & 7 \\ 7 & 7 \end{bmatrix}
	\quad \text{و} \quad
	2 \begin{bmatrix} 1 & 2 \\ 3 & 4 \\ 5 & 6 \end{bmatrix} =
	\begin{bmatrix} 2 & 4 \\ 6 & 8 \\ 10 & 12 \end{bmatrix}
	\]
	ماتریس‌ها دقیقاً مانند بردارها جمع می‌شوند—درایه به درایه. ما حتی می‌توانیم یک بردار ستونی را به عنوان یک ماتریس با تنها یک ستون (بنابراین $n=1$) در نظر بگیریم. ماتریس $-A$ از ضرب در $c=-1$ به دست می‌آید (علامت همه درایه‌ها را معکوس می‌کند). جمع کردن $A$ با $-A$ ماتریس صفر را نتیجه می‌دهد که همه درایه‌هایش صفر هستند. همه این‌ها کاملاً منطقی و بدیهی است.
	
	درایه واقع در سطر $i$ و ستون $j$ را $a_{ij}$ یا $A(i,j)$ می‌نامند. بنابراین درایه‌های سطر اول $a_{11}, a_{12}, \dots, a_{1n}$ هستند. درایه پایین سمت چپ ماتریس $a_{m1}$ و درایه پایین سمت راست $a_{mn}$ است. شماره سطر $i$ از ۱ تا $m$ و شماره ستون $j$ از ۱ تا $n$ تغییر می‌کند.
	
	جمع ماتریسی آسان است. سوال جدی، ضرب ماتریسی است. چه زمانی می‌توانیم $A$ را در $B$ ضرب کنیم و حاصل‌ضرب $AB$ چیست؟ این بخش ۴ روش برای یافتن $AB$ ارائه می‌دهد. اما ما نمی‌توانیم ماتریس‌های ۳ در ۲ را در هم ضرب کنیم. آن‌ها آزمون زیر را پشت سر نمی‌گذارنند:
	
	\begin{quote}
		\textbf{شرط ضرب ماتریسی $AB$:} اگر $A$ دارای $n$ ستون باشد، $B$ باید $n$ سطر داشته باشد.
	\end{quote}
	
	وقتی $A$ یک ماتریس ۳ در ۲ است، ماتریس $B$ می‌تواند ۲ در ۱ (یک بردار)، ۲ در ۲ (مربع) یا ۲ در ۲۰ باشد. هر ستون از $B$ در $A$ ضرب می‌شود. من ضرب ماتریسی را با روش حاصل‌ضرب داخلی شروع می‌کنم و به روش ستونی باز خواهم گشت: $A$ ضربدر ستون‌های $B$. هر دو روش از این قانون پیروی می‌کنند:
	
	\begin{quote}
		\textbf{قانون بنیادی ضرب ماتریسی (قانون شرکت‌پذیری)} \\
		ضرب $(AB)$ در $C$ برابر با ضرب $A$ در $(BC)$ است:
		\[ (AB)C = A(BC) \quad (۱) \]
	\end{quote}
	پرانتزها را می‌توان با خیال راحت جابجا کرد. جبر خطی به این قانون وابسته است.
	
	فرض کنید $A$ یک ماتریس $m$ در $n$ و $B$ یک ماتریس $n$ در $p$ باشد. ما می‌توانیم آن‌ها را ضرب کنیم. حاصل‌ضرب $AB$ یک ماتریس $m$ در $p$ خواهد بود.
	\[ (m \times n)(n \times p) = (m \times p) \]
	\[
	\begin{bmatrix} m \text{ سطر} \\ n \text{ ستون} \end{bmatrix}
	\begin{bmatrix} n \text{ سطر} \\ p \text{ ستون} \end{bmatrix}
	=
	\begin{bmatrix} m \text{ سطر} \\ p \text{ ستون} \end{bmatrix}
	\]
	یک سطر ضربدر یک ستون یک حالت حدی است. در این حالت یک ماتریس ۱ در $n$ در یک ماتریس $n$ در ۱ ضرب می‌شود. نتیجه یک ماتریس ۱ در ۱ خواهد بود. آن تک عدد همان «حاصل‌ضرب داخلی» (dot product) است.
	
	در هر حالتی، ماتریس $AB$ با حاصل‌ضرب‌های داخلی پر می‌شود. برای گوشه بالا، درایه (۱,۱) از $AB$ برابر است با (سطر ۱ از $A$) $\cdot$ (ستون ۱ از $B$). این اولین و معمول‌ترین روش برای ضرب ماتریس‌هاست. حاصل‌ضرب داخلی هر سطر از $A$ را با هر ستون از $B$ بگیرید.
	
	\textbf{۱. درایه واقع در سطر $i$ و ستون $j$ از ماتریس $AB$ برابر است با (سطر $i$ از $A$) $\cdot$ (ستون $j$ از $B$).}
	
	برای مثال، درایه واقع در سطر $i=2$ و ستون $j=3$ ماتریس حاصل‌ضرب $AB$ از ضرب داخلی سطر دوم $A$ در ستون سوم $B$ به دست می‌آید. ماتریس $AB$ به تعداد سطرهای $A$ (یعنی $m$) سطر و به تعداد ستون‌های $B$ (یعنی $p$) ستون خواهد داشت.
	\[ (AB)_{ij} = \sum_{k=1}^{n} a_{ik}b_{kj} \]
	
	\subsubsection*{مثال ۱}
	ماتریس‌های مربعی را می‌توان ضرب کرد اگر و تنها اگر ابعاد یکسانی داشته باشند:
	\[
	\begin{bmatrix} 1 & 1 \\ 0 & 1 \end{bmatrix}
	\begin{bmatrix} 2 & 3 \\ 3 & 0 \end{bmatrix}
	=
	\begin{bmatrix} 1 \cdot 2 + 1 \cdot 3 & 1 \cdot 3 + 1 \cdot 0 \\ 0 \cdot 2 + 1 \cdot 3 & 0 \cdot 3 + 1 \cdot 0 \end{bmatrix}
	=
	\begin{bmatrix} 5 & 3 \\ 3 & 0 \end{bmatrix}
	\]
	اولین حاصل‌ضرب داخلی $1 \cdot 2 + 1 \cdot 3 = 5$ است. سه حاصل‌ضرب داخلی دیگر اعداد ۳، ۳ و ۰ را می‌دهند. هر حاصل‌ضرب داخلی به دو عمل ضرب نیاز دارد—بنابراین در کل هشت عمل ضرب انجام می‌شود.
	
	اگر $A$ و $B$ ماتریس‌های $n$ در $n$ باشند، $AB$ نیز $n$ در $n$ خواهد بود. این ماتریس شامل $n^2$ حاصل‌ضرب داخلی (سطر $A$ در ستون $B$) است. هر حاصل‌ضرب داخلی به $n$ عمل ضرب نیاز دارد، بنابراین محاسبه $AB$ از $n^3$ عمل ضرب مجزا استفاده می‌کند. برای $n=100$ ما یک میلیون بار ضرب می‌کنیم. برای $n=2$ ما $n^3=8$ عمل ضرب داریم.
	
	\textit{(توضیح مترجم: تا همین اواخر، ریاضیدانان تصور می‌کردند که ضرب دو ماتریس $2 \times 2$ قطعاً به $2^3 = 8$ عمل ضرب نیاز دارد. سپس روشی پیدا شد که این کار را با ۷ ضرب (و چند جمع اضافی) انجام می‌داد. این ایده با تقسیم ماتریس‌های بزرگ به قطعه‌های $2 \times 2$، تعداد محاسبات برای ضرب ماتریس‌های بزرگ را نیز کاهش داد. به جای $n^3$ عمل ضرب، این تعداد اکنون به حدود $n^{2.81}$ کاهش یافته است. شاید حتی $n^2$ هم ممکن باشد، اما الگوریتم‌ها به قدری پیچیده هستند که در محاسبات علمی معمولاً از همان روش استاندارد $n^3$ استفاده می‌شود.)}
	
	\subsubsection*{مثال ۲}
	فرض کنید $A$ یک بردار سطری (۱ در ۳) و $B$ یک بردار ستونی (۳ در ۱) باشد. آنگاه $AB$ یک ماتریس ۱ در ۱ است (فقط یک درایه، که همان حاصل‌ضرب داخلی است). از طرف دیگر، ضرب $B$ در $A$ (یک ستون در یک سطر) یک ماتریس کامل ۳ در ۳ است. این ضرب مجاز است!
	
	\textbf{ضرب ستون در سطر} ($ (n \times 1)(1 \times n) = (n \times n) $)
	\[
	\begin{bmatrix} 1 \\ 2 \\ 3 \end{bmatrix} \begin{bmatrix} 4 & 5 & 6 \end{bmatrix}
	= \begin{bmatrix}
		1 \cdot 4 & 1 \cdot 5 & 1 \cdot 6 \\
		2 \cdot 4 & 2 \cdot 5 & 2 \cdot 6 \\
		3 \cdot 4 & 3 \cdot 5 & 3 \cdot 6
	\end{bmatrix}
	= \begin{bmatrix}
		4 & 5 & 6 \\
		8 & 10 & 12 \\
		12 & 15 & 18
	\end{bmatrix}
	\]
	ضرب سطر در ستون یک «ضرب داخلی» (inner product) است—این نام دیگری برای حاصل‌ضرب داخلی است. ضرب ستون در سطر یک «ضرب خارجی» (outer product) است. این‌ها موارد حدی از ضرب ماتریسی هستند.
	
	\subsection*{روش‌های دوم و سوم: سطرها و ستون‌ها}
	در تصویر کلی، $A$ در هر ستون از $B$ ضرب می‌شود. نتیجه یک ستون از $AB$ است. در آن ستون، ما در حال ترکیب ستون‌های $A$ هستیم. هر ستون از $AB$ یک ترکیب خطی از ستون‌های $A$ است. این تصویر ستونی از ضرب ماتریسی است:
	
	\textbf{۲. ماتریس $A$ ضربدر هر ستون از $B$}
	\[ A \begin{bmatrix} \mathbf{b}_1 & \mathbf{b}_2 & \dots \end{bmatrix} = \begin{bmatrix} A\mathbf{b}_1 & A\mathbf{b}_2 & \dots \end{bmatrix} \]
	
	تصویر سطری برعکس است. هر سطر از $A$ در کل ماتریس $B$ ضرب می‌شود. نتیجه یک سطر از $AB$ است. هر سطر از $AB$ یک ترکیب خطی از سطرهای $B$ است:
	
	\textbf{۳. هر سطر از $A$ ضربدر ماتریس $B$}
	\[ \begin{bmatrix} \text{سطر ۱ از } A \\ \text{سطر ۲ از } A \\ \vdots \end{bmatrix} B = \begin{bmatrix} (\text{سطر ۱ از } A)B \\ (\text{سطر ۲ از } A)B \\ \vdots \end{bmatrix} \]
	ما عملیات سطری را در حذف (ماتریس $E$ ضربدر $A$) می‌بینیم. به زودی ستون‌ها را در $AA^{-1}=I$ خواهیم دید. «تصویر سطر-ستون» حاصل‌ضرب‌های داخلی سطرها با ستون‌ها را دارد. حاصل‌ضرب‌های داخلی روش معمول برای ضرب ماتریس‌ها با دست است: $mnp$ مرحله مجزای ضرب/جمع.
	\[ AB = (m \times n)(n \times p) = (m \times p) \implies mp \text{ حاصل‌ضرب داخلی که هر کدام } n \text{ مرحله دارند} \quad (۲) \]
	
	\subsection*{روش چهارم: ضرب ستون‌ها در سطرها}
	روش چهارمی برای ضرب ماتریس‌ها وجود دارد. افراد زیادی متوجه اهمیت این روش نیستند.
	
	\textbf{۴. ستون‌های ۱ تا $n$ از $A$ را در سطرهای ۱ تا $n$ از $B$ ضرب کنید. سپس آن ماتریس‌ها را با هم جمع کنید.}
	\[ AB = (\text{ستون ۱})(\text{سطر ۱}) + (\text{ستون ۲})(\text{سطر ۲}) + \dots + (\text{ستون } n)(\text{سطر } n) \]
	ستون ۱ از $A$ در سطر ۱ از $B$ ضرب می‌شود. ستون‌های ۲ و ۳ در سطرهای ۲ و ۳ ضرب می‌شوند. سپس جمع کنید:
	
	اگر من ماتریس‌های ۲ در ۲ را با این روش ستون-سطر ضرب کنم، خواهید دید که $AB$ درست است.
	\[
	AB = \begin{bmatrix} a & b \\ c & d \end{bmatrix} \begin{bmatrix} E & F \\ G & H \end{bmatrix}
	= \begin{bmatrix} a \\ c \end{bmatrix} \begin{bmatrix} E & F \end{bmatrix} + \begin{bmatrix} b \\ d \end{bmatrix} \begin{bmatrix} G & H \end{bmatrix}
	= \begin{bmatrix} aE & aF \\ cE & cF \end{bmatrix} + \begin{bmatrix} bG & bH \\ dG & dH \end{bmatrix}
	= \begin{bmatrix} aE+bG & aF+bH \\ cE+dG & cF+dH \end{bmatrix} \quad (۳)
	\]
	ستون $k$ از $A$ در سطر $k$ از $B$ ضرب می‌شود. این یک ماتریس (و نه فقط یک عدد) تولید می‌کند. سپس شما این ماتریس‌ها را برای $k=1, 2, \dots, n$ با هم جمع می‌کنید تا $AB$ تولید شود. اگر $AB$ از ضرب $(m \times n)$ در $(n \times p)$ به دست آید، آنگاه $n$ ماتریس (ستون)(سطر) خواهیم داشت. همه آنها $m$ در $p$ هستند. این روش از همان $mnp$ مرحله‌ای که در حاصل‌ضرب‌های داخلی وجود دارد استفاده می‌کند - اما با ترتیبی جدید.
	
	\subsection*{قوانین عملیات ماتریسی}
	اجازه دهید شش قانونی را که ماتریس‌ها از آن‌ها پیروی می‌کنند ثبت کنم، در حالی که بر قانونی که از آن پیروی نمی‌کنند تأکید می‌کنم. ماتریس‌ها می‌توانند مربع یا مستطیلی باشند و قوانینی که شامل $A+B$ هستند همگی ساده و برقرارند. در اینجا سه قانون جمع آمده است:
	\begin{itemize}
		\item $A+B = B+A$ (قانون جابجایی)
		\item $c(A+B) = cA+cB$ (قانون توزیع‌پذیری)
		\item $A+(B+C) = (A+B)+C$ (قانون شرکت‌پذیری)
	\end{itemize}
	سه قانون دیگر برای ضرب برقرار است، اما $AB=BA$ یکی از آن‌ها نیست:
	\begin{itemize}
		\item $AB \neq BA$ (قانون جابجایی معمولاً نقض می‌شود)
		\item $A(B+C) = AB+AC$ (قانون توزیع‌پذیری از چپ)
		\item $(A+B)C = AC+BC$ (قانون توزیع‌پذیری از راست)
		\item $A(BC)=(AB)C$ (قانون شرکت‌پذیری برای $ABC$) (پرانتز لازم نیست)
	\end{itemize}
	وقتی $A$ و $B$ مربع نباشند، $AB$ اندازه‌ای متفاوت از $BA$ دارد. این ماتریس‌ها نمی‌توانند برابر باشند—حتی اگر هر دو ضرب مجاز باشند. برای ماتریس‌های مربع، تقریباً هر مثالی نشان می‌دهد که $AB$ با $BA$ متفاوت است:
	\[
	AB = \begin{bmatrix} 0 & 1 \\ 0 & 0 \end{bmatrix} \begin{bmatrix} 0 & 0 \\ 1 & 0 \end{bmatrix} = \begin{bmatrix} 1 & 0 \\ 0 & 0 \end{bmatrix}
	\quad \text{اما} \quad
	BA = \begin{bmatrix} 0 & 0 \\ 1 & 0 \end{bmatrix} \begin{bmatrix} 0 & 1 \\ 0 & 0 \end{bmatrix} = \begin{bmatrix} 0 & 0 \\ 0 & 1 \end{bmatrix}
	\]
	درست است که $AI=IA$. همه ماتریس‌های مربع با $I$ و همچنین با $cI$ جابجا می‌شوند. فقط این ماتریس‌های $cI$ با همه ماتریس‌های دیگر جابجا می‌شوند.
	
	قانون $A(B+C)=AB+AC$ ستون به ستون ثابت می‌شود. با $A(\mathbf{b}+\mathbf{c}) = A\mathbf{b}+A\mathbf{c}$ برای ستون اول شروع کنید. این کلید همه چیز است—خطی بودن.
	
	قانون $A(BC)=(AB)C$ به این معناست که می‌توانید ابتدا $BC$ را ضرب کنید یا $AB$ را. اثبات مستقیم آن تا حدی دشوار است (مسئله ۳۷) اما این قانون بسیار مفید است. ما آن را در بالا برجسته کردیم؛ این کلید روش ضرب ماتریس‌هاست.
	
	به حالت خاصی که $A=B=C$ یک ماتریس مربع باشد نگاه کنید. آنگاه ($A$ ضربدر $A^2$) برابر است با ($A^2$ ضربدر $A$). حاصل‌ضرب در هر دو ترتیب $A^3$ است. توان‌های ماتریس $A^p$ از همان قوانین اعداد پیروی می‌کنند:
	\[ A^p = \underbrace{AAA \cdots A}_{p \text{ عامل}} \]
	$A^p A^q = A^{p+q}$ و $(A^p)^q = A^{pq}$.
	این‌ها قوانین عادی توان‌ها هستند. $A^3$ ضربدر $A^4$ برابر با $A^7$ است. اما توان چهارم $A^3$ برابر با $A^{12}$ است. وقتی $p$ و $q$ صفر یا منفی باشند، این قوانین همچنان برقرارند، به شرطی که $A$ یک «توان منفی ۱» داشته باشد—که همان ماتریس معکوس $A^{-1}$ است. آنگاه $A^0=I$ ماتریس همانی است، در قیاس با $2^0=1$.
	
	برای یک عدد، $a^{-1}$ همان $1/a$ است. برای یک ماتریس، معکوس به صورت $A^{-1}$ نوشته می‌شود (این $I/A$ نیست، مگر در MATLAB). هر عددی به جز $a=0$ معکوس دارد. تصمیم‌گیری در مورد اینکه چه زمانی $A$ معکوس دارد، یک مسئله محوری در جبر خطی است. بخش ۲.۵ پاسخ به این سوال را آغاز خواهد کرد.
	
	\subsection*{ماتریس‌های قطعه‌ای و ضرب قطعه‌ای}
	باید یک چیز دیگر در مورد ماتریس‌ها بگوییم. آن‌ها را می‌توان به قطعات (که ماتریس‌های کوچک‌تری هستند) تقسیم کرد. این کار اغلب به طور طبیعی اتفاق می‌افتد. در اینجا یک ماتریس ۴ در ۶ به قطعاتی با اندازه ۲ در ۲ تقسیم شده است—در این مثال هر قطعه فقط ماتریس همانی $I$ است:
	\[
	A = \left[ \begin{array}{cc|cc|cc}
		1 & 0 & 1 & 0 & 1 & 0 \\
		0 & 1 & 0 & 1 & 0 & 1 \\ \hline
		1 & 0 & 1 & 0 & 1 & 0 \\
		0 & 1 & 0 & 1 & 0 & 1
	\end{array} \right] =
	\begin{bmatrix} I & I & I \\ I & I & I \end{bmatrix}
	\]
	یک ماتریس ۴ در ۶ به یک ماتریس قطعه‌ای ۲ در ۳ تبدیل شده است. اگر $B$ نیز ۴ در ۶ باشد و اندازه‌های قطعات مطابقت داشته باشند، می‌توانید $A+B$ را قطعه به قطعه جمع کنید.
	
	شما قبلاً ماتریس‌های قطعه‌ای را دیده‌اید. بردار سمت راست $\mathbf{b}$ در «ماتریس الحاقی» کنار $A$ قرار گرفت. آنگاه $[A \ \mathbf{b}]$ دو قطعه با اندازه‌های مختلف دارد. ضرب در یک ماتریس حذف، $[EA \ E\mathbf{b}]$ را نتیجه می‌دهد. ضرب قطعه در قطعه، زمانی که ابعادشان اجازه دهد، مشکلی ندارد.
	
	\begin{quote}
		\textbf{ضرب قطعه‌ای} اگر قطعات $A$ بتوانند در قطعات $B$ ضرب شوند، آنگاه ضرب قطعه‌ای $AB$ مجاز است. برش‌های بین ستون‌های $A$ باید با برش‌های بین سطرهای $B$ مطابقت داشته باشند.
		\[
		\begin{bmatrix} A_{11} & A_{12} \\ A_{21} & A_{22} \end{bmatrix}
		\begin{bmatrix} B_{11} & B_{12} \\ B_{21} & B_{22} \end{bmatrix}
		= \begin{bmatrix} A_{11}B_{11}+A_{12}B_{21} & A_{11}B_{12}+A_{12}B_{22} \\ A_{21}B_{11}+A_{22}B_{21} & A_{21}B_{12}+A_{22}B_{22} \end{bmatrix} \quad (۴)
		\]
	\end{quote}
	این معادله همانند زمانی است که قطعات، اعداد بودند (که قطعات ۱ در ۱ هستند). ما مراقب هستیم که $A$ها را جلوی $B$ها نگه داریم، زیرا $BA$ می‌تواند متفاوت باشد.
	
	\textbf{نکته اصلی:} وقتی ماتریس‌ها به قطعات تقسیم می‌شوند، اغلب ساده‌تر است که ببینیم چگونه عمل می‌کنند. ماتریس قطعه‌ای متشکل از $I$ها در بالا بسیار واضح‌تر از ماتریس اصلی ۴ در ۶ است.
	
	\subsubsection*{مثال ۳ (حالت خاص مهم)}
	فرض کنید قطعات $A$ ستون‌های آن و قطعات $B$ سطرهای آن باشند. آنگاه ضرب قطعه‌ای $AB$ حاصل‌ضرب ستون‌ها در سطرها را جمع می‌زند:
	\[ AB = \begin{bmatrix} \mathbf{a}_1 & \mathbf{a}_2 & \dots & \mathbf{a}_n \end{bmatrix} \begin{bmatrix} \mathbf{b}_1^* \\ \mathbf{b}_2^* \\ \vdots \\ \mathbf{b}_n^* \end{bmatrix} = \mathbf{a}_1\mathbf{b}_1^* + \mathbf{a}_2\mathbf{b}_2^* + \dots + \mathbf{a}_n\mathbf{b}_n^* \quad (۵) \]
	این همان قاعده ۴ برای ضرب ماتریس‌هاست.
	
	\subsubsection*{مثال ۴ (حذف با استفاده از قطعات)}
	فرض کنید ستون اول $A$ شامل ۱، ۳ و ۴ باشد. برای تبدیل ۳ و ۴ به ۰، سطر لولا را در ۳ و ۴ ضرب کرده و تفریق می‌کنیم. این عملیات سطری در واقع ضرب در ماتریس‌های حذف $E_{21}$ و $E_{31}$ است. ایده «قطعه‌ای» این است که هر دو حذف را با یک ماتریس $E$ انجام دهیم. آن ماتریس کل ستون اول $A$ را در زیر لولای $a=1$ صفر می‌کند.
	
	حذف قطعه‌ای، سطر اول قطعه‌ای $[A \ B]$ را در $CA^{-1}$ ضرب می‌کند و از سطر دوم کم می‌کند تا یک بلوک صفر در ستون اول ایجاد شود. همچنین از $D$ کم می‌شود تا $S = D - CA^{-1}B$ به دست آید. این همان حذف معمولی است که ستون به ستون با استفاده از قطعات انجام می‌شود. قطعه لولا $A$ است. قطعه نهایی $D - CA^{-1}B$ است، درست مانند $d - (c/a)b$. این عبارت \textbf{مکمل شور (Schur complement)} نامیده می‌شود.
	
	\textit{(توضیح مترجم: مکمل شور کاربردهای فراوانی در علوم مهندسی، آمار و حل دستگاه‌های معادلات خطی بزرگ دارد. این مفهوم اجازه می‌دهد تا با معکوس کردن یک قطعه کوچک‌تر ($A$)، اطلاعاتی درباره معکوس کل ماتریس به دست آوریم و مسئله را به مسائل کوچک‌تر تقسیم کنیم.)}
	
	\subsection*{مروری بر ایده‌های کلیدی}
	\begin{enumerate}
		\item درایه $(i,j)$ از $AB$ برابر است با (سطر $i$ از $A$) $\cdot$ (ستون $j$ از $B$).
		\item یک ماتریس $m \times n$ ضربدر یک ماتریس $n \times p$ از $mnp$ عمل ضرب مجزا استفاده می‌کند.
		\item $A(BC)$ برابر است با $(AB)C$ (بسیار مهم).
		\item $AB$ همچنین برابر با مجموع این $n$ ماتریس است: (ستون $j$ از $A$) ضربدر (سطر $j$ از $B$).
		\item ضرب قطعه‌ای زمانی مجاز است که ابعاد قطعات به درستی با هم مطابقت داشته باشند.
		\item حذف قطعه‌ای مکمل شور $D-CA^{-1}B$ را تولید می‌کند.
	\end{enumerate}
	
	\subsection*{مثال‌های حل شده}
	\subsubsection*{مثال ۲.۴ الف}
	یک گراف یا شبکه $n$ گره دارد. ماتریس مجاورت آن $S$ یک ماتریس $n \times n$ است. این یک ماتریس ۰-۱ است که $s_{ij}=1$ است اگر گره‌های $i$ و $j$ با یک یال به هم متصل باشند. برای گراف‌های بدون جهت، این ماتریس متقارن است.
	
	ماتریس $S^2$ تفسیر مفیدی دارد. درایه $(S^2)_{ij}$ تعداد مسیرهای به طول ۲ بین گره $i$ و گره $j$ را می‌شمارد. برای مثال، بین گره‌های ۲ و ۳ در یک گراف چهارضلعی، دو مسیر به طول ۲ وجود دارد: یکی از طریق گره ۱ و دیگری از طریق گره ۴.
	
	سوال اصلی این است که چرا $S^N$ تعداد تمام مسیرهای به طول $N$ بین دو گره را می‌شمارد. با $S^2$ شروع می‌کنیم و به ضرب ماتریسی با حاصل‌ضرب داخلی نگاه می‌کنیم:
	\[ (S^2)_{ij} = (\text{سطر } i \text{ از } S) \cdot (\text{ستون } j \text{ از } S) = \sum_{k=1}^{n} s_{ik}s_{kj} \]
	اگر یک مسیر ۲ مرحله‌ای $i \to k \to j$ وجود داشته باشد، آنگاه $s_{ik}=1$ و $s_{kj}=1$ است و حاصلضرب $s_{ik}s_{kj}$ برابر ۱ می‌شود. اگر چنین مسیری از طریق گره $k$ وجود نداشته باشد، حداقل یکی از این دو مقدار صفر است و حاصلضرب نیز صفر می‌شود. بنابراین، $(S^2)_{ij}$ با جمع کردن ۱‌ها برای تمام گره‌های میانی ممکن $k$، تعداد کل مسیرهای ۲ مرحله‌ای از $i$ به $j$ را می‌شمارد. به همین ترتیب، $S^N$ مسیرهای به طول $N$ را می‌شمارد، زیرا ضرب ماتریسی دقیقاً برای شمارش مسیرها در یک گراف مناسب است.
	
	\subsubsection*{مثال ۲.۴ ب}
	برای ماتریس‌های زیر، چه زمانی $AB=BA$؟ چه زمانی $BC=CB$؟ چه زمانی $A(BC)$ برابر با $(AB)C$ است؟ شرایط را بر حسب درایه‌های $p, q, r, z$ بیان کنید.
	\[ A = \begin{bmatrix} p & q \\ 0 & r \end{bmatrix}, \quad B = \begin{bmatrix} 1 & 1 \\ 0 & 1 \end{bmatrix}, \quad C = \begin{bmatrix} 1 & z \\ 0 & 1 \end{bmatrix} \]
	اگر $p,q,r,z$ به جای اعداد، قطعات $4 \times 4$ باشند، آیا پاسخ‌ها تغییر می‌کنند؟
	
	\textbf{راه حل} \\
	اول از همه، $A(BC)$ همیشه برابر با $(AB)C$ است. پرانتزها لازم نیستند: $A(BC)=(AB)C=ABC$. اما باید ماتریس‌ها را به همین ترتیب نگه داریم.
	
	معمولاً $AB \neq BA$:
	\[ AB = \begin{bmatrix} p & p+q \\ 0 & r \end{bmatrix}, \quad BA = \begin{bmatrix} p & q+r \\ 0 & r \end{bmatrix} \implies AB=BA \text{ اگر و تنها اگر } p+q=q+r \text{ یعنی } p=r. \]
	
	برحسب اتفاق $BC=CB$:
	\[ BC = \begin{bmatrix} 1 & z+1 \\ 0 & 1 \end{bmatrix}, \quad CB = \begin{bmatrix} 1 & 1+z \\ 0 & 1 \end{bmatrix} \implies BC=CB \text{ همیشه برقرار است.} \]
	بخشی از توضیح این است که قطر اصلی $B$ و $C$ ماتریس همانی است که با بسیاری از ماتریس‌ها جابجا می‌شود.
	
	اگر $p,q,r,z$ قطعات $4 \times 4$ باشند و عدد ۱ به ماتریس همانی $I$ تبدیل شود، تمام این حاصل‌ضرب‌ها صحیح باقی می‌مانند. بنابراین پاسخ‌ها یکسان هستند.
	
	\newpage
	\section*{مجموعه مسائل ۲.۴}
	\subsection*{مسائل ۱-۳۸}
	\begin{enumerate}
		\item \textbf{(مسائل ۱-۱۶ درباره قوانین ضرب ماتریسی هستند.)} \\
		$A$ ماتریس ۳ در ۵، $B$ ماتریس ۵ در ۳، $C$ ماتریس ۵ در ۱ و $D$ ماتریس ۳ در ۱ است. همه درایه‌ها ۱ هستند. کدام یک از این عملیات ماتریسی مجاز است و نتایج چه هستند؟
		\[ BA \quad AB \quad ABD \quad DC \]
		
		\item برای یافتن موارد زیر چه سطرها یا ستون‌ها یا ماتریس‌هایی را ضرب می‌کنید؟
		\begin{itemize}
			\item[(الف)] ستون دوم $AB$؟
			\item[(ب)] سطر اول $AB$؟
			\item[(ج)] درایه سطر ۳، ستون ۵ از $AB$؟
			\item[(د)] درایه سطر ۱، ستون ۱ از $CDE$؟
		\end{itemize}
		
		\item $AB$ را به $AC$ اضافه کرده و با $A(B+C)$ مقایسه کنید:
		\[ A = \begin{bmatrix} 1 & 2 \\ 3 & 4 \end{bmatrix}, \quad B = \begin{bmatrix} 1 & 0 \\ 0 & 1 \end{bmatrix}, \quad C = \begin{bmatrix} 2 & 3 \\ 1 & 0 \end{bmatrix} \]
		
		\item در مسئله ۳، $A$ را در $BC$ ضرب کنید. سپس $AB$ را در $C$ ضرب کنید.
		
		\item $A^2$ و $A^3$ را محاسبه کنید. برای $A^5$ و $A^n$ پیش‌بینی کنید:
		\[ A = \begin{bmatrix} 1 & 1 \\ 0 & 1 \end{bmatrix} \quad \text{و} \quad A = \begin{bmatrix} 2 & 2 \\ 0 & 0 \end{bmatrix} \]
		
		\item نشان دهید که $(A+B)^2$ با $A^2+2AB+B^2$ متفاوت است، وقتی:
		\[ A = \begin{bmatrix} 1 & 0 \\ 0 & 0 \end{bmatrix} \quad \text{و} \quad B = \begin{bmatrix} 0 & 1 \\ 0 & 0 \end{bmatrix} \]
		قانون صحیح برای $(A+B)(A+B) = A^2 + \_\_\_ + B^2$ را بنویسید.
		
		\item درست یا غلط. در صورت غلط بودن یک مثال خاص بزنید:
		\begin{itemize}
			\item[(الف)] اگر ستون‌های ۱ و ۳ از $B$ یکسان باشند، ستون‌های ۱ و ۳ از $AB$ نیز یکسانند.
			\item[(ب)] اگر سطرهای ۱ و ۳ از $A$ یکسان باشند، سطرهای ۱ و ۳ از $AB$ نیز یکسانند.
			\item[(ج)] اگر سطرهای ۱ و ۳ از $A$ یکسان باشند، سطرهای ۱ و ۳ از $ABC$ نیز یکسانند.
			\item[(د)] $(AB)^2 = A^2B^2$.
		\end{itemize}
		
		\item هر سطر از $DA$ و $EA$ چگونه به سطرهای $A$ مرتبط است، وقتی:
		\[ D = \begin{bmatrix} 3 & 0 \\ 0 & 5 \end{bmatrix}, \quad E = \begin{bmatrix} 0 & 1 \\ 0 & 0 \end{bmatrix}, \quad A = \begin{bmatrix} a & b \\ c & d \end{bmatrix} \]
		هر ستون از $AD$ و $AE$ چگونه به ستون‌های $A$ مرتبط است؟
		
		\item سطر ۱ از $A$ به سطر ۲ اضافه می‌شود. این $EA$ را در زیر نتیجه می‌دهد. سپس ستون ۱ از $EA$ به ستون ۲ اضافه می‌شود تا $(EA)F$ تولید شود:
		\[ E = \begin{bmatrix} 1 & 0 \\ 1 & 1 \end{bmatrix}, \quad A = \begin{bmatrix} a & b \\ c & d \end{bmatrix}, \quad F = \begin{bmatrix} 1 & 1 \\ 0 & 1 \end{bmatrix} \]
		\[ EA = \begin{bmatrix} a & b \\ a+c & b+d \end{bmatrix}, \quad (EA)F = \begin{bmatrix} a & a+b \\ a+c & a+c+b+d \end{bmatrix} \]
		\begin{itemize}
			\item[(الف)] این مراحل را به ترتیب مخالف انجام دهید. ابتدا ستون ۱ از $A$ را به ستون ۲ با $AF$ اضافه کنید، سپس سطر ۱ از $AF$ را به سطر ۲ با $E(AF)$ اضافه کنید.
			\item[(ب)] با $(EA)F$ مقایسه کنید. چه قانونی توسط ضرب ماتریسی رعایت می‌شود؟
		\end{itemize}
		
		\item سطر ۱ از $A$ دوباره به سطر ۲ اضافه می‌شود تا $EA$ تولید شود. سپس $F$ سطر ۲ از $EA$ را به سطر ۱ اضافه می‌کند. نتیجه $F(EA)$ است:
		\[ F(EA) = \begin{bmatrix} 1 & 1 \\ 0 & 1 \end{bmatrix} \begin{bmatrix} a & b \\ a+c & b+d \end{bmatrix} = \begin{bmatrix} 2a+c & 2b+d \\ a+c & b+d \end{bmatrix} \]
		\begin{itemize}
			\item[(الف)] این مراحل را به ترتیب مخالف انجام دهید: ابتدا سطر ۲ را به سطر ۱ با $FA$ اضافه کنید، سپس سطر ۱ از $FA$ را به سطر ۲ اضافه کنید.
			\item[(ب)] چه قانونی توسط ضرب ماتریسی رعایت می‌شود یا نمی‌شود؟
		\end{itemize}
		
		\item (ماتریس‌های ۳ در ۳) تنها ماتریس $B$ را انتخاب کنید به طوری که برای هر ماتریس $A$:
		\begin{itemize}
			\item[(الف)] $BA = 4A$
			\item[(ب)] $BA = 4B$
			\item[(ج)] $BA$ سطرهای ۱ و ۳ از $A$ را معکوس کرده و سطر ۲ بدون تغییر است
			\item[(د)] همه سطرهای $BA$ مانند سطر ۱ از $A$ هستند.
		\end{itemize}
		
		\item فرض کنید $AB=BA$ و $AC=CA$ برای این دو ماتریس خاص $B$ و $C$:
		\[ A = \begin{bmatrix} a & b \\ c & d \end{bmatrix} \text{ با } B = \begin{bmatrix} 1 & 0 \\ 0 & 0 \end{bmatrix} \text{ و } C = \begin{bmatrix} 0 & 1 \\ 0 & 0 \end{bmatrix} \text{ جابجا می‌شود.} \]
		ثابت کنید که $a=d$ و $b=c=0$. آنگاه $A$ مضربی از $I$ است. تنها ماتریس‌هایی که با $B$ و $C$ و همه ماتریس‌های ۲ در ۲ دیگر جابجا می‌شوند، $A$ مضربی از $I$ است.
		
		\item کدام یک از ماتریس‌های زیر تضمین می‌شود که با $(A-B)^2$ برابر باشند:
		$A^2 - B^2$, $(B-A)^2$, $A^2 - 2AB + B^2$, $A(A-B) - B(A-B)$, $A^2 - AB - BA + B^2$؟
		
		\item درست یا غلط:
		\begin{itemize}
			\item[(الف)] اگر $A^2$ تعریف شده باشد، $A$ لزوماً مربع است.
			\item[(ب)] اگر $AB$ و $BA$ تعریف شده باشند، $A$ و $B$ مربع هستند.
			\item[(ج)] اگر $AB$ و $BA$ تعریف شده باشند، $AB$ و $BA$ مربع هستند.
			\item[(د)] اگر $AB=B$ آنگاه $A=I$.
		\end{itemize}
		
		\item اگر $A$ ماتریس $m \times n$ باشد، چه تعداد عمل ضرب مجزا در موارد زیر وجود دارد؟
		\begin{itemize}
			\item[(الف)] $A$ یک بردار $x$ با $n$ مؤلفه را ضرب می‌کند؟
			\item[(ب)] $A$ یک ماتریس $n \times p$ به نام $B$ را ضرب می‌کند؟
			\item[(ج)] $A$ خودش را ضرب می‌کند تا $A^2$ را تولید کند؟ اینجا $m=n$.
		\end{itemize}
		
		\item \textbf{(مسائل ۱۷-۲۰ از نماد $a_{ij}$ استفاده می‌کنند.)}\\
		برای $A = \begin{bmatrix} 2 & -1 \\ 3 & 4 \end{bmatrix}$ و $B = \begin{bmatrix} 0 & 1 & 2 \\ 5 & 6 & 7 \end{bmatrix}$، این پاسخ‌ها را محاسبه کنید و نه بیشتر:
		\begin{itemize}
			\item[(الف)] ستون دوم $AB$
			\item[(ب)] سطر دوم $AB$
			\item[(ج)] سطر دوم $AA = A^2$
			\item[(د)] سطر دوم $AAA = A^3$.
		\end{itemize}
		
		\item ماتریس ۳ در ۳ $A$ را بنویسید که درایه‌های آن عبارتند از:
		\begin{itemize}
			\item[(الف)] $a_{ij} = \min(i, j)$
			\item[(ب)] $a_{ij} = (-1)^{i+j}$
			\item[(ج)] $a_{ij} = i/j$.
		\end{itemize}
		
		\item برای توصیف هر یک از این کلاس‌های ماتریس از چه کلماتی استفاده می‌کنید؟ یک مثال ۳ در ۳ در هر کلاس ارائه دهید. کدام ماتریس به هر چهار کلاس تعلق دارد؟
		\begin{itemize}
			\item[(الف)] $a_{ij} = 0$ اگر $i \neq j$ (قطری)
			\item[(ب)] $a_{ij} = 0$ اگر $i < j$ (پایین مثلثی)
			\item[(ج)] $a_{ij} = a_{ji}$ (متقارن)
			\item[(د)] $a_{ij} = -a_{ji}$ (پادمتقارن)
		\end{itemize}
		
		\item درایه‌های $A$ برابر $a_{ij}$ هستند. با فرض اینکه صفری ظاهر نشود، موارد زیر چیست؟
		\begin{itemize}
			\item[(الف)] اولین لولا؟
			\item[(ب)] مضرب $l_{31}$ از سطر ۱ که باید از سطر ۳ کم شود؟
			\item[(ج)] درایه جدیدی که پس از آن تفریق جایگزین $a_{32}$ می‌شود؟
			\item[(د)] لولای دوم؟
		\end{itemize}
		
		\item \textbf{(مسائل ۲۱-۲۴ شامل توان‌های ماتریس A هستند.)}\\
		$A^2, A^3, A^4$ و همچنین $A\mathbf{v}, A^2\mathbf{v}, A^3\mathbf{v}, A^4\mathbf{v}$ را برای موارد زیر محاسبه کنید:
		\[ A = \begin{bmatrix} 0 & 1 & 1 & 0 \\ 0 & 0 & 1 & 0 \\ 0 & 0 & 0 & 1 \\ 0 & 0 & 0 & 0 \end{bmatrix}, \quad \mathbf{v} = \begin{bmatrix} 1 \\ 1 \\ 1 \\ 1 \end{bmatrix} \]
		
		\item با آزمون و خطا ماتریس‌های حقیقی غیرصفر ۲ در ۲ پیدا کنید به طوری که:
		\[ A^2 = -I \quad BC=0 \quad DE = -ED \text{ (بدون اینکه } DE=0 \text{ باشد)} \]
		
		\item (الف) یک ماتریس غیرصفر $A$ پیدا کنید که $A^2=0$ باشد.
		(ب) یک ماتریس پیدا کنید که $A^2 \neq 0$ باشد اما $A^3=0$.
		
		\item با آزمایش برای $n=2$ و $n=3$، $A^n$ را برای این ماتریس‌ها پیش‌بینی کنید:
		\[ A = \begin{bmatrix} 1 & b \\ 0 & 1 \end{bmatrix} \quad \text{و} \quad A = \begin{bmatrix} 2 & 1 \\ 0 & 1 \end{bmatrix} \]
		
		\item \textbf{(مسائل ۲۵-۳۱ از ضرب ستون-سطر و ضرب قطعه‌ای استفاده می‌کنند.)}\\
		$A$ را در $I$ با استفاده از (ستون‌های $A$) ضربدر (سطرهای $I$) ضرب کنید (برای ماتریس ۳ در ۳).
		
		\item $AB$ را با استفاده از ضرب ستون در سطر محاسبه کنید:
		\[ AB = \begin{bmatrix} 1 & 0 \\ 2 & 1 \end{bmatrix} \begin{bmatrix} 3 & 3 & 0 \\ 1 & 2 & 1 \end{bmatrix} = \begin{bmatrix} 1 \\ 2 \end{bmatrix} [3 \ 3 \ 0] + \begin{bmatrix} 0 \\ 1 \end{bmatrix} [1 \ 2 \ 1] = \_\_\_ \]
		
		\item نشان دهید که حاصلضرب ماتریس‌های بالا مثلثی همیشه بالا مثلثی است:
		\[ AB = \begin{bmatrix} x & x & x \\ 0 & x & x \\ 0 & 0 & x \end{bmatrix} \begin{bmatrix} x & x & x \\ 0 & x & x \\ 0 & 0 & x \end{bmatrix} = \begin{bmatrix} \_ & \_ & \_ \\ \_ & \_ & \_ \\ \_ & \_ & \_ \end{bmatrix} \]
		\textbf{اثبات با استفاده از حاصل‌ضرب داخلی (سطر در ستون):} (سطر ۲ از A) $\cdot$ (ستون ۱ از B) = ۰. کدام حاصل‌ضرب‌های داخلی دیگر صفر می‌دهند؟ \\
		\textbf{اثبات با استفاده از ماتریس‌های کامل (ستون در سطر):} در (ستون ۲ از A) ضربدر (سطر ۲ از B) درایه‌های صفر و غیرصفر (x) را رسم کنید. همچنین (ستون ۳ از A) ضربدر (سطر ۳ از B) را نشان دهید.
		
		\item برش‌ها را در $A$ (۲ در ۳) و $B$ (۳ در ۴) و $AB$ رسم کنید تا نشان دهید چگونه هر یک از چهار قانون ضرب در واقع یک ضرب قطعه‌ای است:
		\begin{itemize}
			\item[(1)] ماتریس $A$ ضربدر ستون‌های $B$. (ستون‌های $AB$)
			\item[(2)] سطرهای $A$ ضربدر ماتریس $B$. (سطرهای $AB$)
			\item[(3)] سطرهای $A$ ضربدر ستون‌های $B$. (حاصل‌ضرب‌های داخلی (اعداد در $AB$))
			\item[(4)] ستون‌های $A$ ضربدر سطرهای $B$. (حاصل‌ضرب‌های خارجی (ماتریس‌هایی که جمعشان $AB$ می‌شود))
		\end{itemize}
		
		\item کدام ماتریس‌های $E_{21}$ و $E_{31}$ در مکان‌های (۲,۱) و (۳,۱) از $E_{21}A$ و $E_{31}A$ صفر تولید می‌کنند؟
		\[ A = \begin{bmatrix} 1 & 2 & 3 \\ 4 & 5 & 6 \\ 7 & 8 & 9 \end{bmatrix} \]
		ماتریس واحد $E = E_{31}E_{21}$ را پیدا کنید که هر دو صفر را به یکباره تولید می‌کند. $EA$ را ضرب کنید.
		
		\item ضرب قطعه‌ای می‌گوید که ستون ۱ با این عملیات حذف می‌شود:
		\[ EA = \begin{bmatrix} 1 & 0 \\ -\mathbf{c}a^{-1} & I \end{bmatrix} \begin{bmatrix} a & \mathbf{b} \\ \mathbf{c} & D \end{bmatrix} = \begin{bmatrix} a & \mathbf{b} \\ 0 & D-\mathbf{c}a^{-1}\mathbf{b} \end{bmatrix} \]
		در مسئله ۲۹، چه چیزی در $\mathbf{c}$ و $D$ قرار می‌گیرد و $D-\mathbf{c}a^{-1}\mathbf{b}$ چیست؟
		
		\item با $i^2=-1$، حاصلضرب $(A+iB)$ و $(x+iy)$ برابر است با $Ax+iBx+iAy-By$. از قطعات برای جدا کردن بخش حقیقی (بدون $i$) از بخش موهومی (که در $i$ ضرب می‌شود) استفاده کنید:
		\[ \begin{bmatrix} A & -B \\ B & A \end{bmatrix} \begin{bmatrix} \mathbf{x} \\ \mathbf{y} \end{bmatrix} = \begin{bmatrix} A\mathbf{x}-B\mathbf{y} \\ B\mathbf{x}+A\mathbf{y} \end{bmatrix} \quad \begin{array}{l} \leftarrow \text{بخش حقیقی} \\ \leftarrow \text{بخش موهومی} \end{array} \]
		
		\item \textbf{(مسائل ۳۲-۳۸ مسائل چالشی هستند.)}\\
		(بسیار مهم) فرض کنید شما $A\mathbf{x}=\mathbf{b}$ را برای سه سمت راست خاص $\mathbf{b}$ حل می‌کنید:
		\[ A\mathbf{x}_1 = \begin{bmatrix} 1 \\ 0 \\ 0 \end{bmatrix} \quad A\mathbf{x}_2 = \begin{bmatrix} 0 \\ 1 \\ 0 \end{bmatrix} \quad A\mathbf{x}_3 = \begin{bmatrix} 0 \\ 0 \\ 1 \end{bmatrix} \]
		اگر سه جواب $\mathbf{x}_1, \mathbf{x}_2, \mathbf{x}_3$ ستون‌های ماتریس $X$ باشند، $A$ ضربدر $X$ چیست؟
		
		\item اگر سه جواب در مسئله ۳۲ عبارتند از $\mathbf{x}_1=(1,1,1)^T$، $\mathbf{x}_2=(0,1,1)^T$ و $\mathbf{x}_3=(0,0,1)^T$، معادله $A\mathbf{x}=\mathbf{b}$ را وقتی $\mathbf{b}=(3,5,8)^T$ است حل کنید. مسئله چالشی: $A$ چیست؟
		
		\item همه ماتریس‌های $A = \begin{bmatrix} a & b \\ c & d \end{bmatrix}$ را پیدا کنید که در رابطه $A \begin{bmatrix} 1 & 1 \\ 1 & 1 \end{bmatrix} = \begin{bmatrix} 1 & 1 \\ 1 & 1 \end{bmatrix} A$ صدق کنند.
		
		\item فرض کنید یک «گراف دایره‌ای» ۴ گره دارد که با یال‌هایی در اطراف یک دایره (در هر دو جهت) به هم متصل شده‌اند. ماتریس مجاورت $S$ آن از مثال حل شده ۲.۴ الف چیست؟ $S^2$ چیست؟ تمام مسیرهای ۲ مرحله‌ای پیش‌بینی شده توسط $S^2$ را پیدا کنید.
		
		\item \textbf{سوال کاربردی:} فرض کنید $A$ ماتریس $m \times n$، $B$ ماتریس $n \times p$ و $C$ ماتریس $p \times q$ باشد. آنگاه تعداد ضرب‌ها برای $(AB)C$ برابر $mnp+mpq$ است. همان ماتریس از $A(BC)$ با $npq+mnq$ عمل ضرب مجزا به دست می‌آید.
		\begin{itemize}
			\item[(الف)] اگر $A$ ماتریس $2 \times 4$، $B$ ماتریس $4 \times 7$ و $C$ ماتریس $7 \times 10$ باشد، آیا شما $(AB)C$ را ترجیح می‌دهید یا $A(BC)$؟
			\item[(ب)] با بردارهای $N$-مؤلفه‌ای، آیا $(\mathbf{u}^T\mathbf{v})\mathbf{w}^T$ را انتخاب می‌کنید یا $\mathbf{u}^T(\mathbf{v}\mathbf{w}^T)$؟
			\item[(ج)] با تقسیم بر $mnpq$ نشان دهید که $(AB)C$ سریع‌تر است وقتی $n^{-1}+q^{-1} < m^{-1}+p^{-1}$.
		\end{itemize}
		
		\item برای اثبات اینکه $(AB)C = A(BC)$، از بردارهای ستونی $\mathbf{b}_1, \dots, \mathbf{b}_n$ از $B$ استفاده کنید. ابتدا فرض کنید $C$ تنها یک ستون $\mathbf{c}$ با درایه‌های $c_1, \dots, c_n$ دارد:
		$AB$ ستون‌های $A\mathbf{b}_1, \dots, A\mathbf{b}_n$ را دارد و سپس $(AB)\mathbf{c}$ برابر است با $c_1A\mathbf{b}_1 + \dots + c_nA\mathbf{b}_n$.
		$B\mathbf{c}$ یک ستون $c_1\mathbf{b}_1 + \dots + c_n\mathbf{b}_n$ دارد و سپس $A(B\mathbf{c})$ برابر است با $A(c_1\mathbf{b}_1 + \dots + c_n\mathbf{b}_n)$.
		خطی بودن برابری این دو مجموع را نتیجه می‌دهد. این $(AB)\mathbf{c} = A(B\mathbf{c})$ را ثابت می‌کند. همین امر برای تمام ستون‌های دیگر $C$ نیز صادق است. بنابراین $(AB)C=A(BC)$. این را بر روی معکوس‌ها اعمال کنید:
		اگر $BA=I$ و $AC=I$ باشد، ثابت کنید که معکوس چپ $B$ برابر با معکوس راست $C$ است.
		
		\item (الف) فرض کنید $A$ سطرهای $\mathbf{a}_1^T, \dots, \mathbf{a}_m^T$ را دارد. چرا $A^TA$ برابر با $\mathbf{a}_1\mathbf{a}_1^T + \dots + \mathbf{a}_m\mathbf{a}_m^T$ است؟
		(ب) اگر $C$ یک ماتریس قطری با $c_1, \dots, c_m$ روی قطرش باشد، یک مجموع مشابه از ستون‌ها ضربدر سطرها برای $A^TCA$ پیدا کنید. ابتدا یک مثال با $m=n=2$ انجام دهید.
	\end{enumerate}
	
\end{document}