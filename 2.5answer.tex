%%%%%%%%%%%%%%%%%%%%%%%%%%%%%%%%%%%%%%%%%%%%%%%%%%%%
%           LaTeX Code for Problem Set 2.5           %
%             Translated to Persian (Farsi)          %
%%%%%%%%%%%%%%%%%%%%%%%%%%%%%%%%%%%%%%%%%%%%%%%%%%%%

\documentclass[12pt,a4paper]{article}

% --- Preamble ---
\usepackage{xepersian}
\settextfont{XB Niloofar}
\setdigitfont{XB Niloofar}

\usepackage{amsmath}
\usepackage{amssymb}
\usepackage{amsfonts}
\usepackage[left=2.5cm, right=2.5cm, top=2.5cm, bottom=2.5cm]{geometry}

% Title setup
\title{ترجمه پاسخنامه مجموعه مسائل ۲.۵}
\author{صفحه ۹۲}
\date{}


% --- Document Body ---
\begin{document}
	\maketitle
	\RTL{
		
		\section*{مجموعه مسائل ۲.۵، صفحه ۹۲}
		\begin{enumerate}
			\item $A^{-1}= \begin{bmatrix} 0 & 1/4 \\ 1/3 & 0 \end{bmatrix}$ و $B^{-1}= \begin{bmatrix} 1/2 & 0 \\ -1 & 1/2 \end{bmatrix}$ و $C^{-1}= \begin{bmatrix} 7 & -4 \\ -5 & 3 \end{bmatrix}$.
			
			\item برای ماتریس اول، یک جابجایی ساده سطرها $P^2=I$ را نتیجه می‌دهد، بنابراین $P^{-1}=P$. برای ماتریس دوم، $P^{-1} = \begin{bmatrix} 0 & 0 & 1 \\ 1 & 0 & 0 \\ 0 & 1 & 0 \end{bmatrix}$. همیشه $P^{-1}$ برابر با «ترانهاده»ی P است که در بخش ۲.۷ خواهد آمد.
			
			\item معادلات به صورت $\begin{bmatrix} x \\ y \end{bmatrix} = \begin{bmatrix} 0.5 \\ -0.2 \end{bmatrix}$ و $\begin{bmatrix} t \\ z \end{bmatrix} = \begin{bmatrix} -0.2 \\ 0.1 \end{bmatrix}$ هستند، بنابراین $A^{-1} = \frac{1}{10}\begin{bmatrix} 5 & -2 \\ -2 & 1 \end{bmatrix}$. این سوال $AA^{-1}=I$ را ستون به ستون حل کرد که ایده اصلی حذف گاوس-جردن است. برای یک ماتریس متفاوت مانند $A=\begin{bmatrix} 1 & 1 \\ 0 & 0 \end{bmatrix}$، می‌توانستید ستون اول $A^{-1}$ را پیدا کنید اما نه ستون دوم را - بنابراین A منفرد (بدون معکوس) خواهد بود.
			
			\item معادلات $x+2y=1$ و $3x+6y=0$ هستند. جوابی وجود ندارد زیرا ۳ برابر معادله اول، $3x+6y=3$ را نتیجه می‌دهد.
		\end{enumerate}
		
		
		\begin{enumerate}
			\setcounter{enumi}{4}
			\item یک ماتریس بالا مثلثی U که در آن $U^2=I$ باشد، برای هر مقدار a به صورت $U=\begin{bmatrix} 1 & a \\ 0 & -1 \end{bmatrix}$ است. و همچنین $-U$.
			
			\item (الف) $AB=AC$ را از سمت چپ در $A^{-1}$ ضرب کنید تا $B=C$ به دست آید (زیرا A معکوس‌پذیر است). (ب) تا زمانی که $B-C$ به شکل $\begin{bmatrix} x & y \\ -x & -y \end{bmatrix}$ باشد، برای ماتریس $A=\begin{bmatrix} 1 & 1 \\ 1 & 1 \end{bmatrix}$ خواهیم داشت $AB=AC$.
			
			\item (الف) در $Ax=(1,0,0)$، معادله ۱ + معادله ۲ - معادله ۳ برابر با $0=1$ است. (ب) سمت راست باید در $b_1+b_2=b_3$ صدق کند. (ج) سطر ۳ به یک سطر صفر تبدیل می‌شود — محور سوم وجود ندارد.
			
			\item (الف) بردار $x=(1,1,-1)$ معادله $Ax=0$ را حل می‌کند. (ب) پس از حذف، ستون‌های ۱ و ۲ به صفر ختم می‌شوند. در نتیجه ستون ۳ که برابر با ستون ۱ + ۲(ستون ۲) است نیز چنین است: محور سوم وجود ندارد.
			
			\item بله، B معکوس‌پذیر است (A فقط در یک ماتریس جایگشت P ضرب شده است). اگر سطرهای ۱ و ۲ از A را برای رسیدن به B جابجا کنید، ستون‌های ۱ و ۲ از $A^{-1}$ را برای رسیدن به $B^{-1}$ جابجا می‌کنید. در نماد ماتریسی، $B=PA$ دارای $B^{-1}=A^{-1}P^{-1}=A^{-1}P$ برای این P است.
			
			\item $A^{-1} = \begin{bmatrix} 0 & 0 & 0 & 1/5 \\ 0 & 0 & 1/4 & 0 \\ 0 & 1/3 & 0 & 0 \\ 1/2 & 0 & 0 & 0 \end{bmatrix}$ و $B^{-1} = \begin{bmatrix} 3 & -2 & 0 & 0 \\ -4 & 3 & 0 & 0 \\ 0 & 0 & 6 & -5 \\ 0 & 0 & -7 & 6 \end{bmatrix}$ (هر بلوک از B را معکوس کنید).
			
			\item (الف) اگر $B=-A$ باشد، آنگاه قطعاً $A+B=$ ماتریس صفر، معکوس‌پذیر نیست.
			(ب) $A=\begin{bmatrix} 1 & 0 \\ 0 & 0 \end{bmatrix}$ و $B=\begin{bmatrix} 0 & 0 \\ 0 & 1 \end{bmatrix}$ هر دو منفرد هستند اما $A+B=I$ معکوس‌پذیر است.
			
			\item $C=AB$ را از سمت چپ در $A^{-1}$ و از سمت راست در $C^{-1}$ ضرب کنید. آنگاه $A^{-1}=BC^{-1}$.
			
			\item $M^{-1} = C^{-1}B^{-1}A^{-1}$ بنابراین از سمت چپ در C و از سمت راست در A ضرب کنید: $B^{-1}=CM^{-1}A$.
			
			\item $B^{-1} = A^{-1} \begin{bmatrix} 1 & 0 \\ 1 & 1 \end{bmatrix}^{-1} = A^{-1} \begin{bmatrix} 1 & 0 \\ -1 & 1 \end{bmatrix}$: ستون دوم $A^{-1}$ را از ستون اول آن کم کنید.
			
			\item اگر A یک ستون صفر داشته باشد، BA نیز چنین است. در این صورت $BA=I$ غیرممکن است. $A^{-1}$ وجود ندارد.
		\end{enumerate}
		
		
		\begin{enumerate}
			\setcounter{enumi}{15}
			\item $\begin{bmatrix} a & b \\ c & d \end{bmatrix} \begin{bmatrix} d & -b \\ -c & a \end{bmatrix} = \begin{bmatrix} ad-bc & 0 \\ 0 & ad-bc \end{bmatrix}$. معکوس هر ماتریس برابر با دیگری تقسیم بر $ad-bc$ است.
			
			\item $E_{32}E_{31}E_{21} = \begin{bmatrix} 1 & & \\ & 1 & \\ & -1 & 1 \end{bmatrix} \begin{bmatrix} 1 & & \\ & 1 & \\ -1 & & 1 \end{bmatrix} \begin{bmatrix} 1 & & \\ -1 & 1 & \\ & & 1 \end{bmatrix} = \begin{bmatrix} 1 & & \\ -1 & 1 & \\ 0 & -1 & 1 \end{bmatrix} = E$. ترتیب را معکوس کرده و علامت‌ها را عوض کنید تا معکوس‌ها به دست آیند: $E_{21}^{-1}E_{31}^{-1}E_{32}^{-1} = \begin{bmatrix} 1 & & \\ 1 & 1 & \\ 1 & 1 & 1 \end{bmatrix} = L=E^{-1}$. توجه کنید که با ضرب معکوس‌ها در این ترتیب، ۱ ها بدون تغییر باقی می‌مانند.
			
			\item $A^2B=I$ را می‌توان به صورت $A(AB)=I$ نیز نوشت. بنابراین، $A^{-1}$ برابر با $AB$ است.
			
			\item درایه (۱,۱) نیاز به $4a-3b=1$ دارد؛ درایه (۱,۲) نیاز به $2b-a=0$ دارد. آنگاه $b=1/5$ و $a=2/5$. برای حالت ۵×۵، $5a-4b=1$ و $2b=a$ نتیجه می‌دهد $b=1/6$ و $a=2/6$.
			
			\item $A \times \text{ones}(4,1) = A(\text{ستون یک‌ها})$ برابر با بردار صفر است، بنابراین A نمی‌تواند معکوس‌پذیر باشد.
			
			\item شش ماتریس از شانزده ماتریس ۰-۱ معکوس‌پذیر هستند: I، P و هر چهار ماتریسی که سه درایه ۱ دارند.
			
			\item $\begin{bmatrix} 1 & 3 & | & 1 & 0 \\ 2 & 7 & | & 0 & 1 \end{bmatrix} \to \begin{bmatrix} 1 & 3 & | & 1 & 0 \\ 0 & 1 & | & -2 & 1 \end{bmatrix} \to \begin{bmatrix} 1 & 0 & | & 7 & -3 \\ 0 & 1 & | & -2 & 1 \end{bmatrix} = [I | A^{-1}]$.
			$\begin{bmatrix} 1 & 4 & | & 1 & 0 \\ 3 & 9 & | & 0 & 1 \end{bmatrix} \to \begin{bmatrix} 1 & 4 & | & 1 & 0 \\ 0 & -3 & | & -3 & 1 \end{bmatrix} \to \begin{bmatrix} 1 & 0 & | & -3 & 4/3 \\ 0 & 1 & | & 1 & -1/3 \end{bmatrix} = [I | A^{-1}]$.
			
			\item $[A|I] = \begin{bmatrix} 2 & 1 & 0 & | & 1 & 0 & 0 \\ 1 & 2 & 1 & | & 0 & 1 & 0 \\ 0 & 1 & 2 & | & 0 & 0 & 1 \end{bmatrix} \to \dots \to \begin{bmatrix} 1 & 0 & 0 & | & 3/4 & -1/2 & 1/4 \\ 0 & 1 & 0 & | & -1/2 & 1 & -1/2 \\ 0 & 0 & 1 & | & 1/4 & -1/2 & 3/4 \end{bmatrix} = [I | A^{-1}]$.
			
			\item $\begin{bmatrix} 1 & a & b & | & 1 & 0 & 0 \\ 0 & 1 & c & | & 0 & 1 & 0 \\ 0 & 0 & 1 & | & 0 & 0 & 1 \end{bmatrix} \to \begin{bmatrix} 1 & a & 0 & | & 1 & 0 & -b \\ 0 & 1 & 0 & | & 0 & 1 & -c \\ 0 & 0 & 1 & | & 0 & 0 & 1 \end{bmatrix} \to \begin{bmatrix} 1 & 0 & 0 & | & 1 & -a & ac-b \\ 0 & 1 & 0 & | & 0 & 1 & -c \\ 0 & 0 & 1 & | & 0 & 0 & 1 \end{bmatrix}$.
			
			\item $\begin{bmatrix} 2 & 1 & 1 \\ 1 & 2 & 1 \\ 1 & 1 & 2 \end{bmatrix}^{-1} = \frac{1}{4}\begin{bmatrix} 3 & -1 & -1 \\ -1 & 3 & -1 \\ -1 & -1 & 3 \end{bmatrix}$; $B \begin{bmatrix} 1\\1\\1 \end{bmatrix} = \begin{bmatrix} 2 & -1 & -1 \\ -1 & 2 & -1 \\ -1 & -1 & 2 \end{bmatrix} \begin{bmatrix} 1\\1\\1 \end{bmatrix} = \begin{bmatrix} 0\\0\\0 \end{bmatrix}$ بنابراین $B^{-1}$ وجود ندارد.
			
			\item $E_{21}A = \begin{bmatrix} 1 & 0 \\ -2 & 1 \end{bmatrix} \begin{bmatrix} 1 & 2 \\ 2 & 6 \end{bmatrix} = \begin{bmatrix} 1 & 2 \\ 0 & 2 \end{bmatrix}$. $E_{12}E_{21}A = \begin{bmatrix} 1 & -1 \\ 0 & 1 \end{bmatrix} \begin{bmatrix} 1 & 2 \\ 0 & 2 \end{bmatrix} = \begin{bmatrix} 1 & 0 \\ 0 & 2 \end{bmatrix}$. با ضرب در $D=\begin{bmatrix} 1 & 0 \\ 0 & 1/2 \end{bmatrix}$ به $DE_{12}E_{21}A=I$ می‌رسیم. آنگاه $A^{-1} = DE_{12}E_{21} = \frac{1}{2}\begin{bmatrix} 6 & -2 \\ -2 & 1 \end{bmatrix}$.
			
			\item $A^{-1} = \begin{bmatrix} 1 & 0 & 0 \\ -2 & 1 & -3 \\ 0 & 0 & 1 \end{bmatrix}$ (به تغییر علامت‌ها توجه کنید)؛ $A^{-1} = \begin{bmatrix} 2 & -1 & 0 \\ -1 & 2 & -1 \\ 0 & -1 & 1 \end{bmatrix}$.
			
			\item $\begin{bmatrix} 0 & 2 & | & 1 & 0 \\ 2 & 2 & | & 0 & 1 \end{bmatrix} \to \begin{bmatrix} 2 & 2 & | & 0 & 1 \\ 0 & 2 & | & 1 & 0 \end{bmatrix} \to \begin{bmatrix} 2 & 0 & | & -1 & 1 \\ 0 & 2 & | & 1 & 0 \end{bmatrix} \to \begin{bmatrix} 1 & 0 & | & -1/2 & 1/2 \\ 0 & 1 & | & 1/2 & 0 \end{bmatrix}$. این $[I|A^{-1}]$ است: جابجایی سطرها قطعاً در روش گاوس-جردن مجاز است.
		\end{enumerate}
		
		
		\begin{enumerate}
			\setcounter{enumi}{28}
			\item (الف) صحیح (اگر A یک سطر صفر داشته باشد، هر AB نیز چنین است و $AB=I$ غیرممکن است).
			(ب) غلط (ماتریس با تمام درایه‌های یک، حتی با ۱ روی قطر، منفرد است).
			(ج) صحیح (معکوس $A^{-1}$ برابر A و معکوس $A^2$ برابر $(A^{-1})^2$ است).
			
			\item حذف، محورهای a، a-b و a-b را تولید می‌کند. $A^{-1} = \frac{1}{a(a-b)}\begin{bmatrix} a & -b & 0 \\ -a & a & 0 \\ 0 & -a & a \end{bmatrix}$. ماتریس C اگر $c=0$ یا $c=7$ یا $c=2$ باشد، معکوس‌پذیر نیست.
			
			\item $A^{-1} = \begin{bmatrix} 1 & -1 & 1 & -1 \\ 0 & 1 & -1 & 1 \\ 0 & 0 & 1 & -1 \\ 0 & 0 & 0 & 1 \end{bmatrix}$ و $x=A^{-1}\begin{bmatrix} 1\\1\\1\\1 \end{bmatrix} = \begin{bmatrix} 0\\0\\0\\1 \end{bmatrix}$. زمانی که ماتریس مثلثی A روی قطرهایش به تناوب ۱ و -۱ داشته باشد، $A^{-1}$ روی قطر و اولین اَبَرقطر (superdiagonal) خود ۱ دارد.
			
			\item $x=(1,1,...,1)$ دارای $x=Px=Qx$ است بنابراین $(P-Q)x=0$. ماتریس‌های جایگشت این بردار تمام-یک را تغییر نمی‌دهند.
			
			\item $\begin{bmatrix} I & 0 \\ -C & I \end{bmatrix}$ و $\begin{bmatrix} A^{-1} & 0 \\ -D^{-1}CA^{-1} & D^{-1} \end{bmatrix}$ و $\begin{bmatrix} -D & I \\ I & 0 \end{bmatrix}$.
		\end{enumerate}
		
		
		\begin{enumerate}
			\setcounter{enumi}{33}
			\item A می‌تواند با درایه‌های قطری صفر معکوس‌پذیر باشد (مثالی برای پیدا کردن). B منفرد است زیرا مجموع هر سطر آن صفر است. بردار تمام-یک x دارای $Bx=0$ است.
			
			\item معادله $LD L^T D = I$ می‌گوید که $LD = \text{pascal}(4,1)$ معکوس خودش است.
			
			\item دستور hilb(6) در متلب ماتریس دقیق هیلبرت را نمی‌سازد زیرا کسرها گرد می‌شوند. بنابراین inv(hilb(6)) نیز معکوس دقیق آن نیست.
			
			\item سه ماتریس پاسکال دارای $P=LU=LL^T$ هستند و سپس $inv(P)=inv(L^T) \times inv(L)$.
			
			\item معادله $Ax=b$ زمانی که $A=\text{ones}(4,4)$ (منفرد) و $b=\text{ones}(4,1)$ باشد، جواب‌های زیادی دارد. دستور $A \setminus b$ در متلب کوتاه‌ترین جواب یعنی $x=(1,1,1,1)/4$ را انتخاب می‌کند. این تنها جوابی است که ترکیبی از سطرهای A است (بعداً از «شبه‌معکوس» $A^+ = \text{pinv}(A)$ به دست می‌آید که جایگزین $A^{-1}$ در زمان منفرد بودن A می‌شود). هر برداری که $Ax=0$ را حل کند، می‌تواند به این جواب خاص x اضافه شود.
			
			\item معکوس $A = \begin{bmatrix} 1 & -a & 0 & 0 \\ 0 & 1 & -b & 0 \\ 0 & 0 & 1 & -c \\ 0 & 0 & 0 & 1 \end{bmatrix}$ برابر است با $A^{-1} = \begin{bmatrix} 1 & a & ab & abc \\ 0 & 1 & b & bc \\ 0 & 0 & 1 & c \\ 0 & 0 & 0 & 1 \end{bmatrix}$. (این مثال خوبی برای فرمول کهاد $A^{-1}=C^T/\det A$ در بخش ۵.۳ خواهد بود).
			
			\item در این ترتیب ضرب، ضرایب a, b, c, d, e, f در حاصلضرب بدون تغییر باقی می‌مانند (این برای تجزیه $A=LU$ در بخش ۲.۶ مهم است).
			
			\item ماتریس ۴×۴ که هنوز $T_{11}=1$ دارد، دارای محورهای ۱,۱,۱,۱ است؛ برعکس کردن به $T^*=UL$ باعث می‌شود $T^*_{44}=1$ شود.
			$T = \begin{bmatrix} 1 & -1 & 0 & 0 \\ -1 & 2 & -1 & 0 \\ 0 & -1 & 2 & -1 \\ 0 & 0 & -1 & 2 \end{bmatrix}$ و $T^{-1} = \begin{bmatrix} 4 & 3 & 2 & 1 \\ 3 & 3 & 2 & 1 \\ 2 & 2 & 2 & 1 \\ 1 & 1 & 1 & 1 \end{bmatrix}$.
			
			\item معادلات $Cx=b$ را جمع کنید تا $0 = b_1+b_2+b_3+b_4$ به دست آید. بنابراین C منفرد است. همین امر برای $Fx=b$ نیز صادق است.
			
			\item محورهای بلوکی A و $S=D-CA^{-1}B$ هستند (و $d-cb/a$ محور دوم صحیح یک ماتریس معمولی ۲×۲ است). مسئله مثال دارای مکمل شور $S = \begin{bmatrix} 1 & 0 \\ 0 & 1 \end{bmatrix} - \begin{bmatrix} 4 \\ 4 \end{bmatrix} \frac{1}{2} [3, 3] = \begin{bmatrix} -5 & -6 \\ -6 & -5 \end{bmatrix}$ است.
			
			\item با معکوس‌گیری از رابطه $A(I+BA)=(I+AB)A$ به $(I+BA)^{-1}A^{-1}=A^{-1}(I+AB)^{-1}$ می‌رسیم. بنابراین $I+BA$ و $I+AB$ هر دو معکوس‌پذیر یا هر دو منفرد هستند زمانی که A معکوس‌پذیر باشد. (این موضوع زمانی که A منفرد است نیز صادق است: فصل ۶ نشان خواهد داد که AB و BA مقادیر ویژه غیرصفر یکسانی دارند، و ما در اینجا به دنبال مقدار ویژه ۱- هستیم.)
		\end{enumerate}
		
	}
\end{document}