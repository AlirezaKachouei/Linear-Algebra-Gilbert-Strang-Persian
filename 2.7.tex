\documentclass[12pt, a4paper]{book}

% فراخوانی بسته‌های لازم
\usepackage{amsmath}         % برای فرمول‌های پیشرفته ریاضی
\usepackage{amsfonts}        % بسته برای فونت‌های ریاضی مانند \mathbb
\usepackage{amssymb}         % برای نمادهای بیشتر ریاضی
\usepackage{graphicx}        % برای افزودن تصاویر
\usepackage{xepersian}       % بسته اصلی برای پارسی‌نویسی
\usepackage{geometry}        % برای تنظیم حاشیه‌ها
\usepackage{setspace}        % برای تنظیم فاصله خطوط
\usepackage{array}           % برای امکانات پیشرفته در جدول‌ها و آرایه‌ها
\usepackage{enumitem}        % برای کنترل بیشتر بر لیست‌ها

% تنظیم حاشیه‌های صفحه
\geometry{
	a4paper,
	total={170mm,257mm},
	left=20mm,
	top=20mm,
}

% تنظیم فونت‌های نوشتاری و ریاضی
% توجه: این فونت‌ها باید روی سیستم شما نصب باشند
\settextfont{XB Niloofar}
\setdigitfont{XB Niloofar}
\setmathdigitfont{XB Niloofar}

\begin{document}
	
	% اعمال فاصله 1.5 بین خطوط برای خوانایی بهتر
	\onehalfspacing
	
	\section*{۲.۷ ترانه‌ها و جایگشت‌ها}
	
	\begin{enumerate}
		\item ترانهاده‌های $A\mathbf{x}$ و $AB$ و $A^{-1}$ به ترتیب $\mathbf{x}^T A^T$ و $B^T A^T$ و $(A^T)^{-1}$ هستند.
		\item حاصل‌ضرب داخلی (dot product) برابر $\mathbf{x} \cdot \mathbf{y} = \mathbf{x}^T\mathbf{y}$ است. این یک ماتریس $(1 \times n)(n \times 1) = (1 \times 1)$ است.
		\item حاصل‌ضرب خارجی برابر $\mathbf{xy}^T$ است = (ستون ضربدر سطر) = $(n \times 1)(1 \times n)$ = ماتریس $n \times n$.
		\item ایده پشت $A^T$ این است که $A\mathbf{x} \cdot \mathbf{y}$ برابر با $\mathbf{x} \cdot A^T\mathbf{y}$ است زیرا $(A\mathbf{x})^T\mathbf{y} = \mathbf{x}^TA^T\mathbf{y} = \mathbf{x}^T(A^T\mathbf{y})$.
		\item یک ماتریس متقارن دارای $S^T=S$ است (و حاصل‌ضرب $A^TA$ همیشه متقارن است).
		\item یک ماتریس متعامد (orthogonal) دارای $Q^T=Q^{-1}$ است. ستون‌های $Q$ بردارهای واحد متعامد هستند.
		\item یک ماتریس جایگشت $P$ همان سطرهای $I$ را (به هر ترتیبی) دارد. $n!$ ترتیب مختلف وجود دارد.
		\item آنگاه $P\mathbf{x}$ مؤلفه‌های $x_1, x_2, \dots, x_n$ را در آن ترتیب جدید قرار می‌دهد و $P^T$ برابر با $P^{-1}$ است.
	\end{enumerate}
	
	ما به یک ماتریس دیگر نیاز داریم و خوشبختانه این ماتریس بسیار ساده‌تر از معکوس است. این ماتریس «ترانهاده» (transpose) $A$ است که با $A^T$ نمایش داده می‌شود. ستون‌های $A^T$ همان سطرهای $A$ هستند.
	وقتی $A$ یک ماتریس $m \times n$ است، ترانهاده آن $n \times m$ است:
	
	\textbf{ترانهاده}
	\[ \text{اگر } A = \begin{bmatrix} 1 & 2 & 3 \\ 4 & 5 & 6 \end{bmatrix} \text{ آنگاه } A^T = \begin{bmatrix} 1 & 4 \\ 2 & 5 \\ 3 & 6 \end{bmatrix} \]
	شما می‌توانید سطرهای $A$ را در ستون‌های $A^T$ بنویسید. یا می‌توانید ستون‌های $A$ را در سطرهای $A^T$ بنویسید. ماتریس حول قطر اصلی خود «برمی‌گردد». درایه واقع در سطر $i$ و ستون $j$ از $A^T$ از سطر $j$ و ستون $i$ ماتریس اصلی $A$ می‌آید:
	
	\textbf{تعویض سطرها و ستون‌ها} \quad $(A^T)_{ij} = A_{ji}$
	
	ترانهاده یک ماتریس پایین‌مثلثی، یک ماتریس بالامثلثی است. (اما معکوس آن همچنان پایین‌مثلثی است.) ترانهاده $A^T$ خود $A$ است.
	
	\textit{(توضیح مترجم: نماد متلب برای ترانهاده $A$، `A'` است. تایپ کردن `[1 2 3]` یک بردار سطری می‌دهد و بردار ستونی `v = [1 2 3]'` است. برای وارد کردن یک ماتریس $M$ که ستون دوم آن `w = [4 5 6]'` باشد، می‌توانید `M = [v w]` را تعریف کنید. راه سریع‌تر، وارد کردن سطرها و سپس ترانهاده کردن کل ماتریس است: `M = [1 2 3; 4 5 6]'`.)}
	
	قوانین ترانهاده بسیار مستقیم هستند. ما می‌توانیم $A+B$ را ترانهاده کنیم تا $(A+B)^T$ را به دست آوریم. یا می‌توانیم $A$ و $B$ را جداگانه ترانهاده کرده و سپس $A^T+B^T$ را جمع کنیم—که نتیجه یکسان است.
	سوالات جدی در مورد ترانهاده حاصل‌ضرب $AB$ و معکوس $A^{-1}$ هستند:
	
	\begin{itemize}
		\item \textbf{جمع:} ترانهاده $A+B$ برابر $A^T+B^T$ است. \quad (۱)
		\item \textbf{حاصل‌ضرب:} ترانهاده $AB$ برابر $B^TA^T$ است. \quad (۲)
		\item \textbf{معکوس:} ترانهاده $A^{-1}$ برابر $(A^{-1})^T = (A^T)^{-1}$ است. \quad (۳)
	\end{itemize}
	
	به خصوص توجه کنید که چگونه $B^TA^T$ به ترتیب معکوس می‌آید. برای معکوس‌ها، بررسی این ترتیب معکوس سریع بود: $B^{-1}A^{-1}$ ضربدر $AB$ برابر $I$ می‌شود. برای درک $(AB)^T = B^TA^T$ ، با $(A\mathbf{x})^T = \mathbf{x}^TA^T$ شروع کنید وقتی $B$ فقط یک بردار $\mathbf{x}$ است:
	\begin{quote}
		$A\mathbf{x}$ ستون‌های $A$ را ترکیب می‌کند در حالی که $\mathbf{x}^TA^T$ سطرهای $A^T$ را ترکیب می‌کند.
	\end{quote}
	این همان ترکیب از همان بردارهاست! در $A$ آنها ستون هستند، در $A^T$ آنها سطر هستند. بنابراین ترانهاده ستون $A\mathbf{x}$ همان سطر $\mathbf{x}^TA^T$ است. این با فرمول ما $(A\mathbf{x})^T = \mathbf{x}^TA^T$ مطابقت دارد. اکنون می‌توانیم فرمول $(AB)^T = B^TA^T$ را ثابت کنیم، زمانی که $B$ چندین ستون دارد.
	اگر $B = [\mathbf{x}_1 \ \mathbf{x}_2]$ دو ستون داشته باشد، همین ایده را برای هر ستون اعمال کنید. ستون‌های $AB$ عبارتند از $A\mathbf{x}_1$ و $A\mathbf{x}_2$. ترانهاده‌های آنها به درستی در سطرهای $B^TA^T$ ظاهر می‌شوند:
	\[ (AB)^T = \begin{bmatrix} (A\mathbf{x}_1)^T \\ (A\mathbf{x}_2)^T \end{bmatrix} = \begin{bmatrix} \mathbf{x}_1^T A^T \\ \mathbf{x}_2^T A^T \end{bmatrix} = \begin{bmatrix} \mathbf{x}_1^T \\ \mathbf{x}_2^T \end{bmatrix} A^T = B^T A^T \quad (۴) \]
	پاسخ صحیح $B^TA^T$ سطر به سطر بیرون می‌آید. در اینجا اعداد در $(AB)^T = B^TA^T$ آمده است:
	\[ A = \begin{bmatrix} 1 & 0 \\ 1 & 1 \end{bmatrix}, B = \begin{bmatrix} 3 & 3 \\ 2 & 2 \end{bmatrix} \implies AB = \begin{bmatrix} 3 & 3 \\ 5 & 5 \end{bmatrix}, (AB)^T = \begin{bmatrix} 3 & 5 \\ 3 & 5 \end{bmatrix} \]
	\[ B^TA^T = \begin{bmatrix} 3 & 2 \\ 3 & 2 \end{bmatrix} \begin{bmatrix} 1 & 1 \\ 0 & 1 \end{bmatrix} = \begin{bmatrix} 3 & 5 \\ 3 & 5 \end{bmatrix} \]
	قانون ترتیب معکوس به سه یا چند عامل گسترش می‌یابد: $(ABC)^T$ برابر $C^TB^TA^T$ است.
	اگر $A=LDU$ آنگاه $A^T=U^TD^TL^T$. ماتریس لولا دارای $D=D^T$ است.
	
	حال این قانون حاصل‌ضرب را با ترانهاده کردن دو طرف $A^{-1}A=I$ اعمال کنید. در یک طرف، $I^T$ همان $I$ است. ما قانونی را تأیید می‌کنیم که $(A^{-1})^T$ معکوس $A^T$ است. حاصل‌ضرب آنها $I$ است:
	\textbf{ترانهاده معکوس} \quad $(A^T)(A^{-1})^T = I$ \quad (۵)
	به طور مشابه $AA^{-1}=I$ منجر به $(A^{-1})^TA^T=I$ می‌شود. ما می‌توانیم ترانهاده را معکوس کنیم یا معکوس را ترانهاده کنیم. به خصوص توجه کنید: $A^T$ دقیقاً زمانی معکوس‌پذیر است که $A$ معکوس‌پذیر باشد.
	
	\textbf{مثال ۱}
	معکوس $A = \begin{bmatrix} 1 & 0 \\ 6 & 1 \end{bmatrix}$ برابر $A^{-1} = \begin{bmatrix} 1 & 0 \\ -6 & 1 \end{bmatrix}$ است. ترانهاده آن $A^T = \begin{bmatrix} 1 & 6 \\ 0 & 1 \end{bmatrix}$ است.
	$(A^{-1})^T$ و $(A^T)^{-1}$ هر دو برابر $\begin{bmatrix} 1 & -6 \\ 0 & 1 \end{bmatrix}$ هستند.
	
	\subsection*{معنای حاصل‌ضرب‌های داخلی}
	ما حاصل‌ضرب داخلی (dot product) $\mathbf{x}$ و $\mathbf{y}$ را می‌شناسیم. این مجموع اعداد $x_iy_i$ است. اکنون روش بهتری برای نوشتن $\mathbf{x} \cdot \mathbf{y}$ داریم، بدون استفاده از آن نقطه غیرحرفه‌ای. به جای آن از نماد ماتریسی استفاده کنید:
	\begin{itemize}
		\item \textbf{حاصل‌ضرب داخلی یا dot product برابر $\mathbf{x}^T\mathbf{y}$ است.} ($T$ در داخل است) \quad $(1 \times n)(n \times 1)$
		\item \textbf{حاصل‌ضرب خارجی یا outer product برابر $\mathbf{xy}^T$ است.} ($T$ در خارج است) \quad $(n \times 1)(1 \times n)$
	\end{itemize}
	$\mathbf{x}^T\mathbf{y}$ یک عدد است، $\mathbf{xy}^T$ یک ماتریس است. مکانیک کوانتومی اینها را به صورت $\langle x|y \rangle$ (داخلی) و $|x \rangle \langle y|$ (خارجی) می‌نویسد. شاید جهان توسط جبر خطی اداره می‌شود. در اینجا سه مثال دیگر وجود دارد که حاصل‌ضرب داخلی در آنها معنا دارد:
	\begin{itemize}
		\item \textbf{از مکانیک:} کار = (جابجایی‌ها) $\cdot$ (نیروها) = $\mathbf{x}^T\mathbf{f}$
		\item \textbf{از مدارها:} اتلاف گرما = (افت ولتاژها) $\cdot$ (جریان‌ها) = $\mathbf{e}^T\mathbf{y}$
		\item \textbf{از اقتصاد:} درآمد = (مقدارها) $\cdot$ (قیمت‌ها) = $\mathbf{q}^T\mathbf{p}$
	\end{itemize}
	ما واقعاً به قلب ریاضیات کاربردی نزدیک هستیم، و یک نکته دیگر برای تأکید وجود دارد. این ارتباط عمیق‌تر بین حاصل‌ضرب‌های داخلی و ترانهاده $A$ است.
	ما $A^T$ را با برگرداندن ماتریس حول قطر اصلی آن تعریف کردیم. این ریاضیات نیست. راه بهتری برای نزدیک شدن به ترانهاده وجود دارد. $A^T$ ماتریسی است که این دو حاصل‌ضرب داخلی را برای هر $\mathbf{x}$ و $\mathbf{y}$ برابر می‌کند:
	\begin{quote}
		$(A\mathbf{x})^T\mathbf{y} = \mathbf{x}^T(A^T\mathbf{y})$ \quad حاصل‌ضرب داخلی $A\mathbf{x}$ با $\mathbf{y}$ = حاصل‌ضرب داخلی $\mathbf{x}$ با $A^T\mathbf{y}$
	\end{quote}
	\textit{(توضیح مترجم: این رابطه تعریف ریاضیاتی و بنیادی ترانهاده است. این رابطه نشان می‌دهد که ماتریس $A^T$ دقیقاً همان ماتریسی است که اثر $A$ را در یک ضرب داخلی، از یک بردار به بردار دیگر منتقل می‌کند.)}
	با $A=\begin{bmatrix} -1 & 1 & 0 \\ 0 & -1 & 1 \end{bmatrix}$ و $\mathbf{y}=\begin{bmatrix} y_1 \\ y_2 \end{bmatrix}$ شروع کنید.
	در یک طرف ما $A\mathbf{x}$ را داریم که در $\mathbf{y}$ ضرب می‌شود: $(x_2-x_1)y_1 + (x_3-x_2)y_2$.
	این همان $x_1(-y_1) + x_2(y_1-y_2) + x_3(y_2)$ است. اکنون $\mathbf{x}$ در $A^T\mathbf{y}$ ضرب می‌شود.
	$A^T\mathbf{y}$ باید $\begin{bmatrix} -y_1 \\ y_1-y_2 \\ y_2 \end{bmatrix}$ باشد که $A^T=\begin{bmatrix} -1 & 0 \\ 1 & -1 \\ 0 & 1 \end{bmatrix}$ را همانطور که انتظار می‌رفت تولید می‌کند.
	
	\subsection*{ماتریس‌های متقارن}
	برای یک ماتریس متقارن، ترانهاده کردن $A$ به $A^T$ تغییری ایجاد نمی‌کند. آنگاه $A^T$ برابر $A$ است. درایه $(j,i)$ آن در آن سوی قطر اصلی با درایه $(i,j)$ آن برابر است. به نظر من، اینها مهمترین ماتریس‌ها هستند. ما به ماتریس‌های متقارن حرف ویژه $S$ را می‌دهیم.
	\textbf{تعریف:} یک ماتریس متقارن دارای $S^T=S$ است. این بدان معناست که $s_{ji}=s_{ij}$.
	
	\textbf{ماتریس‌های متقارن} \quad $S = \begin{bmatrix} 1 & 2 & 5 \\ 2 & 8 & 6 \\ 5 & 6 & 3 \end{bmatrix} \quad D = \begin{bmatrix} 1 & 0 \\ 0 & 10 \end{bmatrix}$
	
	معکوس یک ماتریس متقارن نیز متقارن است. ترانهاده $S^{-1}$ برابر است با $(S^{-1})^T = (S^T)^{-1} = S^{-1}$. این می‌گوید $S^{-1}$ متقارن است (وقتی $S$ معکوس‌پذیر باشد):
	\textbf{معکوس‌های متقارن}
	\[ S = \begin{bmatrix} 1 & 2 \\ 2 & 5 \end{bmatrix} \quad S^{-1} = \begin{bmatrix} 5 & -2 \\ -2 & 1 \end{bmatrix} \]
	اکنون ما با ضرب هر ماتریس $A$ در $A^T$ یک ماتریس متقارن $S$ تولید می‌کنیم.
	
	\textbf{حاصل‌ضرب‌های متقارن $A^TA$ و $AA^T$ و $LDL^T$}
	هر ماتریس $A$ (احتمالاً مستطیلی) را انتخاب کنید. $A^T$ را در $A$ ضرب کنید. آنگاه حاصل‌ضرب $S=A^TA$ به طور خودکار یک ماتریس مربعی متقارن است:
	\begin{quote}
		ترانهاده $A^TA$ برابر $A^T(A^T)^T$ است که دوباره همان $A^TA$ می‌شود. \quad (۶)
	\end{quote}
	این یک اثبات سریع برای تقارن $A^TA$ است. ما می‌توانیم به درایه $(i,j)$ از $A^TA$ نگاه کنیم. این حاصل‌ضرب داخلی سطر $i$ از $A^T$ (ستون $i$ از $A$) با ستون $j$ از $A$ است. درایه $(j,i)$ همان حاصل‌ضرب داخلی است، ستون $j$ با ستون $i$. بنابراین $A^TA$ متقارن است.
	
	ماتریس $AA^T$ نیز متقارن است (شکل $A$ و $A^T$ اجازه ضرب را می‌دهد). اما $AA^T$ ماتریسی متفاوت از $A^TA$ است. در تجربه ما، اکثر مسائل علمی که با یک ماتریس مستطیلی $A$ شروع می‌شوند، به $A^TA$ یا $AA^T$ یا هر دو ختم می‌شوند. مانند روش کمترین مربعات.
	
	\textbf{مثال ۲} $A = \begin{bmatrix} 1 & 2 \\ -1 & 0 \end{bmatrix}$ و $A^T = \begin{bmatrix} 1 & -1 \\ 2 & 0 \end{bmatrix}$ را در هر دو ترتیب ضرب کنید.
	\[ AA^T = \begin{bmatrix} 5 & -1 \\ -1 & 1 \end{bmatrix} \quad \text{و} \quad A^TA = \begin{bmatrix} 2 & 2 \\ 2 & 4 \end{bmatrix} \text{ هر دو ماتریس متقارن هستند.} \]
	حاصل‌ضرب $A^TA$ یک ماتریس $n \times n$ است. در ترتیب مخالف، $AA^T$ یک ماتریس $m \times m$ است. هر دو متقارن هستند، با درایه‌های قطری مثبت (چرا؟). اما حتی اگر $m=n$ باشد، بسیار محتمل است که $A^TA \neq AA^T$. تساوی ممکن است رخ دهد، اما غیرعادی است.
	
	\textbf{ماتریس‌های متقارن در حذف} $S^T=S$ حذف را سریع‌تر می‌کند، زیرا می‌توانیم با نیمی از ماتریس (به علاوه قطر) کار کنیم. درست است که $U$ بالامثلثی احتمالاً متقارن نیست. تقارن در حاصل‌ضرب سه‌گانه $S=LDU$ است. به یاد بیاورید که چگونه ماتریس قطری $D$ از لولاها را می‌توان فاکتور گرفت تا ۱ روی قطر $L$ و $U$ باقی بماند:
	\begin{itemize}
		\item $LU$ تقارن $S$ را از دست می‌دهد.
		\item $LDL^T$ تقارن را حفظ می‌کند.
	\end{itemize}
	حالا $U$ ترانهاده $L$ است.
	
	وقتی $S$ متقارن است، فرم معمول $A=LDU$ به $S=LDL^T$ تبدیل می‌شود. $U$ نهایی (با ۱ روی قطر) ترانهاده $L$ است (که آن هم روی قطر ۱ دارد). ماتریس قطری $D$ که حاوی لولاهاست، به خودی خود متقارن است.
	\begin{quote}
		اگر $S=S^T$ بدون تعویض سطر به $LDU$ تجزیه شود، آنگاه $U$ دقیقاً $L^T$ است. تجزیه متقارن یک ماتریس متقارن $S=LDL^T$ است.
	\end{quote}
	توجه کنید که ترانهاده $LDL^T$ به طور خودکار $(L^T)^TD^TL^T$ است که دوباره همان $LDL^T$ می‌شود. کار حذف از $n^3/3$ ضرب به $n^3/6$ کاهش می‌یابد. فضای ذخیره‌سازی نیز اساساً به نصف کاهش می‌یابد. ما فقط $L$ و $D$ را نگه می‌داریم، نه $U$ را که فقط $L^T$ است.
	
	\subsection*{ماتریس‌های جایگشت}
	ترانهاده نقش ویژه‌ای برای یک ماتریس جایگشت دارد. این ماتریس $P$ در هر سطر و هر ستون یک «۱» دارد. آنگاه $P^T$ نیز یک ماتریس جایگشت است—شاید همان $P$ یا شاید متفاوت. هر حاصل‌ضرب $P_1P_2$ دوباره یک ماتریس جایگشت است.
	
	ما اکنون هر $P$ را از ماتریس همانی، با مرتب‌سازی مجدد سطرهای $I$ ایجاد می‌کنیم.
	ساده‌ترین ماتریس جایگشت $P=I$ است (بدون تعویض). ساده‌ترین‌های بعدی، ماتریس‌های تعویض سطر $P_{ij}$ هستند. اینها با تعویض دو سطر $i$ و $j$ از $I$ ساخته می‌شوند. جایگشت‌های دیگر سطرهای بیشتری را بازآرایی می‌کنند. با انجام تمام تعویض‌های سطری ممکن روی $I$، ما تمام ماتریس‌های جایگشت ممکن را به دست می‌آوریم:
	\textbf{تعریف:} یک ماتریس جایگشت $P$ سطرهای ماتریس همانی $I$ را به هر ترتیبی دارد.
	
	\textbf{مثال ۳} شش ماتریس جایگشت ۳ در ۳ وجود دارد. در اینجا آنها بدون صفرها آمده‌اند:
	\[ I = \begin{bmatrix} 1 & & \\ & 1 & \\ & & 1 \end{bmatrix} \ P_{21} = \begin{bmatrix} & 1 & \\ 1 & & \\ & & 1 \end{bmatrix} \ P_{32} = \begin{bmatrix} 1 & & \\ & & 1 \\ & 1 & \end{bmatrix} \ P_{13} = \begin{bmatrix} & & 1 \\ & 1 & \\ 1 & & \end{bmatrix} \]
	\[ P_{32}P_{21} = \begin{bmatrix} & 1 & \\ & & 1 \\ 1 & & \end{bmatrix} \ P_{21}P_{32} = \begin{bmatrix} & & 1 \\ 1 & & \\ & 1 & \end{bmatrix} \]
	$n!$ ماتریس جایگشت از مرتبه $n$ وجود دارد. نماد $n!$ به معنای «اِن فاکتوریل»، حاصل‌ضرب اعداد $(1)(2)\cdots(n)$ است. بنابراین $3!=(1)(2)(3)$ که ۶ است. ۲۴ ماتریس جایگشت از مرتبه $n=4$ وجود خواهد داشت و ۱۲۰ جایگشت از مرتبه ۵.
	
	تنها دو ماتریس جایگشت از مرتبه ۲ وجود دارد، یعنی $\begin{bmatrix} 1 & 0 \\ 0 & 1 \end{bmatrix}$ و $\begin{bmatrix} 0 & 1 \\ 1 & 0 \end{bmatrix}$.
	\textbf{مهم:} $P^{-1}$ نیز یک ماتریس جایگشت است. در میان شش $P$ ۳ در ۳ که در بالا نمایش داده شد، چهار ماتریس سمت چپ معکوس خودشان هستند. دو ماتریس سمت راست معکوس یکدیگرند. در همه موارد، یک تعویض سطر منفرد معکوس خودش است. اگر تعویض را تکرار کنیم به $I$ برمی‌گردیم. اما برای $P_{32}P_{21}$، معکوس‌ها مثل همیشه به ترتیب مخالف می‌آیند. معکوس آن $P_{21}P_{32}$ است.
	
	\textbf{مهم‌تر:} $P^{-1}$ همیشه همان $P^T$ است. دو ماتریس سمت راست ترانهاده—و معکوس—یکدیگرند. وقتی ما $PP^T$ را ضرب می‌کنیم، «۱» در سطر اول $P$ به «۱» در ستون اول $P^T$ برخورد می‌کند (زیرا سطر اول $P$ همان ستون اول $P^T$ است). این «۱»ها را در تمام ستون‌های دیگر از دست می‌دهد. بنابراین $PP^T=I$.
	اثبات دیگر $P^T=P^{-1}$ به $P$ به عنوان حاصل‌ضرب تعویض‌های سطر نگاه می‌کند. هر تعویض سطر، ترانهاده و معکوس خودش است. $P^T$ و $P^{-1}$ هر دو از حاصل‌ضرب تعویض‌های سطر به ترتیب معکوس به دست می‌آیند. بنابراین $P^T$ و $P^{-1}$ یکسان هستند.
	\begin{quote}
		جایگشت‌ها (تعویض‌های سطر قبل از حذف) به $PA=LU$ منجر می‌شوند.
	\end{quote}
	
	\subsection*{تجزیه $PA=LU$ با تعویض‌های سطر}
	مطمئناً شما $A=LU$ را به یاد دارید. این با $A = (E_{21}^{-1} \cdots E_{32}^{-1} \cdots)U$ شروع شد. هر گام حذف توسط یک $E_{ij}$ انجام می‌شد و با $E_{ij}^{-1}$ معکوس می‌شد. آن معکوس‌ها در یک ماتریس $L$ فشرده شدند. $L$ پایین‌مثلثی روی قطر اصلی ۱ دارد و نتیجه $A=LU$ است.
	
	این یک تجزیه عالی است، اما همیشه کار نمی‌کند. گاهی برای تولید لولا به تعویض سطر نیاز است. آنگاه $A = (E^{-1} \cdots P^{-1} \cdots E^{-1} \cdots P^{-1} \cdots)U$. هر تعویض سطر توسط یک $P_{ij}$ انجام و با همان $P_{ij}$ معکوس می‌شود. ما اکنون آن تعویض‌های سطر را در یک ماتریس جایگشت منفرد $P$ فشرده می‌کنیم. این یک تجزیه برای هر ماتریس معکوس‌پذیر $A$ به ما می‌دهد—که طبیعتاً آن را می‌خواهیم.
	
	سوال اصلی این است که $P_{ij}$ها را کجا جمع کنیم. دو امکان خوب وجود دارد: تمام تعویض‌ها را قبل از حذف انجام دهیم، یا بعد از $E_{ij}$ها. روش اول $PA=LU$ را می‌دهد. روش دوم یک ماتریس جایگشت $P_1$ در وسط دارد.
	\begin{enumerate}
		\item تعویض‌های سطر را می‌توان از قبل انجام داد. حاصل‌ضرب آنها $P$ سطرهای $A$ را به ترتیب صحیح قرار می‌دهد، به طوری که برای $PA$ هیچ تعویضی لازم نیست. آنگاه $PA=LU$.
		\item اگر تعویض‌های سطر را تا بعد از حذف نگه داریم، سطرهای لولا در ترتیبی عجیب قرار دارند. $P_1$ آنها را در ترتیب مثلثی صحیح در $U_1$ قرار می‌دهد. آنگاه $A=L_1P_1U_1$.
	\end{enumerate}
	$PA=LU$ به طور مداوم در تمام محاسبات استفاده می‌شود. ما روی این فرم تمرکز خواهیم کرد.
	تجزیه $A=L_1P_1U_1$ شاید زیباتر باشد. اگر هر دو را ذکر می‌کنیم، به این دلیل است که تفاوت آنها به خوبی شناخته شده نیست. احتمالاً شما زمان زیادی را روی هیچکدام صرف نخواهید کرد. لطفاً نکنید. مهمترین حالت $P=I$ است، زمانی که $A$ برابر $LU$ بدون هیچ تعویضی باشد.
	
	این ماتریس $A$ با $a_{11}=0$ شروع می‌شود. سطرهای ۱ و ۲ را تعویض کنید تا لولای اول به جای معمول خود بیاید. سپس حذف را روی $PA$ انجام دهید:
	\[ A = \begin{bmatrix} 0 & 1 & 1 \\ 1 & 2 & 1 \\ 2 & 7 & 9 \end{bmatrix} \to PA = \begin{bmatrix} 1 & 2 & 1 \\ 0 & 1 & 1 \\ 2 & 7 & 9 \end{bmatrix} \xrightarrow{l_{31}=2} \begin{bmatrix} 1 & 2 & 1 \\ 0 & 1 & 1 \\ 0 & 3 & 7 \end{bmatrix} \xrightarrow{l_{32}=3} \begin{bmatrix} 1 & 2 & 1 \\ 0 & 1 & 1 \\ 0 & 0 & 4 \end{bmatrix} = U \]
	ماتریس $PA$ سطرهای خود را به ترتیب خوبی دارد و طبق معمول به $LU$ تجزیه می‌شود:
	\[ PA = \begin{bmatrix} 1 & 2 & 1 \\ 0 & 1 & 1 \\ 2 & 7 & 9 \end{bmatrix} = \begin{bmatrix} 1 & 0 & 0 \\ 0 & 1 & 0 \\ 2 & 3 & 1 \end{bmatrix} \begin{bmatrix} 1 & 2 & 1 \\ 0 & 1 & 1 \\ 0 & 0 & 4 \end{bmatrix} = LU \quad (۷) \]
	ما با $A$ شروع کردیم و به $U$ ختم شدیم. تنها شرط لازم، معکوس‌پذیری $A$ است.
	\begin{quote}
		اگر $A$ معکوس‌پذیر باشد، یک جایگشت $P$ سطرهای آن را به ترتیب صحیح قرار می‌دهد تا $PA=LU$ تجزیه شود. برای اینکه $A$ معکوس‌پذیر باشد، باید پس از تعویض سطرها، یک مجموعه کامل از لولاها وجود داشته باشد.
	\end{quote}
	در MATLAB، دستور `A([r k], :) = A([k r], :)` سطر $k$ را با سطر $r$ زیر آن تعویض می‌کند (جایی که لولای $k$-ام پیدا شده است). سپس کد `lu` ماتریس‌های $L$ و $P$ و علامت $P$ را به‌روزرسانی می‌کند:
	\begin{verbatim}
		% این بخشی از کد [L, U, P] = lu(A) است
		A([r k], :) = A([k r], :);      % تعویض سطرها
		L([r k], 1:k-1) = L([k r], 1:k-1); % به‌روزرسانی L
		P([r k], :) = P([k r], :);      % به‌روزرسانی P
		sign = -sign;                   % تغییر علامت
	\end{verbatim}
	«علامت» $P$ می‌گوید که آیا تعداد تعویض‌های سطر زوج است (علامت = ۱+) یا خیر. تعداد فرد تعویض‌های سطر، علامت = ۱- را تولید می‌کند. در ابتدا، $P=I$ و علامت = ۱+ است. وقتی یک تعویض سطر وجود دارد، علامت معکوس می‌شود. مقدار نهایی علامت، دترمینان $P$ است و به ترتیب تعویض‌های سطر بستگی ندارد.
	
	برای $PA$ ما به $LU$ آشنا برمی‌گردیم. در واقعیت، کدی مانند `lu(A)` اغلب از اولین لولای موجود استفاده نمی‌کند. از نظر ریاضی ما می‌توانیم یک لولای کوچک را بپذیریم—هر چیزی جز صفر. همه کدهای خوب برای یافتن بزرگترین لولا، ستون را به پایین جستجو می‌کنند.
	بخش ۱۱.۱ توضیح می‌دهد که چرا این «لولایابی جزئی» خطای گردکردن را کاهش می‌دهد. آنگاه $P$ ممکن است شامل تعویض‌های سطری باشد که از نظر جبری ضروری نیستند. با این حال هنوز $PA=LU$ است.
	
	توصیه ما این است که جایگشت‌ها را بفهمید اما اجازه دهید کامپیوتر کار را انجام دهد. محاسبات $A=LU$ برای انجام با دست کافی است، بدون $P$. کد آموزشی `splu(A)` ماتریس $PA=LU$ را تجزیه می‌کند و `splv(A,b)` دستگاه $A\mathbf{x}=\mathbf{b}$ را برای هر ماتریس معکوس‌پذیر $A$ حل می‌کند. برنامه `splu` در وب‌سایت اگر نتواند لولایی در ستون $k$ پیدا کند متوقف می‌شود. آنگاه $A$ معکوس‌پذیر نیست.
	
	\subsection*{مروری بر ایده‌های کلیدی}
	\begin{enumerate}
		\item ترانهاده، سطرهای $A$ را در ستون‌های $A^T$ قرار می‌دهد. آنگاه $(A^T)_{ij}=A_{ji}$.
		\item ترانهاده $AB$ برابر $B^TA^T$ است. ترانهاده $A^{-1}$ معکوس $A^T$ است.
		\item حاصل‌ضرب داخلی $\mathbf{x} \cdot \mathbf{y} = \mathbf{x}^T\mathbf{y}$ است. آنگاه $(A\mathbf{x})^T\mathbf{y}$ برابر با حاصل‌ضرب داخلی $\mathbf{x}^T(A^T\mathbf{y})$ است.
		\item وقتی $S$ متقارن است ($S^T=S$)، تجزیه $LDU$ آن نیز متقارن است: $S=LDL^T$.
		\item یک ماتریس جایگشت $P$ در هر سطر و ستون یک ۱ دارد و $P^T=P^{-1}$ است.
		\item $n!$ ماتریس جایگشت به اندازه $n$ وجود دارد. نیمی زوج، نیمی فرد.
		\item اگر $A$ معکوس‌پذیر باشد، آنگاه یک جایگشت $P$ سطرهای آن را برای تجزیه $PA=LU$ بازآرایی می‌کند.
	\end{enumerate}
	
	\subsection*{مثال‌های حل شده}
	\subsubsection*{مثال ۲.۷ الف}
	اعمال جایگشت $P$ بر سطرهای $S$ تقارن آن را از بین می‌برد:
	\[ P = \begin{bmatrix} 0 & 1 & 0 \\ 0 & 0 & 1 \\ 1 & 0 & 0 \end{bmatrix}, \quad S = \begin{bmatrix} 1 & 4 & 5 \\ 4 & 2 & 6 \\ 5 & 6 & 3 \end{bmatrix}, \quad PS = \begin{bmatrix} 4 & 2 & 6 \\ 5 & 6 & 3 \\ 1 & 4 & 5 \end{bmatrix} \]
	چه جایگشت $Q$ که بر ستون‌های $PS$ اعمال شود، تقارن را در $PSQ$ بازیابی می‌کند؟
	اعداد ۱، ۲، ۳ باید به قطر اصلی بازگردند (نه لزوماً به ترتیب). نشان دهید که $Q=P^T$ است، به طوری که تقارن با $PSP^T$ حفظ می‌شود.
	
	\textbf{راه حل:}
	برای بازیابی تقارن و بازگرداندن «۲» به قطر، ستون ۲ از $PS$ باید به ستون ۱ منتقل شود. ستون ۳ از $PS$ (حاوی «۳») باید به ستون ۲ منتقل شود. آنگاه «۱» به موقعیت (۳,۳) منتقل می‌شود. ماتریسی که ستون‌ها را جایگشت می‌دهد $Q$ است:
	\[ PSQ = \begin{bmatrix} 4 & 2 & 6 \\ 5 & 6 & 3 \\ 1 & 4 & 5 \end{bmatrix} \begin{bmatrix} 0 & 0 & 1 \\ 1 & 0 & 0 \\ 0 & 1 & 0 \end{bmatrix} = \begin{bmatrix} 2 & 6 & 4 \\ 6 & 3 & 5 \\ 4 & 5 & 1 \end{bmatrix} \quad \text{متقارن است.} \]
	ماتریس $Q$ همان $P^T$ است. این انتخاب همیشه تقارن را بازیابی می‌کند، زیرا $PSP^T$ تضمین شده است که متقارن باشد. (ترانهاده آن دوباره $PSP^T$ است.) ماتریس $Q$ همچنین $P^{-1}$ است، زیرا معکوس هر ماتریس جایگشت، ترانهاده آن است.
	
	اگر $D$ یک ماتریس قطری باشد، در می‌یابیم که $PDP^T$ نیز قطری است. وقتی $P$ سطر ۱ را به سطر ۳ منتقل می‌کند، $P^T$ در سمت راست ستون ۱ را به ستون ۳ منتقل می‌کند. درایه (۱,۱) به (۳,۱) و سپس به (۳,۳) منتقل می‌شود.
	
	\subsubsection*{مثال ۲.۷ ب}
	تجزیه متقارن $S=LDL^T$ را برای ماتریس $S$ بالا بیابید.
	
	\textbf{راه حل:} برای تجزیه $S$ به $LDL^T$ ما طبق معمول حذف را انجام می‌دهیم تا به $U$ برسیم:
	\[ S = \begin{bmatrix} 1 & 4 & 5 \\ 4 & 2 & 6 \\ 5 & 6 & 3 \end{bmatrix} \to \begin{bmatrix} 1 & 4 & 5 \\ 0 & -14 & -14 \\ 0 & -14 & -22 \end{bmatrix} \to \begin{bmatrix} 1 & 4 & 5 \\ 0 & -14 & -14 \\ 0 & 0 & -8 \end{bmatrix} = U \]
	مضارب $l_{21}=4$ و $l_{31}=5$ و $l_{32}=1$ بودند. لولاهای ۱، ۱۴-، ۸- به $D$ می‌روند.
	وقتی سطرهای $U$ را بر این لولاها تقسیم کنیم، $L^T$ باید ظاهر شود:
	\[ \textbf{تجزیه متقارن وقتی } S=S^T \quad S = \begin{bmatrix} 1 & 0 & 0 \\ 4 & 1 & 0 \\ 5 & 1 & 1 \end{bmatrix} \begin{bmatrix} 1 & 0 & 0 \\ 0 & -14 & 0 \\ 0 & 0 & -8 \end{bmatrix} \begin{bmatrix} 1 & 4 & 5 \\ 0 & 1 & 1 \\ 0 & 0 & 1 \end{bmatrix} = LDL^T \]
	این ماتریس $S$ معکوس‌پذیر است زیرا سه لولا دارد. معکوس آن $(L^T)^{-1}D^{-1}L^{-1}$ است و $S^{-1}$ نیز متقارن است. اعداد ۱۴ و ۸ در مخرج‌های $S^{-1}$ ظاهر خواهند شد. «دترمینان» $S$ حاصل‌ضرب لولاهاست: $(1)(-14)(-8) = 112$.
	
	\subsubsection*{مثال ۲.۷ ج}
	برای یک ماتریس مستطیلی $A$، این ماتریس نقطه زینی $S$ متقارن و مهم است:
	\[ S = \begin{bmatrix} I & A \\ A^T & 0 \end{bmatrix} = S^T \text{ اندازه } m+n \text{ دارد.} \]
	حذف قطعه‌ای را برای یافتن تجزیه قطعه‌ای $S=LDL^T$ اعمال کنید. سپس معکوس‌پذیری را بیازمایید:
	\begin{quote}
		$S$ معکوس‌پذیر است $\iff$ $A^TA$ معکوس‌پذیر است $\iff$ $A\mathbf{x} \neq \mathbf{0}$ هرگاه $\mathbf{x} \neq \mathbf{0}$
	\end{quote}
	\textbf{راه حل:} اولین لولای قطعه‌ای $I$ است. $A^T$ برابر سطر ۱ را از سطر ۲ کم کنید:
	\textbf{حذف قطعه‌ای}
	\[ S = \begin{bmatrix} I & A \\ A^T & 0 \end{bmatrix} \quad \text{به} \quad \begin{bmatrix} I & A \\ 0 & -A^TA \end{bmatrix} \text{ می‌رود. این } U \text{ است.} \]
	ماتریس لولای قطعه‌ای $D$ شامل $I$ و $-A^TA$ است. آنگاه $L$ و $L^T$ شامل $A^T$ و $A$ هستند:
	\[ S = \begin{bmatrix} I & 0 \\ A^T & I \end{bmatrix} \begin{bmatrix} I & 0 \\ 0 & -A^TA \end{bmatrix} \begin{bmatrix} I & A \\ 0 & I \end{bmatrix} = LDL^T \]
	$L$ قطعاً معکوس‌پذیر است، با ۱‌های قطری. معکوس ماتریس میانی شامل $(A^TA)^{-1}$ است. بخش ۴.۲ به یک سوال کلیدی در مورد ماتریس $A^TA$ پاسخ می‌دهد:
	چه زمانی $A^TA$ معکوس‌پذیر است؟ پاسخ: $A$ باید ستون‌های مستقل داشته باشد.
	آنگاه $A\mathbf{x}=\mathbf{0}$ فقط اگر $\mathbf{x}=\mathbf{0}$ باشد. در غیر این صورت $A\mathbf{x}=\mathbf{0}$ به $A^TA\mathbf{x}=\mathbf{0}$ منجر خواهد شد.
	
	\newpage
	\section*{مجموعه مسائل ۲.۷}
	\begin{enumerate}
		\item[] \textbf{سوالات ۱-۷ در مورد قوانین ماتریس‌های ترانهاده هستند.}
		\item $A^T$ و $A^{-1}$ و $(A^{-1})^T$ و $(A^T)^{-1}$ را برای ماتریس‌های زیر بیابید:
		\[ A = \begin{bmatrix} 1 & 0 \\ 9 & 3 \end{bmatrix} \quad \text{و همچنین} \quad A = \begin{bmatrix} 1 & c \\ c & 0 \end{bmatrix} \]
		
		\item تأیید کنید که $(AB)^T$ برابر $B^TA^T$ است اما این‌ها با $A^TB^T$ متفاوت هستند:
		\[ A = \begin{bmatrix} 1 & 0 \\ 1 & 1 \end{bmatrix}, \quad B = \begin{bmatrix} 1 & 3 \\ 0 & 1 \end{bmatrix}, \quad AB = \begin{bmatrix} 1 & 3 \\ 1 & 4 \end{bmatrix} \]
		همچنین نشان دهید که $AA^T$ با $A^TA$ متفاوت است. اما هر دوی این ماتریس‌ها \underline{متقارن} هستند.
		
		\item (الف) ماتریس $((AB)^{-1})^T$ از $(A^{-1})^T$ و $(B^{-1})^T$ به دست می‌آید. به چه ترتیبی؟
		(ب) اگر $U$ بالامثلثی باشد، آنگاه $(U^{-1})^T$ \underline{پایین‌مثلثی} است.
		
		\item نشان دهید که $A^2=0$ ممکن است اما $A^TA=0$ ممکن نیست (مگر اینکه $A$ ماتریس صفر باشد).
		
		\item (الف) بردار سطری $\mathbf{x}^T$ ضربدر $A$ ضربدر بردار ستونی $\mathbf{y}$ چه عددی را تولید می‌کند؟
		\[ \mathbf{x}^TA\mathbf{y} = \begin{bmatrix} 0 & 1 \end{bmatrix} \begin{bmatrix} 1 & 2 & 3 \\ 4 & 5 & 6 \end{bmatrix} \begin{bmatrix} 0 \\ 1 \\ 0 \end{bmatrix} = \underline{5} \]
		(ب) این برابر است با سطر $\mathbf{x}^TA = \underline{\begin{bmatrix} 4 & 5 & 6 \end{bmatrix}}$ ضربدر ستون $\mathbf{y}=(0,1,0)^T$.
		(ج) این برابر است با سطر $\mathbf{x}^T = \begin{bmatrix} 0 & 1 \end{bmatrix}$ ضربدر ستون $A\mathbf{y} = \underline{\begin{bmatrix} 2 \\ 5 \end{bmatrix}}$.
		
		\item ترانهاده یک ماتریس قطعه‌ای $M = \begin{bmatrix} A & B \\ C & D \end{bmatrix}$ برابر $M^T = \underline{\begin{bmatrix} A^T & C^T \\ B^T & D^T \end{bmatrix}}$ است. یک مثال را بیازمایید. تحت چه شرایطی بر $A,B,C,D$ این ماتریس قطعه‌ای متقارن است؟
		
		\item درست یا غلط:
		\begin{itemize}
			\item[(الف)] ماتریس قطعه‌ای $\begin{bmatrix} 0 & A \\ A^T & 0 \end{bmatrix}$ به طور خودکار متقارن است.
			\item[(ب)] اگر $A$ و $B$ متقارن باشند، حاصل‌ضرب آنها $AB$ متقارن است.
			\item[(ج)] اگر $A$ نامتقارن باشد، $A^{-1}$ نیز نامتقارن است.
			\item[(د)] وقتی $A, B, C$ متقارن هستند، ترانهاده $ABC$ برابر $CBA$ است.
		\end{itemize}
		
		\item[] \textbf{سوالات ۸-۱۵ در مورد ماتریس‌های جایگشت هستند.}
		\item چرا $n!$ ماتریس جایگشت از مرتبه $n$ وجود دارد؟
		
		\item اگر $P_1$ و $P_2$ ماتریس‌های جایگشت باشند، $P_1P_2$ نیز چنین است. این ماتریس هنوز سطرهای $I$ را به ترتیبی دارد. مثال‌هایی با $P_1P_2 \neq P_2P_1$ و $P_3P_4 = P_4P_3$ ارائه دهید.
		
		\item ۱۲ جایگشت «زوج» از $(1,2,3,4)$ وجود دارد که با تعداد زوجی تعویض به دست می‌آیند. دو تای آنها $(1,2,3,4)$ بدون تعویض و $(4,3,2,1)$ با دو تعویض هستند. ده تای دیگر را فهرست کنید. به جای نوشتن هر ماتریس ۴ در ۴، فقط ترتیب اعداد را بنویسید.
		
		\item کدام جایگشت $PA$ را بالامثلثی می‌کند؟ کدام جایگشت‌ها $P_1AP_2$ را پایین‌مثلثی می‌کنند؟ ضرب $A$ از سمت راست در $P_2$ \underline{ستون‌های} $A$ را تعویض می‌کند.
		\[ A = \begin{bmatrix} 0 & 0 & 6 \\ 1 & 2 & 3 \\ 0 & 4 & 5 \end{bmatrix} \]
		
		\item توضیح دهید چرا حاصل‌ضرب داخلی $\mathbf{x}$ و $\mathbf{y}$ برابر با حاصل‌ضرب داخلی $P\mathbf{x}$ و $P\mathbf{y}$ است. آنگاه $(P\mathbf{x})^T(P\mathbf{y})=\mathbf{x}^T\mathbf{y}$ به ما می‌گوید که $P^TP=I$ برای هر جایگشت. با $\mathbf{x}=(1,2,3)$ و $\mathbf{y}=(1,4,2)$، یک $P$ انتخاب کنید تا نشان دهید $P\mathbf{x} \cdot \mathbf{y}$ همیشه با $\mathbf{x} \cdot P\mathbf{y}$ برابر نیست.
		
		\item (الف) یک ماتریس جایگشت ۳ در ۳ بیابید که $P^3=I$ باشد (اما $P \neq I$).
		(ب) یک جایگشت ۴ در ۴ $P$ بیابید که $P^4 \neq I$.
		
		\item اگر $P$ روی پادقطر (antidiagonal) از $(1,n)$ تا $(n,1)$ دارای ۱ باشد، $PAP$ را توصیف کنید. توجه کنید که $P=P^T$.
		
		\item تمام ماتریس‌های تعویض سطر متقارن هستند: $P^T=P$. آنگاه $P^TP=P^2=I$. ماتریس‌های جایگشت دیگر ممکن است متقارن باشند یا نباشند.
		\begin{itemize}
			\item[(الف)] اگر $P$ سطر ۱ را به سطر ۴ بفرستد، آنگاه $P^T$ سطر \underline{۴} را به سطر \underline{۱} می‌فرستد. وقتی $P^T=P$ باشد، تعویض‌های سطر به صورت جفت‌های بدون همپوشانی انجام می‌شوند.
			\item[(ب)] یک مثال ۴ در ۴ با $P^T=P$ بیابید که هر چهار سطر را جابجا کند.
		\end{itemize}
		
		\item[] \textbf{سوالات ۱۶-۲۱ در مورد ماتریس‌های متقارن و تجزیه‌های آنها هستند.}
		\item اگر $A=A^T$ و $B=B^T$ باشد، کدام یک از این ماتریس‌ها قطعاً متقارن هستند؟
		(الف) $A^2-B^2$ \quad (ب) $(A+B)(A-B)$ \quad (ج) $ABA$ \quad (د) $ABAB$.
		
		\item ماتریس‌های متقارن ۲ در ۲ $S=S^T$ با این ویژگی‌ها بیابید:
		\begin{itemize}
			\item[(الف)] $S$ معکوس‌پذیر نیست.
			\item[(ب)] $S$ معکوس‌پذیر است اما نمی‌توان آن را به $LU$ تجزیه کرد (تعویض سطر لازم است).
			\item[(ج)] $S$ را می‌توان به $LDL^T$ تجزیه کرد اما نه به $LL^T$ (به دلیل $D$ منفی).
		\end{itemize}
		
		\item (الف) چند درایه از $S$ را می‌توان به طور مستقل انتخاب کرد، اگر $S=S^T$ یک ماتریس ۵ در ۵ باشد؟
		(ب) چگونه $L$ و $D$ (همچنان ۵ در ۵) همان تعداد انتخاب را در $LDL^T$ می‌دهند؟
		(ج) اگر $A$ پادمتقارن باشد ($A^T=-A$) چند درایه را می‌توان انتخاب کرد؟
		
		\item فرض کنید $A$ مستطیلی ($m \times n$) و $S$ متقارن ($m \times m$) باشد.
		\begin{itemize}
			\item[(الف)] $A^TSA$ را ترانهاده کنید تا تقارن آن را نشان دهید. شکل این ماتریس چیست؟
			\item[(ب)] نشان دهید چرا $A^TA$ هیچ عدد منفی روی قطر اصلی خود ندارد.
		\end{itemize}
		
		\item این ماتریس‌های متقارن را به $S=LDL^T$ تجزیه کنید. ماتریس لولا $D$ قطری است:
		\[ S = \begin{bmatrix} 1 & 3 \\ 3 & 2 \end{bmatrix} \quad \text{و} \quad S = \begin{bmatrix} 1 & b \\ b & c \end{bmatrix} \quad \text{و} \quad S = \begin{bmatrix} 2 & -1 & 0 \\ -1 & 2 & -1 \\ 0 & -1 & 2 \end{bmatrix} \]
		
		\item پس از اینکه حذف ستون ۱ را زیر لولای اول پاک کرد، ماتریس متقارن ۲ در ۲ را که در گوشه پایین سمت راست ظاهر می‌شود، بیابید:
		\[ S = \begin{bmatrix} 2 & 4 & 8 \\ 4 & 3 & 9 \\ 8 & 9 & 0 \end{bmatrix} \quad \text{و} \quad S = \begin{bmatrix} 1 & b & c \\ b & d & e \\ c & e & f \end{bmatrix} \]
		
		\item[] \textbf{سوالات ۲۲-۲۴ در مورد تجزیه‌های $PA=LU$ و $A=L_1P_1U_1$ هستند.}
		\item تجزیه‌های $PA=LU$ را برای ماتریس‌های زیر بیابید (و آنها را بررسی کنید):
		\[ A = \begin{bmatrix} 0 & 1 & 1 \\ 1 & 0 & 1 \\ 2 & 3 & 4 \end{bmatrix} \quad \text{و} \quad A = \begin{bmatrix} 1 & 2 & 0 \\ 2 & 4 & 1 \\ 1 & 1 & 1 \end{bmatrix} \]
		
		\item یک ماتریس جایگشت ۴ در ۴ (آن را $A$ بنامید) بیابید که برای رسیدن به انتهای حذف به ۳ تعویض سطر نیاز دارد. برای این ماتریس، عامل‌های $P, L, U$ آن چه هستند؟
		
		\item ماتریس زیر را به $PA=LU$ تجزیه کنید. آن را همچنین به $A=L_1P_1U_1$ تجزیه کنید (تعویض سطر ۳ را تا زمانی که ۳ برابر سطر ۱ از سطر ۲ کم شود، نگه دارید):
		\[ A = \begin{bmatrix} 0 & 1 & 2 \\ 0 & 3 & 8 \\ 2 & 1 & 1 \end{bmatrix} \]
		
		\item[] \textbf{سوالات چالشی}
		
		\item ثابت کنید که ماتریس همانی نمی‌تواند حاصل‌ضرب سه تعویض سطر (یا پنج تا) باشد. این می‌تواند حاصل‌ضرب دو تعویض (یا چهار تا) باشد.
		
		\item (الف) $E_{21}$ را برای حذف ۳ زیر لولای اول انتخاب کنید. سپس $E_{21}SE_{21}^T$ را ضرب کنید تا هر دو ۳ حذف شوند:
		\[ S = \begin{bmatrix} 1 & 3 & 0 \\ 3 & 11 & 4 \\ 0 & 4 & 9 \end{bmatrix} \text{ به سمت } D = \begin{bmatrix} 1 & 0 & 0 \\ 0 & 2 & 0 \\ 0 & 0 & 1 \end{bmatrix} \text{ می‌رود.} \]
		(ب) $E_{32}$ را برای حذف ۴ زیر لولای دوم انتخاب کنید. سپس $S$ با $E_{32}E_{21}SE_{21}^TE_{32}^T=D$ به $D$ کاهش می‌یابد. $E$ها را معکوس کنید تا $L$ را در $S=LDL^T$ بیابید.
		
		\item اگر هر سطر از یک ماتریس ۴ در ۴ شامل اعداد ۰, ۱, ۲, ۳ به ترتیبی باشد، آیا ماتریس می‌تواند متقارن باشد؟
		
		\item ثابت کنید که هیچ بازآرایی سطرها و بازآرایی ستون‌ها نمی‌تواند یک ماتریس نوعی را ترانهاده کند. (به درایه‌های قطری توجه کنید.)
		
		\item[] \textbf{کاربردها}
		
		\item سیم‌هایی بین بوستون، شیکاگو و سیاتل کشیده شده‌اند. این شهرها در ولتاژهای $x_B, x_C, x_S$ هستند. با مقاومت‌های واحد بین شهرها، جریان‌های بین شهرها در $\mathbf{y}$ هستند:
		\[ \mathbf{y} = A\mathbf{x} \text{ یعنی } \begin{bmatrix} y_{BC} \\ y_{CS} \\ y_{BS} \end{bmatrix} = \begin{bmatrix} 1 & -1 & 0 \\ 0 & 1 & -1 \\ 1 & 0 & -1 \end{bmatrix} \begin{bmatrix} x_B \\ x_C \\ x_S \end{bmatrix} \]
		\begin{itemize}
			\item[(الف)] جریان‌های کل $A^T\mathbf{y}$ را که از سه شهر خارج می‌شوند، بیابید.
			\item[(ب)] تأیید کنید که $(A\mathbf{x})^T\mathbf{y}$ با $\mathbf{x}^T(A^T\mathbf{y})$ موافق است—شش جمله در هر دو.
		\end{itemize}
		
		\item تولید $x_1$ کامیون و $x_2$ هواپیما به $x_1+50x_2$ تن فولاد، $40x_1+1000x_2$ پوند لاستیک و $2x_1+50x_2$ ماه کار نیاز دارد. اگر هزینه‌های واحد $y_1, y_2, y_3$ به ترتیب ۷۰۰ دلار برای هر تن، ۳ دلار برای هر پوند و ۳۰۰۰ دلار برای هر ماه باشند، ارزش یک کامیون و یک هواپیما چقدر است؟ اینها مؤلفه‌های $A^T\mathbf{y}$ هستند.
		
		\item $A\mathbf{x}$ مقادیر فولاد، لاستیک و کار را برای تولید $\mathbf{x}$ در مسئله ۳۰ می‌دهد. $A$ را بیابید. آنگاه $A\mathbf{x} \cdot \mathbf{y}$ \underline{هزینه} ورودی‌ها است در حالی که $\mathbf{x} \cdot A^T\mathbf{y}$ ارزش \underline{خروجی‌ها} است.
		
		\item ماتریس $P$ که $(x,y,z)$ را ضرب می‌کند تا $(z,x,y)$ را بدهد، یک ماتریس دوران نیز هست. $P$ و $P^3$ را بیابید. محور دوران $\mathbf{a}=(1,1,1)$ حرکت نمی‌کند، یعنی $P\mathbf{a}=\mathbf{a}$. زاویه دوران از $\mathbf{v}=(2,3,-5)$ به $P\mathbf{v}=(-5,2,3)$ چقدر است؟
		
		\item $A = \begin{bmatrix} 1 & 2 \\ 4 & 9 \end{bmatrix}$ را به صورت حاصل‌ضرب $ES$ از یک ماتریس عملیات سطری مقدماتی $E$ و یک ماتریس متقارن $S$ بنویسید.
		
		\item در اینجا یک تجزیه جدید از $A$ به $LS$ آمده است: مثلثی (با ۱) ضربدر متقارن.
		از $A=LDU$ شروع کنید. آنگاه $A$ برابر $L(U^T)^{-1}$ ضربدر $S=U^TDU$ است.
		چرا $L(U^T)^{-1}$ مثلثی است؟ قطر آن همگی ۱ است. چرا $U^TDU$ متقارن است؟
		
		\item یک گروه از ماتریس‌ها شامل $AB$ و $A^{-1}$ است اگر شامل $A$ و $B$ باشد. «حاصل‌ضرب‌ها و معکوس‌ها در گروه باقی می‌مانند.» کدام یک از این مجموعه‌ها گروه هستند؟
		ماتریس‌های پایین‌مثلثی $L$ با ۱ روی قطر، ماتریس‌های متقارن $S$، ماتریس‌های مثبت $M$، ماتریس‌های قطری معکوس‌پذیر $D$، ماتریس‌های جایگشت $P$، ماتریس‌های با $Q^T=Q^{-1}$. دو گروه ماتریسی دیگر ابداع کنید.
		
		\item[] \textbf{مسائل بسیار چالشی}
		
		\item یک ماتریس مربع شمال-غربی $B$ در گوشه جنوب-شرقی، زیر پادقطر که $(1,n)$ را به $(n,1)$ متصل می‌کند، صفر است. آیا $B^T$ و $B^2$ ماتریس‌های شمال-غربی خواهند بود؟ آیا $B^{-1}$ شمال-غربی خواهد بود یا جنوب-شرقی؟ شکل $BC$ = (شمال-غربی) ضربدر (جنوب-شرقی) چیست؟
		
		\item اگر توان‌های یک ماتریس جایگشت را بگیرید، چرا سرانجام یک $P^k$ برابر $I$ می‌شود؟ یک جایگشت ۵ در ۵ $P$ بیابید به طوری که کوچکترین توانی که برابر $I$ شود $P^6$ باشد.
		
		\item (الف) هر ماتریس ۳ در ۳ $M$ را بنویسید. $M$ را به $S+A$ تجزیه کنید که در آن $S=S^T$ متقارن و $A=-A^T$ پادمتقارن است.
		(ب) فرمول‌هایی برای $S$ و $A$ شامل $M$ و $M^T$ بیابید. ما می‌خواهیم $M=S+A$.
		
		\item فرض کنید $Q^T=Q^{-1}$ (ترانهاده برابر معکوس، بنابراین $Q^TQ=I$).
		\begin{itemize}
			\item[(الف)] نشان دهید که ستون‌های $\mathbf{q}_1, \dots, \mathbf{q}_n$ بردارهای واحد هستند: $\|\mathbf{q}_i\|^2=1$.
			\item[(ب)] نشان دهید که هر دو ستون $Q$ بر هم عمود هستند: $\mathbf{q}_1^T\mathbf{q}_2=0$.
			\item[(ج)] یک مثال ۲ در ۲ با درایه اول $q_{11}=\cos\theta$ بیابید.
		\end{itemize}
		
		\item[] \textbf{ترانهاده یک مشتق}
		
		آیا به من اجازه کمی حساب دیفرانسیل و انتگرال را می‌دهید؟ این بسیار مهم است وگرنه جبر خطی را ترک نمی‌کردم. (این واقعاً جبر خطی برای توابع $x(t)$ است.) ماتریس به یک مشتق تغییر می‌کند، بنابراین $A=d/dt$. برای یافتن ترانهاده این $A$ غیرمعمول، باید حاصل‌ضرب داخلی بین دو تابع $x(t)$ و $y(t)$ را تعریف کنیم.
		
		حاصل‌ضرب داخلی از مجموع $x_ky_k$ به انتگرال $x(t)y(t)$ تغییر می‌کند.
		\textbf{حاصل‌ضرب داخلی توابع}
		\[ \mathbf{x}^T\mathbf{y} = (x,y) = \int_{-\infty}^{\infty} x(t)y(t) dt \]
		از این حاصل‌ضرب داخلی، ما شرط لازم برای $A^T$ را می‌دانیم. کلمه «الحاقی» (adjoint) صحیح‌تر از «ترانهاده» است وقتی با مشتقات کار می‌کنیم.
		
		ترانهاده یک ماتریس دارای $(A\mathbf{x})^T\mathbf{y} = \mathbf{x}^T(A^T\mathbf{y})$ است. الحاقی $A=d/dt$ دارای:
		\[ (Ax,y) = \int_{-\infty}^{\infty} \frac{dx}{dt} y(t) dt = \int_{-\infty}^{\infty} x(t) \left(-\frac{dy}{dt}\right) dt = (x, A^Ty) \]
		امیدوارم انتگرال‌گیری جزء به جزء را بشناسید. مشتق از تابع اول $x(t)$ به تابع دوم $y(t)$ منتقل می‌شود. در طول این انتقال، یک علامت منفی ظاهر می‌شود. این به ما می‌گوید که ترانهاده مشتق، منفیِ مشتق است.
		
		مشتق پادمتقارن است: $A=d/dt$ و $A^T=-d/dt$. ماتریس‌های متقارن $S^T=S$ دارند، ماتریس‌های پادمتقارن $A^T=-A$ دارند. جبر خطی شامل مشتقات و انتگرال‌ها در فصل ۸ می‌شود، زیرا هر دو خطی هستند.
		
		این پادتقارن مشتق در مورد ماتریس‌های تفاضل مرکزی نیز صدق می‌کند.
		\[ A = \begin{bmatrix} 0 & 1 & 0 \\ -1 & 0 & 1 \\ 0 & -1 & 0 \end{bmatrix} \text{ به } A^T = \begin{bmatrix} 0 & -1 & 0 \\ 1 & 0 & -1 \\ 0 & 1 & 0 \end{bmatrix} = -A \text{ ترانهاده می‌شود.} \]
		و یک ماتریس تفاضل پیشرو به یک ماتریس تفاضل پسرو، ضرب در ۱- ترانهاده می‌شود. در معادلات دیفرانسیل، مشتق دوم (شتاب) متقارن است. مشتق اول (میرایی متناسب با سرعت) پادمتقارن است.
		
	\end{enumerate}
	
\end{document}