\documentclass[12pt]{article}

\usepackage{amsmath}
\usepackage{amsfonts}
\usepackage{amssymb}
\usepackage{xepersian}

\settextfont{XB Niloofar}
\setdigitfont{XB Niloofar}
\setmathdigitfont{XB Niloofar}


\begin{document}
	
	\begin{enumerate}
		\item معادله ۱ را در $ℓ_{21} = \frac{10}{2} = 5$ ضرب کرده و از معادله ۲ کم کنید تا $2x+3y=1$ (بدون تغییر) و $−6y=6$ به دست آید. محورهایی که باید دایره‌ای دورشان کشیده شود ۲ و ۶- هستند.
		\item از $−6y=6$ نتیجه می‌شود $y=−1$. سپس $2x+3y=1$ نتیجه می‌دهد $x=2$. ضرب کردن سمت راست $(1,11)$ در ۴، جواب را در ۴ ضرب می‌کند تا جواب جدید $(x,y)=(8,−4)$ به دست آید.
		\item $-\frac{1}{2}$ (یا $\frac{1}{2}$) برابر معادله ۱ را کم (یا اضافه) کنید. معادله دوم جدید $3y=3$ است. آنگاه $y=1$ و $x=5$. اگر سمت راست علامت عوض کند، جواب نیز چنین می‌کند: $(x,y)=(−5,−1)$.
		\item $ℓ=\frac{c}{a}$ برابر معادله ۱ را از معادله ۲ کم کنید. محور دوم جدیدی که در $y$ ضرب می‌شود $d−(\frac{cb}{a})$ یا $\frac{(ad−bc)}{a}$ است. آنگاه $y=\frac{(ag−cf)}{(ad−bc)}$. به «دترمینان A» = $ad−bc$ توجه کنید. برای این تقسیم باید غیرصفر باشد.
		\item $6x+4y$ دو برابر $3x+2y$ است. جوابی وجود ندارد مگر اینکه سمت راست $2 \cdot 10 = 20$ باشد. آنگاه تمام نقاط روی خط $3x+2y = 10$ جواب هستند، شامل $(0,5)$ و $(4,−1)$. دو خط در تصویر سطری یک خط یکسان هستند که شامل تمام جواب‌هاست.
		\item سیستم منفرد است اگر $b = 4$ باشد، زیرا $4x + 8y$ دو برابر $2x+4y$ است. آنگاه $g = 32$ باعث می‌شود خطوط $2x+4y=16$ و $4x+8y=32$ یکسان شوند: بی‌نهایت جواب مانند $(8,0)$ و $(0,4)$.
		\item اگر $a = 2$ باشد، حذف باید با شکست مواجه شود (دو خط موازی در تصویر سطری). معادلات جوابی ندارند. با $a = 0$، حذف برای تعویض سطر متوقف می‌شود. آنگاه $3y = −3$ نتیجه می‌دهد $y = −1$ و $4x+6y = 6$ نتیجه می‌دهد $x = 3$.
		\item اگر $k = 3$ باشد حذف باید با شکست مواجه شود: جوابی وجود ندارد. اگر $k = −3$ باشد، حذف به $0 = 0$ در معادله ۲ می‌رسد: بی‌نهایت جواب. اگر $k = 0$ باشد نیاز به تعویض سطر است: یک جواب.
		\item در سمت چپ، $6x−4y$ دو برابر $(3x−2y)$ است. بنابراین در سمت راست به $b_2 = 2b_1$ نیاز داریم. آنگاه بی‌نهایت جواب وجود خواهد داشت (دو خط موازی به یک خط واحد در تصویر سطری تبدیل می‌شوند). تصویر ستونی هر دو ستون را در امتداد یک خط دارد.
		\item معادله $y = 1$ از حذف به دست می‌آید (کم کردن $x+y=5$ از $x+2y=6$). آنگاه $x=4$ و $5x−4y = 20−4=c=16$.
		\item (الف) جواب دیگر $\frac{1}{2}(x+X, y+Y, z+Z)$ است. (ب) اگر ۲۵ صفحه در دو نقطه تلاقی کنند، در تمام خطی که از آن دو نقطه می‌گذرد تلاقی می‌کنند.
		\item حذف به این سیستم بالا مثلثی منجر می‌شود؛ سپس جایگزینی معکوس انجام می‌شود. \\
		$2x +3y + z=8$ \\
		$y +3z =4$ \\
		$8z = 8$
		\item 
		$2x−3y=3$ \\
		$4x −5y + z=7$ \\
		$2x − y−3z=5$ \\
		نتیجه می‌دهد \\
		$x =2$ \\
		$y =1$ اگر یک صفر در ابتدای سطر ۲ یا ۳ باشد، \\
		$z =1$ از یک عملیات سطری جلوگیری می‌کند. \\
		$2x −3y =3$ \\
		$y + z=1$ \\
		$2y +3z =2$ \\
		نتیجه می‌دهد \\
		$2x −3y =3$ \\
		$y + z=1$ \\
		$−5z =0$ \\
		و \\
		$x =3$ \\
		$y =1$ \\
		$z =0$ \\
		و \\
		اینجا مراحل ۱، ۲، ۳ آمده است: ۲ × سطر ۱ را از سطر ۲ کم کنید، ۱ × سطر ۱ را از سطر ۳ کم کنید، ۲ × سطر ۲ را از سطر ۳ کم کنید.
		\item ۲ برابر سطر ۱ را از سطر ۲ کم کنید تا به $(d−10)y−z = 2$ برسید. معادله (۳) $y−z = 3$ است. اگر $d=10$ باشد سطرهای ۲ و ۳ را تعویض کنید. اگر $d=11$ باشد سیستم منفرد می‌شود.
		\item موقعیت محور دوم شامل $−2−b$ خواهد بود. اگر $b=−2$ باشد با سطر ۳ تعویض می‌کنیم. اگر $b=−1$ (حالت منفرد) باشد معادله دوم $−y−z=0$ است. اما معادله (۳) همان است، بنابراین یک خط از جواب‌ها $(x,y,z) = (1,1,−1)$ وجود دارد.
		\item (الف) مثالی از ۲ تعویض: \\
		$0x +0y +2z =4$ \\
		$x +2y+2z =5$ \\
		$0x +3y +4z =6$ \\
		(تعویض ۱ و ۲، سپس ۲ و ۳) \\
		(ب) تعویض اما سپس شکست: \\
		$0x +3y +4z =4$ \\
		$x +2y+2z =5$ \\
		$0x +3y +4z =6$ \\
		(سطرهای ۱ و ۳ سازگار نیستند)
		\item اگر سطر ۱ = سطر ۲، آنگاه سطر ۲ پس از مرحله اول صفر است؛ سطر صفر را با سطر ۳ تعویض کنید و سطر ۳ محوری ندارد. اگر ستون ۲ = ستون ۱، آنگاه ستون ۲ محوری ندارد.
		\item مثال $x +2y +3z = 0, 4x+8y+12z = 0, 5x+10y+15z = 0$ دارای ۹ ضریب متفاوت است اما سطرهای ۲ و ۳ به $0=0$ تبدیل می‌شوند: بی‌نهایت جواب برای $Ax=0$ اما تقریباً به طور قطع هیچ جوابی برای $Ax=b$ برای یک $b$ تصادفی وجود ندارد.
		\item سطر ۲ به $3y−4z=5$ تبدیل می‌شود، سپس سطر ۳ به $(q+4)z = t−5$ تبدیل می‌شود. اگر $q=−4$ باشد سیستم منفرد است—محور سومی وجود ندارد. آنگاه اگر $t=5$ باشد معادله سوم $0=0$ است که به بی‌نهایت جواب اجازه می‌دهد. با انتخاب $z=1$، معادله $3y−4z=5$ نتیجه می‌دهد $y=3$ و معادله ۱ نتیجه می‌دهد $x=−9$.
		\item منفرد است اگر سطر ۳ ترکیبی از سطرهای ۱ و ۲ باشد. از نمای انتهایی، سه صفحه یک مثلث تشکیل می‌دهند. این اتفاق می‌افتد اگر سطر ۱ + سطر ۲ = سطر ۳ در سمت چپ باشد اما در سمت راست نه: $x+y+z=0, x−2y−z=1, 2x−y=4$. هیچ صفحه موازی‌ای وجود ندارد اما هنوز جوابی نیست. سه صفحه در تصویر سطری یک تونل مثلثی تشکیل می‌دهند.
		\item (الف) محورها $\frac{5}{4}, \frac{4}{3}, \frac{3}{2}, 2$ در معادلات $2x + y = 0, \frac{3}{2}y +z = 0, \frac{4}{3}z +t = 0, \frac{5}{4}t = 5$ پس از حذف. جایگزینی معکوس نتیجه می‌دهد $t = 4,z = −3,y = 2,x = −1$. (ب) اگر درایه‌های خارج از قطر از +۱ به -۱ تغییر کنند، محورها یکسان هستند. جواب $(1, 2,3,4)$ به جای $(−1,2,−3,4)$ است.
		\item محور پنجم برای هر دو ماتریس (۱ ها یا -۱ ها خارج از قطر) $\frac{6}{5}$ است. محور n-ام $\frac{n+1}{n}$ است.
		\item اگر حذف معمولی به $x+y=1$ و $2y=3$ منجر شود، معادله دوم اصلی می‌توانست $2y+ℓ(x+y)=3+ℓ$ برای هر $ℓ$ باشد. آنگاه $ℓ$ مضربی خواهد بود که با کم کردن $ℓ$ برابر معادله ۱ از معادله ۲، به $2y=3$ می‌رسد.
		\item حذف روی $\begin{bmatrix} a & 2 \\ a & a \end{bmatrix}$ اگر $a=2$ یا $a=0$ باشد با شکست مواجه می‌شود. (می‌توانید توجه کنید که دترمینان $a^2−2a$ برای $a=2$ و $a=0$ صفر است.)
		\item $a=2$ (ستون‌های مساوی)، $a=4$ (سطرهای مساوی)، $a=0$ (ستون صفر).
		\item برای $s=10$ قابل حل است (دو جفت معادله را جمع کنید تا در سمت چپ $a+b+c+d$ و در سمت راست $12$ و $2+s$ به دست آید). بنابراین $12$ باید با $2+s$ برابر باشد، که $s=10$ می‌شود. چهار معادله برای $a,b,c,d$ منفرد هستند! دو جواب عبارتند از
		$\begin{bmatrix} 1 & 3 \\ 1 & 7 \end{bmatrix}$ و $\begin{bmatrix} 0 & 4 \\ 2 & 6 \end{bmatrix}$،
		$A= \begin{bmatrix} 1 & 1 & 0 & 0 \\ 1 & 0 & 1 & 0 \\ 0 & 0 & 1 & 1 \\ 0 & 1 & 0 & 1 \end{bmatrix}$ و $U= \begin{bmatrix} 1 & 1 & 0 & 0 \\ 0 & -1 & 1 & 0 \\ 0 & 0 & 1 & 1 \\ 0 & 0 & 0 & 0 \end{bmatrix}$.
		\item حذف ماتریس قطری $\text{diag}(3,2,1)$ را در $3x=3, 2y=2, z=2$ باقی می‌گذارد. آنگاه $x=1, y=1, z=2$.
		\item دستور $A(2,:)=A(2,:)−3∗A(1,:)$، ۳ برابر سطر ۱ را از سطر ۲ کم می‌کند.
		\item میانگین محورها برای \texttt{rand(3)} بدون تعویض سطر در یک آزمایش $\frac{1}{2}, 5, 10$ بود—اما محورهای ۲ و ۳ می‌توانند به طور دلخواه بزرگ باشند. میانگین آنها در واقع بی‌نهایت است! با تعویض سطر در کد \texttt{lu} متلب، میانگین‌ها $0.75$ و $0.50$ و $0.365$ بسیار پایدارتر هستند (و باید قابل پیش‌بینی باشند، همچنین برای \texttt{randn} با توزیع احتمال نرمال به جای یکنواخت برای اعداد در A).
		\item اگر $A(5,5)$ به جای ۱۱، ۷ باشد، آنگاه محور آخر به جای ۴، ۰ خواهد بود.
		\item سطر $j$ از $U$ ترکیبی از سطرهای $1,...,j$ از $A$ است (وقتی هیچ تعویض سطری وجود ندارد). اگر $Ax=0$ باشد آنگاه $Ux=0$ (اگر $b$ جایگزین $0$ شود درست نیست). $U$ فقط قطر $A$ را نگه می‌دارد وقتی $A$ پایین مثلثی است.
		\item این سوال به ۱۰۰ معادله $Ax=0$ می‌پردازد وقتی $A$ منفرد است.
		(الف) ترکیبی خطی از ۱۰۰ سطر، سطر ۱۰۰ صفر است.
		(ب) ترکیبی خطی از ۱۰۰ ستون، ستون صفرها است.
		(ج) یک ماتریس بسیار منفرد همه درایه‌هایش یک است: $A=\text{ones}(100)$. یک مثال بهتر دارای ۹۹ سطر تصادفی است (یا اعداد $1^i,...,100^i$ در آن سطرها). سطر صدم می‌تواند مجموع ۹۹ سطر اول باشد (یا هر ترکیب دیگری از آن سطرها بدون صفر).
		(د) تصویر سطری دارای ۱۰۰ صفحه است که در امتداد یک خط مشترک از مبدأ می‌گذرند. تصویر ستونی دارای ۱۰۰ بردار است که همگی در یک ابرصفحه ۹۹ بعدی قرار دارند.
		
	\end{enumerate}
	
\end{document}