\documentclass[12pt]{article}

% Import necessary packages
\usepackage{amsmath} % For advanced math environments like bmatrix
\usepackage{amsfonts}
\usepackage{amssymb}
\usepackage{xepersian} % For Persian language support

% Set the Persian font
\settextfont{XB Niloofar}
\setdigitfont{XB Niloofar}
\setmathdigitfont{XB Niloofar}


\begin{document}
	
	\begin{enumerate}
		\item تصویر سطری برای $A = I$ دارای ۳ صفحه عمود بر هم $x = 2$ و $y = 3$ و $z = 4$ است. این صفحات بر محورهای x و y و z عمود هستند: $z = 4$ یک صفحه افقی در ارتفاع ۴ است. بردارهای ستونی $i = (1,0,0)$ و $j = (0,1,0)$ و $k = (0,0,1)$ هستند. آنگاه $b = (2, 3,4)$ ترکیب خطی $2i + 3j + 4k$ است.
		
		\item صفحات در تصویر سطری یکسان هستند: $2x = 4$ همان $x = 2$ است، $3y = 9$ همان $y = 3$ است، و $4z = 16$ همان $z = 4$ است. جواب همان نقطه $X = x$ است. سه بردار ستونی تغییر کرده‌اند؛ اما همان ترکیب (ضرایب z)، بردار $b = (4,9,16)$ را تولید می‌کند.
		
		\item جواب تغییر نمی‌کند! صفحه دوم و سطر ۲ ماتریس و تمام ستون‌های ماتریس (بردارها در تصویر ستونی) تغییر می‌کنند.
		
		\item اگر $z = 2$ باشد، آنگاه $x+y = 0$ و $x-y = 2$ نقطه $(x,y,z) = (1,-1,2)$ را نتیجه می‌دهند. اگر $z = 0$ باشد، آنگاه $x+y = 6$ و $x-y = 4$ نقطه $(5,1,0)$ را تولید می‌کنند. نقطه میانی بین این دو $(3,0,1)$ است.
		
		\item اگر $x,y,z$ در دو معادله اول صدق کنند، در معادله سوم که مجموع دو معادله اول است نیز صدق می‌کنند. خط L جواب‌ها شامل $v = (1,1,0)$ و $w = (\frac{1}{2}, 1, \frac{1}{2})$ و $u = \frac{1}{2}v + \frac{1}{2}w$ و تمام ترکیبات $cv+dw$ با $c+d = 1$ است. (به شرط $c+d=1$ توجه کنید. اگر تمام مقادیر c و d مجاز باشند، یک صفحه به دست می‌آید.)
		
		\item معادله ۱ + معادله ۲ - معادله ۳ اکنون $0 = -4$ است. خط تقاطع L صفحات ۱ و ۲ از صفحه ۳ عبور نمی‌کند: جوابی وجود ندارد.
		
		\item ستون ۳ = ستون ۱ ماتریس را منفرد می‌کند. برای $b = (2,3,5)$ جواب‌ها $(x, y,z) = (1,1,0)$ یا $(0,1,1)$ هستند و شما می‌توانید هر مضربی از $(-1,0,1)$ را به آن اضافه کنید. برای قابل حل بودن $b = (4,6,c)$، باید $c = 10$ باشد (آنگاه b در صفحه ستون‌ها قرار می‌گیرد و مجموع سه معادله $0 = 0$ می‌شود).
		
		\item چهار صفحه در فضای ۴ بعدی به طور معمول در یک نقطه تلاقی می‌کنند. جواب $Ax = (3, 3,3,2)$، $x = (0,0,1,2)$ است اگر A دارای ستون‌های $(1,0,0,0)$، $(1,1,0,0)$، $(1,1,1,0)$، $(1,1,1,1)$ باشد. معادلات $x+y+z+t = 3$، $y+z+t = 3$، $z+t = 3$، $t = 2$ هستند. آن‌ها را به ترتیب معکوس حل کنید!
		
		\item (الف) $Ax=(18,5,0)$ و (ب) $Ax=(3,4,5,5)$.
		
		\item ضرب به عنوان ترکیبات خطی ستون‌ها همان $Ax=(18,5,0)$ و $(3,4,5,5)$ را می‌دهد. با سطرها یا با ستون‌ها: ۹ ضرب جداگانه وقتی A یک ماتریس ۳ در ۳ است.
		
		\item $Ax$ برابر است با $(14,22)$ و $(0,0)$ و $(9,7)$.
		
		\item $Ax$ برابر است با $(z,y,x)$ و $(0,0,0)$ و $(3,3,6)$.
		
		\item (الف) $x$ دارای $n$ مؤلفه و $Ax$ دارای $m$ مؤلفه است. (ب) صفحات حاصل از هر معادله در $Ax=b$ در فضای $n$ بعدی قرار دارند. ستون‌های A در فضای $m$ بعدی قرار دارند.
		
		\item معادله $2x+3y+z+5t=8$ همان $Ax=b$ با ماتریس ۱ در ۴ $A=[2 \ 3 \ 1 \ 5]$ است: یک سطر. جواب‌ها $(x,y,z,t)$ یک «صفحه» سه بعدی را در فضای ۴ بعدی پر می‌کنند. این را می‌توان یک ابرصفحه نامید.
		
		\item (الف) $I= \begin{bmatrix} 1 & 0 \\ 0 & 1 \end{bmatrix}$ = «همانی» (ب) $P= \begin{bmatrix} 0 & 1 \\ 1 & 0 \end{bmatrix}$ = «جایگشت»
		
		\item دوران ۹۰ درجه از $R= \begin{bmatrix} 0 & 1 \\ -1 & 0 \end{bmatrix}$، دوران ۱۸۰ درجه از $R^2= \begin{bmatrix} -1 & 0 \\ 0 & -1 \end{bmatrix} =-I$.
		
		\item ماتریس $P= \begin{bmatrix} 0 & 1 & 0 \\ 0 & 0 & 1 \\ 1 & 0 & 0 \end{bmatrix}$ بردار $\begin{bmatrix} y \\ z \\ x \end{bmatrix}$ را تولید می‌کند و $Q= \begin{bmatrix} 0 & 0 & 1 \\ 1 & 0 & 0 \\ 0 & 1 & 0 \end{bmatrix}$ بردار $\begin{bmatrix} x \\ y \\ z \end{bmatrix}$ را بازیابی می‌کند. $Q$ معکوس $P$ است. بعداً خواهیم نوشت $QP=I$ و $Q=P^{-1}$.
		
		\item ماتریس $E= \begin{bmatrix} 1 & 0 \\ -1 & 1 \end{bmatrix}$ و $E= \begin{bmatrix} 1 & 0 & 0 \\ -1 & 1 & 0 \\ 0 & 0 & 1 \end{bmatrix}$ مؤلفه اول را از مؤلفه دوم کم می‌کنند.
		
		\item ماتریس $E= \begin{bmatrix} 1 & 0 & 0 \\ 0 & 1 & 0 \\ 1 & 0 & 1 \end{bmatrix}$ و $E^{-1}= \begin{bmatrix} 1 & 0 & 0 \\ 0 & 1 & 0 \\ -1 & 0 & 1 \end{bmatrix}$، $Ev= \begin{bmatrix} 3 \\ 4 \\ 8 \end{bmatrix}$ و $E^{-1}Ev$ بردار $\begin{bmatrix} 3 \\ 4 \\ 5 \end{bmatrix}$ را بازیابی می‌کند.
		
		\item ماتریس $P_1= \begin{bmatrix} 1 & 0 \\ 0 & 0 \end{bmatrix}$ بر روی محور x تصویر می‌کند و $P_2= \begin{bmatrix} 0 & 0 \\ 0 & 1 \end{bmatrix}$ بر روی محور y تصویر می‌کند. بردار $v= \begin{bmatrix} 5 \\ 7 \end{bmatrix}$ به $P_1v= \begin{bmatrix} 5 \\ 0 \end{bmatrix}$ تصویر می‌شود و $P_2P_1v= \begin{bmatrix} 0 \\ 0 \end{bmatrix}$.
		
		\item ماتریس $R= \frac{1}{2} \begin{bmatrix} \sqrt{2} & -\sqrt{2} \\ \sqrt{2} & \sqrt{2} \end{bmatrix}$ تمام بردارها را به اندازه $45^\circ$ می‌چرخاند. ستون‌های R نتایج دوران بردارهای $(1,0)$ و $(0,1)$ هستند!
		
		\item حاصلضرب نقطه‌ای $Ax=[1 \ 4 \ 5] \begin{bmatrix} x \\ y \\ z \end{bmatrix}$ (یک ماتریس ۱ در ۳ در یک ماتریس ۳ در ۱) برای نقاط $(x,y,z)$ روی یک صفحه در فضای سه بعدی صفر است. ۳ ستون A بردارهای یک بعدی هستند.
		
		\item $A=[1 \ 2 ; 3 \ 4]$ و $x=[5 \ -2]'$ یا $[5 ; -2]$ و $b=[1 \ 7]'$ یا $[1 ; 7]$. دستور $r=b-A*x$ دو صفر را چاپ می‌کند.
		
		\item $A*v=[3 \ 4 \ 5]'$ و $v'*v=50$. اما $v*A$ به دلیل ضرب یک ماتریس ۳ در ۱ در یک ماتریس ۳ در ۳ پیغام خطا می‌دهد.
		
		\item دستور $\text{ones}(4,4)*\text{ones}(4,1)$ برابر است با بردار ستونی $[4 \ 4 \ 4 \ 4]'$؛ $B*w=[10 \ 10 \ 10 \ 10]'$.
		
		\item تصویر سطری دو خط را نشان می‌دهد که در جواب $(4,2)$ تلاقی می‌کنند. تصویر ستونی $4(1,1)+2(-2,1)=4(\text{ستون ۱})+2(\text{ستون ۲}) = \text{طرف راست } (0,6)$ را خواهد داشت.
		
		\item تصویر سطری ۲ صفحه را در فضای ۳ بعدی نشان می‌دهد. تصویر ستونی در فضای ۲ بعدی است. جواب‌ها به طور معمول یک خط را در فضای ۳ بعدی پر می‌کنند.
		
		\item تصویر سطری چهار خط را در صفحه دو بعدی نشان می‌دهد. تصویر ستونی در فضای چهار بعدی است. جوابی وجود ندارد مگر اینکه طرف راست ترکیبی از دو ستون باشد.
		
		\item $u_2= \begin{bmatrix} .7 \\ .3 \end{bmatrix}$ و $u_3= \begin{bmatrix} .65 \\ .35 \end{bmatrix}$. مجموع مؤلفه‌ها ۱ است. آنها همیشه مثبت هستند. مجموع مؤلفه‌های آنها همچنان ۱ است.
		
		\item $u_7$ و $v_7$ مؤلفه‌هایی دارند که مجموع آنها ۱ است؛ آنها به $s=(.6,.4)$ نزدیک هستند.
		$\begin{bmatrix} .8 & .3 \\ .2 & .7 \end{bmatrix} \begin{bmatrix} .6 \\ .4 \end{bmatrix} = \begin{bmatrix} .6 \\ .4 \end{bmatrix}$ = حالت پایدار. با ضرب در $\begin{bmatrix} .8 & .3 \\ .2 & .7 \end{bmatrix}$ تغییری نمی‌کند.
		
		\item $M= \begin{bmatrix} 8 & 3 & 4 \\ 1 & 5 & 9 \\ 6 & 7 & 2 \end{bmatrix} = \begin{bmatrix} 5+u & 5-u+v & 5-v \\ 5-u-v & 5 & 5+u+v \\ 5+v & 5+u-v & 5-u \end{bmatrix}$؛ $M_3(1,1,1)=(15,15,15)$؛ $M_4(1,1,1,1)=(34,34,34,34)$ زیرا $1+2+...+16=136$ که برابر با $4(34)$ است.
		
		\item ماتریس A زمانی منفرد است که ستون سوم آن $w$ ترکیبی از ستون‌های اول به صورت $cu+dv$ باشد. یک تصویر ستونی معمول دارای $b$ خارج از صفحه $u,v,w$ است. یک تصویر سطری معمول دارای خط تقاطع دو صفحه موازی با صفحه سوم است. در این صورت جوابی وجود ندارد.
		
		\item $w=(5,7)$ برابر است با $5u+7v$. آنگاه $Aw$ برابر است با ۵ برابر $Au$ به علاوه ۷ برابر $Av$. خطی بودن به این معناست: وقتی $w$ ترکیبی از $u$ و $v$ باشد، آنگاه $Aw$ همان ترکیب از $Au$ و $Av$ است.
		
		\item معادله $\begin{bmatrix} 2 & -1 & 0 & 0 \\ -1 & 2 & -1 & 0 \\ 0 & -1 & 2 & -1 \\ 0 & 0 & -1 & 2 \end{bmatrix} \begin{bmatrix} x_1 \\ x_2 \\ x_3 \\ x_4 \end{bmatrix} = \begin{bmatrix} 1 \\ 2 \\ 3 \\ 4 \end{bmatrix}$ دارای جواب $\begin{bmatrix} x_1 \\ x_2 \\ x_3 \\ x_4 \end{bmatrix} = \begin{bmatrix} 4 \\ 7 \\ 8 \\ 6 \end{bmatrix}$ است.
		
		\item $x=(1,...,1)$ نتیجه می‌دهد $Sx$ = مجموع هر سطر = $1+...+9=45$ برای ماتریس‌های سودوکو. ۶ ترتیب سطر $(1,2,3), (1,3,2), (2,1,3), (2,3,1), (3,1,2), (3,2,1)$ در بخش ۲.۷ آمده است. همان ۶ جایگشت بلوک‌های سطرها ماتریس‌های سودوکو تولید می‌کنند، بنابراین $6^4=1296$ ترتیب از ۹ سطر همگی سودوکو باقی می‌مانند. (و همچنین ۱۲۹۶ جایگشت از ۹ ستون).
		
	\end{enumerate}
	
\end{document}